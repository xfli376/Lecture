\documentclass[11pt,twoside]{article}

\usepackage{2in1}% 两页放到一页上

\usepackage{xeCJK}

\usepackage{lastpage}

%\usepackage{times} %use the Times New Roman fonts

\usepackage{color}

\usepackage{placeins}

\usepackage{ulem}

\usepackage{titlesec}

\usepackage{graphicx}

\usepackage{caption,subcaption}

\usepackage{colortbl}

\usepackage{listings}

\usepackage{makecell}

\usepackage{indentfirst}

\usepackage{fancyhdr}

\usepackage{setspace} % 行间距

\usepackage{bm}%\boldsymbol 粗体

\usepackage{amsmath,amsfonts,amsmath,times}

\usepackage{amssymb}% 数学

\usepackage{enumerate}% 编号

%\usepackage[paperwidth=18.4cm,paperheight=26cm,top=1.5cm,bottom=2cm,right=2cm]{geometry} % 单页

\usepackage[paperwidth=36.8cm,paperheight=26cm,top=2.5cm,bottom=2cm,right=2cm]{geometry}

\lstset{language=C,keywordstyle=\color{red},showstringspaces=false,rulesepcolor=\color{green}}

\oddsidemargin=0.5cm %奇数页页边距

\evensidemargin=0.5cm %偶数页页边距

%\textwidth=14.5cm %文本的宽度 单页

\textwidth=30cm %文本的宽度 双页

\newsavebox{\zdx}

\newcommand{\putzdx}{\marginpar{

\parbox{1cm}{\vspace{-1.6cm}

\rotatebox[origin=c]{90}{

\usebox{\zdx}

}}

}}

\newcommand{\blank}{\uline{\textcolor{white}{a}\ \textcolor{white}{a}\ \textcolor{white}{a}\ \textcolor{white}{a}\ \textcolor{white}{a}\ \textcolor{white}{a}\ \textcolor{white}{a}\ \textcolor{white}{a}\ \textcolor{white}{a}\ \textcolor{white}{a}\ \textcolor{white}{a}}}

\newcommand{\me}{\mathrm{e}} %定义 对数常数e,虚数符号i,j以及微分算子d为直立体。

\newcommand{\mi}{\mathrm{i}}

\newcommand{\mj}{\mathrm{j}}

\newcommand{\dif}{\mathrm{d}}

\newcommand{\bs}{\boldsymbol}%数学黑体

%通常我们使用的分数线是系统自己定义的分数线,即分数线的长度的预设值是分子或分母所占的最大宽度,如何让分数线的长度变长成,我们%可以在分子分母添加间隔来实现。如中文分式的命令可以定义为:

%\newcommand{\chfrac[2]}{\cfrac{;#1;}{;#2;}}

%\frac{1}{2} \qquad \chfrac{1}{2}

%选择题

\newcommand{\fourch}[4]{\\begin{tabular}{{4}{@{}p{3.5cm}}}(A)~#1 & (B)~#2 & (C)~#3 & (D)~#4\end{tabular}} % 四行

\newcommand{\twoch}[4]{\
\begin{tabular}{{2}{@{}p{7cm}}}(A)~#1 & (B)~#2\end{tabular}
\\begin{tabular}{*{2}{@{}p{7cm}}}(C)~#3 &

(D)~#4\end{tabular}} %两行

\newcommand{\onech}[4]{\ (A)~#1 \ (B)~#2 \ (C)~#3 \ (D)~#4} % 一行

\renewcommand{\headrulewidth}{0pt}

\pagestyle{fancy}

\begin{document}

\fancyhf{}

\fancyfoot[CO,CE]{\vspace{1mm}第,\thepage,页 , 共 ~\pageref{LastPage} 页}

\sbox{\zdx}

{\parbox{27cm}{

\centering 班~级\underline{\makebox[34mm][c]{}}~\CJKfamily{song}学~号\underline{\makebox[34mm][c]{}}~\CJKfamily{song}姓~名\underline{\makebox[34mm][c]{}} ~ 试卷序号~\underline{\makebox[34mm][c]{}}\

\vspace{3mm}

密封位置线内不要答题,填写姓名、学号、班级和试卷序号。\

%答题时学号

\vspace{1mm}

\dotfill{} 装\dotfill{}订\dotfill{}线\dotfill{} \

}}

\reversemarginpar

\begin{spacing}{1.25}

\begin{center}

\begin{LARGE}

\CJKfamily{li}世界一流大学\

\underline{2016~-- 2017 }学年第~\underline{1} 学期期末考试\

\underline{《电磁学》(150667)}试卷(A) 答题时间:\underline{120}分钟\

使用班级:\underline{1550116} \quad 考试方式:\underline{闭卷}\

\end{LARGE}

\vspace{0.5cm}

\begin{tabular}{|p{0.05\textwidth}|{7}{p{0.05\textwidth}|}p{0.08\textwidth}|}

\hline

\centering ~题~号 & \centering 一 & \centering 二 & \centering 三 %& \centering 六 & \centering 七 & \centering 八 & \centering 九 & \centering 十

& ~~ 总~分 \rule{0pt}{6mm} \

\hline

\centering ~分~数 & & & & % & % & & & &

\rule{0pt}{6mm} \\hline

% \centering 计 & & & & & & & & & & & \

% \centering 分 & & & & & & & & & & & \

% \centering 人 & & & & & & & & & & & \ \hline

\end{tabular}

\end{center}

\end{spacing}

\setlength{\marginparsep}{1.7cm}

\putzdx %%装订线--奇页数

\begin{spacing}{1.3}

\section{\hspace{5cm} 一、填空题~(每空~2 分,共~18 分)}

\vspace{-2cm}

\begin{tabular}{|p{0.05\textwidth}|p{0.05\textwidth}|}

\hline

% after \: \hline or \cline{col1-col2} \cline{col3-col4} ...

\centering 阅卷人& \

\hline

\centering 得~~分 & \

\hline

\end{tabular}

\begin{enumerate}%[(1)]

\item 一线圈与电阻组成的LR电路中,线圈自感为$L$,电阻为$R$,电路的时间常量为~\blank .%$L/R$

\item 一载流圆线圈,电流为$I$,半径为$a$,则在圆心处磁感应强度大小为~\blank.%$\mu_0 I/(2a)$

\item 电量和符号都相同的三个点电荷$q$ 放在等边三角形的顶点上.为了不使它们由于斥力的作用而散开,可在三角形的中心放一符号相反的点电荷,则此点电荷的电量为~\blank.%$-q/\sqrt{3}$

\item 质子束以$3.0\times 10^5 \mathrm {m/s}$的速率穿过磁感应强度大小为2.0T的匀强磁场,速度方向与磁感应强度方向夹角为30°,则一个质子受力为~\blank. %$4.8\times 10^{-14}\mathrm N$

\item 纸筒长30cm,半径为3.0cm,密绕500匝线圈,则线圈的自感为~\blank.

\item 电荷量为~$q$的自由电荷放入电容率为~$\varepsilon$的无穷大的电介质中,距此电荷~$r$处的电场强度的大小为~\blank .% $\frac{q}{4\pi\varepsilon r^2}$

\item 已知电量为~$Q$的均匀带电圆环在轴线上的电势为$U=\frac{Q}{4\pi\varepsilon_0 \sqrt{x^2+a^2 }}$,其中~$a$为带电圆环半径,$x$ 为场点到带电圆环中心的距离,则场点$x$处电场强度为~\blank .%$\frac{Qx}{4π\varepsilon_0 (x^2+a^2 )^{3/2}}$

\item 已知一螺绕环匝数为$N$,半径为$R$,电流为$I$,则内部磁感应强度为~\blank.%$\mu_0NI/(2\pi R)$

\item 如图所示,通电螺线管外绕有一匝线圈,通电螺线管电流随时间变化,可以发现线圈中有电流产生,即线圈中会产生电动势,问此电动势的非静电力的来源为~\blank .

\begin{figure}[h]

\centering

% Requires \usepackage{graphicx}

\includegraphics[scale=0.5]{Induced-electric-field.png}\

\caption{第9题图}

\end{figure}

\end{enumerate}

\section{\hspace{5cm} 二、选择题~(每题~3 分, 共~ 27 分)}

\vspace{-2cm}

\begin{tabular}{|p{0.05\textwidth}|p{0.05\textwidth}|}

\hline

% after \: \hline or \cline{col1-col2} \cline{col3-col4} ...

\centering 阅卷人& \

\hline

\centering 得~~分 & \

\hline

\end{tabular}

\begin{enumerate}\setcounter{enumi}{9}

\item 有两条长直导线,各载有5A的电流,分别沿x、y轴正方向流动,则在点(0.4m, 0.2m)处磁感应强$B$大小为~(\hspace{4em})%\[5pt]

\fourch{$3.5\times 10^{-6}\mathrm T$ }{$2.5\times 10^{-6}\mathrm T$ }{$4.5\times 10^{-6}\mathrm T$ }{$5.5\times 10^{-6}\mathrm T$ }%B

\begin{figure}[h]

\centering

% Requires \usepackage{graphicx}

\includegraphics[scale=0.2]{Eflux.png}\

\caption{第11题图}

\end{figure}

\item 如图所示,闭合曲面$S$ 内有一点电荷$q$ ,$P$为$S$ 面上一点,在$S$ 面外$A$点有一点电荷$q'$ ,若将$q'$ 移至$B$ 点,则~(\hspace{4em})

\onech{穿过$S$ 面的电通量改变,$P$ 点的电场强度不变.}{穿过$S$ 面的电通量不变,$P $点的电场强度改变.}{穿过$S $面的电通量和$P$ 点的电场强度都不变.}{穿过$S$ 面的电通量和$P$ 点的电场强度都改变.}%B

\newpage

\putzdx %偶数页装订线

\item 如图所示,导体棒$AB$在均匀磁场中绕$OO’$轴匀速转动,转轴方向与磁感应强度方向平行。$OO’$轴与导体棒$AB$垂直,且通过其中点C。 导体棒端点A和B与中点C的电势分别为$u_A$、$u_B$、$u_C$,则下列关系成立的是~(\hspace{4em})

\begin{figure}[h]

\centering

% Requires \usepackage{graphicx}

\includegraphics[scale=0.2]{rodrotate.png}\

\caption{第12题图}

\end{figure}

\fourch{$u_A>u_B$.}{$u_B>u_C$.}{$u_C>u_A$.}{$u_A=u_B=u_C$.}%B

\item 当一个带电体达到静电平衡时:~(\hspace{4em})

\onech{表面上电荷密度较大处电势较高 }{表面曲率较大处电势较高 }{导体内部的电势比导体表面的电势高 }{导体内任一点与其表面上任一点的电势差等于零}%D

\item 两个闭合的金属环,穿在一光滑的绝缘杆上,如习题图所示.当条形磁铁N极自右向左插向圆环时,两圆环的运动是:~(\hspace{4em})

\twoch{边向左移动边分开}{边向右移动边合拢}{边向左移动边合拢}{边向右移动边分开}\

\begin{figure}[h]

\centering

% Requires \usepackage{graphicx}

\includegraphics[scale=0.2]{tworings.png}\

\caption{第14题图}

\end{figure}

\item 下列物质是铁磁物质的是: ~(\hspace{4em})

\fourch{锰 }{铜} {铁 }{水}%C

\item 如图所示,在点电荷$q$的电场中有1, 2, 3 三个点, $U_{1,2}$表示点1和2之间的电势差,$U_{1,3}$表示点1和3之间的电势差,$U_{2,3}$ 表示点2和3之间的电势差,则以下关系成立的是: ~(\hspace{4em})

\twoch{$U_{1,2}<U_{1,3}=U_{2,3}$ }{$U_{1,3}>U_{2,3}>U_{1,2}$} {$U_{1,2}>U_{1,3}>U_{2,3}$ }{$ U_{2,3}>U_{1,3}>U_{1,2}$}

\begin{figure}[h]

\centering

% Requires \usepackage{graphicx}

\includegraphics[scale=0.3]{q123.png}\

\caption{第16题图}

\end{figure}

\item 带负电粒子垂直进入一个磁场,速度和受力方向如图所示,则磁感应强度方向为: ~(\hspace{4em})

\fourch{水平向左}{竖直向上}{垂直纸面向里}{垂直纸面向外} %C

\begin{figure}[h]

\begin{minipage}[t]{0.5\linewidth}

\centering

\includegraphics[scale=0.3]{vFB.png}\

\caption{第17题图}

\end{minipage}

\begin{minipage}[t]{0.5\linewidth}

\centering

\includegraphics[scale=0.3]{rectB.png}\

\caption*{第18题图}

\end{minipage}

\end{figure}

\item 如图所示,一矩形导体线框,有一半处于磁场中,从某时刻开始,磁感应强度大小开始增大,方向不变,则以下关于导体线框说法正确的是 ~(\hspace{4em})

\twoch{导体线框会向纸面上方运动}{导体线框会向纸面下方运动}{导体线框会向右运动}{导体线框不运动}%C

\end{enumerate}

\newpage

\putzdx %%装订线--奇页数

\section{\hspace{5cm}三、计算题~(共~55 分。)}

\vspace{-2cm}

\begin{tabular}{|p{0.05\textwidth}|p{0.05\textwidth}|}

\hline

% after \: \hline or \cline{col1-col2} \cline{col3-col4} ...

\centering 阅卷人& \

\hline

\centering 得~~分 & \

\hline

\end{tabular}

\begin{enumerate}\setcounter{enumi}{18}

\item (本题14分)如图所示,长直载流直导线与直角三角形回路abc共面,初始时刻,长为~$l_1$~的直角边与直导线距离为~$d$~,直角边bc 长为~$l_2$。 现在使回路以速率~$v$~向右运动,求在时刻~$t$~回路中的动生电动势。% 大学物理学习指导

\begin{figure}[h]

\raggedleft%centering

% Requires \usepackage{graphicx}

\includegraphics[scale=0.2]{triangleB.png}

%\caption{第19题图}

\end{figure}

%\vspace{2cm}

\item (本题14分)如图所示,一均匀带电的无限长直线段,电荷线密度$\lambda_1$ , 另有一均匀带电直线段, 长度为$l$ ,电荷密度为$\lambda_2$ ,两线互相垂直且共面,若带电线段近端距长直导线为$a$。求两带电线之间的相互作用力。%电磁学复习题与答案

\begin{figure}[h]

\raggedleft%centering

% Requires \usepackage{graphicx}

\includegraphics[scale=0.5]{twochargedrod.png}

%\caption*{第19题图}

\end{figure}

\end{enumerate}

\newpage

\begin{enumerate}\setcounter{enumi}{20}

\item (本题14分)如图,一半径为$R_1$ 的长直导线的外面,套有内半径为$R_2$ 的同轴薄导体圆筒,二者构成圆柱形电容器,它

们之间充以相对介电常数为$\varepsilon_r$ 的均匀电介质.设导线和圆筒都均匀带电,且沿轴

线单位长度带电量分别为$+\lambda$ 和$-\lambda$.求(1)电场的空间分布,(2)导线与圆筒之间的电势差,(3)单位长度的电容。

\begin{figure}[h]

\raggedleft%centering

% Requires \usepackage{graphicx}

\includegraphics[scale=0.5]{cylinder.png}

%\caption{第19题图}

\end{figure}

\item (本题13分)电缆由导体圆柱和一同轴的导体圆筒构成,使用时电流$I$ 从导体流出,从另一导体流回,电流均匀分布在横截面上,如图所示,设圆柱体的半径为$r_1$ ,圆筒的内外半径分别为$r_2$和$r_3$,根据安培环路定理,求磁感应强度的空间分布.%大学物理学习指导PDF216

\begin{figure}[h]

\raggedleft%centering

% Requires \usepackage{graphicx}

\includegraphics[scale=0.2]{ampere.png}

%\caption{第19题图}

\end{figure}

\end{enumerate}

\end{spacing}

\clearpage

\end{document}