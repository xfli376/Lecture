
\section{2.厄密算符性质}

\begin{frame}
    \frametitle{前情回顾}
    \begin{itemize}
        \Item 波函数描述体系的状态
        \Item 体系的物理量用厄密算符表示
    \end{itemize}
\end{frame} 

\subsection{运算性质}

\begin{frame}
    \frametitle{运算性质}
    \begin{enumerate}
        \Item 两厄米算符之和仍为厄米算符
        \Item 当且仅当两厄米算符对易时,它们之积才是厄米算符。
        \Item 无论两厄米算符是否对易,算符$\dfrac{1}{2}(AB+BA)$ 及$\dfrac{1}{2i}(AB-BA) $  都是厄米算符。
        \Item 任意算符总可以分解成$A=A_+ +iA_-$,且$A_+$和$A_-$,都是厄米算符
    \end{enumerate}
\end{frame} 

\begin{frame} [allowframebreaks=]
    \frametitle{}
    \begin{tcolorbox1}{命题1.}
     试证明两厄米算符之和仍为厄米算符 
    \end{tcolorbox1}
    \alert{证明:}设A,B为厄米算符,对于任意态,有\\
    $$(\Psi, A\psi ) = (A\Psi, \psi), \qquad (\Psi, B\psi ) = (B\Psi, \psi)$$
    它们的和的内积有: \\
    \begin{equation*}
        \begin{split}
            (\Psi, (A+B)\psi ) &= (\Psi, A\psi ) + (\Psi, B\psi ) \\  
            &=(A\Psi, \psi ) + (B\Psi, \psi ) \\
            &=((A+B)\Psi, \psi ) 
         \end{split}
    \end{equation*}  
    证毕!
\end{frame} 

\begin{frame} [allowframebreaks=]
    \frametitle{}
    \begin{tcolorbox1}{命题2、}
        当且仅当两厄米算符对易时,它们之积才是厄米算符。
    \end{tcolorbox1}
    \alert{证明:}设A,B为厄米算符,对于任意态,
    \begin{equation*}
        \begin{split}
            (\Psi, (AB)\psi ) &= (\Psi, A(B\psi) ) \\  
            &=((A \Psi), (B\psi) )  \\
            &=(B(A \Psi), \psi )  \\
            &=( (BA) \Psi, \psi )  \\
            &=( (AB) \Psi, \psi )  \\
         \end{split}
    \end{equation*}  
    证毕!
\end{frame}  

\begin{frame} [allowframebreaks=]
    \frametitle{}
    \begin{tcolorbox1}{命题3、}
        无论两厄米算符是否对易,算符$\dfrac{1}{2}(AB+BA)$ 及 $\dfrac{1}{2i}(AB-BA) $ 都是厄米算符
    \end{tcolorbox1}
    \alert{证明:}设A,B为厄米算符,对于任意态,
    \begin{equation*}
        \begin{split}
            I.~ (\Psi, \dfrac{1}{2}(AB+BA)\psi ) &=\dfrac{1}{2}(\Psi, AB\psi) + \dfrac{1}{2}(\Psi, BA\psi)  \\
            &=\dfrac{1}{2}(A\Psi, B\psi) + \dfrac{1}{2}(B\Psi, A\psi)  \\
            &=\dfrac{1}{2}(BA\Psi, \psi) + \dfrac{1}{2}(AB\Psi, \psi)  \\
            &=\dfrac{1}{2}((BA+AB)\Psi, \psi) =(\dfrac{1}{2}(AB+BA)\Psi, \psi)\\
         \end{split}
    \end{equation*}  
    \begin{equation*}
        \begin{split}
            II.~ (\Psi, \dfrac{1}{2i}(AB-BA)\psi ) &= (\Psi, \dfrac{1}{2i}AB\psi) - (\Psi, \dfrac{1}{2i}BA\psi)\\  
            &=\dfrac{1}{2i}(\Psi, AB\psi) - \dfrac{1}{2i}(\Psi, BA\psi)  \\
            &=\dfrac{1}{2i}(A\Psi, B\psi) - \dfrac{1}{2i}(B\Psi, A\psi)  \\
            &=\dfrac{1}{2i}(BA\Psi, \psi) - \dfrac{1}{2i}(AB\Psi, \psi)  \\
            &=-(\dfrac{1}{2i}BA\Psi, \psi) +(\dfrac{1}{2i}AB\Psi, \psi)  \\
            &=(\dfrac{1}{2i}(AB-BA)\Psi, \psi) \\
         \end{split}
    \end{equation*}  
    证毕!
\end{frame}  

\begin{frame} [allowframebreaks=]
    \frametitle{}
    \begin{tcolorbox1}{命题4、}
       任意算符总可以分解成$A=A_+ +iA_-$,且$A_+$和$A_-$都是厄米算符
    \end{tcolorbox1}
    \alert{证明:}令:
    $A_+=\dfrac{1}{2} (A+A^\dagger), \quad A_-=\dfrac{1}{2i} (A-A^\dagger) $,有$A=A_+ +iA_-$\\
    问题转化为求证$\dfrac{1}{2} (A+A^\dagger), \quad \dfrac{1}{2i} (A-A^\dagger) $是厄米算符\\
    \begin{equation*}
        \begin{split}
            (\Psi, \dfrac{1}{2} (A+A^\dagger)\psi ) &=\dfrac{1}{2}(\Psi, (A)\psi) + \dfrac{1}{2}(\Psi, (A^\dagger)\psi) \\
            &= \dfrac{1}{2}((A^\dagger)\Psi, \psi) + \dfrac{1}{2}((A^\dagger)^\dagger\Psi, \psi) \\
            &= \dfrac{1}{2}((A^\dagger)\Psi, \psi) + \dfrac{1}{2}(A\Psi, \psi)= ( \dfrac{1}{2}(A^\dagger + A) \Psi, \psi ) \\
         \end{split}
    \end{equation*}  
\end{frame} 

\subsection{本征性质}

\begin{frame}{本征性质}
    \begin{enumerate}
        \Item 厄米算符的本征值为实数
        \Item 任意态下平均值为实数的算符必为厄米算符
        \Item 厄米算符属于不同本征值的本征函数正交
        \Item 简并的本征函数可通过重组变得正交
        \Item 厄米算符的本征函数系具有完备性
        \Item 厄米算符的本征函数系具有封闭性
    \end{enumerate}
\end{frame} 

\begin{frame}
    \frametitle{}
    \begin{tcolorbox1}{本征方程:}
        设厄密算符A的本征方程为
        $$ A\psi=a \psi $$
        则称a是算符A的本征值,$\psi$是属于本征值a的本征函数
     \end{tcolorbox1}   
     \begin{quote}
        《量子化就是本征值问题》\\
        ~~\\
        \rightline{薛定谔(1926)\hspace{9em}}     
     \end{quote}
\end{frame} 

\begin{frame} [allowframebreaks=]
    \frametitle{}
    \begin{tcolorbox1}{命题 1、}
    厄米算符的本征值为实数
    \end{tcolorbox1}
    \alert{证明:}设A为厄米算符,有如下本征方程
    $$A\psi=a\psi $$
    \begin{equation*}
        (\psi, A\psi)=(\psi, a\psi)=a(\psi, \psi)=a
    \end{equation*}  
    由厄米性有:
    \begin{equation*}
        (\psi, A\psi)=(A\psi, \psi)=(a\psi, \psi)= a^* (\psi, \psi)=a^*
    \end{equation*}
    有:
    \begin{equation*}
        a= a^* 
    \end{equation*}
    所以,本征值 a必为实数。
\end{frame} 

\begin{frame} [allowframebreaks=]
    \frametitle{}
    \begin{tcolorbox1}{命题 2、}
    任意态下平均值为实数的算符必为厄米算符
    \end{tcolorbox1}
    \alert{证明:}任意态$\Psi$下,F的平均值
    $$(\Psi,F\Psi)=\bar{F}=\bar{F}^*=(\Psi,F\Psi)^*=(F\Psi,\Psi), \qquad (1) $$
    令 $\Psi= \psi_1+c\psi_2 $, 代入上式,得:
    $$([\psi_1+c\psi_2],F [\psi_1+c\psi_2])=(F[\psi_1+c\psi_2],[\psi_1+c\psi_2]) $$
    进行积分,得:
    $$
    \begin{array}{r}
    \left(\psi_{1}, F \psi_{1}\right)+c^{*}\left(\psi_{2}, F \psi_{1}\right)+c\left(\psi_{1}, F \psi_{2}\right)+|c|^{2}\left(\psi_{2}, \hat{F} \psi_{2}\right) \\
    =\left(F \psi_{1}, \psi_{1}\right)+c^{*}\left(F \psi_{2}, \psi_{1}\right)+c\left(F \psi_{1}, \psi_{2}\right)+|c|^{2}\left(\hat{F} \psi_{2}, \psi_{2}\right)
    \end{array}
    $$
    由(1)有: 
    $$(\psi_1,F\psi_1)=(F \psi_1, \psi_1), \qquad (\psi_2,F\psi_2)=(F \psi_2, \psi_2) $$
    上式可消去第一、四项,变为:
    $$\begin{array}{r}
        c^{*}\left(\psi_{2}, F \psi_{1}\right)+c\left(\psi_{1}, F \psi_{2}\right) \\
        =c^{*}\left(F \psi_{2}, \psi_{1}\right)+c\left(F \psi_{1}, \psi_{2}\right)
    \end{array}$$
    分别取$c=1$, $c=i$代入,得到两个等式:
    $$  \left(\psi_{2}, F \psi_{1}\right)+\left(\psi_{1}, F \psi_{2}\right) = 
    \left(\hat{F} \psi_{2}, \psi_{1}\right)+\left(\hat{F} \psi_{1}, \psi_{2}\right) , \cdots (2)
    $$
    $$
    -i\left(\psi_{2}, F \psi_{1}\right)+i\left(\psi_{1}, F \psi_{2}\right) 
    =-i\left(\hat{F} \psi_{2}, \psi_{1}\right)+i\left(F \psi_{1}, \psi_{2}\right)
    $$
    第二式乘以$i$,得:
    $$
    \left(\psi_{2}, F \psi_{1}\right)-\left(\psi_{1}, F \psi_{2}\right) 
    =\left(\hat{F} \psi_{2}, \psi_{1}\right)-\left(F \psi_{1}, \psi_{2}\right), \cdots (3)
    $$
    (2)+(3),并两边除以2,得
    $$
    \left(\psi_{2}, F \psi_{1}\right) =\left(F \psi_{2}, \psi_{1}\right)
    $$
    证毕!
\end{frame} 
\begin{frame} [allowframebreaks=]
    \frametitle{}
    \begin{tcolorbox1}{命题 3、}
    厄米算符属于不同本征值的本征函数正交
     \end{tcolorbox1}
    \alert{证明:}设$\psi_a$、$\psi_b$分别是厄米算符A属于本征值a、b的本征函数
    \begin{equation*}
        (\psi_a, A\psi_b)=(\psi_a, b\psi_b)=b(\psi_a, \psi_b)
    \end{equation*}  
    由于厄米性,有:
    \begin{equation*}
        (\psi_a, A\psi_b)=(A\psi_a, \psi_b)=a(\psi_a, \psi_b)
    \end{equation*}
    由于$a\neq b$,有
    \begin{equation*}
        (\psi_a, \psi_b)=0
    \end{equation*}
   证毕!
\end{frame} 

\begin{frame} [allowframebreaks=]
    \frametitle{}
    \alert{本征函数的正交归一性}\\
    设$\psi_n$、$\psi_m$都是厄米算符A的本征函数\\
    归一性:
    \begin{equation*}
        (\psi_n, \psi_m)=(\psi_n, \psi_n)=1, \qquad (n=m)
    \end{equation*}  
    正交性:
    \begin{equation*}
        (\psi_n, \psi_m)=1, \qquad (n\neq m)
    \end{equation*}
    定义$\delta$函数:
    \begin{equation*}
        \delta_{n m}= 
        \begin{cases}1, & n=m \\ 
            0, & n \neq m
        \end{cases}
        \end{equation*}
    正交归一性:
    \begin{equation*}
        (\psi_n, \psi_m)=\delta_{nm}
    \end{equation*}
\end{frame} 

\begin{frame} [allowframebreaks=]
    \frametitle{}
    \begin{tcolorbox1}{命题 4、}
       简并的本征函数可通过重组变得正交
     \end{tcolorbox1}
    \alert{证明:}设厄米算符A属于本征值a的本征函数有f个
    \begin{equation*}
        A\psi_{na}=a\psi_{na}, \qquad (n=1,2,3,\cdots, f)
    \end{equation*}  
    由这f函数构成如下线性叠加态
    \begin{equation*}
        \Psi_a=\sum_{n=1}^{f} c_n \psi_{na} \qquad (n=1,2,3,\cdots, f)
    \end{equation*}
    这样的叠加态也有f个
    \begin{equation*}
        \Psi_{\beta a}=\sum_{n=1}^{f} c_{\beta n} \psi_{na} \qquad (\beta=1,2,3,\cdots, f)
    \end{equation*}
    \begin{equation*}
        A\Psi_{\beta a}=\sum_{n=1}^{f} c_{\beta n} A\psi_{na} =a \Psi_{\beta a}
    \end{equation*}
   说明叠加态也是属于本征值a的本征函数。\\
   选择系数$c_{\beta n}$,让这f个新的本征态正交归一
   \begin{equation*}
    (\Psi_{\beta a}, \Psi_{\beta' a})=\delta_{\beta\beta'}
    \end{equation*}
    正交条件式数目 $\dfrac{1}{2}f(f-1)$, 归一条件式数目 $f$\\
    系数$c_{\beta n}$的数目为$f^2$,有:$$ f^2\ge \dfrac{1}{2}f(f-1)+f$$
    因此,总可以找到一组系数$c_{\beta n}$,使其满足正交归一化条件。
\end{frame} 

\begin{frame} [allowframebreaks=]
    \frametitle{}
    \例[1.试采用Schmidt正交化方案使能量E的三个简并函数($\Psi_1, \Psi_2, \Psi_3$)正交归一]{}
    \解~取$\psi_1=\dfrac{\Psi_1}{(\Psi_1, \Psi_1)}$\\
    设 $\psi_2'=\Psi_2-(\psi_1, \Psi_2)\psi_1$\\
    \begin{equation*}
        (\psi_1, \psi_2')=(\psi_1, \Psi_2)-(\psi_1, \Psi_2)(\psi_1, \psi_1)=0
    \end{equation*}  
    取$\psi_2=\dfrac{\psi_2'}{(\psi_2', \psi_2')}$\\
    设 $\psi_3'=\Psi_3-(\psi_1, \Psi_3)\psi_1-(\psi_2, \Psi_3)\psi_2$\\
    \begin{equation*}
        (\psi_1, \psi_3')=(\psi_1, \Psi_3)-(\psi_1, \Psi_3)(\psi_1, \psi_1)-(\psi_2, \Psi_3)(\psi_1, \psi_2)=0
    \end{equation*}
    \begin{equation*}
        (\psi_2, \psi_3')=(\psi_2, \Psi_3)-(\psi_1, \Psi_3)(\psi_2, \psi_1)-(\psi_2, \Psi_3)(\psi_2, \psi_2)=0
    \end{equation*}
    取$\psi_3=\dfrac{\psi_3'}{(\psi_3', \psi_3')}$\\
    则$\psi_1, \psi_2, \psi_3$构成正交归一化组。\\ \vspace{0.6em}
    现在求它们的本征值$\dots$\\
    $$ H\psi_1= H \dfrac{\Psi_1}{(\Psi_1, \Psi_1)} =  \dfrac{E\Psi_1}{(\Psi_1, \Psi_1)} = E \psi_1$$
    $$ H\psi_2= H \dfrac{\Psi_2-(\psi_1, \Psi_2)\psi_1}{(\psi_2', \psi_2')} =  \dfrac{H\Psi_2-(\psi_1, \Psi_2)H\psi_1}{(\psi_2', \psi_2')}=E\psi_2$$
    同理,有$ H\psi_3=E\psi_3$\\
    它们依然是简并的!
\end{frame} 

\begin{frame} [allowframebreaks=]
    \frametitle{}
    \begin{tcolorbox1}{命题 5、}
        厄米算符的本征函数系具有完备性
     \end{tcolorbox1}
    \alert{完备性定义:}设某个体系的厄米算符A具有本征方程
    \begin{equation*}
        A\psi_{n}=a_n\psi_{n}, 
    \end{equation*}  
    则这个体系的任意态函数都可以在A的本征函数系上展开,
    \begin{equation*}
        \Psi=\sum_n c_n \psi_{n} \qquad (n=1,2,3,\cdots)
    \end{equation*}
    本征函数系的这种性质称为完备性。\\
    完备性证明: 见文献《厄米算符本征函数完备性的一般证明》,大学物理,2012, 31(9): 16-19.
\end{frame} 

\begin{frame} 
    \begin{exampleblock}{推论 1、}
        展开系数就是态矢量在对应本征基矢上的投影
    \end{exampleblock}
    \begin{equation*}
        c_n=\sum_m c_m\delta_{nm} = \sum_m c_m(\psi_n, \psi_m)= (\psi_n, \sum_m c_m\psi_m) =(\psi_n, \Psi)
    \end{equation*}
\end{frame} 

\begin{frame} 
    \begin{exampleblock}{推论 2、}
        展开系数的模方$|c_n|^2$就是测得相应本征值$a_n$的概率
    \end{exampleblock}
    \begin{equation*}
        \begin{split}
            \bar{A}&=(\Psi, A\Psi)=(\Psi, A\sum_n c_n \psi_{n})=(\Psi, \sum_n c_n A\psi_{n})\\
            &=(\sum_m c_m \psi_{m}, \sum_n c_n a_n \psi_{n})\\
            &=\sum_{m,n} c_m^* c_n a_n (\psi_m, \psi_n)\\
            &=\sum_{m,n} c_m^* c_n a_n \delta_{mn} \\
            &=\sum_{n} c_n^* c_n a_n  \\
            &=\sum_{n} |c_n|^2 a_n 
        \end{split}
    \end{equation*}
\end{frame} 

\begin{frame} [allowframebreaks=]
    \frametitle{}
    \begin{tcolorbox1}{命题 6、}
        厄米算符的本征函数系具有封闭性
     \end{tcolorbox1}
    \begin{equation*}
        \begin{split}
            \Psi(x)&=\sum_n c_n \psi_{n}(x) \\
            &=\sum_n (\psi_n(x'), \Psi(x')) \psi_{n}(x)\\
            &= (\sum_n\psi_{n} ^* (x)\psi_n(x'), \Psi(x')) \\
            \to &\sum_n\psi_{n} ^* (x)\psi_n(x')=\delta(x-x')\\
            \to &(\psi_{n}(x),\psi_n(x'))=\delta(x-x')
        \end{split} 
    \end{equation*}
\end{frame} 

\subsection{希尔伯特空间}

\begin{frame} 
    \frametitle{希尔伯特空间}
    系数矩阵:\\ \vspace{0.3em}
    \begin{equation*}
        \begin{split}
            \vec{P}&=\sum_i{x_i\vec{e_i}}, \qquad i=1,2,3 \\
            \Psi&=\sum_n c_n \psi_n, \qquad n=1,2,\cdots 
        \end{split}  
    \end{equation*}
    数学上,把正交归一的基矢组{$\vec{e_i}, i=1,2,3$}张开的空间叫三维矢量空间, \\
    有 $\vec{P}\Leftrightarrow(x_1,x_2,x_3)$\\
    把正交归一的本征函数系{$\psi_n, n=1,2,\cdots$}张开的空间叫Hilbert空间,\\
    有 $\Psi\Leftrightarrow(c_1,c_2,\cdots)^T$\\
\end{frame} 

\begin{frame} 
    \frametitle{严格定义}
    \begin{equation*}
        \begin{split}
            \text{1、定义加法} \quad  &\xi=\psi+\varphi\\
            &\psi+\varphi=\varphi+\psi \qquad (\text{交换律})\\
            &(\psi+\varphi)+\xi=\psi+(\varphi+\xi) \qquad (\text{结合律})\\
            &\psi+\text{O}= \psi \qquad (\text{零元})\\
            &\psi+\varphi= \text{O} \qquad (\text{逆元})\\
        \end{split}  
    \end{equation*}
    \begin{equation*}
        \begin{split}
            \text{2、定义数乘} \quad &\varphi=\psi a\\
            &\psi 1= \psi \qquad (\text{1元})\\
            &(\psi a)b=\psi (ab) \qquad (\text{结合律})\\
            &\psi(a+b)= \psi a+ \psi b \qquad (\text{第一分配律})\\
            &(\psi+\varphi) a= \psi a +\varphi a \qquad (\text{第二分配律})\\
        \end{split}  
    \end{equation*}
\end{frame} 

\begin{frame} 
    \begin{equation*}
        \begin{split}
            \text{3、定义内积} \quad &c=(\psi, \varphi)\\
            &(\psi, \varphi)= (\varphi,\psi)^* \\
            &(\psi, \varphi+\xi)= (\psi, \varphi) + (\psi, \xi)\qquad (\text{分配律})\\
            &(\psi, \varphi a)= (\psi, \varphi )a \\
            &\Rightarrow (\psi a, \varphi )= (\psi, \varphi )a^* \\
            &(\psi,\psi)= c\ge 0\\
        \end{split}  
    \end{equation*}
\end{frame} 

\begin{frame}
    4、定义空间\\
   \begin{itemize}
       \Item 矢量空间:满足加法和数乘两种运算的集合
       \Item 内积空间:满足加法、数乘和内积三种运算的集合
       \Item 希尔伯特空间:  完全的内积空间\\
       ~~ \\
       *完全性:对给定任意小的实数$\varepsilon$,总有数N存在,当m, n>N时,有\\
       $$ (\psi_m -\psi_n, \psi_m -\psi_n )< \varepsilon $$
   \end{itemize} 
\end{frame} 

\begin{frame}
   5、描述量子力学\\
   \begin{itemize}
       \Item 希尔伯特空间的态矢量描述体系的状态
       \Item 希尔伯特空间的厄米算符描述体系的物理量
       \Item 物理量的取值是相应算符的本征值
       \Item 取某本征值的概率是态矢量按算符本征函数系展开时的对应本征矢前展开系数的模方成正比
       \Item 态矢量随时间的演化服从薛定谔方程。
   \end{itemize}
   ~~ \\ \vspace{1.0em}
   \begin{quote}
    《量子化就是本征值问题》\\
    ~~\\
    \rightline{薛定谔(1926)\hspace{9em}}     
 \end{quote}
\end{frame} 


