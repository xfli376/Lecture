%%%%%%%%%%%%%%%%%%%%%%%%%%%%%%%%%%%%%%%%%%
\begin{frame}
    \frametitle{}
    \begin{center}
    { {\huge 第八讲、本征方程}}
    \end{center}    
\end{frame}
%%%%%%%%%%%%%%%%%%%%%%%%%%%%%%%%%%%%%


\section{前情回顾}

\begin{frame}
    \frametitle{前情回顾}
    \begin{itemize}
        \item 希尔伯特空间的态矢量描述体系的状态
        \item 希尔伯特空间的线性厄米算符给出体系的物理量
        \item 任一厄米算符的本征函数系是希尔伯特空间的一个正交归一完全基
    \end{itemize}   
\end{frame} 

\section{动量算符}
\begin{frame} [allowframebreaks=]
    \frametitle{1. 动量}
    算符:  $\hat{\vec p}=-i\hbar \nabla$ \\
    本征方程: $\hat{\vec p}\psi_{\vec p}=\vec p \psi_{\vec p}$ \\
    解本征方程:
    \begin{equation*}
        \begin{split}
            \hat{p}_x\psi_{p_x}&=p_x \psi_{p_x} \\
            -i\hbar\frac{\partial}{\partial x} psi_{\vec p} &= p_x \psi_{p_x}\\
            \frac{1}{\psi_{p_{x}}} \frac{\partial}{\partial x} \psi_{p_{x}}&=\frac{i p_{x}}{\hbar}\\
            \psi_{p_{x}}&=Ae^{\frac{i}{\hbar}p_x x} \\
            \psi_{p_{x}}&=\frac{1}{\sqrt{2\pi\hbar}}e^{\frac{i}{\hbar}p_x x}
        \end{split} 
    \end{equation*}
    本征函数: $$ \psi_{\vec{p}}(\vec{r})=\frac{1}{(2\pi\hbar)^{3/2}}e^{\frac{i}{\hbar}\vec{p}\cdot \vec{x}}   $$
    本征值谱: 连续
        $$ p \in (-\infty, +\infty) $$
    正交归一性:
        $$ (\psi_{\vec{p}'}, \psi_{\vec{p}}) =\delta(\vec{p}'-\vec{p})$$
    完备性:
    $$ \Psi(\vec{r},t)=\iiint\limits_{-\infty}^{+\infty}c(\vec{p},t) \psi_{\vec{p}}(\vec{r}) dp_xdp_ydp_z $$
    封闭性:$$ (\psi_{\vec{p}}(\vec{r}''), \psi_{\vec{p}}(\vec{r}')) =\delta(\vec{r}''-\vec{r}')$$
\end{frame} 

\section{位置算符}
\begin{frame} [allowframebreaks=]
    \frametitle{2. 位置}
    算符:  $\hat{\vec r}=\vec{r}$ \\
    本征方程: $\hat{\vec r}\psi_{\vec \lambda}=\vec \lambda \psi_{\vec \lambda}$ \\
    解本征方程:
    \begin{equation*}
        \begin{split}
            \hat{\vec r}\psi_{\vec \lambda}&=\vec \lambda \psi_{\vec \lambda} \\
            \vec{r}\psi_{\vec \lambda}&=\vec \lambda \psi_{\vec \lambda} \\
        \end{split} 
    \end{equation*}
    分析,$\vec \lambda$是本征值(常数),所以除 $\vec r =\vec \lambda $这一点外,$\psi_{\vec \lambda}$其他位置
    处处为零,有:
    本征函数: $$ \psi_{\vec \lambda}(\vec{r})= A \delta(\vec{r}-\vec{\lambda})= \delta(\vec{r}-\vec{\lambda})$$
    本征值谱: 连续
        $$ \lambda \in (-\infty, +\infty) $$
    正交归一性:
        $$ (\psi_{\vec{\lambda}'}, \psi_{\vec{\lambda}}) =\delta(\vec{\lambda}'-\vec{\lambda})$$
    完备性:
    $$ \Psi(\vec{r},t)=\iiint\limits_{-\infty}^{+\infty}c_(\vec{\lambda}(\vec{r},t) \psi_{\vec{\lambda}}(\vec{r}') dx'dy'dz' $$
    封闭性:$$ (\psi_{\vec{\lambda}}(\vec{r}''), \psi_{\vec{\lambda}}(\vec{r}')) =\delta(\vec{r}''-\vec{r}')$$
\end{frame} 
\begin{frame} [allowframebreaks=]
    \begin{tcolorbox2}{课堂作业:}
        已知某算符为  $\hat{F}=-ie^{ix}\frac{d}{dx}$,求本征函数 \\  
     \end{tcolorbox2}
\end{frame} 

\begin{frame} [allowframebreaks=]
    \frametitle{}
    \alert{解:} 本征方程为 $\hat{F}\psi_f(x)=f\psi_f(x)$\\
    代入算符的具体形式:
    \begin{equation*}
        \begin{split}
            -ie^{ix}\frac{d}{dx}\psi_f(x)&=f\psi_f(x) \\
            \frac{d\psi_f(x)}{\psi_f(x)}&=ife^{-ix} dx \\
            \frac{d\psi_f(x)}{\psi_f(x)}&=d(-fe^{-ix}) \\
           \ln{\psi_f(x)}&=-fe^{-ix}+\ln c \\
           \psi_f(x)&=c e^{-fe^{-ix}}
        \end{split} 
    \end{equation*}
\end{frame} 

\section{角动量算符}
\begin{frame} [allowframebreaks=]
    \frametitle{3. 角动量算符}
    \begin{tcolorbox1}{角动量:}
        已知角动量的经典定义为:
    $$\vec{L}=\vec{r}\times\vec{p}$$ 
    求其算符形式,并求本征方程     
     \end{tcolorbox1}
    \alert{解:} 根据对应原则,有:
    $$\hat{\vec{L}}=\hat{\vec{r}}\times\hat{\vec{p}}= -i\hbar \vec{r}\times\nabla$$
    (I)在直角坐标系中,\\
    $$
    \left \{
    \begin{array}{l} 
        \hat{L}_x=y\hat{p}_z-z\hat{p}_y  \\ 
        \hat{L}_y=z\hat{p}_x-x\hat{p}_z  \\ 
        \hat{L}_z=x\hat{p}_y-y\hat{p}_x 
    \end{array}
    \right.
    $$
    $$ \hat{L}^2= \hat{L}_x ^2+ \hat{L}_y ^2 +\hat{L}_z ^2  $$

    (II)在球坐标系中,\\
    $$
    \left\{\begin{array}{l}
        \hat{L}_{x}=i \hbar\left[\sin \varphi \frac{\partial}{\partial \theta}+\cot \theta \cos \varphi \frac{\partial}{\partial \varphi}\right] \\
        \hat{L}_{y}=-i \hbar\left[\cos \varphi \frac{\partial}{\partial \theta}+\cot \theta \sin \varphi \frac{\partial}{\partial \varphi}\right] \\
        \hat{L}_{z}=-i \hbar \frac{\partial}{\partial \varphi}
        \end{array}\right.
    $$
    $$ \hat{L}^{2}=-\hbar^{2}\left[\frac{1}{\sin \theta} \frac{\partial}{\partial \theta}\left(\sin \theta \frac{\partial}{\partial \theta}\right)+\frac{1}{\sin ^{2} \theta} \frac{\partial^{2}}{\partial \varphi^{2}}\right] $$
    相对比较简单,可求解本征方程
\end{frame} 

\begin{frame} [allowframebreaks=]
    \frametitle{解角动量本征方程}
    $\hat{L}_z$ 的本征方程为 $$\hat{L}_z\Phi(\varphi)=l_z\Phi(\varphi)$$
    代入算符的具体形式:
    \begin{equation*}
        \begin{split}
            -i \hbar \frac{\partial}{\partial \varphi}\Phi(\varphi)&=l_z\Phi(\varphi) \\
            \frac{1}{\Phi(\varphi)}\frac{\partial}{\partial \varphi}\Phi(\varphi)&=\frac{i}{\hbar}l_z \\
            \Phi(\varphi)&=A e ^{\frac{i}{\hbar}l_z\varphi}
        \end{split} 
    \end{equation*}
     根据周期性边界条件:$\Phi(\varphi)=\Phi(2\pi+\varphi)$\\
    \begin{equation*}
        \frac{A e ^{\frac{i}{\hbar}l_z(2\pi+\varphi)}}{A e ^{\frac{i}{\hbar}l_z\varphi}}=1
    \end{equation*} 
    \begin{equation*}
        e ^{\frac{i}{\hbar}l_z2\pi}=1
    \end{equation*} 
    $$
    \cos \left(2 \pi l_{z} / \hbar\right)+i \sin \left(2 \pi l_{z} / \hbar\right)=1
    $$
    $$ 2 \pi l_{z} / \hbar=2\pi m, \qquad (m=0,\pm 1,  \pm 2, \cdots) $$
    $$\to l_z=m\hbar$$
    $$\to \Phi_m(\varphi)=Ae^{im\varphi}$$
    归一化:
    \begin{equation*}
        \begin{split}
            \int_0 ^{2\pi} |\Phi|^2 d \varphi &= \int_0 ^{2\pi} \Phi^*\Phi  d \varphi \\
            &= A^2 \int_0 ^{2\pi} e^{im\varphi-im\varphi} d \varphi \\
            &= A^2 2\pi \\
            &= 1
        \end{split}   
    \end{equation*}
    $$ \to A= \frac{1}{\sqrt{2\pi}} $$
    $$ \to \Phi_m(\varphi)=\frac{1}{\sqrt{2\pi}}e^{im\varphi}$$
    (小结)算符:  $\hat{L}_{z}=-i \hbar \frac{\partial}{\partial \varphi}$ \\
    本征方程: $$\hat{L}_z\Phi_m(\varphi)=m\hbar \Phi_m (\varphi)$$
    本征函数: $$ \Phi_m(\varphi)=\frac{1}{\sqrt{2\pi}}e^{im\varphi}$$
    本征值谱:  分立
        $$ l_z=m\hbar, \qquad (m=0,\pm 1,  \pm 2, \cdots) $$
    正交归一性:
        $$ (\Phi_{m'}(\varphi), \Phi_m(\varphi)) =\delta_{m'm}$$
    完备性与封闭性:$$ (\Phi_m(\varphi''), \Phi_m(\varphi')) =\delta(\varphi''-\varphi')$$
\end{frame} 

\begin{frame} [allowframebreaks=]
    \frametitle{}
    $\hat{L}^2$ 的本征方程为 $$\hat{L}^2Y(\theta,\varphi)=l^2 Y(\theta,\varphi)$$
    代入算符的具体形式,并令本征值为$\lambda\hbar^2$,有:
    \begin{equation*}
        \begin{split}
            -\hbar^{2}\left[\frac{1}{\sin \theta} \frac{\partial}{\partial \theta}\left(\sin \theta \frac{\partial}{\partial \theta}\right)+\frac{1}{\sin ^{2} \theta} \frac{\partial^{2}}{\partial \varphi^{2}}\right] Y(\theta,\varphi)&=\lambda\hbar^2 Y(\theta,\varphi) \\
            \left[\frac{1}{\sin \theta} \frac{\partial}{\partial \theta}\left(\sin \theta \frac{\partial}{\partial \theta}\right)+\frac{1}{\sin ^{2} \theta} \frac{\partial^{2}}{\partial \varphi^{2}}\right] Y(\theta,\varphi)&=-\lambda Y(\theta,\varphi) \\
        \end{split} 
    \end{equation*}
    这是球谐方程,利用分离变量及幂级数法可解:\\
    本征值: $$\lambda=l(l+1), \qquad (l= 0,1,2,\cdots)$$
    $\hat{L}^2$的本征值:$$l^2=\lambda\hbar^2=l(l+1)\hbar^2$$
    本征函数:
    $$
    \mathrm{Y}_{l m}(\theta, \varphi)=(-1)^{m} \sqrt{\frac{(2 l+1)(l-m) !}{4 \pi(l+m) !}} \mathrm{P}_{l}^{m}(\cos \theta) \Phi_{m}(\varphi)
    $$ 
    式中:
    $$ m=0,\pm 1,  \pm 2, \cdots, \pm l)  $$
    $\mathrm{P}_{l}^{m} $ 是勒上德多项式
    $$
    P_{l}^{m}(\cos \theta)=(-1)^{l+m} \frac{1}{2^{l} l !} \sqrt{\frac{(2 l+1)}{4 \pi} \frac{(l+m) !}{(l-m) !} \frac{1}{\sin ^{m} \theta}\left(\frac{d}{d \cos \theta}\right)^{l-m} \sin ^{2 l} \theta}
    $$
    (小结)算符:  $$ \hat{L}^{2}=-\hbar^{2}\left[\frac{1}{\sin \theta} \frac{\partial}{\partial \theta}\left(\sin \theta \frac{\partial}{\partial \theta}\right)+\frac{1}{\sin ^{2} \theta} \frac{\partial^{2}}{\partial \varphi^{2}}\right] $$
    本征方程: $$\hat{L}^2Y(\theta,\varphi)=l(l+1)\hbar^2 Y(\theta,\varphi)$$
    本征函数:     $$
    \mathrm{Y}_{l m}(\theta, \varphi)=(-1)^{m} \sqrt{\frac{(2 l+1)(l-m) !}{4 \pi(l+m) !}} \mathrm{P}_{l}^{m}(\cos \theta) \Phi_{m}(\varphi)
    $$ 
    本征值谱:  分立
    $$l^2=l(l+1)\hbar^2, \qquad (l= 0,1,2,\cdots, n) $$
    正交归一性:
    $$
    \int_{0}^{\pi} \int_{0}^{2 \pi} Y_{l m}(\theta, \varphi) Y_{l^{\prime} m^{\prime}}^{*}(\theta, \varphi) \sin \theta \mathrm{d} \theta \mathrm{d} \varphi=\delta_{l l^{\prime}} \delta_{m m^{\prime}}
    $$
    完备性与封闭性:
    $$\psi(\theta, \varphi)=\sum_{l=0}^{n} \sum_{m=-l}^{l} C_{l m} \mathrm{Y}_{l m}(\theta, \varphi)$$

    简并度: $2l+1$\\
    对于本征值$l(l+1)\hbar^2$,有$2l+1$个本征函数 $\mathrm{Y}_{l m}$与之对应。
\end{frame} 

\begin{frame} 
    \frametitle{角动量量子化}
    \begin{wrapfigure} {b} {0.4\textwidth} %;图在右
        \includegraphics[width=0.35\textwidth]{figs/LandL2.png}   
    \end{wrapfigure}
    角动量大小:$\sqrt{l(l+1)\hbar}, \qquad (l=1,2,\cdots, n)$,对应球的半径。\\
    角动量在Z轴上的投影大小 $l_z=m\hbar, \qquad (m=0,\pm 1,\pm 2, \cdots, \pm l)$
\end{frame} 

\section{能量算符}

\begin{frame} 
    \frametitle{}
    
    \begin{tcolorbox2}{例1:自由粒子能量本征值}
    求质量为$\mu$的一维自由粒子的能量本征值和本征态  
    \end{tcolorbox2}
    \alert{解:} 能量算子即哈密顿算子:
    $$ \hat{H}=\hat{T}+\hat{V}=\frac{\hat{p}_x ^2 }{2\mu} = -\frac{\hbar^2}{2\mu}\frac{d^2}{dx^2} $$
    建立能量本征方程:
    $$ \hat{H} \psi =E \psi $$
    $$ -\frac{\hbar^2}{2\mu}\frac{d^2}{dx^2} \psi =E \psi $$
    $$ \frac{d^2}{dx^2} \psi = -\frac{2\mu E}{\hbar^2} \psi $$
\end{frame}

\begin{frame} 
    $$  \psi'' + k^2 \psi =0 $$
    特征方程的根为:$\lambda_{1,2}=\pm k$ \\
    方程的解(本征函数)为: $\psi \sim e^{\pm ikx}$  \\
    能量本征值:$ E= \frac{\hbar^2 k^2 }{2\mu} \le 0 $
\end{frame}


\begin{frame} 
    \frametitle{ }
    \begin{tcolorbox2}{例2:转子的能量本征值}
    求转动惯量为I的平面转子的能量本征值和本征态 
    \end{tcolorbox2}
    \alert{解:} 能量算子即哈密顿算子:
    $$ \hat{H}=\hat{T}+\hat{V}=\frac{\hat{l}_z ^2 }{2I} = -\frac{\hbar^2}{2I}\frac{\partial^2}{\partial\varphi^2} $$
    建立能量本征方程:
    $$ \hat{H} \psi =E \psi $$
    $$ -\frac{\hbar^2}{2I}\frac{\partial^2}{\partial\varphi^2} \psi =E \psi $$
\end{frame}

\begin{frame} 
    能量本征函数: $$\psi_m(\varphi)=\frac{1}{\sqrt{2\pi}}e^{im\varphi}, \qquad (m=0,\pm 1, \pm 2, \cdots)$$
    能量本征值: $$ E_m=\frac{m^2\hbar^2}{2I} $$ 
    
\end{frame}

\begin{frame}    
    \begin{tcolorbox2}{讨论:}
        求转动惯量为I的平面转子的能量本征值和本征态 如何求空间转子的能量本征值和本征态?  
    \end{tcolorbox2}
\end{frame}