%%%%%%%%%%%%%%%%%%%%%%%%%%%%%%%%%%%%%%%%%%
\begin{frame}
    \frametitle{}
    \begin{center}
    { {\huge 第六讲、力学量算符}}
    \end{center}    
\end{frame}
%%%%%%%%%%%%%%%%%%%%%%%%%%%%%%%%%%%%%


\section{前情回顾}

\begin{frame}
    \frametitle{前情回顾}
    \begin{itemize}
        \item 波粒二象性
        \item 波函数假说
        \item 波函数的统计解释
        \item 态叠加原理
        \item 薛定谔方程
    \end{itemize}
    以上都是关于量子态的问题,下面讲解量子力学的物理量问题
\end{frame} 

\section{动量算符}

\begin{frame} [allowframebreaks=]
    \frametitle{动量算符}
    \begin{exampleblock}{}
        已知粒子的位置波函数$\psi(x,t)$,求动量的期望值   
    \end{exampleblock}
    \alert{解:} 由概率解释知,位置的期望值为
    \begin{equation*}
        \bar{x}=\int x|\psi(x, t)|^{2} d x=\int \psi^{*}(x, t) x \psi(x, t) d x
    \end{equation*}
    对于动量波函数 $c(p,t)$, 动量的期望值为
    \begin{equation*}
        \bar{p_x}=\int p_x|c(p_x, t)|^{2} d p_x=\int c^{*}(p_x, t) p c(p_x, t) d p_x
    \end{equation*}
    很明显,对于已知位置波函数$\psi(x,t)$的条件下,
    \begin{equation*}
        \bar{p_x}\neq\int p_x|\psi(x, t)|^{2} d p_x
    \end{equation*}
    但可以通过如下变换求解(注:为了方法,用$p \to p_x$)
    \begin{equation*}
        \begin{split}
            \bar{p}&=\int c^{*}(p) p c(p) d p \\  
            &=\int (\frac{1}{\sqrt{2 \pi \hbar}} \int \psi^{*}(x) e^{\frac{i}{\hbar} p\cdot x} d x) p c\left(p\right) d p \\
            &=\frac{1}{\sqrt{2 \pi \hbar}} \int \int \psi^{*}(x) (e^{\frac{i}{\hbar} p\cdot x}  p) c\left(p\right) d xd p \\
            &=\frac{1}{\sqrt{2 \pi \hbar}} \int \int \psi^{*}(x) (-i\hbar\frac{d}{d x} e^{\frac{i}{\hbar} p\cdot x}) c(p) d xd p \\
            &=\int \psi^{*}(x) (-i\hbar\frac{d}{d x}) (\frac{1}{\sqrt{2 \pi \hbar}} \int e^{\frac{i}{\hbar} p\cdot x} c(p) d p)  d x\\
            &=\int \psi^{*}(x) (-i\hbar\frac{d}{d x}) \psi(x)  d x\\
         \end{split}
    \end{equation*}  
    定义如下计算符号:
    $$ \hat{p}_x= -i\hbar\frac{d}{d x} $$ 
    上式变为:         
    $$\bar{p}_x=\int \psi^{*}(x) \hat{p}_x \psi(x) d x $$
    我们称$ \hat{p}_x= -i\hbar\dfrac{d}{d x} $ 是位置表象里的动量算符($x$分量)\\
    如果有任意力学量F的算符$\hat{F}$,则可求任意力学量的平均值。\\
    $$\bar{F}=\int \psi^{*}(x) \hat{F} \psi(x) d x $$
    由此, 我们发现:
    \begin{tcolorbox}[colback=yellow!10,colframe=red!75!black,title=Basic assumption 3/5]
    如果体系的状态用波函数描述,则力学量用算符描述。
    \end{tcolorbox}
\end{frame} 

\section{一般算符}

\begin{frame} [allowframebreaks=]
    \frametitle{各种算符定义}
    既然量子力学中,算符这么重要,我们先给出一般性的定义:
    \begin{definition}
        算符:作用于态函数,使它变成另一个态函数
        $$ \hat{F} \Psi=\psi$$
    \end{definition}
    (*),在不引起不明意义的条件下,为了简单起见,略去算符的帽子
    \begin{definition}
        单位算符:
        $$ I\Psi=\Psi $$
    \end{definition}
    \begin{definition}
        算符相等 : 对任意波函数,有
        $$ A\Psi=B\Psi \to A=B $$
    \end{definition}
    \begin{definition}
        算符的和 : 
        $$ (A+B)\Psi=A\Psi+B\Psi $$
        存在交换律和结合律\\
        A+B=B+A\\
        (A+B)+C=A+(B+C)
    \end{definition}
    \begin{definition}
        算符的积: 
        $$ (AB)\Psi=A(B\Psi) $$
        不存在交换律
        $AB=BA$ 或 $AB\ne BA$ 都有可能
    \end{definition}
    \begin{definition}
        对易子: 
        $$ [A,B]=AB-BA$$
        若[A,B]=0,称两算符对易,否则不对易
    \end{definition}
    \begin{definition}
        逆算符: 
        $$ F\Psi=\psi $$
        $$ F^{-1}\psi=\Psi $$
    \end{definition}
    \begin{definition}
        内积: 
        $$ \int\Psi^*\psi d \tau$$
        称为两函数的内积,可以简写成
        $$ (\Psi,\psi)$$ 
        $$ <\Psi|\psi>$$
        称 $ <\psi|$为左矢, $|\psi>$为右矢\\
        内积性质:
        $$ (\Psi,\psi)^*=(\psi,\Psi)$$ 
        $$ (\Psi,c_1\psi_1+c_2\psi_2)=(\Psi,c_1\psi_1)+(\Psi,c_2\psi_2)$$ 
        $$ (\Psi,c\psi)=c(\Psi,\psi)$$ 
        $$ (c\Psi,\psi)=c^*(\Psi,\psi)$$ 
    \end{definition} 
    \begin{definition}
        伴算符: 
        $$ F|\Psi> = |\psi> $$
        $$ <\psi| = <\Psi|F^{\dagger} $$
        内积形式:
        $$ (\psi,F\Psi)=(\psi,\psi)$$ 
        $$ (F^\dagger \Psi,\psi)=(\psi,\psi)$$ 
    \end{definition}   
    \begin{definition}
        自伴算符: 
        $$ F^{+} = F $$
        有性质:
        $$ (\Psi,F\psi)=(F\Psi,\psi)$$ 
        称为厄密性
    \end{definition} 
    \begin{definition}
        线性算符:对任意函数,有\\
        $$F(c_1\psi_1+c_2\psi_2 ) = c_1(F\psi_1)+c_2(F\psi_2 )$$
    \end{definition}
    \begin{definition}
        幺正(酉)算符: 
        $$ F^{+}F = FF^{+}=I $$
        性质:
        $$ F^{+}=F^{-1}$$ 
    \end{definition}         
\end{frame} 

\section{力学量算符}

\begin{frame} [allowframebreaks=]
    \frametitle{力学量算符}
    (位置表象)
    \begin{itemize}
        \item  位置算符: $ \hat{\vec{r}} =\vec{r} $
        \item  动量算符: $ \hat{\vec{p}} =-i\hbar(\dfrac{d}{d x}+ \dfrac{d}{d y} + \dfrac{d}{d z})=-i\hbar \nabla $
    \end{itemize}
    力学量F在经典物理学中,一般是位置与动量的函数$F(\vec{r},\vec{p})$\\
    则其在量子力学中的算符:$$ \hat{F}=F(\hat{\vec{r}},\hat{\vec{p}})$$
    \begin{exampleblock}{}
        \begin{itemize}
            \item  动能: $ T=\frac{p^2}{2\mu} \to \hat{T}= \frac{\hat{p}^2}{2\mu} $
            \item  哈密顿量: $ H=T+U(\vec{r} ) \to \hat{H}= \hat{T}+ U(\hat{\vec{r}})$
            \item  角动量:$ \vec{L}=\vec{r}\times\vec{p} \to \hat{\vec{L}}=\hat{\vec{r}}\times \hat{\vec{p}}$
        \end{itemize}
    \end{exampleblock}
\end{frame} 
\begin{frame}   
    TIPS:上述法则有例外情况
    \begin{itemize}
        \item  $F(\vec{r},\vec{p})$中含$\vec{r}^m\cdot\vec{p}^n$的项 
        \item  纯量子力学力学量,比如:自旋(S),宇称(P),\dots
    \end{itemize}
\end{frame} 

\section{力学量算符的性质}

\begin{frame} [allowframebreaks=]
    \frametitle{1、都是线性算符}
    \begin{definition}
        对任意波函数,满足如下运算法则的算符称为线性算符\\
        $$\hat{F}(c_1\psi_1+c_2\psi_2 ) = c_1(\hat{F}\psi_1)+c_2(\hat{F}\psi_2 )$$
    \end{definition}
    试判断下列算符哪些是线性算符:\\
    $$4x^2+\frac{d^2}{dx^2} \hspace{1cm}  []^2 \hspace{1cm} \sum\limits_{n=1}^{N}$$
\end{frame} 

\begin{frame} [allowframebreaks=]
    \begin{tcolorbox}[colback=yellow!10,colframe=red!75!black,title=命题]
      力学量算符都是线性算符  
    \end{tcolorbox}
    \begin{proof}
        设$\psi_1, \psi_2$ 是算符$\hat{F}$的属于本征值$f$的两个解,有\\
        $$\hat{F}\psi_1=f\psi_1, \to c_1\hat{F}\psi_1=c_1f\psi_1 $$
        $$\hat{F}\psi_2=f\psi_2, \to c_2\hat{F}\psi_2=c_2f\psi_2 $$
        $$f(c_1f\psi_1+c_2f\psi_2)=c_1\hat{F}\psi_1+c_2\hat{F}\psi_2$$
        $$\hat{F}(c_1f\psi_1+c_2f\psi_2)=c_1\hat{F}\psi_1+c_2\hat{F}\psi_2$$
    \end{proof}
\end{frame} 

\begin{frame} [allowframebreaks=]
    \frametitle{2、都是厄密算符}
    \begin{definition}
    对任意波函数,满足如下运算法则的算符称为厄密算符\\
        $$\int \psi^* \hat{F} \Psi d\tau =\int (\hat{F}\psi)^* \Psi d\tau $$
    \end{definition}
    为了简单起见,一般令: $$\int \psi^* \hat{F} \Psi d\tau= (\psi,\hat{F}\Psi)$$
    $$\int (\hat{F}\psi)^* \Psi d\tau = (\hat{F}\psi,\Psi) $$
    (*)厄密算符定义式: $$(\psi,\hat{F}\Psi)= (\hat{F}\psi,\Psi) $$
    (*)归一化公式: $$ \int \psi^{*} \psi d \tau=1 \to (\psi,\psi)=1$$
    (*)期望值公式: $$ \bar{F}=\int \psi^{*} \hat{F} \psi d \tau \to \bar{F}= (\psi,\hat{F}\psi) $$
    试判断下列算符哪些是厄密算符:\\
    $$\frac{d}{dx} \hspace{1cm}  i\frac{d}{dx} \hspace{1cm} 4\frac{d^2}{dx^2} $$
\end{frame} 

\begin{frame} [allowframebreaks=]
    \begin{tcolorbox}[colback=yellow!10,colframe=red!75!black,title=命题]
    力学量算符都是厄密算符  
    \end{tcolorbox}
    \begin{proof}
        对任意波函数$\Psi$, 力学量算符$F$的期望值为\\
        $$\bar{F}=(\Psi,\hat{F} \Psi) $$
        $$\bar{F}^*=(\Psi, \hat{F} \Psi)^* = (\hat{F}\Psi, \Psi) $$
        由于可观测力学量的期望值都是实数,\\
        $$(\Psi,\hat{F}\Psi)=(\hat{F} \Psi, \Psi) $$
    \end{proof}
\end{frame} 

\begin{frame} [allowframebreaks=]
        取 $\Psi= \psi_1+c\psi_2 $, 代入上式,得:
        $$([\psi_1+c\psi_2],\hat{F} [\psi_1+c\psi_2])=(\hat{F}[\psi_1+c\psi_2],[\psi_1+c\psi_2]) $$
        进行积分,得:
        $$
        \begin{array}{r}
        \left(\psi_{1}, \hat{F} \psi_{1}\right)+c^{*}\left(\psi_{2}, \hat{F} \psi_{1}\right)+c\left(\psi_{1}, \hat{F} \psi_{2}\right)+|c|^{2}\left(\psi_{2}, \hat{F} \psi_{2}\right) \\
        =\left(\hat{F} \psi_{1}, \psi_{1}\right)+c^{*}\left(\hat{F} \psi_{2}, \psi_{1}\right)+c\left(\hat{F} \psi_{1}, \psi_{2}\right)+|c|^{2}\left(\hat{F} \psi_{2}, \psi_{2}\right)
        \end{array}
        $$
        算符的平均值都是实数,即 
        $$(\psi_1,\hat{F}\psi_1)=(\hat{F} \psi_1, \psi_1), \qquad (\psi_2,\hat{F}\psi_2)=(\hat{F} \psi_2, \psi_2) $$
        上式可消去第一、四项,变为:
        $$\begin{array}{r}
            c^{*}\left(\psi_{2}, \hat{F} \psi_{1}\right)+c\left(\psi_{1}, \hat{F} \psi_{2}\right) \\
            =c^{*}\left(\hat{F} \psi_{2}, \psi_{1}\right)+c\left(\hat{F} \psi_{1}, \psi_{2}\right)
        \end{array}$$
        分别取$c=1$, $c=i$代入,得到两个等式:
        $$  \left(\psi_{2}, \hat{F} \psi_{1}\right)+\left(\psi_{1}, \hat{F} \psi_{2}\right) = 
        \left(\hat{F} \psi_{2}, \psi_{1}\right)+\left(\hat{F} \psi_{1}, \psi_{2}\right) , \cdots (1)
        $$
        $$
        -i\left(\psi_{2}, \hat{F} \psi_{1}\right)+i\left(\psi_{1}, \hat{F} \psi_{2}\right) 
        =-i\left(\hat{F} \psi_{2}, \psi_{1}\right)+i\left(\hat{F} \psi_{1}, \psi_{2}\right)
        $$
        第二式乘以$i$,得:
        $$
        \left(\psi_{2}, \hat{F} \psi_{1}\right)-\left(\psi_{1}, \hat{F} \psi_{2}\right) 
        =\left(\hat{F} \psi_{2}, \psi_{1}\right)-\left(\hat{F} \psi_{1}, \psi_{2}\right), \cdots (2)
        $$
        (1)+(2),并两边除以2,得
        $$
        \left(\psi_{2}, \hat{F} \psi_{1}\right) =\left(\hat{F} \psi_{2}, \psi_{1}\right)
        $$
        证毕!
\end{frame} 

\begin{frame} [allowframebreaks=]
    THE END
\end{frame} 