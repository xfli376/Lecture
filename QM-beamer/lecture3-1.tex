%%%%%%%%%%%%%%%%%%%%%%%%%%%%%%%%%%%55%%
\begin{frame} [plain]
    \frametitle{}
    \Background[1] 
    \begin{center}
    { {\huge 第三章、量子力学中的力学量 \\ (12学时)}}
    \end{center}  
    \addtocounter{framenumber}{-1}   
\end{frame}
%%%%%%%%%%%%%%%%%%%%%%%%%%%%%%%%%%

\section{1.力学量算符表示}

\begin{frame}
    \frametitle{前情回顾}
    \begin{itemize}
        \Item 波粒二象性
        \Item 波函数假说
        \Item 波函数的统计解释
        \Item 态叠加原理
        \Item 薛定谔方程
    \end{itemize}
\end{frame} 

\begin{frame}    
    \begin{tcolorbox4}[Basic assumption 1/5]
    物体的状态用希尔伯特空间的态矢量表示
    \end{tcolorbox4}
    \begin{tcolorbox4}[Basic assumption 2/5]
    封闭体系的态函数随时间的演化服从薛定谔方程
    \end{tcolorbox4}
    ~~\\
    \hspace{2em}\alert{TIPS:}这些都是有关量子态的问题,那物理量又如何呢?
\end{frame} 
\begin{frame} 
    \frametitle{}
    \begin{tcolorbox4}[Basic assumption 3/5]
    体系的力学量用希尔伯特空间的厄密算符表示
    \end{tcolorbox4}
\end{frame} 

\subsection{希尔伯特空间}

\begin{frame} 
    \frametitle{希尔伯特空间}
    \begin{equation*}
        \begin{split}
            \text{1、定义加法} \quad  &\xi=\psi+\varphi\\
            &\psi+\varphi=\varphi+\psi \qquad (\text{交换律})\\
            &(\psi+\varphi)+\xi=\psi+(\varphi+\xi) \qquad (\text{结合律})\\
            &\psi+\text{O}= \psi \qquad (\text{零元})\\
            &\psi+\varphi= \text{O} \qquad (\text{逆元})\\
        \end{split}  
    \end{equation*}
\end{frame} 

\begin{frame} 
    \begin{equation*}
        \begin{split}
            \text{2、定义数乘} \quad &\varphi=\psi a\\
            &\psi 1= \psi \qquad (\text{1元})\\
            &(\psi a)b=\psi (ab) \qquad (\text{结合律})\\
            &\psi(a+b)= \psi a+ \psi b \qquad (\text{第一分配律})\\
            &(\psi+\varphi) a= \psi a +\varphi a \qquad (\text{第二分配律})\\
            ~~\\
        \text{3、定义内积} \quad &(\psi, \varphi)=c\\
        &(\varphi,\psi)=c^*
        \end{split} 
    \end{equation*}
\end{frame} 

\begin{frame} 
    \frametitle{}
    \例 [1. 有定义在3维矢量空间的两矢量,求它们的内积]
    {\[ \psi=x_1\vec{i}+y_1\vec{j}+z_1\vec{k}, \qquad \varphi=x_2\vec{i}+y_2\vec{j}+z_2\vec{k}\] }
    \解 ~ \[(\psi, \varphi) = \psi \cdot \varphi= x_1x_2+y_1y_2+z_1z_2=c\]
\end{frame} 

\begin{frame} 
    \frametitle{}
    \例 [2. 有定义在$C^3$空间的列矩阵,求内积]
    { \[\psi=
        \begin{pmatrix}
                a_1\\
                a_2\\
                a_3
        \end{pmatrix}, \qquad 
        \varphi =\begin{pmatrix}
            b_1\\
            b_2\\
            b_3
    \end{pmatrix}
     \] 
    }
    \解 ~ \[(\psi, \varphi) = \begin{pmatrix}
        a_1 ^* &
        a_2 ^* &
        a_3 ^*
    \end{pmatrix}
        \begin{pmatrix}
        b_1\\
        b_2\\
        b_3
    \end{pmatrix}
    =a_1 ^* b_1 +a_2 ^* b_2 +a_3 ^* b_3
    =c 
    \]
    ~ \[(\varphi,\psi) = \begin{pmatrix}
        b_1 ^* &
        b_2 ^* &
        b_3 ^*
    \end{pmatrix}
        \begin{pmatrix}
        a_1\\
        a_2\\
        a_3
    \end{pmatrix}
    =b_1 ^* a_1 +b_2 ^* a_2 +b_3 ^* a_3
    =c^* 
    \]
\end{frame} 

\begin{frame} 
    \frametitle{}
    \例 [3. 求定义在实数空间的两函数的内积]{\[\psi(x)=\sin nx, \varphi(x)=\sin mx\]}
    \解 ~ \[(\psi, \varphi)=\int_{-l} ^{l} \sin nx \sin mx dx=\begin{cases}
         0, \qquad n\neq m \\
        \frac{l}{2} , \qquad n=m
    \end{cases}
    \]
\end{frame} 

\begin{frame} 
    \frametitle{}
    \例 [4. 求定义在复数空间的两函数的内积]{}
    \解 ~ \[(\psi, \varphi)=\int_a ^b \psi^*(x)  \varphi(x) dx=c\]
    \[(\varphi,\psi)=\int_a ^b \varphi^*(x)\psi(x) dx = (\int_a ^b \varphi(x)\psi^*(x) dx) ^* =c^*\]
\end{frame} 

\begin{frame}
    内积性质:
    \[(\psi, \varphi)= (\varphi,\psi)^* \]
    \[(\psi, \varphi+\xi)= (\psi, \varphi) + (\psi, \xi)\qquad (\text{分配律})\]
    \[(\psi, \varphi a)= (\psi, \varphi )a \]
    \[(\psi a, \varphi )= (\psi, \varphi )a^* \]    
    \[(\Psi,c_1\psi_1+c_2\psi_2)=(\Psi,c_1\psi_1)+(\Psi,c_2\psi_2)\]
    \[(\psi,\psi)= c\ge 0\]
\end{frame}

\begin{frame}{}
    *左矢与右矢\\ \vspace{0.6em}
    考察内积: $(\psi,\psi)=\int\psi^*\psi d\tau$ \\
    同一波函数放在左边还是右边,意义有所不同: \\
    右边是线性的:  $(\psi,a\psi)=a (\psi,\psi)$ \\
    左边是反线性的:   $(a\psi,\psi)=a^* (\psi,\psi)$  \\
    为了清楚地描述这种线性反线性特点,定义左矢和右矢
    $$\langle \psi |, \qquad |\psi \rangle $$ 
    内积的又一形式:\[(\psi,\varphi)\equiv \langle \psi | \varphi \rangle\]
    线性与反线性$$\langle a\psi | = \langle \psi |a^* ,\qquad |a\psi \rangle = a|\psi \rangle$$ 
\end{frame}

\begin{frame}
    \frametitle{希尔伯特空间定义}
    4、空间定义\\
   \begin{itemize}
       \Item 矢量空间:满足加法和数乘两种运算的集合
       \Item 内积空间:满足加法、数乘和内积三种运算的集合
       \Item 希尔伯特空间:  完全的内积空间\\
       ~~ \\
       *完全性:对给定任意小的实数$\varepsilon$,总有数N存在,当m, n>N时,有\\
       $$ (\psi_m -\psi_n, \psi_m -\psi_n )< \varepsilon $$
   \end{itemize} 
   \Tips ~ 物体的状态用希尔伯特空间的矢量描述
\end{frame} 

\subsection{算符表示}

\begin{frame}
    \frametitle{算符的定义:}
    \begin{definition}[算符]
    描述态矢量之间的映射关系,即算符作用于一个态矢量,映射到另一个态矢量。
        \[\hat{F} \rs{\Psi}=\rs{\psi}\]
    \end{definition}
    \Tips~在不引起不明意义的条件下,为简单见可略去算符的帽子
    \begin{definition}[单位算符]
    \[I\Psi=\Psi \]
    \end{definition}   
\end{frame} 

\begin{frame}
    \begin{definition}[算符相等]
        ~~\\
    对任意波函数,有
        $$ A\Psi=B\Psi \to A=B $$
    \end{definition}
    \begin{definition}[算符的和]
        $$ (A+B)\Psi=A\Psi+B\Psi $$
        存在交换律和结合律\\
        A+B=B+A\\
        (A+B)+C=A+(B+C)
    \end{definition}
\end{frame} 

\begin{frame}
    \begin{definition} [算符的积]
        $$ (AB)\Psi=A(B\Psi) $$
        不存在交换律,即 \\
        $AB=BA$ 或 $AB\ne BA$ 都有可能
    \end{definition}
    \begin{definition}[对易子]
        $$ [A,B]=AB-BA$$
        若[A,B]=0,称两算符对易,否则不对易
    \end{definition}
\end{frame} 

\begin{frame}
    \begin{definition}[逆算符]   
        $$ F\Psi=\psi $$
        $$ F^{-1}\psi=\Psi $$
    \end{definition}
    \begin{definition}[伴算符] 
        $$ F|\Psi> = |\psi> $$
        $$ <\psi| = <\Psi|F^{\dagger} $$
        内积形式:
        $$ (\varphi,F\Psi)=(\varphi,\psi)$$ 
        $$ (F^\dagger \Psi,\varphi)=(\psi,\varphi)$$ 
    \end{definition}  
\end{frame} 

\begin{frame} 
    \begin{definition}[自伴算符] 
        $$ F^{\dagger} = F $$
        性质:
        $$ (\Psi,F\psi)=(F\Psi,\psi), \quad \lcr{\Psi}{F^\dagger}{\psi}=(\lcr{\psi}{F}{\Psi})^*, $$ 
        也称厄密性,{\color{red}自伴算符即厄密算符}
    \end{definition} 
    \begin{definition}[线性算符]
    ~~\\
    对任意函数,有\\
        $$F(c_1\psi_1+c_2\psi_2 ) = c_1(F\psi_1)+c_2(F\psi_2 )$$
    \end{definition}
\end{frame} 

\begin{frame}
    \begin{definition}[幺正(酉)算符] 
        $$ F^{\dagger}F = FF^{\dagger}=I $$
        性质:
        $$ F^{\dagger}=F^{-1}$$ 
    \end{definition}     
    \Tips ~\\   
    厄密算符~~~~~~~~~ $ F=F^{\dagger}$  \\
    幺正算符~~~~~~~~~ $ F^{-1}=F^{\dagger}$  \\
    幺正厄密算符~~ $ F=F^{-1}=F^{\dagger}$  \\
\end{frame} 

\begin{frame} 
    \frametitle{力学量算符的获取}
    1. 位置算符和动量算符
    \例[1.已知粒子的位置波函数$\psi(x,t)$,求动量的期望值]{}   
    \解~ 由概率诠释,位置期望值为
    \begin{equation*}
        \bar{x}=\int x|\psi(x, t)|^{2} d x=\int \psi^{*}(x, t) x \psi(x, t) d x
    \end{equation*}
    对于动量波函数 $c(p,t)$, 动量期望值为
    \begin{equation*}
        \bar{p_x}=\int p_x|c(p_x, t)|^{2} d p_x=\int c^{*}(p_x, t) p c(p_x, t) d p_x
    \end{equation*}
    很明显,有
    \begin{equation*}
        \bar{p_x}\neq\int p_x|\psi(x, t)|^{2} d p_x
    \end{equation*}
\end{frame} 

\begin{frame}
    变换求解(注:为了方便,用$p \to p_x$)
    \begin{equation*}
        \begin{split}
            \bar{p}&=\int c^{*}(p) p c(p) d p \\  
            &=\int (\frac{1}{\sqrt{2 \pi \hbar}} \int \psi^{*}(x) e^{\frac{i}{\hbar} p\cdot x} d x) p c\left(p\right) d p \\
            &=\frac{1}{\sqrt{2 \pi \hbar}} \int \int \psi^{*}(x) (e^{\frac{i}{\hbar} p\cdot x}  p) c\left(p\right) d xd p \\
            &=\frac{1}{\sqrt{2 \pi \hbar}} \int \int \psi^{*}(x) \Myitem{t1}{red}{(-i\hbar\frac{d}{d x} e^{\frac{i}{\hbar} p\cdot x})} c(p) d xd p \\
            &=\int \psi^{*}(x) (-i\hbar\frac{d}{d x}) (\frac{1}{\sqrt{2 \pi \hbar}} \int e^{\frac{i}{\hbar} p\cdot x} c(p) d p)  d x\\
            &=\int \psi^{*}(x) (-i\hbar\frac{d}{d x}) \psi(x)  d x\\
         \end{split}
    \end{equation*}  
\end{frame} 

\begin{frame}
    定义如下计算符号:
    $$ \boxed{\hat{p}_x= -i\hbar\frac{d}{d x}} $$ 
    上式变为:         
    $$\boxed{\bar{p}_x=\int \psi^{*}(x) \hat{p}_x \psi(x) d x} $$
    称$ \hat{p}_x= -i\hbar\dfrac{d}{d x} $ 为位置表象里的动量算符表示($p_x$分量)\\
    同理,称 $\hat{x}= x $ 为位置表象里的位置算符表示($x$分量)\\ \vspace{0.3em}
    设任意力学量F有算符表示$\hat{F}$,有平均值公式\\
    $$\boxed{\bar{F}=\int \psi^{*}(x) \hat{F} \psi(x) d x =(\psi, \hat{F} \psi)}$$
\end{frame} 

\begin{frame} 
    \frametitle{}
    2.任意力学量的算符表示
    \begin{tcolorbox1}{命题:}
    已知位置、动量的算符表示如下,
    \begin{itemize}
        \Item  位置算符(三维): $ \hat{\vec{r}} =\vec{r} $
        \Item  动量算符(三维): $ \hat{\vec{p}} =-i\hbar(\dfrac{d}{d x}+ \dfrac{d}{d y} + \dfrac{d}{d z})=-i\hbar \nabla $
    \end{itemize}
    求任意其他力学量的算符表示
    \end{tcolorbox1}
    \alert{Bohm规则(1954):} 经典物理学存在力学量F,它若是位置与动量的函数
    \[F(\vec{r},\vec{p})\]
    则其量子力学算符为:
    \[\hat{F}=F(\hat{\vec{r}},\hat{\vec{p}})\]
\end{frame} 

\begin{frame}    
    \begin{exampleblock}{例如}
        \begin{itemize}
            \Item  动能: $ T=\dfrac{p^2}{2\mu} \to \hat{T}= \dfrac{\hat{p}^2}{2\mu} $
            \Item  哈密顿量: $ H=T+U(\vec{r} ) \to \hat{H}= \hat{T}+ U(\hat{\vec{r}})$
            \Item  角动量:$ \vec{L}=\vec{r}\times\vec{p} \to \hat{\vec{L}}=\hat{\vec{r}}\times \hat{\vec{p}}$
        \end{itemize}
    \end{exampleblock}  
    \alert{TIPS:} 若 $F(\vec{r},\vec{p})$ 中存在连乘项: 
    \[\vec{r}^m\cdot\vec{p}^n\] 
    则采用如下方式进行取代
    \[\frac{1}{2}(\hat{\vec{r}}^m\cdot\hat{\vec{p}}^n+\hat{\vec{p}}^n\cdot\hat{\vec{r}}^m)\]
\end{frame} 

\begin{frame} 
    \例[2. 求经典物理量$F=x^2p_x$的量子力学算符表示] {} 
    \解~ 根据Bohm规则,有:
    \[\begin{aligned}
        \hat{F}=\frac{1}{2} (\hat{x}^2 \hat{p}_x + \hat{p}_x \hat{x}^2 ) 
    \end{aligned}\]   
\end{frame}     

\begin{frame}
    \frametitle{算符的本征方程}
    在算符定义式中
    \[\hat{F} \rs{\Psi}=\rs{\varphi}\]
    若$\rs{\varphi}=f\rs{\Psi}$,有:\\
    \[\boxed{\hat{F} \rs{\Psi}=f\rs{\Psi}}\]
    ~~\\
    则称上式为算符$\hat{F}$的本征方程\\
    其中$f$是$\hat{F}$的本征值,$\rs{\Psi}$是属于本征值$f$的本征函数。
\end{frame}

\begin{frame} 
    \frametitle{力学量用希尔伯特空间的厄密算符表示}
    \begin{tcolorbox1}{证明命题1:}
      一切可观测力学量算符都是线性算符  
    \end{tcolorbox1}
    \alert{证明:}
        设$\psi_1, \psi_2$ 是算符$\hat{F}$的属于本征值$f$的两个解,有\\
        $$\hat{F}\psi_1=f\psi_1, \quad \to \quad c_1\hat{F}\psi_1=c_1f\psi_1 \cdots (1)$$
        $$\hat{F}\psi_2=f\psi_2, \quad \to \quad c_2\hat{F}\psi_2=c_2f\psi_2 \cdots (2)$$
        (1)+(2)
        $$(c_1f\psi_1+c_2f\psi_2)=c_1\hat{F}\psi_1+c_2\hat{F}\psi_2$$
        $$f(c_1\psi_1+c_2\psi_2)=c_1\hat{F}\psi_1+c_2\hat{F}\psi_2\cdots (3)$$
\end{frame} 

\begin{frame} 
    由于$\psi_1, \psi_2$ 都是属于本征值$f$的解,则它们的线性组合$c_1\psi_1+c_2\psi_2$也是属于本征值$f$的解, 即
    $$\hat{F}(c_1\psi_1+c_2\psi_2)=f(c_1\psi_1+c_2\psi_2)\cdots (4)$$
    联立(3)(4),有
    $$\hat{F}(c_1\psi_1+c_2\psi_2)=c_1\hat{F}\psi_1+c_2\hat{F}\psi_2$$
    证毕!
\end{frame} 

\begin{frame} 
    \begin{tcolorbox1}{证明命题2:}
        可观测力学量算符都是厄密算符  
    \end{tcolorbox1}
    \alert{证明:}
        对任意波函数$\Psi$, 力学量算符$F$的期望值为\\
        $$\bar{F}=\int \Psi^{*}(x) \hat{F} \Psi(x) d x=(\Psi,\hat{F} \Psi) $$
        $$\bar{F}^*=(\Psi, \hat{F} \Psi)^* = (\hat{F}\Psi, \Psi) $$
        可观测力学量的期望值都是实数,有:\\
        $$(\Psi,\hat{F}\Psi)=(\hat{F} \Psi, \Psi) $$
\end{frame} 

\begin{frame} [allowframebreaks=]
        取 $\Psi= \psi_1+c\psi_2 $, 代入上式,得:
        $$([\psi_1+c\psi_2],\hat{F} [\psi_1+c\psi_2])=(\hat{F}[\psi_1+c\psi_2],[\psi_1+c\psi_2]) $$
        积分:
        $$
        \begin{array}{r}
        \left(\psi_{1}, \hat{F} \psi_{1}\right)+c^{*}\left(\psi_{2}, \hat{F} \psi_{1}\right)+c\left(\psi_{1}, \hat{F} \psi_{2}\right)+|c|^{2}\left(\psi_{2}, \hat{F} \psi_{2}\right) \\
        =\left(\hat{F} \psi_{1}, \psi_{1}\right)+c^{*}\left(\hat{F} \psi_{2}, \psi_{1}\right)+c\left(\hat{F} \psi_{1}, \psi_{2}\right)+|c|^{2}\left(\hat{F} \psi_{2}, \psi_{2}\right)
        \end{array}
        $$
        算符的平均值都是实数,即 
        $$(\psi_1,\hat{F}\psi_1)=(\hat{F} \psi_1, \psi_1), \qquad (\psi_2,\hat{F}\psi_2)=(\hat{F} \psi_2, \psi_2) $$
        上式可消去第一、四项,变为:
        $$\begin{array}{r}
            c^{*}\left(\psi_{2}, \hat{F} \psi_{1}\right)+c\left(\psi_{1}, \hat{F} \psi_{2}\right) \\
            =c^{*}\left(\hat{F} \psi_{2}, \psi_{1}\right)+c\left(\hat{F} \psi_{1}, \psi_{2}\right)
        \end{array}$$
        分别取$c=1$, $c=i$代入,得到两个等式:
        $$  \left(\psi_{2}, \hat{F} \psi_{1}\right)+\left(\psi_{1}, \hat{F} \psi_{2}\right) = 
        \left(\hat{F} \psi_{2}, \psi_{1}\right)+\left(\hat{F} \psi_{1}, \psi_{2}\right) , \cdots (1)
        $$
        $$
        -i\left(\psi_{2}, \hat{F} \psi_{1}\right)+i\left(\psi_{1}, \hat{F} \psi_{2}\right) 
        =-i\left(\hat{F} \psi_{2}, \psi_{1}\right)+i\left(\hat{F} \psi_{1}, \psi_{2}\right),\cdots (2)
        $$
        第二式乘以$i$,得:
        $$
        \left(\psi_{2}, \hat{F} \psi_{1}\right)-\left(\psi_{1}, \hat{F} \psi_{2}\right) 
        =\left(\hat{F} \psi_{2}, \psi_{1}\right)-\left(\hat{F} \psi_{1}, \psi_{2}\right), \cdots (3)
        $$
        (1)+(3),两边除以2,得
        $$
        \left(\psi_{2}, \hat{F} \psi_{1}\right) =\left(\hat{F} \psi_{2}, \psi_{1}\right)
        $$
        证毕!
\end{frame} 

%%%%%%%%%%%%%%%%%%%%%%%%%%%%%%%%%%%%%%%%%%%%%%%%%%%%%%%%%%%%%%%%%%%
\begin{frame}
    \frametitle{课外作业3-1}
    \begin{enumerate}
        \item 试证明如下两个态正交
        \[\rs{\psi_1}=\frac{1}{\sqrt{2}}
        \begin{bmatrix}
            1 \\ 
            0 \\  
            0 \\ 
            1 
        \end{bmatrix}, \qquad  \rs{\psi_2}=\frac{1}{\sqrt{2}}
        \begin{bmatrix}
            0 \\ 
            1 \\  
            1 \\ 
            0 
        \end{bmatrix} \]
        \item 试证明如下两个态正交
        \[ \rs{\psi_1}=\sin nx,\qquad \rs{\psi_2}=\sin mx,\qquad  \left|x\right|<l      
        \]
        \item 试证明处于定态的粒子的动量平均值不随时间变化
        \item 设氢原子处于基态$\psi_{100}$,求径向位置r,动量和动能的平均值. 
    \end{enumerate}
    
\end{frame}
%%%%%%%%%%%%%%%%%%%%%%%%%%%%%%%%%%%%%%%%%%%%%%%%%%%%%%%%%%%%%%%%%%%

