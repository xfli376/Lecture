%%%%%%%%%%%%%%%%%%%%%%%%%%%%%%%%%%%55%%
\begin{frame} [plain]
    \frametitle{}
    \Background[1] 
    \begin{center}
    { {\huge 第三章、量子力学中的力学量 \\ (12学时)}}
    \end{center}  
    \addtocounter{framenumber}{-1}   
\end{frame}
%%%%%%%%%%%%%%%%%%%%%%%%%%%%%%%%%%

\section{力学量算符表示}

\begin{frame}
    \frametitle{前情回顾}
    \begin{itemize}
        \item 波粒二象性
        \item 波函数假说
        \item 波函数的统计解释
        \item 态叠加原理
        \item 薛定谔方程
    \end{itemize}
    ~~\\ \vspace{1.0em}
    \hspace{2em}\alert{TIPS:}这些都是有关量子态的问题,那物理量又如何呢?
\end{frame} 

\subsection{力学量算符表示假说}

\begin{frame}    
    \begin{tcolorbox4}[Basic assumption 3/5]
    力学量用厄密算符表示
    \end{tcolorbox4}
\end{frame} 

\begin{frame} 
    \frametitle{}
    \begin{exampleblock}{考察平均值问题}
        已知粒子的位置波函数$\psi(x,t)$,求动量的期望值   
    \end{exampleblock}
    \alert{解:} 由概率解释知,位置的期望值为
    \begin{equation*}
        \bar{x}=\int x|\psi(x, t)|^{2} d x=\int \psi^{*}(x, t) x \psi(x, t) d x
    \end{equation*}
    对于动量波函数 $c(p,t)$, 动量的期望值为
    \begin{equation*}
        \bar{p_x}=\int p_x|c(p_x, t)|^{2} d p_x=\int c^{*}(p_x, t) p c(p_x, t) d p_x
    \end{equation*}
    很明显,对于已知位置波函数$\psi(x,t)$的条件下,
    \begin{equation*}
        \bar{p_x}\neq\int p_x|\psi(x, t)|^{2} d p_x
    \end{equation*}
\end{frame} 

\begin{frame}
    但可以通过变换求解(注:为了方便,用$p \to p_x$)
    \begin{equation*}
        \begin{split}
            \bar{p}&=\int c^{*}(p) p c(p) d p \\  
            &=\int (\frac{1}{\sqrt{2 \pi \hbar}} \int \psi^{*}(x) e^{\frac{i}{\hbar} p\cdot x} d x) p c\left(p\right) d p \\
            &=\frac{1}{\sqrt{2 \pi \hbar}} \int \int \psi^{*}(x) (e^{\frac{i}{\hbar} p\cdot x}  p) c\left(p\right) d xd p \\
            &=\frac{1}{\sqrt{2 \pi \hbar}} \int \int \psi^{*}(x) (-i\hbar\frac{d}{d x} e^{\frac{i}{\hbar} p\cdot x}) c(p) d xd p \\
            &=\int \psi^{*}(x) (-i\hbar\frac{d}{d x}) (\frac{1}{\sqrt{2 \pi \hbar}} \int e^{\frac{i}{\hbar} p\cdot x} c(p) d p)  d x\\
            &=\int \psi^{*}(x) (-i\hbar\frac{d}{d x}) \psi(x)  d x\\
         \end{split}
    \end{equation*}  
\end{frame} 

\begin{frame}
    定义如下计算符号:
    $$ \hat{p}_x= -i\hbar\frac{d}{d x} $$ 
    上式变为:         
    $$\bar{p}_x=\int \psi^{*}(x) \hat{p}_x \psi(x) d x $$
    称$ \hat{p}_x= -i\hbar\dfrac{d}{d x} $ 是位置表象里的动量算符($x$分量)\\
    同理,任意力学量F的也有算符$\hat{F}$,其平均值为\\
    $$\bar{F}=\int \psi^{*}(x) \hat{F} \psi(x) d x $$
    表明算符与物理量关系密切
\end{frame} 

\begin{frame} 
    \frametitle{一般力学量算符的获取}
    \begin{tcolorbox1}{命题:}
    已知两个基本力学量的算符形式
    \begin{itemize}
        \item  位置算符: $ \hat{\vec{r}} =\vec{r} $
        \item  动量算符: $ \hat{\vec{p}} =-i\hbar(\dfrac{d}{d x}+ \dfrac{d}{d y} + \dfrac{d}{d z})=-i\hbar \nabla $
    \end{itemize}
    求其他一般力学量的算符形式
    \end{tcolorbox1}
    \alert{Bohm规则(1954):} 经典物理学存在力学量F,它若是位置与动量的函数
    \[F(\vec{r},\vec{p})\]
    则其在量子力学中的算符为:
    \[\hat{F}=F(\hat{\vec{r}},\hat{\vec{p}})\]
\end{frame} 

\begin{frame}    
    \begin{exampleblock}{例如:}
        \begin{itemize}
            \item  动能: $ T=\dfrac{p^2}{2\mu} \to \hat{T}= \dfrac{\hat{p}^2}{2\mu} $
            \item  哈密顿量: $ H=T+U(\vec{r} ) \to \hat{H}= \hat{T}+ U(\hat{\vec{r}})$
            \item  角动量:$ \vec{L}=\vec{r}\times\vec{p} \to \hat{\vec{L}}=\hat{\vec{r}}\times \hat{\vec{p}}$
        \end{itemize}
    \end{exampleblock}  
    \alert{TIPS:} 若 $F(\vec{r},\vec{p})$ 中存在连乘项: 
    \[\vec{r}^m\cdot\vec{p}^n\] 
    则采用如下方式进行取代
    \[\frac{1}{2}(\hat{\vec{r}}^m\cdot\hat{\vec{p}}^n+\hat{\vec{p}}^n\cdot\hat{\vec{r}}^m)\]
\end{frame} 

\begin{frame} 
    \alert{例1:} 求经典物理量$F=x^2p_x$的量子力学算符形式 \\ 
    \alert{解:} 根据Bohm规则,有:
    \[\begin{aligned}
        \hat{F}&=\frac{1}{2} (\hat{x}^2 \hat{p}_x + \hat{p}_x \hat{x}^2 ) \\
    \end{aligned}\]   
\end{frame}     

\subsection{力学算符是线性厄密算符}

\begin{frame} 
    \frametitle{}
    算符的一般性定义:
    \begin{definition}
        算符:描述态函数之间的对应关系,即算符作用于一个态函数,使它变成另一个态函数。
        $$ \hat{F} \Psi=\psi$$
    \end{definition}
    注:,在不引起不明意义的条件下,为简单见可略去算符的帽子
    \begin{definition}
        单位算符:
        $$ I\Psi=\Psi $$
    \end{definition}
\end{frame} 

\begin{frame}
    \begin{definition}
        算符相等 : 对任意波函数,有
        $$ A\Psi=B\Psi \to A=B $$
    \end{definition}
    \begin{definition}
        算符的和 : 
        $$ (A+B)\Psi=A\Psi+B\Psi $$
        存在交换律和结合律\\
        A+B=B+A\\
        (A+B)+C=A+(B+C)
    \end{definition}
\end{frame} 

\begin{frame}
    \begin{definition}
        算符的积: 
        $$ (AB)\Psi=A(B\Psi) $$
        不存在交换律
        $AB=BA$ 或 $AB\ne BA$ 都有可能
    \end{definition}
    \begin{definition}
        对易子: 
        $$ [A,B]=AB-BA$$
        若[A,B]=0,称两算符对易,否则不对易
    \end{definition}
\end{frame} 

\begin{frame}
    \begin{definition}
        逆算符: 
        $$ F\Psi=\psi $$
        $$ F^{-1}\psi=\Psi $$
    \end{definition}
    \begin{definition}
        内积: 
        $$ \int\Psi^*\psi d \tau$$
        称为两函数的内积,
    \end{definition} 
        内积可以简写成
        $$ (\Psi,\psi)$$ 
        $$ <\Psi|\psi>$$
        并称 $ <\psi|$为左矢, $|\psi>$为右矢\\
    \end{frame} 

    \begin{frame}        
        内积性质:
        $$ (\Psi,\psi)^*=(\psi,\Psi)$$ 
        $$ (\Psi,c_1\psi_1+c_2\psi_2)=(\Psi,c_1\psi_1)+(\Psi,c_2\psi_2)$$ 
        $$ (\Psi,c\psi)=c(\Psi,\psi)$$ 
        $$ (c\Psi,\psi)=c^*(\Psi,\psi)$$ 
  
    \begin{definition}
        伴算符: 
        $$ F|\Psi> = |\psi> $$
        $$ <\psi| = <\Psi|F^{\dagger} $$
        内积形式:
        $$ (\psi,F\Psi)=(\psi,\psi)$$ 
        $$ (F^\dagger \Psi,\psi)=(\psi,\psi)$$ 
    \end{definition}  
\end{frame} 

\begin{frame} 
        \begin{definition}
        线性算符:对任意函数,有\\
        $$F(c_1\psi_1+c_2\psi_2 ) = c_1(F\psi_1)+c_2(F\psi_2 )$$
    \end{definition}
    \begin{definition}
        自伴算符: 
        $$ F^{+} = F $$
        性质:
        $$ (\Psi,F\psi)=(F\Psi,\psi)$$ 
        也称厄密性,即厄密算符就是自伴算符
    \end{definition} 
\end{frame} 

\begin{frame}
    \begin{definition}
        幺正(酉)算符: 
        $$ F^{\dagger}F = FF^{\dagger}=I $$
        性质:
        $$ F^{\dagger}=F^{-1}$$ 
    \end{definition}         
\end{frame} 

\begin{frame} 
    \frametitle{力学量算符的性质}
    \begin{tcolorbox1}{命题1:}
      可观测力学量算符都是线性算符  
    \end{tcolorbox1}
    \alert{证明:}
        设$\psi_1, \psi_2$ 是算符$\hat{F}$的属于本征值$f$的两个解,有\\
        $$\hat{F}\psi_1=f\psi_1, \to c_1\hat{F}\psi_1=c_1f\psi_1 $$
        $$\hat{F}\psi_2=f\psi_2, \to c_2\hat{F}\psi_2=c_2f\psi_2 $$
        $$f(c_1f\psi_1+c_2f\psi_2)=c_1\hat{F}\psi_1+c_2\hat{F}\psi_2$$
        $$\hat{F}(c_1f\psi_1+c_2f\psi_2)=c_1\hat{F}\psi_1+c_2\hat{F}\psi_2$$
    证毕!
\end{frame} 

\begin{frame} 
    \begin{tcolorbox1}{命题2:}
        可观测力学量算符都是厄密算符  
    \end{tcolorbox1}
    \alert{证明:}
        对任意波函数$\Psi$, 力学量算符$F$的期望值为\\
        $$\bar{F}=(\Psi,\hat{F} \Psi) $$
        $$\bar{F}^*=(\Psi, \hat{F} \Psi)^* = (\hat{F}\Psi, \Psi) $$
        可观测力学量的期望值都是实数,有:\\
        $$(\Psi,\hat{F}\Psi)=(\hat{F} \Psi, \Psi) $$
\end{frame} 

\begin{frame} [allowframebreaks=]
        取 $\Psi= \psi_1+c\psi_2 $, 代入上式,得:
        $$([\psi_1+c\psi_2],\hat{F} [\psi_1+c\psi_2])=(\hat{F}[\psi_1+c\psi_2],[\psi_1+c\psi_2]) $$
        进行积分,得:
        $$
        \begin{array}{r}
        \left(\psi_{1}, \hat{F} \psi_{1}\right)+c^{*}\left(\psi_{2}, \hat{F} \psi_{1}\right)+c\left(\psi_{1}, \hat{F} \psi_{2}\right)+|c|^{2}\left(\psi_{2}, \hat{F} \psi_{2}\right) \\
        =\left(\hat{F} \psi_{1}, \psi_{1}\right)+c^{*}\left(\hat{F} \psi_{2}, \psi_{1}\right)+c\left(\hat{F} \psi_{1}, \psi_{2}\right)+|c|^{2}\left(\hat{F} \psi_{2}, \psi_{2}\right)
        \end{array}
        $$
        算符的平均值都是实数,即 
        $$(\psi_1,\hat{F}\psi_1)=(\hat{F} \psi_1, \psi_1), \qquad (\psi_2,\hat{F}\psi_2)=(\hat{F} \psi_2, \psi_2) $$
        上式可消去第一、四项,变为:
        $$\begin{array}{r}
            c^{*}\left(\psi_{2}, \hat{F} \psi_{1}\right)+c\left(\psi_{1}, \hat{F} \psi_{2}\right) \\
            =c^{*}\left(\hat{F} \psi_{2}, \psi_{1}\right)+c\left(\hat{F} \psi_{1}, \psi_{2}\right)
        \end{array}$$
        分别取$c=1$, $c=i$代入,得到两个等式:
        $$  \left(\psi_{2}, \hat{F} \psi_{1}\right)+\left(\psi_{1}, \hat{F} \psi_{2}\right) = 
        \left(\hat{F} \psi_{2}, \psi_{1}\right)+\left(\hat{F} \psi_{1}, \psi_{2}\right) , \cdots (1)
        $$
        $$
        -i\left(\psi_{2}, \hat{F} \psi_{1}\right)+i\left(\psi_{1}, \hat{F} \psi_{2}\right) 
        =-i\left(\hat{F} \psi_{2}, \psi_{1}\right)+i\left(\hat{F} \psi_{1}, \psi_{2}\right)
        $$
        第二式乘以$i$,得:
        $$
        \left(\psi_{2}, \hat{F} \psi_{1}\right)-\left(\psi_{1}, \hat{F} \psi_{2}\right) 
        =\left(\hat{F} \psi_{2}, \psi_{1}\right)-\left(\hat{F} \psi_{1}, \psi_{2}\right), \cdots (2)
        $$
        (1)+(2),并两边除以2,得
        $$
        \left(\psi_{2}, \hat{F} \psi_{1}\right) =\left(\hat{F} \psi_{2}, \psi_{1}\right)
        $$
        证毕!
\end{frame} 
