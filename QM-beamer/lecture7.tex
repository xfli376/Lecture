%%%%%%%%%%%%%%%%%%%%%%%%%%%%%%%%%%%%%%%%%%
\begin{frame}
    \frametitle{}
    \begin{center}
    { {\huge 第七讲、厄密算符的性质}}
    \end{center}    
\end{frame}
%%%%%%%%%%%%%%%%%%%%%%%%%%%%%%%%%%%%%


\section{前情回顾}

\begin{frame}
    \frametitle{前情回顾}
    \begin{itemize}
        \item 波函数描述体系的状态
        \item 算符给出体系的物理量
        \item 力学量算符是线性厄密算符
    \end{itemize}
\end{frame} 

\section{厄密算符运算性质}

\begin{frame}
    \frametitle{运算性质}
    \begin{enumerate}
        \item 两厄米算符之和仍为厄米算符
        \item 当且仅当两厄米算符对易时,它们之积才是厄米算符。
        \item 无论两厄米算符是否对易,算符$\dfrac{1}{2}(AB+BA)$ 及$\dfrac{1}{2i}(AB-BA) $  都是厄米算符。
        \item 任意算符总可以分解成$A=A_+ +iA_-$,且$A_+$和$A_-$,都是厄米算符
    \end{enumerate}
\end{frame} 

\begin{frame} [allowframebreaks=]
    \frametitle{}
    \begin{exampleblock}{}
        1、试证明两厄米算符之和仍为厄米算符
    \end{exampleblock}
    \alert{证明:}A,B为厄米算符,则对于任意态,有\\
    $$(\psi, A\psi ) = (A\psi, \psi), \qquad (\psi, B\psi ) = (B\psi, \psi)$$
    对于它们的和,有: \\
    \begin{equation*}
        \begin{split}
            (\psi, (A+B)\psi ) &= (\psi, A\psi ) + (\psi, B\psi ) \\  
            &=(A\psi, \psi ) + (B\psi, \psi ) \\
            &=((A+B)\psi, \psi ) 
         \end{split}
    \end{equation*}  
    证毕!
\end{frame} 

\begin{frame} [allowframebreaks=]
    \frametitle{}
    \begin{exampleblock}{}
        2、当且仅当两厄米算符对易时,它们之积才是厄米算符。
    \end{exampleblock}
    \alert{证明:}A,B为厄米算符,则对于任意态,
    \begin{equation*}
        \begin{split}
            (\psi, (AB)\psi ) &= (\psi, A(B\psi) ) \\  
            &=((A \psi), (B\psi) )  \\
            &=(B(A \psi), \psi )  \\
            &=( (BA) \psi, \psi )  \\
            &=( (AB) \psi, \psi )  \\
         \end{split}
    \end{equation*}  
    证毕!
\end{frame}  

\begin{frame} [allowframebreaks=]
    \frametitle{}
    \begin{exampleblock}{}
        3、无论两厄米算符是否对易,算符$\dfrac{1}{2}(AB+BA)$ 及 $\dfrac{1}{2i}(AB-BA) $ 都是厄米算符
    \end{exampleblock}
    \alert{证明:}A,B为厄米算符,则对于任意态,
    \begin{equation*}
        \begin{split}
            (\psi, \dfrac{1}{2}(AB+BA)\psi ) &=\dfrac{1}{2}(\psi, AB\psi) + \dfrac{1}{2}(\psi, BA\psi)  \\
            &=\dfrac{1}{2}(A\psi, B\psi) + \dfrac{1}{2}(B\psi, A\psi)  \\
            &=\dfrac{1}{2}(BA\psi, \psi) + \dfrac{1}{2}(AB\psi, \psi)  \\
            &=\dfrac{1}{2}((BA+AB)\psi, \psi) \\
            &=(\dfrac{1}{2}(BA+AB)\psi, \psi) \\
            &=(\dfrac{1}{2}(AB+BA)\psi, \psi) 
         \end{split}
    \end{equation*}  
    \begin{equation*}
        \begin{split}
            (\psi, \dfrac{1}{2i}(AB-BA)\psi ) &= (\psi, \dfrac{1}{2i}AB\psi) - (\psi, \dfrac{1}{2i}BA\psi)\\  
            &=\dfrac{1}{2i}(\psi, AB\psi) - \dfrac{1}{2i}(\psi, BA\psi)  \\
            &=\dfrac{1}{2i}(A\psi, B\psi) - \dfrac{1}{2i}(B\psi, A\psi)  \\
            &=\dfrac{1}{2i}(BA\psi, \psi) - \dfrac{1}{2i}(AB\psi, \psi)  \\
            &=-(\dfrac{1}{2i}BA\psi, \psi) +(\dfrac{1}{2i}AB\psi, \psi)  \\
            &=(\dfrac{1}{2i}(AB-BA)\psi, \psi) \\
         \end{split}
    \end{equation*}  
    证毕!
\end{frame}  

\begin{frame} [allowframebreaks=]
    \frametitle{}
    \begin{exampleblock}{}
        4、任意算符总可以分解成$A=A_+ +iA_-$,且$A_+$和$A_-$,都是厄米算符
    \end{exampleblock}
    \alert{证明:}令:
    $$A_+=\dfrac{1}{2} (A+A^+), \qquad A_+=\dfrac{1}{2i} (A-A^+) $$
    有$A=A_+ +iA_-$, 问题转化为证$\dfrac{1}{2} (A+A^+), \qquad \dfrac{1}{2i} (A-A^+) $是厄米算符\\
    \begin{equation*}
        \begin{split}
            (\psi, \dfrac{1}{2} (A+A^+)\psi ) &=\dfrac{1}{2}(\psi, (A)\psi) + \dfrac{1}{2}(\psi, (A^+)\psi) \\
            &= \dfrac{1}{2}((A^+)\psi, \psi) + \dfrac{1}{2}((A^+)^+\psi, \psi) \\
            &= \dfrac{1}{2}((A^+)\psi, \psi) + \dfrac{1}{2}(A\psi, \psi) \\
            &= \dfrac{1}{2}( (A^+ + A) \psi, \psi ) =( \dfrac{1}{2}(A+A^+) \psi, \psi ) 
         \end{split}
    \end{equation*}  
\end{frame} 

\section{厄密算符本征性质}

\begin{frame}
    \frametitle{本征方程}
    厄密算符A的本征方程为
    $$ A\psi=a \psi $$
    称a是算符A的本征值,$\psi$是属于本征值a的本征函数
\end{frame} 

\begin{frame}
    \frametitle{本征性质}
    \begin{enumerate}
        \item 厄米算符的本征值为实数
        \item 任意态下平均值为实数的算符必为厄米算符
        \item 厄米算符属于不同本征值的本征函数正交
        \item 简并的本征函数可重组成正交
        \item 厄米算符的本征函数系具有完备性
        \item 厄米算符的本征函数系具有封闭性
    \end{enumerate}
\end{frame} 


\section{希尔伯特空间}