%%%%%%%%%%%%%%%%%%%%%%%%%%%%%%%%%%%%%%%%%%
\begin{frame}
    \frametitle{}
    \begin{center}
    { {\huge 第九讲、对易关系求解}}
    \end{center}    
\end{frame}
%%%%%%%%%%%%%%%%%%%%%%%%%%%%%%%%%%%%%


\section{前情回顾}

\begin{frame}
    \frametitle{前情回顾}
    \begin{itemize}
        \item 希尔伯特空间的态矢量描述体系状态
        \item 希尔伯特空间的算符给出体系的物理量
        \item 算符的本征函数系构成正交归一完全基
        \item 常见算符本征方程求解
    \end{itemize}   
\end{frame} 

\section{对易子及运算法则}
\begin{frame} 
    \frametitle{对易子定义}
    \begin{definition}
        定义对易子:$$ [F,G]\equiv FG-GF $$ \\
        若$[F,G]=0$,则对易 \\
        若$[F,G]\neq0$,则不对易
    \end{definition}
\end{frame} 

\begin{frame} 
    \frametitle{运算法则}
    \begin{enumerate}
        \item  $[A,B]=-[B,A]$
        \item  $[A,A]=0$
        \item  $[A,c]=0$
        \item  $[A,B+C]=[A,B]+[A,C]$
        \item  $[A,BC]=B[A,C]+[A,B]C$
        \item  $[AB,C]=A[B,C]+[A,C]B$
    \end{enumerate}
\end{frame} 
\begin{frame}
    \begin{proof}{}   
        \begin{equation*}
            \begin{split} 
            [AB,C]&=ABC-CAB \\
            A[B,C]+[A,C]B&=A(BC-CB)+(AC-CA)B\\
            &=ABC-ACB+ACB-CAB\\
            &=ABC-CAB
            \end{split}  
        \end{equation*}  
        $$ [AB,C]=A[B,C]+[A,C]B $$
    \end{proof}
\end{frame} 

\begin{frame} 
    \begin{tcolorbox}[colback=yellow!5,colframe=red!75!black,title=推论:]
        \begin{itemize}
            \item 如果A与B、C对易,则A与$B+C$对易
            \item 如果A与B、C对易,则A与$B^n+C^m$对易
            \item 如果A与B、C对易,则A与$BC$对易
            \item 如果A与B、C对易,则A与$B^nC^m$对易
            \item 如果A与B、C对易,则A与$B^nC^m+B^{n'}C^{m'}$对易
        \end{itemize}
    \end{tcolorbox}
\end{frame} 

\begin{frame} [allowframebreaks=]
    \begin{proof}{}
        \begin{equation*}
            \begin{split} 
             [A,B^{n}]&=[A,BB^{n-1}] \\
             &=B[A,B^{n-1}]+[A,B]B^{n-1}\\
             &=B[A,B^{n-1}] \\
             &=B^2[A,B^{n-2}]\\
             &=\cdots\\
             &=B^{n-1}[A,B]\\
             &= 0\\
            \end{split}  
        \end{equation*}  
    \end{proof}
    \begin{proof}{}
        同理:$[A,C^{m}]=0$\\
        \begin{equation*}
            \begin{split} 
            [A,B^{n}+C^{m}] &= [A,B^{n}]+[A,C^{m}]\\
            &=0 \\
            [A,B^{n}C^{m}] &= B^{n}[A,C^{m}] + [A,B^{n}] C^{m}\\
            &=0 \\
        \end{split}  
        \end{equation*}
        同理:$[A,B^{n'}C^{m'}]=0$\\
        \begin{equation*}
            \begin{split} 
            [A,B^{n}C^{m}+B^{n'}C^{m'}] &= [A,B^{n}C^{m}]+[A,B^{n'}C^{m'}]\\
            &=0\\
        \end{split}  
        \end{equation*}
    \end{proof}
\end{frame} 

\section{对易关系}
\begin{frame} [allowframebreaks=]
    \frametitle{1.位置-动量对易关系}
    \begin{exampleblock}{}
     求位置动量对易关系 $[x,p_x]=?$,  $[x,p_y]=?$
    \end{exampleblock}
    \alert{解:} 对任意波函数$\psi$,有
    \begin{equation*}
        \begin{split}
        xp_x\psi&= x(-i\hbar \frac{\partial}{\partial x})\psi \\
        &=-i\hbar x \frac{\partial}{\partial x}\psi\\
        p_x x \psi&= -i\hbar \frac{\partial}{\partial x} (x\psi) \\
        &=-i\hbar\psi - i\hbar x \frac{\partial}{\partial x}\psi \\
        \end{split}  
    \end{equation*}
    两式相减,$$(xp_x-p_x x)\psi= i\hbar\psi$$
    得 $xp_x-p_x x= i\hbar$ \\
    即 $[x,p_x]= i\hbar$\\
    同理,有:\\
    $\begin{cases}
        [x,p_x]= i\hbar  \\ 
        [y,p_y]= i\hbar  \\ 
        [z,p_z]= i\hbar  
    \end{cases}$
    $\begin{cases}
        [x,p_y]= 0  \\ 
        [y,p_z]= 0  \\ 
        [z,p_x]= 0  
    \end{cases}$
    $\begin{cases}
        [p_x,p_y]= 0  \\ 
        [p_y,p_z]= 0  \\ 
        [p_z,p_x]= 0  
    \end{cases}$
    $\begin{cases}
        [x,y]= 0  \\ 
        [y,z]= 0  \\ 
        [z,x]= 0  
    \end{cases}$ \\
    \begin{tcolorbox}[colback=yellow!5,colframe=red!75!black,title=量子力学基本对易关系]
    $\begin{cases}
        [x_\alpha,x_\beta]= 0  \\ 
        [p_\alpha,p_\beta]= 0  \\ 
        [x_\alpha,p_\beta]= i\hbar \delta_{\alpha\beta}  \\ 
    \end{cases}$
    \end{tcolorbox}
\end{frame} 

\begin{frame} [allowframebreaks=]
    \frametitle{2.角动量-位置}
    \begin{exampleblock}{}
     试证明角动量-位置对易关系 $[L_x,y]=i\hbar z,  [L_x,x]=0$
    \end{exampleblock}
    \alert{证明:} 
    \begin{equation*}
        \begin{split}
        [L_x,y]&= [yp_z-zp_y,y]\\
        &=-[y,yp_z-zp_y]\\
        &= -[y,yp_z] + [y,zp_y]\\
        &=-y[y,p_z] -[y,y]p_z + z[y,p_y] + [y,z]p_y\\
        &=-0 -0 + z i\hbar + 0\\
        &=i\hbar z \\
        \end{split}  
    \end{equation*}
    同理,有:\\
    $\begin{cases}
        [L_x,y]= i\hbar z  \\ 
        [L_x,x]= 0  \\ 
        [L_x,z]= -i\hbar y 
    \end{cases}$
    $\begin{cases}
        [L_y,z]= i\hbar x  \\ 
        [L_y,y]= 0  \\ 
        [L_y,x]= -i\hbar z 
    \end{cases}$
    $\begin{cases}
        [L_z,x]= i\hbar y  \\ 
        [L_z,z]= 0  \\ 
        [L_z,y]= -i\hbar x 
    \end{cases}$
    \begin{tcolorbox}[colback=yellow!5,colframe=red!75!black,title=角动量-位置对易关系]
        $$ [L_\alpha,x_\beta]= \varepsilon_{\alpha\beta\gamma} i\hbar x_\gamma $$ 
    \end{tcolorbox}
\end{frame} 

\begin{frame} [allowframebreaks=]
    \frametitle{3.角动量-动量}
    \begin{exampleblock}{}
     试证明角动量-动量对易关系 $[L_x,p_y]=i\hbar p_z,  [L_x,p_x]=0$
    \end{exampleblock}
    \alert{证明:} 
    \begin{equation*}
        \begin{split}
        [L_x,p_y]&= [yp_z-zp_y,p_y]\\
        &=-[p_y,yp_z-zp_y]\\
        &=-[p_y,yp_z] + [p_y,zp_y]\\
        &=-y[p_y,p_z] -[p_y,y]p_z + z[p_y,p_y] + [p_y,z]p_y\\
        &=-y[p_y,p_z] +[y,p_y]p_z + z[p_y,p_y] + [p_y,z]p_y\\
        &=-0 + i\hbar p_z + 0+0\\
        &=i\hbar p_z \\
        \end{split}  
    \end{equation*}
    同理,有:\\
    $\begin{cases}
        [L_x,p_y]= i\hbar p_z  \\ 
        [L_x,p_x]= 0  \\ 
        [L_x,p_z]= -i\hbar p_y 
    \end{cases}$
    $\begin{cases}
        [L_y,p_z]= i\hbar p_x  \\ 
        [L_y,p_y]= 0  \\ 
        [L_y,p_x]= -i\hbar p_z 
    \end{cases}$
    $\begin{cases}
        [L_z,p_x]= i\hbar p_y  \\ 
        [L_z,p_z]= 0  \\ 
        [L_z,p_y]= -i\hbar p_x 
    \end{cases}$
    \begin{tcolorbox}[colback=yellow!5,colframe=red!75!black,title=角动量-动量对易关系]
        $$ [L_\alpha,p_\beta]= \varepsilon_{\alpha\beta\gamma} i\hbar p_\gamma $$ 
    \end{tcolorbox}
\end{frame} 

\begin{frame} [allowframebreaks=]
    \frametitle{4.角动量-角动量}
    \begin{exampleblock}{}
     试证明位置角动量对易关系 $[L_x,L_y]=i\hbar L_z,  [L_z, L^2]= 0$
    \end{exampleblock}
    \alert{证明1:} 基于运算法则
    \begin{equation*}
        \begin{split}
        [L_x,L_y]= &[yp_z-zp_y,zp_x-xp_z]\\
        =&[yp_z,zp_x-xp_z] - [zp_y,zp_x-xp_z]\\
        =&y[p_z,zp_x-xp_z]+[y,zp_x-xp_z]p_z- z[p_y,zp_x-xp_z]-[z,zp_x-xp_z]p_y\\
        =&y[p_z,zp_x]-y[p_z,xp_z]+[y,zp_x]p_z-[y,xp_z]p_z \\ &-z[p_y,zp_x] +z[p_y,xp_z]- [z,zp_x]p_y+[z,xp_z]p_y\\
        =&yz[p_z,p_x]+y[p_z,z]p_x-yx[p_z,p_z]-y[p_z,x]p_z \\ & +z[y,p_x]p_z+[y,z]p_xp_z-x[y,p_z]p_z -[y,x]p_zp_z \\ & -zz[p_y,p_x] -z[p_y,z]p_x +zx[p_y,p_z] +z[p_y,x]p_z \\ & -z[z,p_x]p_y -[z,z]p_xp_y+x[z,p_z]p_y +[z,x]p_zp_y\\  
    \end{split}  
    \end{equation*}
    \begin{equation*}
        \begin{split}
        [L_x,L_y]=&0-yi\hbar p_x-0-0+0+0-0 -0-0 -0 +0 +0-0 -0+xi\hbar p_y +0\\
        =&i\hbar xp_y-i\hbar y p_x\\
        =&i\hbar (xp_y-y p_x)\\
        =&i\hbar L_z\\    
    \end{split}  
    \end{equation*}
    \alert{证明2:} 基于推论 
    \begin{equation*}
        \begin{split}
        [L_x,L_y]= &[yp_z-zp_y,zp_x-xp_z]\\
        =&[yp_z,zp_x-xp_z] - [zp_y,zp_x-xp_z]\\
        =&y[p_z,zp_x-xp_z]+[y,zp_x-xp_z]p_z- z[p_y,zp_x-xp_z]-[z,zp_x-xp_z]p_y\\
        =&y[p_z,zp_x]+0- 0-[z,-xp_z]p_y\\
        =&yz[p_z,p_x]+y[p_z,z]p_x+x[z,p_z]p_y+[z,x]p_zp_y\\
        =&0-yi\hbar p_x+x i\hbar p_y+0\\
        =&i\hbar L_z\\
    \end{split}  
    \end{equation*}
    同理,有:\\
    $\begin{cases}
        [L_x,L_y]= i\hbar L_z  \\ 
        [L_x,L_x]= 0  \\ 
        [L_x,L_z]= -i\hbar L_y 
    \end{cases}$
    $\begin{cases}
        [L_y,L_z]= i\hbar L_x  \\ 
        [L_y,L_y]= 0  \\ 
        [L_y,L_x]= -i\hbar L_z 
    \end{cases}$
    $\begin{cases}
        [L_z,L_x]= i\hbar L_y  \\ 
        [L_z,L_z]= 0  \\ 
        [L_z,L_y]= -i\hbar L_x 
    \end{cases}$
    \begin{tcolorbox}[colback=yellow!5,colframe=red!75!black,title=角动量对易关系]
        $$ [L_\alpha,L_\beta]= \varepsilon_{\alpha\beta\gamma} i\hbar L_\gamma $$ 
    \end{tcolorbox}
\end{frame} 

\begin{frame} [allowframebreaks=]
    \frametitle{}
    \begin{exampleblock}{}
     试证明角动量对易关系 $[L_z,L^2]=0$
    \end{exampleblock}
    \alert{证明:} 
    \begin{equation*}
        \begin{split}
        [L_z,L^2]&= [L_z,L_x ^2+L_y ^2+L_z ^2]\\
        &=[L_z,L_x ^2]+[L_z,L_y ^2]+[L_z,L_z ^2]\\
        &=[L_z,L_x ^2]+[L_z,L_y ^2]\\
        &=L_x[L_z,L_x] +[L_z,L_x]L_x +L_y[L_z,L_y] +[L_z,L_y]L_y\\
        &=i\hbar L_x L_y +i\hbar L_yL_x - i\hbar L_x L_y -i\hbar L_yL_x\\
        &=0 \\
        \end{split}  
    \end{equation*}
    同理,有:\\
    $\begin{cases}
        [L_x,L^2]= 0  \\ 
        [L_y,L^2]= 0  \\ 
        [L_z,L^2]= 0 
    \end{cases}$
    \begin{tcolorbox}[colback=yellow!5,colframe=red!75!black,title=角动量对易关系]
        $$ [L_\alpha,L^2]= 0 $$ 
    \end{tcolorbox}
\end{frame} 

\begin{frame} [allowframebreaks=]
    \frametitle{}
    \begin{tcolorbox}[colback=yellow!5,colframe=yellow!75!black,title=课堂作业]
        定义升降算符:$L_\pm \equiv L_x \pm i L_y$ \\
     试证明如下对易关系  $$[L_\pm,L^2]=0$$
     $$[L_z, L_\pm]= \pm \hbar L_\pm $$
     $$[L_+, L_-]= 2 \hbar L_z $$
    \end{tcolorbox}
\end{frame} 
