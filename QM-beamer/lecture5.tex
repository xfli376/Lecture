
%%%%%%%%%%%%%%%%%%%%%%%%%%%%%%%%%%%%%%%%%%
\begin{frame}
    \frametitle{}
    \begin{center}
    { {\huge 第五讲、薛定谔方程}}
    \end{center}    
\end{frame}
%%%%%%%%%%%%%%%%%%%%%%%%%%%%%%%%%%%%%


\section{前情回顾}

\begin{frame}
    \frametitle{前情回顾}
    \begin{itemize}
        \item 波粒二象性
        \item 波函数假说
        \item 波函数的统计解释
        \item 态叠加原理
    \end{itemize}
\end{frame}  

\section{薛定谔方程}

\begin{frame}
    \frametitle{话分两支}
    \begin{center}
        \includegraphics[width=0.9\textwidth]{figs/2021-12-06-16-22-39.png}\\   
    \end{center}
    ~~\\
    \begin{tcolorbox}[colback=yellow!10,colframe=red!75!black,title=]
    既然你们说粒子有具有波动性,那总得有个波动方程吧\\
    \qquad \qquad \qquad \qquad \qquad --德拜(1925)
    \end{tcolorbox}
\end{frame}

\begin{frame}
    \begin{tcolorbox}[colback=yellow!10,colframe=red!75!black,title=]
        Dear Debye, I find one!\hspace{3cm} --薛定谔 (2 weeks later)
    \end{tcolorbox}
\end{frame}

\begin{frame}
	\frametitle{}
	\begin{alertblock} {方程的来源}  
		\begin{itemize}
			\item 	\textbf{ 观点1:}  最小作用量原理 $\int\limits_{t_1}^{t_2} \delta L d t =0 $\\ 
			\item 	\textbf{ 观点2:}  波动性和粒子性的结合\\ 
			\item 	\textbf{ 观点3:}  基本假设,不能从其他原理推导 \\
		\end{itemize}
	\end{alertblock}
\end{frame}

\begin{frame} [allowframebreaks=]
    \frametitle{方程的建议}
    \bullet Plane wave-function ($\psi(x,t)=\Psi_p(x,t)=e^{\frac{i}{\hbar}(p\cdot x-Et)}$) be a sulotion of Schr$\ddot{o}$dinger equation, obviously
    \begin{equation*}
        \begin{split}
       -i\hbar \nabla \psi(x,t) &=p\psi(x,t) \\
       \hbar^2 \nabla^2 \psi(x,t) &=p^2\psi(x,t) \\
       \frac{\hbar^2}{2\mu} \nabla^2 \psi(x,t) &=\frac{p^2}{2\mu} \psi(x,t) , \qquad \cdots (1)
        \end{split}
    \end{equation*}
    \begin{equation*}
       i\hbar \frac{\partial }{\partial t} \psi(x,t) =E\psi(x,t)  , \qquad \cdots (2)
     \end{equation*}
    (2)-(1)
    \begin{equation*}
        (i\hbar \frac{\partial }{\partial t} - \frac{\hbar^2}{2\mu} \nabla^2 )\psi(x,t) =(E-\frac{p^2}{2\mu})\psi(x,t)=0  
    \end{equation*}
    \begin{equation*}
        i\hbar \frac{\partial }{\partial t} \psi(x,t) = \frac{\hbar^2}{2\mu} \nabla^2 \psi(x,t)
    \end{equation*}
    For general wavefunction, it's a wave packet of plane wave
    \begin{equation*}
        \Psi(x,t)= \int\limits_{p=0} ^{\infty} c(p,t) e^{\frac{i}{\hbar}px}dp
    \end{equation*}
    we get 
    \begin{equation*}
        \begin{split}
        (i\hbar \frac{\partial }{\partial t} - \frac{\hbar^2}{2\mu} \nabla^2 )\Psi(x,t) &= \int\limits_{p=0} ^{\infty} c(p,t) (E-\frac{p^2}{2\mu}) e^{\frac{i}{\hbar}px}dp=0  \\
        i\hbar \frac{\partial }{\partial t} \Psi(x,t) &= \frac{\hbar^2}{2\mu} \nabla^2 \Psi(x,t)
        \end{split}
    \end{equation*}
    For nonfree particle in a potential $U(x)$,
    \begin{equation*}
        i\hbar \frac{\partial }{\partial t} \Psi(x,t) = (\frac{\hbar^2}{2\mu} \nabla^2 +U(x)) \Psi(x,t)
    \end{equation*}
    That is the Schr$\ddot{o}$dinger equation. \\
    ~~\\
    \bullet For N-particles system
   {\small \begin{equation*}
        i\hbar \frac{\partial }{\partial t} \Psi(x_1, x_2, \cdots x_N,t) = [\sum_{i=1} ^{N} \frac{\hbar ^2}{2\mu_i} \nabla^2 +U(x_1, x_2, \cdots x_N)] \Psi(x_1, x_2, \cdots x_N,t)
    \end{equation*}}
\end{frame}

\section{检验正确性}

\begin{frame}
    \frametitle{检验正确性}
    \begin{enumerate}
        \item 自由粒子的解
        \item 氢原子光谱
        \item $\cdots \cdots$
    \end{enumerate}
\end{frame}

\begin{frame}
    \frametitle{评价}
    \begin{enumerate}
        \item 我一阅读完毕整篇论文,就像被一个迷语困惑多时渴慕知道答案的孩童,现在终于听到了解答!\\
        \hspace{5cm} --普朗克
        \item 这著作的灵感如同泉水般源自一位真正的天才!\\
        \hspace{5cm} --爱因斯坦
        \item  波动方程把量子理论推进了关键性的一步\\
        \hspace{5cm} --玻尔
    \end{enumerate}
\end{frame}

\begin{frame}
    \frametitle{薛定谔}
    \begin{wrapfigure} {r} {0.3\textwidth} %;图在右
        \includegraphics[width=0.25\textwidth]{figs/schroginger.png}   
    \end{wrapfigure}
奥地利理论物理学家, 生于维也纳, 量子力学的奠基人之一。薛天才,通灵的人, 1926年提出薛定谔方程,获1933年诺贝尔物理学奖; 1935年提出“薛定谔的猫”,至今还是“养猫人”的猫王;1943年写的《生命是什么》一书,被誉为“唤起生物革命的小册子”。薛定谔:他玉树临风,英俊潇洒,风流倜傥,人见人爱,花见花开,情人无数,江湖人称“段正淳
\end{frame}

\begin{frame}
    \begin{tcolorbox}[colback=yellow!10,colframe=red!75!black,title=Basic assumption 2/5]
        The evolution of wavefunction obeys Schr$\ddot{o}$dinger equation
        \begin{equation*}
            i\hbar \frac{\partial }{\partial t} \Psi (\overrightarrow{r},t ) =\left [ -\frac{\hbar^2}{2\mu }\nabla ^2 + V(\overrightarrow{r},t ) \right ]\Psi (\overrightarrow{r}, t ) 
        \end{equation*}
    \end{tcolorbox}
    如今,它已是量子力学基本方程,地位与经典物理学中的牛顿第二定律相当。有着丰富的内含
\end{frame}

\section{定态薛定谔方程}

\begin{frame} [allowframebreaks=]
    \frametitle{定态问题}
    势函数$V(\vec{r},t ) $若不显含时间 t,时间变量可分离 \\ \vspace{0.3cm}
    方程: { $ \displaystyle i \hbar \frac{\partial }{\partial t} \Psi (\vec{r},t ) =\left [- \frac{\hbar^2}{2\mu }\nabla ^2 + V(\vec{r}) \right ]\Psi (\vec{r},t ) $}  \\  \vspace{0.3cm}
    \alert{解:}  设  $\Psi (\vec{r},t )  = \Psi (\vec{r} ) f(t) $ , 代回方程 \\ 
     { $ \displaystyle i\hbar \Psi (\vec{r})  \frac{\partial }{\partial t} f(t)=f(t) \left [ -\frac{\hbar^2}{2\mu }\nabla ^2 + V(\vec{r}) \right ]\Psi (\vec{r}) $}  \\ 	
     { $ \displaystyle i\hbar \frac{1}{f(t)}  \frac{\partial }{\partial t} f(t)= \frac{1}{\Psi (\vec{r}) } \left [ -\frac{\hbar^2}{2\mu }\nabla ^2 + V(\vec{r}) \right ]\Psi (\vec{r}) =E $}  \\ 
     得两个微分方程:\\  \vspace{0.3cm}
     I、演化方程  $ \displaystyle  i\hbar \frac{1}{f(t)}  \frac{\partial }{\partial t} f(t)=E $  \\ 
        解方程,得:$\displaystyle  f(t) =e^{-iEt/\hbar}$ \\  \vspace{0.3cm}
    II、定态薛定谔方程 $\displaystyle   \left [ -\frac{\hbar^2}{2\mu }\nabla ^2 + V(\vec{r}) \right ]\Psi (\vec{r}) =E \Psi (\vec{r})  $   \\ 
    \begin{definition}
        定态:能量有确定值的态称为定态,用定态波函数描述
        $$ \Psi (\vec{r},t )  = \Psi (\vec{r} ) f(t) = \Psi_E (\vec{r} ) e^{-iEt/\hbar} $$ 
    \end{definition}
    定态薛定谔方程算符形式:$$\displaystyle   \hat{H} \Psi (\vec{r}) =E \Psi (\vec{r})  $$   
    很明显,定态薛定谔方程是哈密顿算符 $\hat{H}$ 的本征方程。结合一定的定解条件,可以得到能量本征值(E)及本征函数 $\Psi_{E_n} (\vec{r} )$ 。\\
    依据态叠加原理,一般的态可表示为:
    $$ \Psi (\vec{r},t ) =\sum\limits_n \Psi_{E_n} (\vec{r} ) e^{-iE_n t/\hbar}  $$
\end{frame}

\section{守恒定律}

\begin{frame} [allowframebreaks=]
    \frametitle{守恒定律 }
    守恒定律关心的是物理量随时间的变化率问题,量子力学中最重要的是概率,我们考虑概率密度的变化率
    $$\omega (\vec{r}, t)=|\Psi(\vec{r}, t)|^{2}=\Psi^{*}(\vec{r}, t) \Psi(\vec{r}, t)$$
    \begin{equation*}
        \begin{split}
            \frac{\partial \omega}{\partial t} &=\Psi^{*} \frac{\partial \Psi}{\partial t}+\frac{\partial \Psi^{*}}{\partial t} \Psi, \cdots (1) \\
            \frac{\partial \Psi}{\partial t} & =\frac{i \hbar}{2 \mu} \nabla^{2} \Psi+\frac{1}{i \hbar} U \Psi, \cdots (2) \\
            \frac{\partial \Psi^{*}}{\partial t} & =-\frac{i \hbar}{2 \mu} \nabla^{2} \Psi^{*}-\frac{1}{i \hbar} U \Psi^{*}, \cdots (3) 
        \end{split}
    \end{equation*}
    把(2)(3)代回(1),得:
    \begin{equation*}
        \begin{split}
        \frac{\partial \omega}{\partial t}& =\frac{i \hbar}{2 \mu}\left(\Psi^{*} \nabla^{2} \Psi-\Psi \nabla^{2} \Psi^{*}\right) \\
        &=\frac{i \hbar}{2 \mu} \nabla \cdot\left(\Psi^{*} \nabla \Psi-\Psi \nabla \Psi^{*}\right)\\
        &=-\nabla \cdot \frac{i \hbar}{2 \mu} \left(\Psi \nabla \Psi^{*}-\Psi^{*} \nabla \Psi\right) \\
        &=-\nabla \cdot \vec{J}
        \end{split}
    \end{equation*}
    上式定义了一个矢量: $\vec{J}=\frac{i \hbar}{2 \mu} \left(\Psi \nabla \Psi^{*}-\Psi^{*} \nabla \Psi\right) $ \\
    \begin{equation*}
        \frac{\partial \omega}{\partial}+ \nabla \cdot \vec{J}=0, \cdots (4)
    \end{equation*}    
    (4) 式具有连续性方程形式,说明矢量 $\vec{J}$ 决定了概率密度的变化率。\\
    在任意空间区域 V, 对(4)式求积分,有:
    \begin{equation*}
        \frac{d}{d t} \int_{V} \omega d \tau =-\int_{S} \vec{J} \cdot d \vec{S}, \cdots (5)
    \end{equation*}
    由 Gauss 定理可知,单位时间内体系V内增加的概率应等于穿过V边界面S进入V内的概率,所以$\vec{J}$是概率流。(4) 式和(5)分别是概率守恒定律的微分和积分形式。
    \begin{equation*}
        \begin{split}
        \frac{d}{d t} \int\limits_{V} \omega d \tau &= \frac{d}{d t} \int\limits_{V} |\Psi(\vec{r}, t)|^{2} \tau ,\qquad \cdots \qquad (V \to \infty)  \\
        &=\frac{d}{d t} 1\\ 
        &=0
        \end{split}
    \end{equation*}
    说明全空间概率不随时间发生变化,即粒子既未产生也未湮灭时,概率守恒定律就是粒子数守恒定律。
    对(4)式,左右两边同乘以粒子的质量$\mu$, 
    \begin{equation*}
        \frac{\partial \mu\omega}{\partial}+ \nabla \cdot \mu\vec{J}=0
    \end{equation*}  
    得质量守恒定律
    \begin{equation*}
        \frac{\partial \omega_\mu}{\partial}+ \nabla \cdot \vec{J_\mu}=0, \cdots (6)
    \end{equation*}  
    对(4)式,左右两边同乘以粒子的电荷$e$, 
    \begin{equation*}
        \frac{\partial e\omega}{\partial}+ \nabla \cdot e\vec{J}=0
    \end{equation*}  
    得电荷守恒定律
    \begin{equation*}
        \frac{\partial \omega_e}{\partial}+ \nabla \cdot \vec{J_e}=0, \cdots (7)
    \end{equation*}  
\end{frame}

\begin{frame} [allowframebreaks=]
    \frametitle{定态的概率流}
    定态的概率密度不随时间变化。
    \begin{equation*}
        \begin{split}
            \omega (\vec{r}, t)&=\Psi^{*}(\vec{r}, t) \Psi(\vec{r}, t) \\
            &=\Psi_{E_n} (\vec{r} ) e^{-iE_n t/\hbar} \Psi_{E_n} ^* (\vec{r} ) e^{iE_n t/\hbar} \\
            &=\Psi_{E_n} (\vec{r} )\Psi_{E_n} ^* (\vec{r} ) \\
            &=|\Psi_{E_n} (\vec{r} )|^2
        \end{split}
    \end{equation*}
    定态的概率流密度也不随时间变化。 
    \begin{equation*}
        \frac{\partial \omega}{\partial}+ \nabla \cdot \vec{J}=0
    \end{equation*}  
    \to
    \begin{equation*}
        \nabla \cdot \vec{J}=-\frac{\partial \omega}{\partial}=0
    \end{equation*}  
\end{frame}

\begin{frame}
    THE END
\end{frame}



