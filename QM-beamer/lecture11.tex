%%%%%%%%%%%%%%%%%%%%%%%%%%%%%%%%%%%%%%%%%%
\begin{frame}
    \frametitle{}
    \begin{center}
    { {\huge 第十一讲、对称与守恒量}}
    \end{center}    
\end{frame}
%%%%%%%%%%%%%%%%%%%%%%%%%%%%%%%%%%%%%


\section{前情回顾}

\begin{frame}
    \frametitle{前情回顾}
    \begin{itemize}
        \item 希尔伯特空间的态矢量描述体系状态
        \item 希尔伯特空间的算符给出体系的物理量
        \item 算符的本征函数系构成正交归一完全基
        \item 常见算符本征方程求解
        \item 算符对易关系及其物理含义 
    \end{itemize}   
\end{frame} 

\section{守恒量}
\begin{frame} 
    \frametitle{守恒量定义}
    \begin{enumerate}
        \item  经典物理中的守恒量与对称条件\\
                守恒量:力学量的值不随时间变化\\
                \begin{itemize}
                    \item 机械能空间平移不变→动量守恒
                    \item 机械能空间转动不变→角动量守恒
                    \item 机械能时间平移不变→能量守恒
                \end{itemize}
        \item  量子力学中的守恒量\\
                守恒量:在任意态下力学量的平均值不随时间变化\\
                $$ \bar{A}(t)=(\Psi(t), A\Psi(t)) =c.  $$
                守恒量及守恒条件... 
    \end{enumerate}
\end{frame} 



