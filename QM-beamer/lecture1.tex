\section{课程简介}

\begin{frame}
    \frametitle{课程目标}
        \begin{enumerate}
            \item Learn the formal theory of Quantum Mechanics
            \item How physical systems are described in Quantum Mechanics.
            \item How to solve problems in Quantum Mechanics.
        \end{enumerate}
\end{frame}
\begin{frame}
    \frametitle{分数构成}
        \begin{enumerate}
            \item Normal results 20\%
            \item Midterm examination results 20\%
            \item Final examination results 60\%
        \end{enumerate}
\end{frame}

\begin{frame}
    \frametitle{参考书目}
        \begin{itemize}
            \item 《量子力学》卷I,II, 曾谨言, 科学出版社, 2008           
            \item Principles of quantum mechanics, shankar
            \item Modern quantum mechanics, shankar
            \item Lectures on quantum mechanics, weinberg
            \item Principles of quantum mechanics, Dirac
        \end{itemize}
\end{frame}

\begin{frame}
    \frametitle{三条军规}
        \begin{enumerate}
            \item Objects are wave-particles and can be in states of superposition
            \item Rule 1 holds as long as you don't measure
            \item Measurement gives random results
        \end{enumerate}
\end{frame}

%%%%%%%%%%%%%%%%%%%%%%%%%%%%%%%%%%%55%%
\begin{frame}
    \frametitle{}
    \begin{center}
    { {\huge 第一讲:普朗克能量子假说 }}
    \end{center}    
\end{frame}
%%%%%%%%%%%%%%%%%%%%%%%%%%%%%%%%%%

\section{伟大成就}

\begin{frame}
    \frametitle{伟大成就}
    Great successes in Classical Physics\\
        \begin{enumerate}
            \item Newtonian mechanics
            \item Maxwell's electromagnetism
            \item Thermodynamic laws
        \end{enumerate}
        "There is nothing new to be discovered in physics now. All that remains is 
        more and more precise measurements"   \ldots Lord Kelvin 1900 \\
        "The beauty and clearness ... is 
        obscured by two small puzzling clouds "  \ldots Lord Kelvin 1900.4
\end{frame}

\begin{frame}
    \frametitle{两朵乌云}
        \begin{itemize} 
            \item Michelson-Morley experiment
            \item Black body radiation
        \end{itemize}
\end{frame}

\begin{frame}
    \frametitle{迈克尔逊-莫雷实验}
    \begin{center}
    \includegraphics[width=0.8\textwidth]{figs/michel.png}
    \end{center}
There is no displacement of the interference bands. \dots 
the Stationary Ether is thus shown to be incorrect
\end{frame}

\begin{frame}
    The theory of relativity is established 
    \begin{center}
        \includegraphics[width=0.4\textwidth]{figs/relativity.jpg}
    \end{center}   
    Greatly changed our view of time and space. Mainly useful in two aspects: high-speed motion, and strong gravitational field. 
\end{frame}

\begin{frame}
    \frametitle{黑体辐射实验}
    \begin{center}
    \includegraphics[width=0.7\textwidth]{figs/2021-12-01-23-47-27.png}
    \end{center}
    No mathematical function to describe the curves exactly 
\end{frame}
\begin{frame}
    Quantum mechanics is established  
    \begin{center}
        \includegraphics[width=0.45\textwidth]{figs/mqm.jpg}
    \end{center}   
    It is a theory about matter.
\end{frame}

\begin{frame}
    \frametitle{现代科学基石}
    \begin{center}
        \includegraphics[width=0.8\textwidth]{figs/stone.png}
    \end{center}   
\end{frame}

%%%%%%%%%%%%%%%%%%%%%%%%%%%%%%%%%%%%
\section{普朗克公式}
%%%%%%%%%%%%%%%%%%%%%%%%%%%%%%%%%%%%

\begin{frame}
    %\frametitle{Black body radiation}
    \begin{definition}
        Black body: absorb all electromagnetic waves in any temperature
    \end{definition}
    \begin{center}
        \includegraphics[width=0.7\textwidth]{figs/blackbody_radn_curves.png}
    \end{center}
    \textbf{\color{deepred} Most interestingly}, what is the mathematical function that describes all of these curves?
\end{frame}

\begin{frame}
    \frametitle{三个经验公式}
    \begin{center}
        \includegraphics[width=0.7\textwidth]{figs/threelaws.png}
    \end{center}
\end{frame}

\begin{frame}
    \frametitle{维恩公式}
    \begin{equation*}
        \rho(\nu) d \nu=c_{1} \nu^{3} e^{-c_{2} \nu / T} d \nu 
    \end{equation*}
    Derived from electromagnetism (1893), but described well only in high frequency region.\\ 
    {\color{deepred} Nobel Prize in physics(1911)}\\
\end{frame}

\begin{frame}
    \frametitle{瑞-金公式}
    \begin{equation*}
        \rho(\nu, T) d \nu=\frac{8 \pi}{c^{3}} \nu^{2} k T d \nu 
    \end{equation*}
    Derived from thermodynamics (1900), but described well only in low frequency region.\\ 
   {\color{deepred} Nobel Prize in physics(1904)}\\
   {\color{deepblue} Ultraviolet Catastrophe:} 
    \begin{equation*}
         \int_0 ^\infty \frac{8 \pi}{c^{3}} \nu^{2} k T \to \infty 
    \end{equation*}
\end{frame}

\begin{frame}
    \frametitle{普朗克公式}
    On 1900-10-19, at the German Physical Society, 
    Max Planck presented a resolution to {\color{deepblue} Ultraviolet Catastrophe} 
    \begin{equation}
        \rho(\nu, T) d \nu=\frac{8 \pi}{c^{3}} \frac{h \nu^{3}}{e^{h \nu / K T}-1} d \nu
    \end{equation}
    Obtained from experimental data via interpolation technique (1900), described well in whole region \\
    {\color{deepred} Nobel Prize in physics(1918)}\\
\end{frame}

\begin{frame}
    \begin{tcolorbox}[colback=yellow!5,colframe=yellow!75!black,title=The problem]
        How to derive the formula from existing theory.
    \end{tcolorbox}
    \begin{tcolorbox}[colback=yellow!10,colframe=red!75!black,title=Solution]
        On 1900-12-14, Planck gives out his solution based on the Energy Quantum Hypothesis  
    \end{tcolorbox}
\end{frame}

%%%%%%%%%%%%%%%%%%%%%%%%%%%%%%%%%%%%
\section{能量子假说}
%%%%%%%%%%%%%%%%%%%%%%%%%%%%%%%%%%%%
\begin{frame}
    \begin{tcolorbox}[colback=yellow!10,colframe=red!75!black,title=Energy quantum hypothesis]
    Assuming the oscillators of the cavity could only radiate at a discrete amounts of energy
    \begin{equation}
        E=n\varepsilon
    \end{equation}
    where, the $\varepsilon$ is the unit of the energy (quanta) determined by the oscillator' frequency 
    \begin{equation}
        \varepsilon=h\nu
    \end{equation}
    and the $h=6.6260693(11)×10^{-34} J\cdot s $ is the Planck constant. 
    \end{tcolorbox}
\end{frame}

\begin{frame}
    Based on Boltzmann distribution law,
    \begin{equation*}
        \frac{N_{i}}{N}=\frac{\exp \left(-\frac{E_{i}}{k T}\right)}{\sum_{i} \exp \left(\frac{-E_{i}}{k T}\right)}
    \end{equation*}
    \bullet when the energy is continuous,the distribution between $E - dE$ should be 
    \begin{equation*}
        \frac{e^{-E / k T}}{\int\limits_{0}^{\infty} e^{-E / k T} d E}
    \end{equation*}  
    the average energy 
    \begin{equation*}
        <E>=\int\limits_{0}^{\infty} E \frac{e^{-E / k T}}{\int\limits_{0}^{\infty} e^{-E / k T} d E} d E
    \end{equation*}
\end{frame}

\begin{frame}
    \begin{equation*}
        \begin{split}
            <E> &= -kT \frac{Ee^{-E / k T}\vert_0 ^\infty-\int\limits_{0}^{\infty} e^{-E / k T} d E } {\int\limits_{0}^{\infty} e^{-E / k T} d E }\\  
                &= kT
        \end{split}  
    \end{equation*} 
    \bullet when the energy is discrete,the distribution should be   
    \begin{equation*}
        \frac{e^{-E / k T}}{\int\limits_{0}^{\infty} e^{-E / k T} d E} 
        \to \frac{e^{-E / k T}}{\sum\limits_{0}^{\infty} e^{-E / k T}} 
        \to \frac{e^{-nh\nu / k T}}{\sum\limits_{0}^{\infty} e^{-nhv / k T}} 
    \end{equation*}    
\end{frame}

\begin{frame}
    the average energy 
    \begin{equation*}
        \begin{split}
            <E> &= \sum\limits_{0}^{\infty} nh\nu\frac{e^{-nh\nu / k T}}{\sum\limits_{0}^{\infty} e^{-nh\nu / k T}} \\
            &= -h\nu \frac{d}{dx} \frac{n e^{-nx}}{\sum\limits_{0}^{\infty} e^{-nx}} \\
            &= \frac{h\nu}{e^{h\nu/kT}-1} 
        \end{split} 
    \end{equation*}
    We get
    \begin{equation*}
        \text{(continuous)} \quad k T \rightarrow \frac{h \nu}{e^{ h \nu / k T}-1} \quad \text{(discrete)} 
    \end{equation*}
\end{frame}

\begin{frame}
    In Rayleigh-Jeans formula
    \begin{equation*}
        \rho(\nu, T) d \nu=\frac{8 \pi}{c^{3}} \nu^{2} k T d \nu 
    \end{equation*}
    the item $kT$ should be replaced by $\frac{h \nu}{e^{ h \nu / k T}-1}$
    \begin{equation*}
        \rho(\nu, T) d \nu=\frac{8 \pi}{c^{3}} \frac{h \nu^{3}}{e^{h \nu / K T}-1} d \nu
    \end{equation*}
    It is the Planck's formula exactly 
\end{frame}

\begin{frame}
    \begin{tcolorbox}[colback=yellow!10,colframe=red!75!black,title=Revolutionary Significance]
        Planck's Energy Quantum Hypothesis broke through the constraints of classical physics and 
        opened the door of quantum mechanics 
    \end{tcolorbox}
\end{frame}

\begin{frame}
    \begin{tcolorbox}[colback=yellow!10,colframe=red!75!black,title=THE END]
        In 1927, **Dirac** got the Planck's formula from Quantum Mechanism.
    \end{tcolorbox}
\end{frame}

\begin{frame}
    \frametitle{学术讨论}
    \begin{center}
        {\color{red} \Large 能量量子化只是一种数学处理技术?}  
    \end{center}
\end{frame}
%\begin{frame}
%    \frametitle{Homework}
%    \begin{enumerate}
%        \item Planck's Energy Quantum Hypothesis
%        \item What's the quanta
%        \item Deriving the Rayleigh-Jeans formula and Wien's formula from Planck's formula
%    \end{enumerate}
%\end{frame}
%%%%%%%%%%%%%%%%%%%%%%%%%%%%%%%%%%%%%%%%%%%%%%%%%%%%%%%%%%%%%%%%%%%