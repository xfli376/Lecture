%%%%%%%%%%%%%%%%%%%%%%%%%%%%%%%%%%%%%%%%%%
\begin{frame}
    \frametitle{}
    \begin{center}
    { {\huge 第十三讲、量子力学公式的矩阵化}}
    \end{center}    
\end{frame}
%%%%%%%%%%%%%%%%%%%%%%%%%%%%%%%%%%%%%

\section{前情回顾}


\begin{frame}
    \frametitle{前情回顾}
    \begin{itemize}
       \done 波函数矩阵表示 :$$ a_n(t)=(u_n(\vec{r}), \Psi(\vec{r},t)) $$ 
       \done 力学量算符矩阵表示 : $$ F_{nm}=(u_n (\vec{r}), Fu_m(\vec{r})) $$   
       \todo 公式的矩阵化:$\cdots$
    \end{itemize}
\end{frame} 



\begin{frame} 
    \frametitle{}
    \begin{tcolorbox}[colback=yellow!5,colframe=yellow!75!black,title=量子力学常用公式]
        \begin{itemize}
            \item 平均值公式
            \item 归一化公式
            \item 本征方程
            \item 薛定谔方程
            \item 运动方程
        \end{itemize}   
    \end{tcolorbox}
\end{frame} 

\section{平均值公式}

\begin{frame} 
    \frametitle{}
    \begin{tcolorbox}[colback=yellow!5,colframe=yellow!75!black,title=1、平均值公式]
        求平均值公式在Q表象中的具体形式(矩阵表示)
        $$ \bar{F}=\int \Psi^* (\vec{r},t) F \Psi(\vec{r},t) d\tau $$
    \end{tcolorbox}
    \alert{解:} 
    \begin{equation*}
        \begin{split}
            \bar{F}&=(\Psi(\vec{r},t), F\Psi(\vec{r},t)) \\
            &= (\sum_n a_n(t) u_n(\vec{r}), \sum_m a_m(t) F u_m(\vec{r}))\\
            &= \sum_{n,m} a_n ^*(t) (u_n(\vec{r}), F u_m(\vec{r})) a_m(t)\\
            &= \sum_{n,m} a_n ^*(t) F_{nm} a_m(t)\\
        \end{split} 
    \end{equation*}
\end{frame}

\begin{frame} 
    取遍$n, m$, 得到如下矩阵形式\\
    $$\bar{F} =(a_1 ^*(t), a_2 ^*(t),\cdots,a_n^*(t) )
    \begin{pmatrix}
       F_{11} & F_{12} & \cdots & F_{1n} \\
       F_{21} & F_{22} & \cdots & F_{2n} \\
       \cdots & \cdots &  \cdots& \cdots\\
        F_{n1} & F_{n2} & \cdots & F_{nn} \\
    \end{pmatrix}
    \begin{pmatrix}
        a_1(t)\\
        a_2(t)\\
        \cdots \\
        a_n(t)
    \end{pmatrix}
    $$ \vspace{1.0em} 
    $$ \large \color{red} \bar{F} = \pmb {\Psi} ^{\dagger } \pmb {F} \pmb {\Psi} $$
\end{frame}

\begin{frame} 
    在自身表象中,有:
    $$\bar{F} =(a_1 ^*(t), a_2 ^*(t),\cdots,a_n^*(t) )
    \begin{pmatrix}
       f_1 & 0 & \cdots & 0 \\
       0& f_2 & \cdots & 0 \\
       \cdots & \cdots &  \cdots& \cdots\\
        0 & 0 & \cdots & f_n \\
    \end{pmatrix}
    \begin{pmatrix}
        a_1(t)\\
        a_2(t)\\
        \cdots \\
        a_n(t)
    \end{pmatrix}
    $$
    \vspace{1.0em} 
    $$ \large \color{red} \bar{F} = \sum_n a_n^*(t) a_n(t) f_n= \sum_n |a_n(t)|^2 f_n $$
\end{frame}

\section{归一化公式}

\begin{frame} 
    \frametitle{}
    \begin{tcolorbox}[colback=yellow!5,colframe=yellow!75!black,title=2、归一化公式]
        求归一化公式在Q表象中的具体形式(矩阵表示)
        $$ \int \Psi^* (\vec{r},t) \Psi(\vec{r},t) d\tau =1 $$
    \end{tcolorbox}
    \alert{解:} 
    \begin{equation*}
        \begin{split}
            1 &=(\Psi(\vec{r},t), \Psi(\vec{r},t)) \\
            &= (\sum_n a_n(t) u_n(\vec{r}), \sum_m a_m(t) u_m(\vec{r}))\\
            &= \sum_{n,m} a_n ^*(t) (u_n(\vec{r}), u_m(\vec{r})) a_m(t)\\
            &= \sum_{n,m} a_n ^*(t) \delta_{nm} a_m(t)\\
        \end{split} 
    \end{equation*}
\end{frame}


\begin{frame} 
    $$  \sum_{n} a_n ^*(t) a_n(t) =1 $$
    取遍$n$, 得到如下矩阵形式\\
    $$ (a_1 ^*(t), a_2 ^*(t),\cdots,a_n^*(t) )
    \begin{pmatrix}
        a_1(t)\\
        a_2(t)\\
        \cdots \\
        a_n(t)
    \end{pmatrix}
    =1 $$ \vspace{1.0em} 
    $$ \large \color{red} \pmb {\Psi} ^{\dagger } \pmb {\Psi} =1 $$

\end{frame}

\section{本征方程}

\begin{frame} 
    \frametitle{}
    \begin{tcolorbox}[colback=yellow!5,colframe=yellow!75!black,title=3、本征方程]
        求本征方程在Q表象中的具体形式(矩阵表示)
        $$ F\psi_n (\vec{r}) =f \psi_n (\vec{r})$$
    \end{tcolorbox}
    \alert{解:} 
    \begin{equation*}
        \begin{split}
            F\psi_m (\vec{r}) &=f \psi_m \\
            \psi_n ^*  F\psi_m &=\psi_n ^* f \psi_m\\
            (\psi_n, F\psi_m )&=(\psi_n, f \psi_m)\\
            (\psi_n, F\psi_m )&=(\psi_n, f \psi_m)\\
            F_{nm} &=f_m \delta_{nm}
        \end{split} 
    \end{equation*}
\end{frame}

\begin{frame} 
    $$ \sum_n (F_{nm} -f \delta_{nm})a_n=0 $$
    取遍$n,m$, 得矩阵形式\\            
    $$\begin{pmatrix}
        F_{11}-f & F_{12} & \cdots & F_{1n} \\
        F_{21} & F_{22}-f & \cdots & F_{2n} \\
        \cdots & \cdots &  \cdots& \cdots\\
         F_{n1} & F_{n2} & \cdots & F_{nn}-f \\
     \end{pmatrix}
     \begin{pmatrix}
         a_1(t)\\
         a_2(t)\\
         \cdots \\
         a_n(t)
     \end{pmatrix}
     =0 \qquad (1)$$
     $$ \color{red} (\pmb F -f \pmb I) \pmb \Psi =0 $$

    有解条件,系数行列式等于零!
\end{frame}

\begin{frame} 
    得久期方程:
    $$\begin{vmatrix}
        F_{11}-f & F_{12} & \cdots & F_{1n} \\
        F_{21} & F_{22}-f & \cdots & F_{2n} \\
        \cdots & \cdots &  \cdots& \cdots\\
         F_{n1} & F_{n2} & \cdots & F_{nn}-f \\
     \end{vmatrix} 
     =0 \qquad (2) $$
     解久期方程, 得本征谱{$f_1,f_2,\cdots, f_n $}\\
     依次把$f_i$ 代回方程(1),解得第i个本征函数。本征方程得解\\
     矩阵化使本征方程从微分方程变为代数方程!
\end{frame}

\begin{frame} 
    \begin{tcolorbox}[colback=yellow!5,colframe=yellow!75!black,title=例1]
        已知算符在Q表象中的矩阵形式如下。
        $$ L_x= \frac{\hbar}{\sqrt{2}}
        \begin{pmatrix}
            0 & 1 & 0  \\
            1 & 0 & 1  \\
            0 & 1 & 0 \\
         \end{pmatrix} $$
        求本征值和归一化本征函数,并将矩阵对角化。
    \end{tcolorbox}
    可选方案:
    \begin{itemize}
        \done 解久期方程,得本征值,然后代入方程(1),得本征函数 ,再直接写出对角阵 
        \todo 直接从数学上对角化,对角元就是本征值,然后代入方程(1),得本征函数。
     \end{itemize}
\end{frame}

\begin{frame} 
    \alert{解:}第一步:解久期方程求本征值
    $$\frac{\hbar}{\sqrt{2}}
    \begin{vmatrix}
       0-f & 1 & 0  \\
       1 & 0-f & 1  \\
       0 & 1 & 0-f \\
    \end{vmatrix} 
    =0 \qquad (2) $$
   $$ -f^3+2f=0 $$
   $$ f_1=\sqrt{2}, f_2=0, f_3=-\sqrt{2} $$
   注意:这只是
   $$\begin{pmatrix}
       0 & 1 & 0  \\
       1 & 0 & 1  \\
       0 & 1 & 0 \\
    \end{pmatrix} $$
    的本征值,$L_x$的本征值为 $\lambda_i=\dfrac{\hbar}{\sqrt{2}} f_i$,即 $ \hbar, 0, -\hbar$
\end{frame}

\begin{frame} 
    第二步:把$f_i$ 代回方程(1)求本征函数
    $$\begin{pmatrix}
        0-f & 1 & 0  \\
        1 & 0-f & 1  \\
        0 & 1 & 0-f \\
     \end{pmatrix} 
     \begin{pmatrix}
         a_1\\
         a_2\\
         a_3
     \end{pmatrix}
     =0 \qquad (1)$$

     $$\begin{matrix}
       f_1=\sqrt{2} & f_2=0  &  f_3=-\sqrt{2}\\
    \begin{pmatrix}
        1/\sqrt{2}a_2\\
        a_2\\
        1/\sqrt{2}a_2
    \end{pmatrix}  
    = 1/\sqrt{2}a_2 \begin{pmatrix}
        1\\
        \sqrt{2}\\
        1
        \end{pmatrix} 
    & 
    \begin{pmatrix}
        a_1\\
        0\\
        a_1
    \end{pmatrix}  
    =  a_1 \begin{pmatrix}
        1\\
        0\\
        1
    \end{pmatrix}
    &
    \begin{pmatrix}
        -1/\sqrt{2}a_2\\
        a_2\\
        -1/\sqrt{2}a_2
    \end{pmatrix} 
    = 1/\sqrt{2}a_2 \begin{pmatrix}
        -1\\
        \sqrt{2}\\
        -1
    \end{pmatrix} 
    \end{matrix}$$           
\end{frame}


\begin{frame} 
    代入归一化公式, $$ \pmb {\Psi} ^{\dagger } \pmb {\Psi} =1 $$

    $$\begin{matrix}
    f_1=\sqrt{2} & f_2=0  &  f_3=-\sqrt{2}\\
    \dfrac{1}{2} a_2 ^2 (1, \sqrt{2}, 1)
    \begin{pmatrix}
    1\\
    {\sqrt{2}}\\
    1
    \end{pmatrix} 
    =1 
    & 
    a_1 ^2 (1, 0, 1)
    \begin{pmatrix}
     1\\
     0\\
     1
    \end{pmatrix} 
    =1 
     &
     \dfrac{1}{2} a_2 ^2 (-1, \sqrt{2}, -1)
    \begin{pmatrix}
     -1\\
     {\sqrt{2}}\\
     -1
    \end{pmatrix} 
    =1 \\
    a_2= \dfrac{1}{\sqrt{2}} &  a_1= \dfrac{1}{\sqrt{2}} &   a_2=\dfrac{1}{\sqrt{2}} 
    \\
    \psi_1=\dfrac{1}{2}
    \begin{pmatrix}
    1\\
    \sqrt{2}\\
    1
    \end{pmatrix}  
    & 
    \psi_2=\dfrac{1}{\sqrt{2}}
    \begin{pmatrix}
    1\\
    0\\
    1
    \end{pmatrix}  
    &
    \psi_3=\dfrac{1}{2}
    \begin{pmatrix}
    -1\\
    \sqrt{2}\\
    -1
    \end{pmatrix}  
    \end{matrix}$$   
\end{frame}

\begin{frame} 
    第三步:写出对角阵:
    $$ L_x= \frac{\hbar}{\sqrt{2}}
        \begin{pmatrix}
            0 & 1 & 0  \\
            1 & 0 & 1  \\
            0 & 1 & 0 \\
         \end{pmatrix} 
    = 
    \begin{pmatrix}
        \hbar & 0 & 0  \\
        0 & 0 &   \\
        0 & 0 & -\hbar \\
     \end{pmatrix} 
    $$
\end{frame}

\section{薛定谔方程}
\begin{frame} 
    \frametitle{}
    \begin{tcolorbox}[colback=yellow!5,colframe=yellow!75!black,title=4、薛定谔方程]
        求薛定谔方程在Q表象中的具体形式(矩阵表示)
        $$ i\hbar \frac{\partial}{\partial t }\psi (\vec{r},t) =H\psi (\vec{r},t)$$
    \end{tcolorbox}
    \alert{解:} 波函数在Q表象展开
    \begin{equation*}
        \begin{split}
            i\hbar \frac{\partial}{\partial t }\sum_n a_n(t) u_n(\vec{r})  &=H\sum_n a_n(t) u_n(\vec{r}) \\
            u_m ^* (\vec{r}) i\hbar \frac{\partial}{\partial t }\sum_n a_n(t) u_n(\vec{r})  &=u_m ^* (\vec{r})H\sum_n a_n(t) u_n(\vec{r}) \\
            i\hbar \frac{\partial}{\partial t }u_m ^* (\vec{r}) \sum_n a_n(t) u_n(\vec{r})  &=u_m ^* (\vec{r})\sum_n a_n(t) Hu_n(\vec{r}) \\      
        \end{split} 
    \end{equation*}
\end{frame}

\begin{frame} 
    \begin{equation*}
        \begin{split}
            i\hbar \frac{\partial}{\partial t }(u_m (\vec{r}), \sum_n a_n(t) u_n(\vec{r}) ) &=(u_m (\vec{r}), \sum_n a_n(t) Hu_n(\vec{r})) \\
            i\hbar \frac{\partial}{\partial t }\sum_n a_n(t)(u_m (\vec{r}),  u_n(\vec{r}) ) &=\sum_n (u_m (\vec{r}),  Hu_n(\vec{r}))a_n(t) \\
            i\hbar \frac{\partial}{\partial t }\sum_n a_n(t)\delta_{mn} &=\sum_n  H_{mn} a_n(t) \\
            i\hbar \frac{\partial}{\partial t } a_n(t) &=\sum_n H_{mn} a_n(t)  \\
        \end{split} 
    \end{equation*}
\end{frame}

\begin{frame} 
    取遍$n,m$, 得矩阵形式\\ 
    $$i\hbar \frac{\partial}{\partial t }  
    \begin{pmatrix}
        a_1(t)\\
        a_2(t)\\
        \cdots \\
        a_n(t)
    \end{pmatrix}
    =         
    \begin{pmatrix}
        H_{11} & H_{12} & \cdots & H_{1n} \\
        H_{21} & H_{22} & \cdots & H_{2n} \\
        \cdots & \cdots &  \cdots& \cdots\\
        H_{n1} & F_{n2} & \cdots & H_{nn} \\
     \end{pmatrix}
     \begin{pmatrix}
         a_1(t)\\
         a_2(t)\\
         \cdots \\
         a_n(t)
     \end{pmatrix}
    $$ \vspace{0.6em }
    $$\color{red} i\hbar \frac{\partial}{\partial t }  \pmb \Psi = \pmb H  \pmb \Psi $$
\end{frame}

\section{算符运动方程}
\begin{frame} 
    \frametitle{}
    \begin{tcolorbox}[colback=yellow!5,colframe=yellow!75!black,title=5、算符运动方程]
        求运动方程在Q表象中的具体形式(矩阵表示)
        $$ \frac{d\overline{F}}{dt}=\overline{\frac{\partial F }{\partial t}}  +\frac{1}{i\hbar} \overline{[F,H]}$$
    \end{tcolorbox}
    \alert{解:} 波函数在Q表象展
    \begin{equation*}
        \begin{split}
            \frac{d(\psi,F \psi )}{dt} &=(\psi,\frac{\partial F }{\partial t} \psi)  +\frac{1}{i\hbar}  ( \psi,[F,H]\psi) \\
            \frac{d(\sum_m a_m u_m,F \sum_n a_n u_n )}{dt} &=(\sum_m a_m u_m,\frac{\partial F }{\partial t} \sum_n a_n u_n) \\  
            &+\frac{1}{i\hbar}  (\sum_m a_m u_m,[F,H]\sum_n a_n u_n)  \\    
        \end{split} 
    \end{equation*}
\end{frame}

\begin{frame} 
    \begin{equation*}
        \begin{split}
            \frac{d\sum_{mn}a_m ^* (u_m,Fu_n )a_n}{dt} &=\sum_{mn} a_m ^*  (u_m,\frac{\partial F }{\partial t} u_n)a_n \\  
            &+\frac{1}{i\hbar} \sum_{mn} a_m ^*  ( u_m,[F,H] u_n)a_n  \\  
            \frac{d\sum_{mn}a_m ^* F_{mn}a_n}{dt} &=\sum_{mn} a_m ^*  \frac{\partial F_{mn} }{\partial t}a_n \\  
            &+\frac{1}{i\hbar} \sum_{mn} a_m ^* [F_{mn},H_{mn}]a_n  \\       
        \end{split} 
    \end{equation*}
\end{frame}

\begin{frame} 
    取遍$n,m$, 得矩阵形式
    \begin{equation*}
        \begin{split} 
            \frac{d \pmb \Psi^{\dagger } \pmb F \pmb \Psi}{dt} &=\pmb \Psi^{\dagger } \frac{\partial \pmb F }{\partial t} \pmb \Psi +\frac{1}{i\hbar} \pmb \Psi^{\dagger } [\pmb F, \pmb H] \pmb \Psi \\ \vspace{0.6em}  
           \color{red} \frac{d \overline{\pmb F}}{dt} & \color{red} =\overline{\frac{\partial \pmb F }{\partial t}} +\frac{1}{i\hbar} \overline{ [\pmb F,\pmb H]} \\      
        \end{split} 
    \end{equation*}
\end{frame}

\begin{frame} 
    \frametitle{}
    \begin{tcolorbox}[colback=yellow!5,colframe=yellow!75!black,title=课堂作业]
    取Q表象为动量表象,试求平均值公式和薛定谔方程的具体形式。
    \end{tcolorbox}
\end{frame} 