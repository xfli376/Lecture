\documentclass[12pt,hyperref,UTF8,aspectratio=169]{beamer} 
\usepackage{mynewbeamer}

\hypersetup{pdfpagemode=FullScreen}

%-------------------正文-------------------------%
%                                               %
%                                               %
\begin{document}                                %
%                                               %
%                                               %
%-----------------------------------------------%

%题目,作者,学校,日期                
\author{\myfont 李小飞}
\title{\textbf{\Huge 量子力学与统计物理}}
\subtitle{Quantum mechanics and statistical physics}
\institute[电子科技大学]{{\large 光电科学与工程学院}}
\date{\today}

	%%%%%%%%%%%%%%%%%%%%%%%%%%%%%%%%%
    \frame[plain]{\titlepage}
    %%%%%%%%%%%%%%%%%%%%%%%%%%%%%%%%%
    \begin{frame}
        \frametitle{总目录}
        \tableofcontents
    \end{frame}
    %%%%%%%%%%%%%%%%%%%%%%%%%%%%%%%%%%

\section{一、基本操作}

\begin{frame}
    \frametitle{}
    第一部:分基本操作  
\end{frame}

\frame{\frametitle{本section目录}\tableofcontents[currentsection]}

\subsection{支持中文}

\begin{frame}
    \frametitle{1、设置中文环境}
    \begin{itemize}
     \item  导入中文包 \textbackslash usepackage\{ xeCJK \}
     \item  运行 xelatex
    \end{itemize}
\end{frame}

\begin{frame}
    \frametitle{2、设置中文字体}
    \begin{itemize}
     \item  {\myfont 中文字体 }
     \item  {\CJKfamily{nicefont} 中文字体}
    \end{itemize}
\end{frame}

\begin{frame}
    \frametitle{3、设置logo背景}

    完成

\end{frame}

\begin{frame}
    \frametitle{colorbox}
    \fcolorbox{gray}{yellow}{test}

\end{frame}


\subsection{公式}
\begin{frame}
    \frametitle{1、在线公式}

    \begin{itemize}
        \item<2-> 表格转换: Excel2\LaTeX~(\href{https://www.ctan.org/tex-archive/support/excel2latex/}{CTAN Excel2\LaTeX});
        \item<2-> 在线公式: \href{https://www.latexlive.com/}{LaTeX公式编辑器},~\href{https://mathpix.com/}{Mathpix}~ \href{https://mathf.itewqq.cn/}{图片在线转LaTeX}.
    \end{itemize}

\end{frame}

\begin{frame}[fragile]{添加线}
	\begin{multicols}{2}
		\verb|\uline|\hfill 下划线\qquad\uline{混}\\
		\verb|\uuline|\hfill 双下划线\qquad\uuline{元}\\
		\verb|\uwave|\hfill 波浪线\qquad\uwave{形}\\
		\verb|\sout|\hfill 删除线\qquad\sout{翼}\\
		\verb|\xout|\hfill 斜删除线\qquad\xout{太}\\
		\verb|\dashuline|\hfill 虚线\qquad\dashuline{极}\\
		\verb|\dotuline|\hfill 加点\qquad\dotuline{门}
	\end{multicols}
\end{frame}

\subsection{3.主题}

\begin{figure}
    \begin{minipage}[t]{0.5\linewidth}
    \centering
    \includegraphics[width=2.2in]{images/uestclogo.png}
    \caption{fig1}
    \label{fig:side:a}
    \end{minipage}%
    \begin{minipage}[t]{0.5\linewidth}
    \centering
    \includegraphics[width=2.2in]{images/uestclogo.png}
    \caption{fig2}
    \label{fig:side:b}
    \end{minipage}
\end{figure}

\subsection{4.主题}

\begin{frame}
	\frametitle{表}
	表格太麻烦了, 掌门说摸摸鱼, 编者觉得不错, 丢一个三线表示例. 当然也可以看看这个手册前面部分表格的源码.
	\begin{table}[htbp]
		\centering
		\caption{一些国风音乐}
		\label{tab:YixieGfyy}
		\begin{tabular}{rlc}
			作曲家 & 歌名 & 门中喜欢的友人 \\
			李志辉 & 小桥流水人家 & 门主 \\
			林海 & 无羁(器乐版) & 初号 \\
			吕秀龄 & 逆伦 & 小初 \\
			麦振鸿 & 从来只有一个人 & 编者(假的) \\
		\end{tabular}
	\end{table}
\end{frame}

\section{二、进阶操作}
\subsection{1.主题}

\begin{frame}[fragile]{代码环境演示}
	\begin{lstlisting}[language=c++]
        #include <iostream>
        int main()
        {
            std::cout << "Hello, World!" << std::endl;
        }  
    \end{lstlisting}
\end{frame}

\subsection{2.主题}

\begin{frame}[fragile,allowframebreaks]{数学环境}
	\begin{proof}{}
		请读者自证.
	\end{proof}
	\begin{definition}{马老卷}{MalaoJ}
		是混元形翼太极门的打砸工, 直系上峰是马凡王, 入门改姓马, 自称老卷, 实则不卷.
	\end{definition}
	\begin{lemma}{卷王森林法则}{JuanwangSlfz}
		源自未知高校学生, 此处略.
	\end{lemma}
	\begin{corollary}{狼人杀的重要性}{LangrenSdzyx}
		编者实习时听公司导师说面试有可能是趣味性游戏, 狼人杀感觉很符合, 所以玩狼人杀吧.
	\end{corollary}
\end{frame}


\begin{frame} [fragile,allowframebreaks]{数学环境}
\begin{algorithm}[H]                           % HERE!!!!!!!!!
\caption{Calculate $y = x^n$}          % give the algorithm a caption
\label{alg1}      % and a label for \ref{} commands later in the document
\begin{algorithmic}  % enter the algorithmic environment
\REQUIRE $n \geq 0 \vee x \neq 0$
\ENSURE $y = x^n$
\STATE $y \Leftarrow 1$
\IF{$n < 0$}
\STATE $X \Leftarrow 1 / x$
\STATE $N \Leftarrow -n$
\ELSE
\STATE $X \Leftarrow x$
\STATE $N \Leftarrow n$
\ENDIF
\WHILE{$N \neq 0$}
\IF{$N$ is even}
\STATE $X \Leftarrow X \times X$
\STATE $N \Leftarrow N / 2$
\ELSE[$N$ is odd]
\STATE $y \Leftarrow y \times X$
\STATE $N \Leftarrow N - 1$
\ENDIF
\ENDWHILE
\end{algorithmic}
\end{algorithm}
\end{frame}

\subsection{3.主题}

\begin{frame}
    \[ x^2+y^2=1 \]
    $$ x^2+y^2=1 $$, $ x^2+y^2=1 $, 
\end{frame}

\subsection{4.主题}

\section{二、高阶操作}
\subsection{1.主题}
\subsection{2.主题}
\subsection{3.主题}
\subsection{4.主题}

\end{document}