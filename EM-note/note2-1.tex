\renewcommand{\thechapter}{}
\chapter{第二章~~线性偏微分方程(6学时)} 
\renewcommand{\thechapter}{2}
\begin{itemize}
\item \textbf{主要内容}:波动方程及求解方法、热传导方程及求解方法、拉普拉斯方程及求解方法。
\item \textbf{重点和难点:} 齐次弦振动方程的建立与分离变量法,固有值问题。
\item \textbf{掌握:}分离变量法的一般理论;熟练掌握直角坐标系一维和二微分离变量法的求解。
\item \textbf{理解:}分离变量法的适用条件。
\item \textbf{了解:}高维问题求解。
\end{itemize}

\begin{definition}[] 偏微分方程(PDE)的未知函数是多元函数,由于这些方程的来源和应用通常具有物理学背景,又称为数学物理方程。
\end{definition}	物理量通常是时间变量和空间变量的函数, 比如\textbf{波动方程}、\textbf{热传导方程}和\textbf{拉普拉斯方程}等,它们通常可以通过分离变量法求解。

\section{波动方程}
\subsection{波动方程的建立}
自然界普遍存在各种振动,振动在媒介中传播就是波,服从统一的方程。现在建立波动方程。
\begin{example} %1
考虑均匀柔软的细弦线,二端固定,受到扰动后在平衡位置作微小运动。分析细弦位移函数 u(x,t)满足的方程。\\
%\begin{tikzpicture}
%%\draw[eaxis] (0,0) -- (8,0) node[below] {$x$};
%%\draw[eaxis] (0,0) -- (0,0.5) node[right] {$y$};
%\draw (0,0) .. controls (2, 1) and (6,1) .. (8,0);
%\filldraw[red] (0,0) circle [radius=1pt];
%\filldraw[red] (8,0) circle [radius=1pt];
%\end{tikzpicture}
\begin{proof} 
建立如下坐标系:\\
\begin{tikzpicture}
	\draw[eaxis] (0,0) -- (8.5,0) node[below] {$x$};
	\draw[eaxis] (0,0) -- (0,1.5) node[right] {$u$};
	\draw (0,0) .. controls (2, 1) and (6,1) .. (8,0);
%	\filldraw[red](2,0.6)circle[radius=1pt];
	\draw[dashed] (2,0.6) --(2,-0.2) node[left] {$x$}; 
%	\filldraw[red](2.3,0.65)circle[radius=1pt];
	\draw[dashed] (2.3,0.65) --(2.3,-0.2) node[right] {$x+dx$}; 
	\draw[red,  line width =1pt] (2,0.6) --(2.3,0.65) node[above] {$ds$};  
	\draw[->, line width =1pt] (2,0.6) --(1.4,0.5) node[above] {$T_1$};  
	\draw[ ->,  line width =1pt] (2.3,0.65) --(2.9,0.75) node[above] {$T_2$};  
\end{tikzpicture}\\
受力分析:取任意微元ds (重力不计),受临近的拉力为$T_1, T_2$,则:\\
则水平合力为零:{ $T_2\cos \alpha _2=T_1\cos \alpha _1=T_0$ }  \\   \vspace{0.3cm} 
竖直合力提供加速度:{$T_2\sin \alpha _2-T_1\sin \alpha _1=ma=\rho ds~ u_{tt} $}   \\   \vspace{0.3cm} 
有:{ $T_0(\tan \alpha _2-\tan \alpha _1)=\rho ds ~u_{tt}$ }   \\   \vspace{0.3cm} 
~~~~$T_0[u_x(x+dx,t)-u_x(x,t)]=\rho dx ~u_{tt}$  \\  \vspace{0.3cm} 
~~~~$\displaystyle \frac{T_0}{\rho}\times\frac{u_x(x+dx,t)-u_x(x,t)}{dx}=u_{tt}$ \\
得波动方程:
\begin{equation*}
 u_{tt}=a^2u_{xx}
\end{equation*}
定解条件:\\
 (1) 初始条件 
	$\displaystyle  u(x,t)|_{t=0}= \psi (x) $, 	 $\displaystyle  u_t(x,t)|_{t=0}= \Psi (x) $\\
(2) 边界条件
	$\displaystyle  u(x,t)|_{x=0}= 0 $, 	 $\displaystyle  u(x,t)|_{x=l}= 0 $\\	\vspace{0.3cm} 
若各质点受到外力作用, 则有:
\begin{equation*}
	u_{tt}=a^2u_{xx} +f(x,t)
\end{equation*}
\end{proof}
\end{example}

\begin{note}
波动方程描述了围绕平衡态小幅震荡的规律,它不仅可以描述琴弦、鼓膜、耳机的震动,也描述着光波、声波、地震波、引力波,甚至弦论中弦的振动。
\end{note}

\subsection{波动方程的求解}
\begin{example} %2
现考虑一维波动方程初边值问题 (傅里叶) \\
   $\displaystyle \begin{cases}
 u_{tt}=a^2u_{xx}\\
u(x,t)|_{t=0}= \psi (x) ,~~~ u_t(x,t)|_{t=0}= \Psi (x) \\
u(x,t)|_{x=0}= 0, ~~~  u(x,t)|_{x=l}= 0 
\end{cases}$ \\
\begin{proof} 方程可分离变量:
设 $\displaystyle  u(x,t)=T(t)X(x) $,代回方程,有  \\
 $\displaystyle  \begin{cases}
  T~^{''}(t)X(x) =a~^2 T(t)X~^{''}(x) \\ \vspace{0.3cm}
 \frac{T~^{''}}{a~^2 T}=\frac{X~^{''} }{X} =-\lambda 
\end{cases}$ 

即偏微分方程转化为两常微分方程 \\
方程(I):\\
 $\displaystyle  \begin{cases}
	X~^{''} +\lambda X=0  ~~,~~ 0<x<l\\
   X(0)=0 ~,~X(l)=0
\end{cases}$ \\	
方程(2):\\
 $\displaystyle  \begin{cases}
	T~^{''} +\lambda {a~^2 T}=0 \\
	......
\end{cases}$ \\	

\begin{remark}
	偏微分方程与常微分方程的分离变量法有何不同?
\end{remark}

解方程(I):有特征方程, 
\begin{equation*}
\mu~^2 +\lambda =0
\end{equation*}
其根为:
 $\displaystyle  \begin{cases}
	\mu~_1=+\sqrt{-\lambda}\\
	\mu~_2=-\sqrt{-\lambda}
\end{cases}$ \\	
分情况讨论:\\
(1) 相异实根($\lambda < 0$)
有通解:	{ $\displaystyle 	X=Aexp~^{\sqrt{-\lambda}x} + Bexp~^{-\sqrt{-\lambda}x} $ } \\ \vspace{0.3cm}
分别取$x=0, x=l$, 得定解方程组:\\
$\left[
\begin{array}{lll}
	1&1\\
	exp~^{\sqrt{-\lambda}~l} &exp~^{-\sqrt{-\lambda}~l}
\end{array}
\right]$
$\left[
\begin{array}{ll}
	A\\
	B
\end{array}
\right]$
=$\left[
\begin{array}{ll}
	0\\
	0
\end{array}
\right]$

有解条件为:
$\begin{vmatrix}
	1&1\\
	exp~^{\sqrt{-\lambda}~l} &exp~^{-\sqrt{-\lambda}~l}
\end{vmatrix}
= 0$\\
很明显,这个行列式不等于0, 所以只有 A=0,~B=0, (零解)   \\

(2) 相同实根($\lambda = 0$),\\
则通解为:	{ $\displaystyle 	X=Ax + B $ } \\ 
分别取$x=0, x=l$, 得定解方程组:
{$\displaystyle \left\{
	\begin{array}{lll}
		B=0\\
		Al+B=0
	\end{array} \right. $}\\
 A=B=0,~也只有零解 \\

(3) 虚根($\lambda >0$),即: $ \mu~_1=i\sqrt{\lambda}~~,~~\mu~_2=-i\sqrt{\lambda}$	\\
则通解为:	{ $\displaystyle 	X=A\cos \sqrt{\lambda}x+ B\sin \sqrt{\lambda}x $ } \\ 
分别取$x=0, x=l$, 得定解方程组:\\
$\left[
\begin{array}{lll}
	1&1\\
	\cos( {\sqrt{\lambda}~l}) &\sin ({\sqrt{\lambda}~l})
\end{array}
\right]$
$\left[
\begin{array}{ll}
	A\\
	B
\end{array}
\right]$
=$\left[
\begin{array}{ll}
	0\\
	0
\end{array}
\right]$\\ 
系数行列式为零, $ \sin ({\sqrt{\lambda}~l})=0$ \\
$ \sqrt{\lambda}~l=n~\pi$ \\ 
固有值为:$\displaystyle  \lambda~_n=\frac{n^2\pi~^2}{l~^2}$ \\ 
固有解:{\large $\displaystyle  X~_n=B~_n \sin \frac{n\pi~}{l} x=B~_n \sin \omega_n x $}

求解方程II : 	$\displaystyle  T~^{''} +\lambda {a~^2 T}=0 $ \\ 
代入$\lambda_n$, 得:
$\displaystyle  T~^{''} +\lambda~_n a~^2 ~T=0 $ \\
 变形成:$\displaystyle  T~^{''} +\omega ~_n ^2 {a~^2 T}=0 $ \\ 
特征方程有虚根\\
通解为 	$\displaystyle 	T~_n=C~_n\cos \omega_n~a~t+ D~_n\sin \omega ~_n~a~t $ \\  \vspace{0.3cm}

原方程的基本解为:\\
   $\begin{array}{llll}
	u_n(x,t) &=& T_n(t)X_n(x)\\
	&=& (a_n\cos \omega_nat+ b_n\sin \omega _nat ) \sin \omega_n x\\
	&=&(a_n\cos\frac{ n\pi at}{l}+ b_n\sin \frac{ n\pi at}{l}) \sin \frac{ n\pi x}{l}
\end{array}$ \\ 

 叠加解 (解函数):\\ 
 $ \displaystyle u(x,t)=\sum_{n=1}^{\infty } u_n(x,t) = \sum_{n=1}^{\infty }  (a_n\cos\frac{ n\pi at}{l}+ b_n\sin \frac{ n\pi at}{l}) \sin \frac{ n\pi x}{l}$


\begin{note}
叠加解的思考与讨论:
\begin{itemize}
\item \textbf {数学理解}: 一个线性方程的解的线性组合,依然是方程的解  
\item \textbf {物理理解}:It is not complicated. It is just a lot of it. 
\item \textbf {核心成果}:傅里叶级数与傅里叶变换 
\end{itemize}
\end{note}

最后一步,确定叠加解的系数,\\
$ \displaystyle u(x,t)=\sum_{n=1}^{\infty } u_n(x,t) = \sum_{n=1}^{\infty }  (a_n\cos\frac{ n\pi at}{l}+ b_n\sin \frac{ n\pi at}{l}) \sin \frac{ n\pi x}{l}$\\
定解条件:\\ 
(1) $ \displaystyle u(x,0)= \varphi (x)$ ~~=> ~~$\varphi (x)=\sum_{n=1}^{\infty } a_n \sin \frac{ n\pi x}{l}$\\  
(2) $ \displaystyle u_t(x,0)= \Psi (x)$ ~~=> ~~$\Psi (x)=\sum_{n=1}^{\infty } b_n \frac{ n\pi a}{l} \sin \frac{ n\pi x}{l}$ \\  \vspace{0.3cm}
由傳里叶变换(非对称公式),得系数:\\  
$ \displaystyle a_n=  \frac{2}{l}\int_{0 }^{l}  \varphi (x) \sin \frac{ n\pi x}{l} dx $\\   
$ \displaystyle b_n= \frac{l} { n\pi a} \frac{2}{l}\int_{0 }^{l}  \Psi  (x) \sin \frac{ n\pi x}{l} dx =  \frac{2} { n\pi a}  \int_{0 }^{l}  \Psi  (x) \sin \frac{ n\pi x}{l} dx$\\   
\end{proof}
\end{example}

\subsection{固有函数正交性问题}
\begin{example} %3
试证明固有函数  $\displaystyle  X~_n= \sin \frac{n\pi~}{l} x  $ 的正交性。
\begin{proof} 
固有函数是固有方程的解:\\
 $\begin{array}{llll}
	&X_n ^{''}+\lambda_n X_n=0\\
	&X_m ^{''}+\lambda_m X_m=0
\end{array}$ \\ 
用$X_m$乘以第一式,$X_n$乘以第二式,\\
 $\begin{array}{llll}
	&X_m X_n ^{''}+\lambda_n X_m X_n=0\\
	&X_nX_m ^{''}+\lambda_m X_n X_m=0
\end{array}$ \\ 
两式相减:\\
 $\begin{array}{llll}
  & (\lambda_n-\lambda_m) X_n X_m= X_nX_m ^{''}-X_mX_n ^{''} 
\end{array}$ \\ 
积分得:\\
  $ \begin{array}{llll}
	(\lambda_n-\lambda_m) \int_{0 }^{l}  X_n X_m dx &= & \int_{0 }^{l}  [X_nX_m ^{''}-X_mX_n ^{''} ] dx\\   \vspace{0.3cm}
	&=&  [X_nX_m ^{'}-X_mX_n ^{'} ]_0 ^{l} - \int_{0 }^{l}  [X_n ^{'} X_m ^{'}-X_m ^{'} X_n ^{'} ] dx =0 \\   
\end{array}$ \\
 由零边界条件可知式中的第一项为零,固有函数求导可得第二项为零:\\
   $ \begin{array}{llll}
 	(\lambda_n-\lambda_m) \int_{0 }^{l}  X_n X_m dx=0 \\   
 \end{array}$ 
$ \lambda_n-\lambda_m\ne=0 (n\ne m)$\\
   $ \begin{array}{llll}
 \int_{0 }^{l}  X_n X_m dx=0 ~~,~~~ (n\ne m)\\   
\end{array}$ \\

当$n= m \ne 0$ \\   
 $ \begin{array}{llll}
	\int_{0 }^{l}  X_n X_m dx&= \int_{0 }^{l}  X_n X_n dx \\
	&= \int_{0 }^{l}   \sin ^2  \frac{n\pi~}{l} x dx \\
	&=  \frac{l}{2}  
\end{array}$ 
\end{proof}
\end{example}

\subsection{波动方程初值问题}
\begin{example} %4
	求解波动方程如下初边值问题\\
	$\displaystyle  \begin{cases}
		u_{tt} =u_{xx} ~~,~~ 0<x<1, t>0\\
		u(0,t) =u(1,t)=0 \\
		u(x,0) =sin \pi x, u_t (x,0)=0 
	\end{cases}$ \\	
	\begin{proof} 
	固有值:$\displaystyle  \lambda~_n=\frac{n^2\pi~^2}{l~^2 }= n^2\pi~^2 $ \\ 
	固有函数:$\displaystyle  X~_n=B~_n \sin \frac{n\pi~}{l} x=B~_n \sin n\pi x $\\
	解函数:\\ 
		$\begin{array}{llll}
			u(x,t)&=\sum_{n=1}^{\infty } u_n(x,t) = \sum_{n=1}^{\infty }  (a_n\cos\frac{ n\pi at}{l}+ b_n\sin \frac{ n\pi at}{l}) \sin \frac{ n\pi x}{l}\\
			        &= \sum_{n=1}^{\infty }  (a_n\cos n\pi t+ b_n\sin n\pi t ) \sin n\pi x \\
		\end{array}$ \\ 
	
		$\begin{array}{lllllllll}
		a_n&=  \frac{2}{l} \int_{0 }^{l}  \varphi (x) \sin \frac{ n\pi x}{l} dx \\
		       &= 2 \int_{0 }^{1}  \sin(\pi x) \sin n\pi x dx \\
		       &= 2 \int_{0 }^{1}  \sin(\pi x) \sin \pi x dx  \\
		       &=1~~  (n=1)   \\
		b_n&= \frac{2} { n\pi a} \int_{0 }^{l}  \Psi  (x) \sin \frac{ n\pi x}{l} dx  \\
		       &= \frac{2} { n\pi} \int_{0 }^{1}  0  \sin n\pi x dx  \\
			   &=0
		\end{array}$ \\ 
		解函数:$\displaystyle  u(x,t) = \cos(\pi t) \sin(\pi x)   $\\
	\end{proof}
\end{example}

\subsection{作业:求解波动方程初边值问题}
	$\begin{array}{lllllllll}

	& \begin{cases}
		u_{tt} =a^2u_{xx} ~~,~~ 0<x<l, t>0\\
		u(0,t) =u(l,t)=0 \\
		u(x,0) =sin \frac{\pi x}{l} ,  u_t (x,0)=\sin \pi x 
	    \end{cases}\\	
	&\begin{cases}
		u_{tt} =a^2u_{xx} ~~,~~ 0<x<l, t>0\\
		u(0,t) =u(l,t)=0 \\
		u(x,0) =3sin \frac{3\pi x}{2l} +6\sin(\frac{5\pi x}{2l}),  u_t (x,0)=0
	\end{cases} \\	
	&\begin{cases}
		u_{tt} =u_{xx} ~~,~~ 0<x<1, t>0\\
		u(0,t) =u(l,t)=0  \\
		u(x,0) =\sin 2\pi x ,  u_t (x,0)=x (1-x) 
	\end{cases} \\	
\end{array}$ \\ 


