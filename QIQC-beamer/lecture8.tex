

%%%%%%%%%%%%%%%%%%%%%%%%%%%%%%%%%%%55%%
\begin{frame} [plain]
    \frametitle{}
    \Background[1] 
    \begin{center}
    {\huge 第8:量子通信(1)    }
    \end{center}  
    \addtocounter{framenumber}{-1}   
\end{frame}
%%%%%%%%%%%%%%%%%%%%%%%%%%%%%%%%%%

\section{1.量子通信基础}
\begin{frame}
    \frametitle{量子通信网络}
    \begin{center}
        \includegraphics[width=0.85\textwidth]{figs/40.png}
    \end{center}
\end{frame}

\begin{frame}
    \frametitle{经典香农熵}
    假设一个事件有N种可能的结果,对应的概率分布为$\{p_k\}$,则香农熵定义为:
    \[H(x)=\sum_{k=1}^{N} p_{k} \log _{2} \frac{1}{p_{k}} \text { (bit) }\]
    代表测得可获取信息量的平均值\\
    对于2态事件,有:
    \[\begin{aligned} H_{2} &=p_{0} \log _{2} \frac{1}{p_{0}}+p_{1} \log _{2} \frac{1}{p_{1}} \\ 
        &=p \log _{2} \frac{1}{p}+(1-p) \log _{2} \frac{1}{(1-p)} \\
        &=1; \quad \text{if} \quad p=\frac{1}{2}
     \end{aligned}\]   
\end{frame}

\begin{frame}
    \frametitle{量子Von Neumann熵}
    设体系第n个本征态出现的概率为$p_n$,则密度矩阵和熵分别定义为:
    \[\rho=\sum_{n=1} p_{n}|n\rangle\langle n| \]
    \[S(\rho)=-\operatorname{tr}\left(\rho \log _{2} \rho\right)\]
    对于双粒子体系:有
    \[S\left(\rho_{A B}\right)=-\operatorname{tr}\left(\rho_{A B} \log _{2} \rho_{A B}\right)\]
    \[S\left(\rho_{A} \otimes \rho_{B}\right)=S\left(\rho_{A}\right)+S\left(\rho_{B}\right)\]
    \[S\left(\rho_{A} ; \rho_{B}\right)=\left[S\left(\rho_{A}\right)+S\left(\rho_{B}\right)\right]-S\left(\rho_{A B}\right)\]
\end{frame}

\begin{frame}
    \frametitle{密钥密码体系}
    \begin{itemize}
        \item  原文: I love you
        \item  密文: M PSZI CSY
        \item  加码函数: $f(x)=x+\alpha \bmod 26$
        \item  解码函数: $f^{-1}(y)=y-\alpha \bmod 26$
        \Item  密钥:$\alpha=4$ \\
        \item  原文序列:abcd{\color{red}{e}}fgh{\color{red}{i}}jk{\color{red}{l}}mn{\color{red}{o}}pqrst{\color{red}{u}}{\color{red}{v}}wx{\color{red}{y}}z \\
        \item  密文序列:ab{\color{green}{c}}defgh{\color{green}{i}}jkl{\color{green}{m}}no{\color{green}{p}}qr{\color{green}{s}}tuvwx{\color{green}{y}}{\color{green}{z}} \\ \vspace{1em}
        请问有多少个可能密钥?
    \end{itemize}
\end{frame}

\begin{frame}
    \frametitle{单时拍密钥方案}
    \begin{itemize}
        \item 原文字符串长度为$n$ \[ a_1a_2a_3\ldots a_n\]
        \item 信息发送者制备一个长度也为$n$的密钥串 \[ b_1b_2b_3\ldots b_n\]
        \item 加密函数 \[ c_i=a_i+b_i \bmod N \]
        \item 密文串 \[ c_1c_2c_3\ldots c_n\]
        \item 解密函数 \[ a_i=c_i-b_i \bmod N \]
    \end{itemize}
\end{frame}

\begin{frame}
    \frametitle{公开钥密方案}
    \begin{itemize}
        \item RSA密码方案: R.L.Rivest,A.Shamir和 L.Adelman于1978年提出的
        \item 依据:“验证两个大素数容易,而将他们的乘积做因数分解则极其困难” 
        \item 原文: $X$
        \item 密文: $Y$
        \item 加密: 使用公钥($e,N$) $Y=X^{e} \bmod N$
        \item 解密:使用私钥($d,N$) $X=Y^{d} \bmod N$
    \end{itemize}
\end{frame}

\begin{frame}
    \frametitle{}
    {\color{red}{制备}}:(e,d,N)
    \begin{enumerate}
        \item 秘密选择两个大素数$p,q$
        \item 计算出$N=p\times q$
        \item 计算出欧拉函数$\Phi(N)=(p-1)\times (q-1)$
        \item 选择一个较小的素数$e$做为公钥,它与$\Phi(N)$互质
        \item 计算私钥$d$, $ed\equiv 1 \mod \Phi(N)$
    \end{enumerate}
     {\color{red}{破解}}:想从$e$得到$d$,必须知道$\Phi(N)$;想要得到$\Phi(N)$,必须知道$p,q$; 想要得到$p,q$,必须做大数质因式分解$N=p \times q$!
\end{frame}


\begin{frame}
    \frametitle{密钥分发}
    \begin{itemize}
        \Item 密使分发 $\qquad$ 怕汉奸!
        \Item 公开分发 $\qquad$ 怕量子算法!
    \end{itemize}
    必须发展量子算法破解不了的量子密钥公开分发方案!
\end{frame}

\section{2.量子密码分发~Quantum key Distribution}
\begin{frame}
    \frametitle{量子不可克隆原理}
    {\Bullet}~克隆(cloning)指在目标系统(B)中产生一个与源系统(A)相同的态,而不改变源系统的态。\\ \vspace{1em}
    \例[1.试证明未知量子态不可克隆]{}
    \证~设已知的本征态可以克隆
    \[ U_c\rs{C}\rs{0}_A\rs{\varphi}_B=\rs{C_0}\rs{0}_A\rs{0}_B\]
    \[ U_c\rs{C}\rs{1}_A\rs{\varphi}_B=\rs{C_1}\rs{1}_A\rs{1}_B\]
    若A处于未知叠加态\[ \rs{\Psi}_A=\alpha\rs{0}+\beta\rs{1}\]
\end{frame}

\begin{frame}
    \frametitle{}
    期望的克隆:
    \[\begin{aligned}
        U_{c}|C\rangle|\psi\rangle_{A}|\varphi\rangle_{B}=&\left|C_{\psi}\right\rangle|\psi\rangle_{A}|\psi\rangle_{B} \\
        =&\left.\left|C_{\psi}\right\rangle(\alpha|0\rangle+\beta)|1\rangle\right)_{A}(\alpha|0\rangle+\beta|1\rangle)_{B} \\
        =&\left|C_{\psi}\right\rangle\left(\alpha^{2}|0\rangle_{A}|0\rangle_{B}+\alpha \beta|0\rangle_{A}|1\rangle_{B}+\alpha \beta|1\rangle_{A}|0\rangle_{B}\right.\\
        &~~\left.+\beta^{2}|1\rangle_{A}|1\rangle_{B}\right)
        \end{aligned} 
    \]
     服从量子力学的过程:
    \[ \begin{aligned}
        U_{c}|C\rangle|\psi\rangle_{A}|\varphi\rangle_{B} &=U_{c}|C\rangle\left(\alpha|0\rangle+\beta|1\rangle\right)_{A}|\varphi\rangle_{B} \\
        &=U_{c}|C\rangle \alpha|0\rangle_{A}|\varphi\rangle_{B}+U_{c}|C\rangle \beta|1\rangle_{A}|\varphi\rangle_{B} \\
        &=\alpha^{2}\left|C_{0}\right\rangle|0\rangle_{A}|0\rangle_{B}+\beta^{2}\left|C_{1}\right\rangle|1\rangle_{A}|1\rangle_{B}
        \end{aligned}
    \]
    ~~\\
    两式相等的必要条件为 $\alpha\beta=0$, 即可克隆的$\rs{\Psi}_A$态不可能是叠加态!\\
    证毕!          
\end{frame}

\begin{frame}
    \frametitle{量子不可删除原理}
    {\Bullet}~删除是指如果目标系统(B)有源系统(A)的量子态副本,把B”置零” 而不改变A系统的状态
    \例[2.试证明未知量子态不可删除]{}
    \证~设已知的本征态可以删除
    \[ U_c\rs{C}\rs{0}_A\rs{0}_B=\rs{C_{00}}\rs{0}_A\rs{R}_B\]
    \[ U_c\rs{C}\rs{1}_A\rs{1}_B=\rs{C_{11}}\rs{0}_A\rs{R}_B\]
    \[ U_c\rs{C}\rs{0}_A\rs{1}_B=\rs{C_{01}}\rs{0}_A\rs{1}_B\]
    \[ U_c\rs{C}\rs{1}_A\rs{0}_B=\rs{C_{10}}\rs{0}_A\rs{0}_B\]
\end{frame}


\begin{frame}
    \frametitle{}
    对于叠加态,期望的删除:
    \[\begin{aligned}
        U_{c}|C\rangle|\psi\rangle_{A}|\psi\rangle_{B}&=\left|C_{\psi}\right\rangle|\psi\rangle_{A}|R\rangle_{B} \\
        &=|C_{\psi}\rangle(\alpha\rs{0}+\beta\rs{1})_{A}|R\rangle_{B} 
        \end{aligned} 
    \]
     服从量子力学的过程:
    \[ \begin{aligned}
        U_{c}|C\rangle|\psi\rangle_{A}|\psi\rangle_{B}=&U_{c}|C\rangle(\alpha|0\rangle+\beta|1\rangle)_{A}(|\alpha| 0\rangle+\beta|1\rangle)_{B} \\
        =& U_{c}|C\rangle \alpha^{2}|0\rangle_{A}|0\rangle_{B}+U_{c}|C\rangle \beta^{2}|1\rangle_{A}|1\rangle_{B} \\
        &~~+U_{c}|C\rangle \alpha \beta|0\rangle_{A}|1\rangle_{B}+U_{c}|C\rangle \alpha \beta|1\rangle_{A}|0\rangle_{B} \\
        =& \alpha^{2}\left|C_{00}\right\rangle|0\rangle_{A}|R\rangle_{B}+\beta^{2}\left|C_{11}\right\rangle|1\rangle_{A}|R\rangle_{B} \\
        &~~+\left|C_{01}\right\rangle \alpha \beta|0\rangle_{A}|1\rangle_{B}+\left|C_{10}\right\rangle \alpha \beta|1\rangle_{A}|0\rangle_{B}
        \end{aligned}
    \]
    ~~\\
    两式根本无法相等,即可删除的$\rs{\Psi}_A$态不可能是叠加态!\\
    证毕!          
\end{frame}

\subsection{BB84量子密码分发协议}
\begin{frame}
    \frametitle{}
    \begin{tcolorbox2}{BB84协议}
    {1984年,C. H. Bennett 和G. Brassard 提出利用偏振光进行量子密钥分发的协议。把一次性密码原理和量子测量原理结合在一起,建立不可破解密码分发方案}
    \end{tcolorbox2}  
\end{frame}

\begin{frame}
    \frametitle{原理}
    \begin{enumerate}
        \item Ailce 由随机数序列a决定选择纵横基还是对角基。设a中的0代表纵横基{$\rs{0},\rs{1}$},1代表对角基{$\rs{+},\rs{-}$} 
        \item Ailce 再由随机数序列a' 决定发射给Bob的光子的具体偏振。设a'中的0代表$\rs{0},\rs{+}$\\
        \hspace{2em} 设有 $a=0010$,$a'=1001$:Ailce发出的光子的偏振序列为$\rs{10+1}$
        \item Bob 得到偏振光子串后,由随机数序列b决定选择纵横基还是对角基,再由随机数序列b' 决定用相应基里的那个基矢波片进行测量\\
        \hspace{2em} 设有 $b=0110$,$b'=1100$:Bob使用的偏振片序列为$\rs{1-+0}$ 
        \item 双方公布第一序列(a=0010 and b=0110),发现第1、3位相同。
        \item 双方就知道各自手里的$a'=1001$和$b'=1100$中的第1、3位相同。整理出来,为$10$,并以此做为私钥d,完成密钥分发。
    \end{enumerate}
\end{frame}

\begin{frame}
    \frametitle{}
    {\color{red}{保密性分析}}:
    \begin{enumerate}
        \item 设有窃听者C,先得到A发给B的每一个光子。这阻断了A、B之间的通信,没有得到d。
        \item C想知道光子的状态,若测量,因不知道具体的基,导致每个光子有$50\%$概率状态改变。
        \item C因不知光子状态,不能进行克隆,强行克隆势必导致原光子状态改变。
        \item C只能窃听Ailce和Bob的通信,得到a和b,但不是a'和b'。
        \item 派内鬼分别去Ailce和Bob的办公室,窃取a'和b',难度大于窃取d。
    \end{enumerate}
\end{frame}

\begin{frame}
    \frametitle{}    
    \begin{columns}[T,onlytextwidth]
        \column{0.49\textwidth} 
        ~\\ \vspace{5em}
        \例 [试求x=11和N=21的阶r]{} 
        \解 (1) 取$r=1,2,3,\cdots$,依次计算 $11^r$ \\
        (2) 依次计算 $\dfrac{11^r}{21}$ 的余数\\
        (3) 第一个余数为1的r=6,得解\\ \vspace{2em}

        左表表明,阶就是余子式(完全集)的自由度
        \column{0.49\textwidth}
        \begin{center}
            \includegraphics[width=0.8\textwidth]{figs/38.png}
        \end{center}
    \end{columns}
\end{frame}

\begin{frame}
    \frametitle{}  
    \例 [试证明求阶问题与因式分解问题等价]{}  
    \证 设N可以作因子分解 $N=n_1\times n_2$
    若阶r为偶数,有
        \[x^r= 1 (mod N)\]
        \[x^r-1= M\times N \]
        \[(x^{r/2}-1) (x^{r/2}+1)=  M\times N = M\times n_1\times n_2 \]
    上式表明,求最大公约数
    \[gcd(x^{r/2}-1, N); \quad gcd(x^{r/2}+1, N)\]  
    很可能就能得到因子 $n_1$ 和 $n_2$ \\
    因此,因式分解问题就转化求阶问题,它们是等价的!
\end{frame}

\begin{frame}
    \frametitle{量子求阶}  
    \begin{itemize}
        \item 阶r就是余子式(完全集)的自由度。
        \item 量子傅里叶变换公式中,N就是基矢数目
        \[\boxed{y_k = \frac{1}{\sqrt{N}} \sum_{j=0} ^{N-1} x_j  e^{i \frac{2\pi}{N} jk} } \]
        \item 量子力学表明这个数目就是体系的自由度,即上式可取$N=r$
    \end{itemize}
    余子式$\{\rs{x^k~mod~N}\} $构成完全集,任意态函数可在其上展开,比如本征态S
    \[\rs{u_s} = \frac{1}{\sqrt{r}} \sum_{k=0} ^{r-1}  e^{-i \frac{2\pi}{r} sk} \rs{x^k~mod~N}  \]
\end{frame}

\begin{frame}
    \frametitle{}  
    对上式做傅里叶逆变换
    \[\rs{x^k~mod~N}   = \frac{1}{\sqrt{r}} \sum_{s=0} ^{r-1}  e^{-i \frac{2\pi}{r} sk} \rs{u_s} \]
    取k=0,得到余子式1态在量子本征函数系 $\{\rs{u_s}\}$的展开式
    \[\rs{x^0~mod~N}   =  \sum_{s=0} ^{r-1} \frac{1}{\sqrt{r}} \rs{u_s} =\sum_{s=0} ^{r-1} a_s \rs{u_s} \]
    其振幅(展开系数)$a_s=\dfrac{1}{\sqrt{r}}$,完全由阶(r)决定!  
\end{frame}

\begin{frame}
    \frametitle{求阶的算法推导} 
    \begin{center}
    \begin{tcolorbox1}[0.86]{$u_s$的本征值}
        试证明相位估计算子U的本征态$u_s$的本征值为$\exp[i2\pi\varphi]=\exp[i2\pi\frac{s}{r}]$
    \end{tcolorbox1}    
    \end{center}
 
     即要证明其本征方程为:
    \[U\rs{u_s}=\exp[i2\pi\frac{s}{r}] \rs{u_s}\]
\end{frame}

\begin{frame}
    \frametitle{}     
    \证~相位估计算子U在余子式有如下作用效果
    \[U\rs{y}=\rs{xy~\mod~N}\]
    \[\begin{aligned}
     \rs{u_s} &= \frac{1}{\sqrt{r}} \sum_{k=0} ^{r-1}  e^{-i \frac{2\pi}{r} sk} \rs{x^k \mod N}  \\
     U\rs{u_s} &= \frac{1}{\sqrt{r}} \sum_{k=0} ^{r-1}  e^{-i \frac{2\pi}{r} sk} U\rs{x^k \mod N}  \\
     &= \frac{1}{\sqrt{r}} \sum_{k=0} ^{r-1}  e^{-i \frac{2\pi}{r} sk} \rs{x^{k+1} \mod N}  \\
     &= \frac{1}{\sqrt{r}} \sum_{k'=1} ^{r}  e^{-i \frac{2\pi}{r} s(k'-1)} \rs{x^{k'} \mod N} 
    \end{aligned}\]   
\end{frame}

\begin{frame}
    \frametitle{}        
    \[\begin{aligned}
    &= \frac{1}{\sqrt{r}} e^{i \frac{2\pi}{r}s }  \sum_{k'=1} ^{r}  e^{-i \frac{2\pi}{r} sk' } \rs{x^{k'} \mod N} \\
    &= \frac{1}{\sqrt{r}} e^{i \frac{2\pi}{r}s }  \sum_{k'=1} ^{r-1}  e^{-i \frac{2\pi}{r} sk' } \rs{x^{k'} \mod N} 
    + \frac{1}{\sqrt{r}} e^{i \frac{2\pi}{r}s }  e^{-i \frac{2\pi}{r} s r } \rs{x^{r} \mod N} \\
    &= \frac{1}{\sqrt{r}} e^{i \frac{2\pi}{r}s }  \sum_{k'=1} ^{r-1}  e^{-i \frac{2\pi}{r} sk' } \rs{x^{k'} \mod N} 
    + \frac{1}{\sqrt{r}} e^{i \frac{2\pi}{r}s }  e^{-i 2\pi s } \rs{x^{0} \mod N} \\
    &= \frac{1}{\sqrt{r}} e^{i \frac{2\pi}{r}s }  \sum_{k'=0} ^{r-1}  e^{-i \frac{2\pi}{r} sk' } \rs{x^{k'} \mod N} \\
    &=\exp[i2\pi\frac{s}{r}] \rs{u_s}
    \end{aligned}\]  
    证毕! 
\end{frame}

\begin{frame}
    \frametitle{}  
    \[U\rs{u_s}=\exp[i2\pi\frac{s}{r}] \rs{u_s} = \exp[i2\pi\varphi] \rs{u_s}\]
    因此
    \[\varphi=\frac{s}{r}\]
    $\varphi$ 可通过傅里叶逆变换求得,则阶$r$可通过连分数算法求得
\end{frame}

\begin{frame}
    \frametitle{连分数算法}  
    \begin{center}
        \includegraphics[width=0.6\textwidth]{figs/39.png}
    \end{center}
    \begin{itemize}
        \item 有理数的连分数表示是有限的
        \item 任一有理数的连分数表示是唯一的
        \item “简单”有理数的连分数表示是简短的
        
    \end{itemize}  
\end{frame}