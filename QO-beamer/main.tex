\documentclass[12pt,UTF8,aspectratio=169]{beamer} 
%aspectratio=1610:160mm,100mm
%aspectratio=169:160mm,90mm
%aspectratio=43:128mm,96mm,默认尺寸
\RequirePackage{mybeamer}
%\includeonlyframes{current} % 只编译 \begin{frame}[label=current] \end{frame}

\usepackage{overpic}
%%% sudo vim /usr/local/texlive/2021/texmf-dist/tex/xelatex/mybeamer
%%%% sudo texhash

%-------------------正文-------------------------%
%                                               %
%                                               %
\begin{document}                                %
%                                               %
%                                               %
%-----------------------------------------------%

%题目,作者,学校,日期                
\author{李小飞}
\title{\textbf{\Huge 量~~子~~光~~学}}
\subtitle{Quantum Optics}
\institute[电子科技大学]{光电科学与工程学院}
\date{\today}

%%%%%%%%%%%%%%%%%%%%%%%%%%%%%%%%%
%\frame[plain]{\titlepage}
\begin{frame} [plain]
    \setbeamertemplate{headline} {} %
    \Background[17] 
    \maketitle
    \addtocounter{framenumber}{-1} 
\end{frame}
%\maketitle
%%%%%%%%%%%%%%%%%%%%%%%%%%%%%%%%%

%-----------章节-------------------------% 
%\begin{frame}
    \frametitle{}
    \例[1、求下方程的解]{
        \[x^2+y^2=z^2\]}
    \解	\\
    \Tips \\
    \证明 \\
\end{frame}

\begin{frame}
    \frametitle{}
    \例[1、求下方程的解]{
        \[x^2+y^2=z^2\]}
    \解	\\
    \Tips \\
    \证明 \\
\end{frame}

\begin{frame}
    \frametitle{}
    \tcbb[0.5]{标题}{内容}
    \tcbb{标题}{内容}
\end{frame}


%%\section{课程简介}
\begin{frame}
    \frametitle{课程简介}
    \begin{enumerate}
        \Item 量子光学: is the study of the interaction of individual photons, in the wavelength range from the infrared to the ultraviolet, with ordinary matter - e.g. atoms, molecules, electrons, etc. - described by nonrelativistic quantum mechanics. 
        \Item 课程目标:       
        \begin{itemize}
          \item Grasp of the basic theory of quantum optics
          \item Know of the common application frontiers of quantum optics
      \end{itemize}
    \end{enumerate}
\end{frame}

\begin{frame} 
  \frametitle{分数构成}
      \begin{enumerate}
          \Item Normal results 20\%
          \Item Group discussion 30\%
          \Item Project final report 50\%
      \end{enumerate}
\end{frame}

\begin{frame}
    \frametitle{教材}
      \begin{itemize}
          \Item 《Quantum optics》 Scully, Zubairy, 1997 (Cambridge)        
          \Item 《Quantum Optics: An Introduction》,Fox, Mark,2006
          \Item 《Introductory Quantum Optics》, Gerry, Knight, 2004 (Cambridge)
          \Item 《Statistical Methods in Quantum Optics》Howard, Carmichael, 1999 
          \Item 《Mathematical Methods of Quantum Optics》Ravinder Rupchand Puri, 2001
          \Item 《量子光学研究前沿》上海交通大学出版社出版,张卫平, 2014
          \Item 《量子光学》科学出版社, 郭光灿, 2022
      \end{itemize}
\end{frame}
%%%%%%%%%%%%%%%%%%%%%%%%%%%%%%%%%%%%%

\begin{frame} 
\frametitle{网络教学资源}

牛津大学 Mark Fox (1)
\href{https://www.bilibili.com/video/BV1PK4y1E79Z?t=3.2}{点这里}\\ {\vspace*{2.3em}}

牛津大学 Mark Fox (2)
\href{https://www.bilibili.com/video/BV1mb4y1j7mu/?spm_id_from=autoNext} {点这里} \\ {\vspace*{2.3em}}

慕尼黑大学 Immanuel Bloch
\href{https://www.bilibili.com/video/BV1Ky4y1W7C7?p=1} {点这里}
     
\end{frame}

\begin{frame}
      \frametitle{量子光学相关专业}
      \begin{itemize}
        \Item 光学 (光物理, 光化学, 光材料, 光谱精细结构)
        \Item 光学工程 (激光,光电器件, 光电探测, 非经典光源,量子成像, 量子雷达)  
        \Item 量子信息学 (量子计算, 量子通信, 量子精密测量, 量子传感,  超冷原子)    
    \end{itemize}  
\end{frame}

\begin{frame}
    \frametitle{光学的发展}
    三大光学: 几何光学, 波动光学(物理光学), 量子光学
    \begin{itemize}
        \Item 经典光学(麦克斯韦方程) \\
        不能解释: 黑体辐射, 光电效应, 康普顿效应, 原子光谱, 自发发射, 受激发射...
        \Item 半经典光学(原子能级量子化+经典光场+光子假说)  \\
        不能解释: 延迟选择实验, 量子擦除实验, 相干态, 压缩态, 量子计算, 量子通信, 量子存储...
        \Item 量子光学(量子化粒子+量子化光场)  \\
        解释当前一切光学(实验)现象. 
    \end{itemize}     
\end{frame}

\begin{frame}
    \frametitle{课程内容}
        \begin{enumerate}
            \Item lecture-1   ~~~~半经典光
            \Item lecture-2and3 ~ 量子力学基础与谐振子
            \Item lecture-3and4 ~ 光场量子化
            \Item lecture-6to9  ~~ 相干态与压缩态
            \Item lecture-10to15 ~ 光场测量与光子统计
            \Item lecture-16to20 ~ 光场中的原子
        \end{enumerate}
  \end{frame}

\begin{frame} [plain]
    \frametitle{}
    \Background[1] 
    \begin{center}
    {\huge 第1讲:经典与半经典光学}
    \end{center}  
    \addtocounter{framenumber}{-1}   
\end{frame}
%%%%%%%%%%%%%%%%%%%%%%%%%%%%%%%%%%

\begin{frame}
      \frametitle{本讲要点}
      \begin{enumerate}
        \Item 经典和半经典光学
        \Item 经典和半经典光学主要成就
        \Item 经典和半经典光学所面临的困难
    \end{enumerate}
      
\end{frame}

\section{1. 经典光学}

\begin{frame}
      \frametitle{经典光学的成果:麦克斯韦方程}
    介质中: 定义电位移矢量$\mathbf{D}$ 和磁场强度 $\mathbf{H}$
\[ \mathbf{D}=\epsilon_0 \mathbf{E} + \mathbf{P} = \epsilon_0 \epsilon_r \mathbf{E} \, (*),  \qquad \mathbf{H}=\frac{1}{\mu_0} \mathbf{B} -\mathbf{M}= \frac{1}{\mu_0\mu_r}\mathbf{B} \, (*)\]
麦克斯韦方程:

\[ \begin{aligned}
        \text{I}~~~ \hspace*{2em}&\nabla \cdot \mathbf{D} =\rho _f  \\  
        \text{II}~~ \hspace*{2em}&\nabla \cdot \mathbf{B} = 0  \\  
        \text{III}~ \hspace*{2em}&\nabla \times  \mathbf{E} = -\cfrac{\partial \mathbf{B}}{\partial t }  \\  
        \text{IV~}~ \hspace*{2em}&\nabla \times  \mathbf{H} = \mathbf{J}_f +  \cfrac{\partial \mathbf{D}}{\partial t } 
    \end{aligned} \]
\end{frame}

\begin{frame}
      \frametitle{电磁波}
    对于真空 ($\rho _f =0, \mathbf{J}_f =0 $),  \[ \mathbf{B} = \mu_0 \mathbf{H}, \qquad  \mathbf{D} = \epsilon_0 \mathbf{E} \]
代入麦克斯韦方程(IV)
 \[ \nabla \times  \mathbf{H} = \mathbf{J}_f +  \cfrac{\partial \mathbf{D}}{\partial t } \]
得: \[ \nabla \times \mathbf{B} = \mu_0\epsilon_0 \cfrac{\partial \mathbf{E}} {\partial t } \]
由麦克斯韦方程(III)
\[    
\begin{aligned}
  \nabla \times (\nabla \times  \mathbf{E}) &= - \nabla \times \cfrac{\partial \mathbf{B}}{\partial t } \\
  &= - \mu_0\epsilon_0  \cfrac{\partial ^2 \mathbf{E}} {\partial t^2 }
\end{aligned} \]  
\end{frame}

\begin{frame}
      \frametitle{}
    由于
  \[
  \begin{aligned}
      \nabla \times (\nabla \times  \mathbf{E}) &=  \nabla (\nabla \cdot  \mathbf{E})- \nabla^2 \mathbf{E} \\
      &= - \nabla^2 \mathbf{E} 
  \end{aligned} \]
  得:
  \[
  \nabla^2 \mathbf{E}= \mu_0\epsilon_0 \cfrac{\partial ^2 \mathbf{E}} {\partial t^2 }\]
  由于$ c= \frac{1}{\sqrt{\mu_0\epsilon_0}} $, 方程可写为
  \[\boxed{\mathbf{E}_{tt} =c^2\nabla^2 \mathbf{E}}\]
  ~~\\
  这是波动方程标准型(见数理方程), 这表明光波就是电磁波 \\ 若给出定解条件, 方程可求解 \\  
\end{frame}

\begin{frame}
 \frametitle{}
 \[\mathbf{E}_{tt} =c^2\nabla^2 \mathbf{E}\]
 这是矢量方程. 描述光的偏振. 
   \begin{center}
        \includegraphics[width=0.8\textwidth]{figs/10.png}
   \end{center}
 考虑标量化.
 \begin{enumerate}
     \item 线偏振 $\mathbf{E}(z,t) = E_x(z,t) \hat{e}_x $ 
     \item 圆偏振 $\mathbf{E}(z,t) = E_x(z,t) \hat{e}_{\sigma}, \qquad \text{with} \qquad \hat{e}_{\pm}= \mp \frac{1}{\sqrt{2}} (\hat{x} \pm \hat{y}) $
 \end{enumerate}
\end{frame}

\begin{frame}
      \frametitle{}
      ~~\\
    \例 [1. 试求一维光学腔中的线偏振电磁场] {
      \begin{center}
           \includegraphics[width=0.4\textwidth]{figs/1.png}
      \end{center}  
      光学腔: $0\leq z\leq L$ \\  
      边界条件: $E_x(0)=E_x(L)=0$  
    }
    \解  设 $\displaystyle  E_x(z,t)=T(t)Z(z) $,代入波动方程 
    \[\mathbf{E}_{tt} =c^2\nabla^2 \mathbf{E}\]
\end{frame}

\begin{frame}
      \frametitle{} 
    得:
	\begin{equation*}
		 T~^{''}(t)Z(z) =c~^2 T(t)Z~^{''}(z) 
	\end{equation*}
	分离变量, 令: {\hspace*{2em}}
      $$ \dfrac{T~^{''}}{c~^2 T}=\dfrac{Z~^{''} }{Z} =-\lambda $$ \\ \vspace{0.3cm}
    转化为两常微分方程 \\ \vspace{0.3cm}
    方程(I):
      $\displaystyle  \begin{cases}
          Z~^{''} +\lambda Z=0  ~~,~~ 0<z<L\\
          Z(0)=0 ~,~Z(L)=0
      \end{cases}$ \\	
    方程(II):
      $\displaystyle  \begin{cases}
          T~^{''} +\lambda {c~^2 T}=0 \\
          ......
      \end{cases}$ \\	  
\end{frame}

\begin{frame}
      \frametitle{}
    特征(辅助)方程法解方程(I)  
    \begin{enumerate}
    \IItem 固有值:$\displaystyle  \lambda_n=\frac{n^2\pi^2}{L^2}= \omega^2 _n, \quad k_n=\frac{\omega_n}{c} $ 
    \IItem 固有解:$\displaystyle  E_n(z)=\sin (\omega_n z) $
    \end{enumerate}
    解方程II : 	\[ T~^{''} +\lambda {c~^2 T}=0 \] \\ 
	代入$\lambda_n$, 得:
	$\displaystyle  T~^{''} +\lambda~_n c~^2 ~T=0 $ \\
	变形为:\[  T~^{''} +\omega ~_n ^2 {c~^2 T}=0 \]
	特征方程有虚根,通解 :\\
	\hspace{3cm}	$\displaystyle 	T~_n=c~_n\cos \omega_n~c~t+ d~_n\sin \omega ~_n~c~t $  \\ \vspace{1em}
\end{frame}

\begin{frame}
    \frametitle{} 
    ~\\
    原方程的基本解:\\ {\vspace*{0.3em}}
	$\begin{array}{llll}
		E_n (z,t) &=& T_n(t)Z_n(z)\\
		&=& (c_n\cos \omega_nct+ d_n\sin \omega _nct ) \sin \omega_n z\\
        &=&a_n \exp(i \omega_n ct) \sin \frac{ n\pi z}{L} \\
	\end{array}$ \\  {\vspace*{0.6em}}    
    \begin{enumerate}
    %\IItem 基本解:$\displaystyle E_{n}(z,t) = a_n q_n (t) \sin (k_n z), \quad \text{with} \quad a_n = \left( \frac{2 \omega ^2}{V \epsilon_0}\right)^{1/2} $
    \IItem 基本解:$\displaystyle E_{n}(z,t) = a_n q_n (t) \sin (k_n z) $
    \IItem 叠加解:$\displaystyle E_{x}(z,t) = \sum\limits_{n=1}^{\infty } a_n q_n (t) \sin (k_n z)$
    \end{enumerate}	
    把解代回由麦克斯韦方程(III), 得磁场叠加解 \\ 
    \[ H_{y}(z,t) = \sum\limits_{n=1}^{\infty } a_n \frac{\epsilon_0}{k_n}q_n ' (t) \cos (k_n z)\] 
\end{frame}

\begin{frame}
      \frametitle{}
    令: 
    \[  L \to \infty \]
    得自由场解 
    \[  E_{x}(z,t) = \frac{1}{2} E_{0x}(z) \exp [i(kz-\omega t)] \]
    ~~\\ 
    电磁场的哈密顿(能量)
    \[ H = \frac{1}{2} \int_V d V \left[ \epsilon_0 \mathbf{E^2}(\mathbf{r},t) + \frac{1}{\mu_0}\mathbf{B^2}(\mathbf{r},t)\right]\]
\end{frame}

\begin{frame} 
\frametitle{波动光学基本结论}
    电磁场是一系列基本振动模式的叠加.\\ 
    \[ E_{x}(z,t) = \sum\limits_{n=1}^{\infty } a_n q_n (t) \sin (k_n z)\]
    \[ H_{y}(z,t) = \sum\limits_{n=1}^{\infty } a_n \frac{\epsilon_0}{k_n}q_n ' (t) \cos (k_n z)\] 
    自由场是自由振动;存在电荷或电流等环境,则是受迫振动, 波动方程为:
    \[\mathbf{E}_{tt} -c^2\nabla^2 \mathbf{E} = \mu_0 \frac{\partial ^2 \mathbf{P}  }{\partial t^2}\]  
\end{frame}

\begin{frame}
      \frametitle{经典光学面临的困难}
      基于麦克斯韦方程的波动光学, 不能解释如下实验
      \begin{itemize}
          \item 黑体辐射, 
          \item 光电效应, 
          \item 康普顿效应, 
          \item 原子光谱, 
          \item 光的发射与吸收...
      \end{itemize}
      * 对上述问题的解释导致量子力学的建立
\end{frame}

\section{2. 半经典光学}

\begin{frame}
      \frametitle{光量子假说}
      1900年, 普朗克提出热辐射能量子假说
      \[E= n \varepsilon , \qquad \varepsilon= h \nu= \hbar \omega\]
      1905年,爱因斯坦提出光量子假说, 揭示光的波粒二象性本质.  \\ {\vspace*{1em}}
      \[E= h \nu= \hbar \omega, \qquad  \mathbf{p}=\frac{h}{\lambda} \mathbf{n} = \hbar \mathbf{k} \]  
      \\  \vspace*{3em}

      基此发展出半经典光学, 可成功解释经典光学所面临的上述困难!
\end{frame}

\begin{frame}
    \frametitle{半经典光学}     
    \begin{itemize}
        \Item 量子化原子能级
        \Item 经典的光学场+光量子假说
    \end{itemize}  
\end{frame}

\begin{frame}
      \frametitle{}
      \例 [2. 试采用半经典方法处理光与原子的相互作用问题] {}
      \解~ 考虑沿z轴传播的单色光 
      \[ \left\{\begin{array}{l}
        E_{x}=E_{0} \cos \left(\frac{2 \pi}{\lambda} z-\omega t\right) \\
        E_{y}=E_{z}=0
        \end{array}\right. \]
     光与原子的相互作用发生在原子内部, 这个尺度的光场可认为是均匀场
     \[E_{x}=E_{0} \cos \left(\omega t\right)  \]
     光波所产生的能量可看做是对原子能级的微扰 
     \[\begin{aligned}
        &\hat{H}^{\prime}=e\mathbf{r}\cdot\mathbf{E}  = ex E_{x} \\
        &=\frac{1}{2} \operatorname{ex} E_{0}\left[e^{i \omega t}+e^{-i \omega t}\right] \\
        &=\hat{F}\left[e^{i \omega t}+e^{-i \omega t}\right]
        \end{aligned}
      \]
\end{frame}

\begin{frame}
      \frametitle{}
      代入含时微扰公式
      \[ \omega_{m \rightarrow k}=\frac{2 \pi}{\hbar}\left|F_{k m}\right|^{2} \delta\left(\varepsilon_{k}-\varepsilon_{m}+\hbar \omega\right) \]
      对于自然光,可得跃迁概率:
      \[ w_{k \rightarrow m}=\frac{4 \pi^{2} e^{2}}{3 \hbar^{2}} I\left(\omega_{m k}\right)\left|\vec{r}_{m k}\right|^{2} = B_{km} I\left(\omega_{m k}\right)\]
      求得爱因斯坦吸收系数 $B_{km}$ \\ 
      同理,得爱因斯坦受激发射系数 $B_{mk}$  \\ 
      代入电磁辐射平衡条件(发射的光子数等于吸收的光子数) 
      \[N_{m}\left[A_{m k}+B_{m k} I\left(\omega_{m k}\right)\right]=N_{k} B_{k m} I\left(\omega_{m k}\right) \]
      得自发发射系数 $A_{mk}$ 
\end{frame}

\begin{frame}
      \frametitle{}
      基于光子数目决定电磁场强度的基本假设, 得辐射场强度
    \[\begin{aligned}
        J_{m k} &=N_{m} A_{m k} \hbar \omega_{m k} \\
        &=N_{m} \frac{4 e^{2} \omega_{m k}^{4}}{3 c^{3}}\left|\vec{r}_{k m}\right|^{2} 
        \end{aligned} \] {\vspace*{2.3em}}
    成功解决辐射场问题, 如: 选择定则, 激发态寿命, 常见光谱, ... \\ \vspace*{2.0em}
    增加自旋, 解决光谱分裂问题 \\
    增加旋-轨耦合,解决复杂光谱问题 \\ 
    增加非线性效应,解决变频问题 \\
\end{frame}

\begin{frame}
    \frametitle{半经典光学面临的困难}
    半经典半量子光学取得了具大成功. \\ 
    但不能解释如下光学现象
    \begin{itemize}
        \item 延迟选择实验
        \item 量子擦除实验
        \item 相干态
        \item 压缩态
        \item 纠缠光子对
        \item 单光子源
        \item 量子隐形传态
    \end{itemize}
    * 这些问题的解释导致第二次量子革命, 1956年后, 发展出非经典光源 (激光, 压缩光, 单光子), 人类进入量子光学时代.
\end{frame}

\begin{frame}
    \frametitle{惠勒延迟选择实验}
    \begin{center}
        \includegraphics[width=0.6\textwidth]{figs/choose.png} \\
    \end{center} 
    {\Bullet} 光子总是处于叠加
    {\Bullet} 光路的说法是不成立的
\end{frame}

%\begin{frame}
%    \frametitle{量子擦除实验}
%    \begin{center}
%        \includegraphics[width=0.6\textwidth]{figs/c1.%png} \\
%    \end{center} 
%\end{frame}
%
%\begin{frame}
%    \frametitle{}
%    \begin{center}
%        \includegraphics[width=0.6\textwidth]{figs/c2.%png} \\
%    \end{center} 
%\end{frame}
%
%\begin{frame}
%    \frametitle{}
%    \begin{center}
%        \includegraphics[width=0.6\textwidth]{figs/c3.%png} \\
%    \end{center} 
%\end{frame}

%%%%%%%%%%%%%%%%%%%%%%%%%%%%%%%%%%%%%%%%%%%%%%%%%%%%%%%%%%%%%%%%%%%
\begin{frame}
    \frametitle{课外作业}
    \begin{enumerate}
        \item 补全例1的计算
        \item 补全例2的计算
        \item 了解非线性光学
        \item 量子擦除实验说明什么?
    \end{enumerate}
\end{frame}
%%%%%%%%%%%%%%%%%%%%%%%%%%%%%%%%%%%%%%%%%%%%%%%%%%%%%%%%%%%%%%%%%%%

       % 半经典光
%
\begin{frame}
    \frametitle{前情回顾}
    学习量子光学的必要性
    \begin{itemize}
        \Item 经典光学(麦克斯韦方程) \\
        <-> 黑体辐射, 光电效应, 康普顿效应, 原子光谱, 自发发射, 受激发射...
        \Item 半经典光学(量子化粒子+经典光场)  \\
        <-> 延迟选择实验, 量子擦除实验, 相干态, 压缩态, 量子计算, 量子通信, 量子存储...
        \Item 量子光学(量子化粒子+量子化光场)   
    \end{itemize}     
\end{frame}

\begin{frame}
    \frametitle{场与粒子}
    物理学基本认识:
    \begin{itemize}
        \item 场是物质存在的基本形式
        \item 所有的粒子都是场的量子
        \item 场量子与量子力学的粒子并不完全一样,在非相对论近似下两者可拟合在一起
        \item 光量子是光场的量子,无非相对论近似,不是量子力学中的粒子
    \end{itemize} 
\end{frame}

%%%%%%%%%%%%%%%%%%%%%%%%%%%%%%%%%%%%%%%%%%%%%%%%%%%%%%%%%%%%%
\begin{frame} [plain]
    \frametitle{}
    \Background[1] 
    \begin{center}
    {\huge 第2-3讲:量子力学基础}
    \end{center}  
    \addtocounter{framenumber}{-1}   
\end{frame}
%%%%%%%%%%%%%%%%%%%%%%%%%%%%%%%%%%%%%%%%%%%%%%%%%%%%%%%%%5%%%

\section{1.量子态与希尔伯特空间}

\begin{frame} 
    \frametitle{希尔伯特空间}
    量子态用希尔伯特空间的矢量描述\\
    \begin{equation*}
        \begin{split}
            \text{1、定义加法} \quad  &\xi=\psi+\varphi\\
            &\psi+\varphi=\varphi+\psi \qquad (\text{交换律})\\
            &(\psi+\varphi)+\xi=\psi+(\varphi+\xi) \qquad (\text{结合律})\\
            &\psi+\text{O}= \psi \qquad (\text{零元})\\
            &\psi+\varphi= \text{O} \qquad (\text{逆元})\\
        \end{split}  
    \end{equation*}
\end{frame} 

\begin{frame} 
    \begin{equation*}
        \begin{split}
            \text{2、定义数乘} \quad &\varphi=\psi a\\
            &\psi 1= \psi \qquad (\text{1元})\\
            &(\psi a)b=\psi (ab) \qquad (\text{结合律})\\
            &\psi(a+b)= \psi a+ \psi b \qquad (\text{第一分配律})\\
            &(\psi+\varphi) a= \psi a +\varphi a \qquad (\text{第二分配律})\\
        \end{split}  
    \end{equation*}
\end{frame} 

\begin{frame} 
    \begin{equation*}
        \begin{split}
            \text{3、定义内积} \quad &c=(\psi, \varphi)\\
            &(\psi, \varphi)= (\varphi,\psi)^* \\
            &(\psi, \varphi+\xi)= (\psi, \varphi) + (\psi, \xi)\qquad (\text{分配律})\\
            &(\psi, \varphi a)= (\psi, \varphi )a \\
            &(\psi a, \varphi )= a^* (\psi, \varphi ) \\
            &(\psi,\psi)= c\ge 0\\
        \end{split}  
    \end{equation*}
\end{frame}

\begin{frame} 
    \例 [0. 有定义在$C^n$空间的列矩阵,求内积]
    { \[\psi=
        \begin{pmatrix}
                a_1\\
                a_2\\
                a_3
        \end{pmatrix}, \qquad 
        \varphi =\begin{pmatrix}
            b_1\\
            b_2\\
            b_3
    \end{pmatrix}
     \] 
    }
    \解 ~ \[(\psi, \varphi) = \begin{pmatrix}
        a_1 ^* &
        a_2 ^* &
        a_3 ^*
    \end{pmatrix}
        \begin{pmatrix}
        b_1\\
        b_2\\
        b_3
    \end{pmatrix}
    =a_1 ^* b_1 +a_2 ^* b_2 +a_3 ^* b_3
    =c 
    \]
    ~ \[(\varphi,\psi) = \begin{pmatrix}
        b_1 ^* &
        b_2 ^* &
        b_3 ^*
    \end{pmatrix}
        \begin{pmatrix}
        a_1\\
        a_2\\
        a_3
    \end{pmatrix}
    =b_1 ^* a_1 +b_2 ^* a_2 +b_3 ^* a_3
    =c^* 
    \]
\end{frame} 

\begin{frame} 
    \例 [1. 求定义在x空间的函数的内积]{}

    \解 ~ \[(\psi, \varphi)=\int_a ^b \psi^*(x)  \varphi(x) dx =c\]
    \[(\varphi,\psi)=\int_a ^b \varphi^*(x)\psi(x) dx = (\int_a ^b \varphi(x)\psi^*(x) dx) ^* =c^*\]
\end{frame} 

\begin{frame}
    4、定义空间\\
   \begin{itemize}
       \Item 矢量空间:满足加法和数乘两种运算的集合
       \Item 内积空间:满足加法、数乘和内积三种运算的集合
       \Item 希尔伯特空间:  完全的内积空间\\
       ~~ \\
       *完全性:对给定任意小的实数$\varepsilon$,总有数N存在,当m, n>N时,有\\
       $$ (\psi_m -\psi_n, \psi_m -\psi_n )< \varepsilon $$
   \end{itemize} 
   \Tips ~ 量子体系的状态用希尔伯特空间的矢量描述
\end{frame} 

\begin{frame}
    5、几个概念\\
   \begin{itemize}
       \Item 模(方):$|\psi|^2= (\psi, \psi)=c$
       \Item 归一化: $|\psi|^2= (\psi, \psi)=c=1$
       \Item 正交性(线性无关):  $(\psi, \varphi)=0 $ \\
       \Item 完全集: 有一组线性无关集,如果空间的任意矢量都可以在其上展开,则称它为一个完全集,记为$\{\phi_i\}$
       \[\psi=\sum_i a_i \phi_i= \sum_i (\phi_i,\psi) \phi_i\]
       \Item 维度:最小完全集所包含矢量的数目相同,称这个数目为空间的维度
       \Item 正交归一完全集:对于一个n维的完全集,有:\[(\phi_i,\phi_j)=\delta_{ij}, \qquad i,j=1,2,3,\cdots, n \]
       \Item 基与基矢:称一个正交归一完全集为空间的一个基,它所含的矢量称不计算基矢
   \end{itemize} 
\end{frame} 

\begin{frame}
 \Tips~ 同一空间可以有不同的基,\\
 $C^2$空间的一个基:
 \[ \rs{0}\equiv\begin{bmatrix}
     1 \\
     0
 \end{bmatrix}; \qquad \rs{1}\equiv\begin{bmatrix}
    0 \\
    1
\end{bmatrix} \]

$C^2$空间的另一个基:
\[ \rs{+}\equiv\frac{1}{\sqrt{2}}\begin{bmatrix}
    1 \\
    1
\end{bmatrix}=\Pstate; \qquad \rs{-}\equiv\frac{1}{\sqrt{2}}\begin{bmatrix}
   1 \\
   -1
\end{bmatrix}=\Mstate \]
它们可以相互转换.
\end{frame} 

\begin{frame} 
    例: 若$\rs{0}, \rs{1}$描述垂直和水平偏振, 则 $\rs{+}, \rs{-}$ 描述正负45度偏振. 
    \begin{center}
        \begin{overpic} [width=0.7\textwidth]{figs/26.png}
            %\includegraphics[width=0.7\textwidth]{figs/26.png}
            \put(72,48){\small\bfseries \color{red}{$|1\rangle$}}
            \put(72,-1){\small\bfseries \color{red}{$|0\rangle$}}
        \end{overpic}   
    \end{center} 
    若$\rs{0}, \rs{1}$描述在Z方向自旋投影, 则 $\rs{+}, \rs{-}$ 描述在X方向的自旋投影. 
\end{frame}

\begin{frame}{}
    6、左矢与右矢\\
    考察内积: $(\psi,\psi)=\int\psi^*\psi d\tau$ \\
    同一波函数放在左边还是右边,意义有所不同: \\
    右边是线性的:  $(\psi,a\psi)=(\psi,\psi)a $ \\
    左边是反线性的:   $(a\psi,\psi)=a^* (\psi,\psi)$  \\
    为清楚描述线性反线性特点,定义左矢和右矢, 写狄拉克记号
    $$\langle \psi |, \qquad |\psi \rangle $$ 
    数乘性质: $$\langle a\psi | = \langle \psi |a^* $$
    $$ |a\psi \rangle = a|\psi \rangle$$ 
    内积:\[(\psi,\varphi)\equiv \langle \psi | \varphi \rangle\]
\end{frame}

\begin{frame}{}
    7、外积\\
    考察展开式: \[\psi=\sum_i a_i \phi_i= \sum_i (\phi_i,\psi) \phi_i\]
    \[\rs{\psi}=\sum_i a_i \rs{\phi_i}= \sum_i \lr{\phi_i}{\psi} \rs{\phi_i} =\sum_i \rl{\phi_i}{\phi_i} \rs{\psi}\]
    ~~\\
    令 $p_i= \rl{\phi_i}{\phi_i} $, 称为外积\\ {\vspace*{1em}}
    完备性:
    \[\sum_i p_i= \sum_i \rl{\phi_i}{\phi_i}=1 \]
\end{frame}

\begin{frame}{}
        \例 [2. 有定义在$C^n$空间的列矩阵,求内积和外积]
        { \[\rs{\psi}=
            \begin{pmatrix}
                    a_1\\
                    a_2\\
                    a_3
            \end{pmatrix}, \qquad 
            \rs{\varphi} =\begin{pmatrix}
                b_1\\
                b_2\\
                b_3
        \end{pmatrix}
         \] 
        }
        \解 ~ \[\lr{\psi}{\varphi} = \begin{pmatrix}
            a_1 ^* &
            a_2 ^* &
            a_3 ^*
        \end{pmatrix}
            \begin{pmatrix}
            b_1\\
            b_2\\
            b_3
        \end{pmatrix}
        =a_1 ^* b_1 +a_2 ^* b_2 +a_3 ^* b_3
        =c 
        \]
        ~ \[\rl{\psi}{\varphi} = \begin{pmatrix}
            a_1  \\
            a_2  \\
            a_3 
        \end{pmatrix}
            \begin{pmatrix}
            b_1 ^* &
            b_2 ^* &
            b_3 ^*
        \end{pmatrix}
        =  \begin{pmatrix}
            a_1b_1 ^* & a_1b_2 ^* & a_1b_3 ^* \\
            a_2b_1 ^* & a_2b_2 ^* & a_2b_3 ^* \\
            a_3b_1 ^* & a_3b_2 ^* & a_3b_3 ^* 
        \end{pmatrix}
        \]
\end{frame}

\begin{frame} 
\frametitle{}
 \例 [3. 求如下波函数的外积]{
 \[\rs{\psi}= a_1 \rs{1} + a_2 \rs{2}, \quad \rs{\varphi}= b_1 \rs{1} + b_2 \rs{2} \]}
 \解 ~(1) 代数形式 
   \[ \begin{aligned}
    \rl{\psi}{\varphi} &= (a_1 \rs{1} + a_2 \rs{2}) (\ls{1}b_1 ^*  + \ls{2} b_2 ^* )\\
      &= a_1 b_1 ^* \rl{1}{1} + a_2 b_2 ^* \rl{2}{2} + a_1 b_2 ^* \rl{1}{2} + a_2 b_1 ^* \rl{2}{1} \\
   \end{aligned}\]   
* 后两项为耦合项\\ 
(2) 矩阵形式
~ \[\rl{\psi}{\varphi} = \begin{pmatrix}
    a_1  \\
    a_2  
\end{pmatrix}
    \begin{pmatrix}
    b_1 ^* &
    b_2 ^* &
\end{pmatrix}
=  \begin{pmatrix}
    a_1b_1 ^* & a_1b_2 ^*  \\
    a_2b_1 ^* & a_2b_2 ^*  
\end{pmatrix}
\]
* 非对称元为耦合项
\end{frame}

\section{2.物理量与算符}

\begin{frame}
    \frametitle{算符}
    物理量用希尔伯特空间的线性厄密算符描述\\
    1. 定义:
    \begin{itemize}
        \Item 算符:描述态矢量之间的映射关系,算符作用于一个态映射到另一态。
        \[F \rs{\Psi}=\rs{\psi}\]
        \Item 算符相等: 对任意态$\rs{\psi}$, 恒有下式, 则 $A= B$ \[ A\rs{\psi}=B\rs{\psi}\]
        \Item 单位算符: \[I\Psi=\Psi \]
        \Item 算符的和:  
        $$ (A+B)\Psi=A\Psi+B\Psi $$   
    \end{itemize}
\end{frame} 

\begin{frame}
 \frametitle{}
 \begin{itemize}
    \Item 算符的积 
    $$ (AB)\Psi=A(B\Psi) $$
    * 不存在交换律,即 $AB=BA$ 或 $AB\ne BA$ 皆有可能 \\
    定义对易子来描述 \\
    $$ [A,B]=AB-BA$$
    若[A,B]=0,称两算符对易(可交换),否则不对易(不可交换)
    \Item 逆算符  
    \[F^{-1}\rs{\psi}=\rs{\Psi} \] 
    \Item 线性算符 \[F (a\rs{\Psi} +b\rs{\psi}) = aF \rs{\Psi} +bF\rs{\psi}\]
\end{itemize}
\end{frame}

\begin{frame}
    \frametitle{}
    \begin{itemize}
        \Item 伴算符   \[\ls{\psi}=\ls{\Psi}F^{\dagger} \]
        \[F^{\dagger} = (F^* _{nm})^T \]
        \Item 自伴(厄密)算符  \[F = F^{\dagger} \] 
        性质(1): $\lr{\Psi F}{\psi}=\lr{\Psi }{F \psi}=\lcr{\Psi }{F} {\psi}$ \\
        性质(2): $\lcr{\Psi }{F} {\psi}=\lcr{\psi }{F} {\Psi}^*$ \\ \vspace{0.6em}
        \Item 幺正(酉)算符    \[F^{-1} = F^{\dagger} \] 性质: $FF^{\dagger}=F^{\dagger}F=I$, 通常写成 $UU^{\dagger}=U^{\dagger}U=I$ \\ \vspace{1.0em}
    \end{itemize}        
\end{frame}

\begin{frame}
    \frametitle{}
    \Tips 幺正(酉)变换实现空间变换: 把一个空间所有矢量都用同一幺正算符作用,得到的另一个新的空间,
    \[U\rs{\Psi} =\rs{\Psi'}\]
    {\Bullet}~新旧空间算符之间的关系: \\
        旧空间的算符: \[F \rs{\Psi}=\rs{\varphi}\]
        新空间的算符: \[F' \rs{\Psi'}=\rs{\varphi'}\]
        关系:
        \[F' U\rs{\Psi}=U\rs{\varphi}\]
        \[F' U\rs{\Psi}=UF \rs{\Psi}\]
        \[U^{\dagger}F' U\rs{\Psi}=U^{\dagger}UF \rs{\Psi}\]
        \[U^{\dagger}F' U=F \]
\end{frame}

\begin{frame}
    \frametitle{}
    \begin{itemize}
    \Item 投影算符: 基矢的外积是一种投影算符,
    \end{itemize}
    对于展开式: 
    \[\rs{\psi}=\sum_i a_i \rs{\phi_i}=\sum_i \rs{\phi_i} \lr{\phi_i}{\psi}=\sum_i p_i \rs{\psi} =\sum_i \rs{\psi_i}\]
    有:\[p_i \rs{\psi} =\rs{\psi_i}\]
    即,外积$\rl{\phi_i}{\phi_i}$作用于$\rs{\psi}$,得到第$i$个基矢态上的投影分量!\\
\end{frame}

\begin{frame}    
    \begin{itemize}
        \Item 测量算子: 常称投影算符为测量算子. 对于$C^2$空间, 定义为:
        \end{itemize}
    \[M_0=\rl{0}{0}, \qquad M_1=\rl{1}{1} \]
    具有$2\times 2$的矩阵形式,
    可以证明:\\
    {\bullet} 测量算子是自伴(厄密)算符 :\[M_m = M_m ^{\dagger} \]
    {\bullet} 平方不变性 :\[M_m ^2 = M_m \]
    {\bullet} 完备性 :\[M_0 + M_1 = M_0 ^2 + M_1 ^2 = M_0 M_0 ^\dagger + M_1 M_0 ^\dagger=I\]
\end{frame}

\begin{frame} 
    \frametitle{}
    {\bullet} 测量后的态函数 
    \[\begin{aligned}
        M_0\rs{\Psi} 
        &= \rl{0}{0}(a_0\rs{0}+a_1\rs{1})  \\ 
        &= \rl{0}{0}a_0\rs{0}  \\ 
        &= a_0\rs{0}  \\ 
        &= \frac{a_0}{|a_0|}\rs{0}  \qquad \text{(归一化)} \\ 
    \end{aligned}\]    
    {\bullet} 测得的概率(密度) 
    \[\begin{aligned}
        \lcr{\Psi}{M_0 ^\dagger M_0}{\Psi} 
        &= \lcr{0 }{a_0 ^* a_0} {0} \\ 
        &= \lr{0 }{0}a_0 ^* a_0 \\ 
        &= |a_0|^2 = p(0) 
    \end{aligned}\] 
    重写测量后的状态 \[  M_m\rs{\Psi} = \frac{a_m\rs{m}}{\sqrt{\lcr{\Psi}{M_m ^\dagger M_m}{\Psi}}} = \frac{M_m\rs{\Psi}}{\sqrt{\lcr{\Psi}{M_m ^\dagger M_m}{\Psi}}}\]
\end{frame}

\begin{frame}
    \frametitle{}
    ~\\
    2. 算符的代数形式 \\ {\vspace*{0.3em}}
 {\Bullet}位置算符和动量算符
\例[4.已知粒子的位置波函数$\psi(x,t)$,求动量的期望值]{}   
\解~ 由概率诠释,位置期望值为
\begin{equation*}
    \bar{x}=\int x|\psi(x, t)|^{2} d x=\int \psi^{*}(x, t) x \psi(x, t) d x
\end{equation*}
对于动量波函数 $c(p,t)$, 动量期望值为
\begin{equation*}
    \bar{p_x}=\int p_x|c(p_x, t)|^{2} d p_x=\int c^{*}(p_x, t) p c(p_x, t) d p_x
\end{equation*}
很明显,有
\begin{equation*}
    \bar{p_x}\neq\int p_x|\psi(x, t)|^{2} d p_x
\end{equation*}
\end{frame} 

\begin{frame}
变换求解
\begin{equation*}
    \begin{split}
        \bar{p}&=\int c^{*}(p) p c(p) d p \\  
        &=\int (\frac{1}{\sqrt{2 \pi \hbar}} \int \psi^{*}(x) e^{\frac{i}{\hbar} p\cdot x} d x) p c\left(p\right) d p \\
        &=\frac{1}{\sqrt{2 \pi \hbar}} \int \int \psi^{*}(x) (e^{\frac{i}{\hbar} p\cdot x}  p) c\left(p\right) d xd p \\
        &=\frac{1}{\sqrt{2 \pi \hbar}} \int \int \psi^{*}(x) \Myitem{t1}{red}{(-i\hbar\frac{d}{d x} e^{\frac{i}{\hbar} p\cdot x})} c(p) d xd p \\
        &=\int \psi^{*}(x) (-i\hbar\frac{d}{d x}) (\frac{1}{\sqrt{2 \pi \hbar}} \int e^{\frac{i}{\hbar} p\cdot x} c(p) d p)  d x\\
        &=\int \psi^{*}(x) (-i\hbar\frac{d}{d x}) \psi(x)  d x\\
     \end{split}
\end{equation*}  
\end{frame} 

\begin{frame}
定义如下计算符号:
$$ \boxed{\hat{p}_x= -i\hbar\frac{d}{d x}} $$ 
上式变为:         
$$\boxed{\bar{p}_x=\int \psi^{*}(x) \hat{p}_x \psi(x) d x} $$
称$ \hat{p}_x= -i\hbar\dfrac{d}{d x} $ 为位置表象里的动量算符表示($p_x$分量)\\
同理,称 $\hat{x}= x $ 为位置表象里的位置算符表示($x$分量)\\ \vspace{0.3em}
存在量子力学基本对易关系 \[ [\hat{x},\hat{p}_x]=i\hbar\] 
\end{frame}

\begin{frame} 
    \frametitle{}
    {\Bullet} 任意力学量的算符
    \begin{tcolorbox1}{命题:}
    已知位置、动量的算符表示如下,
    \begin{itemize}
        \Item  位置算符(三维): $ \hat{\vec{r}} =\vec{r} $
        \Item  动量算符(三维): $ \hat{\vec{p}} =-i\hbar(\dfrac{d}{d x}, \dfrac{d}{d y} , \dfrac{d}{d z})=-i\hbar \nabla $
    \end{itemize}
    求任意其他力学量的算符
    \end{tcolorbox1}
    \alert{Bohm规则(1):} 经典物理学的力学量F,若是位置与动量的函数
    \[F(\vec{r},\vec{p})\]
    则其量子力学算符为:
    \[\hat{F}=F(\hat{\vec{r}},\hat{\vec{p}})\]
\end{frame} 

\begin{frame}
    \frametitle{}     
{\Bullet}正则位置和正则动量
  \[ \frac{\mathrm{d }q}{\mathrm{d }t} = \frac{\partial H(q,p) }{\partial p}, \quad \frac{\mathrm{d }p}{\mathrm{d }t} = -\frac{ \partial H(q,p) }{\partial q} \]
  正则位置和正则动量算符的对易关系
  \[ [\hat{q},\hat{p}]=i\hbar\]
  \alert{Bohm规则(2):} 经典物理学存在力学量F,是正则位置与动量的函数
  \[F(q,p)\]
  则其量子力学算符为:
  \[\hat{F}=F(\hat{q},\hat{p})\]
  量子涨落关系(不确定性原理) \[\overline{(\Delta \hat{q})^2} \cdot \overline{(\Delta \hat{p})^2} \geq  \frac{1}{4} \hbar ^2 \]
\end{frame} 

\begin{frame}
    \frametitle{}
    3. 算符的矩阵形式
    \[\begin{aligned}
        \rs{\psi}&=F \rs{\Psi} \\
        \lr{i}{\psi}&= \lcr{i}{F}{\Psi} \\
        \lr{i}{\psi}&= \sum_j\lcr{i}{F}{j}\lr{j}{\Psi} \\
        \lr{i}{\psi}&= \sum_j F_{ij}\lr{j}{\Psi}
    \end{aligned}\]  
    算符矩阵元公式:\[ F_{ij}=\lcr{i}{F}{j}\]
    伴算符的矩阵等于原算符的厄密共轭:\[ F^{\dagger}=(F_{ij} ^*)^T\]
\end{frame}

\begin{frame}
    \frametitle{}
    4. 算符的本征方程
    \begin{itemize}
        \Item 定义式: \[F \rs{n}=f_n\rs{n}\]
        \Item 相关定理:
        \begin{itemize}
            \IItem 厄密算符的本征值是实数
            \IItem 厄密算符的所有本征矢构成正交归一完全集
            \IItem 当且仅当两厄密算符互相对易时才俱有共同的本征矢完全集
            \IItem 完全确定一个量子态所需要的彼此对易的一组力学量算符的最小集称为力学量完全集,所含力学量数目与体系的自由度数目相同
            \end{itemize}
    \end{itemize}
\end{frame}

\section{3.狄拉克记号量子力学}

\begin{frame} 
    \frametitle{狄拉克记号的量子力学}  
    量子态(无表象): $\hspace{1em}|\Psi \rangle, \quad$ 位置表象:$\hspace{1em} \langle x |\Psi \rangle , \quad$ 动量表象:$\hspace{1em} \langle p_x |\Psi \rangle$ \\ \vspace{0.1em}
    算符: $F, \quad$ 位置表象:$\hspace{1em} F(x, -i\hbar \frac{\partial }{\partial x}) , \quad$ 动量表象:$\hspace{1em} F(i\hbar \frac{\partial }{\partial p_x}, p_x) $ \\ \vspace{0.1em}
    展开式: $\hspace{1em}|\Psi \rangle =\sum\limits_{n=1} ^n a_n |n \rangle$ \\
    内积:   $\hspace{2em}\langle \varphi | \Psi \rangle = (\varphi, \Psi)= \int \varphi^*\Psi d\tau $ \\  \vspace{0.1em}
    归一化: $\hspace{1em}\langle \Psi | \Psi \rangle = (\Psi, \Psi)= \int \Psi^*\Psi d\tau = 1 $ \\ \vspace{0.1em}
    正交归一: $\langle n | m \rangle = \delta_{nm} $ \\ \vspace{0.1em}
    $ \hspace{5em} \langle \lambda | \lambda' \rangle = \delta(\lambda-\lambda') $\\ \vspace{0.2em}
    展开系数: $ a_n= \langle n | \Psi \rangle$ \\ \vspace{0.2em}
    展开系数: $ a_n ^*= \langle \Psi | n \rangle$ \\ \vspace{0.2em}
\end{frame} 
 
\begin{frame} 
    平均值:  $\hspace{1em}\bar{F} = \langle \Psi |F | \Psi \rangle$ \\ \vspace{0.2em}
    矩阵元:  $\hspace{1em}F_{nm} = \langle n |F | m \rangle$ \\ \vspace{0.2em}
    幺正变换:$S_{m\alpha} =\langle m| \alpha \rangle $ \\ \vspace{0.2em}
    投影算符:$p_i = |i\rangle\langle i |, \quad \text{封闭性:} \sum_i |i\rangle\langle i |=1 $ \\ \vspace{0.2em}
    本征方程:$F|n\rangle =f_n |n\rangle$ \\ \vspace{0.2em}
    薛定谔方程:$$ i\hbar \frac{\partial }{\partial t} |\Psi(t)\rangle = H|\Psi(t)\rangle $$ 
    算符运动方程:$$ \frac{d\bar{A}(t)}{dt}=\overline{(\frac{\partial A(t) }{\partial t})}  +\frac{1}{i\hbar} \overline{[A(t),H(t)]}$$
\end{frame} 

\begin{frame} 
    \例 [求薛定谔方程在各表象中的形式]{} 
    $$ \begin{aligned}
    i \hbar \frac{\partial}{\partial t} |\Psi(t) \rangle &= H |\Psi(t) \rangle  , \quad \text{无表象}\\
    i \hbar \frac{\partial}{\partial t} \langle x|\Psi(t) \rangle &= H (x, \hat{p}_x) \langle x|\Psi(t) \rangle  , \quad \text{位置表象}\\
    i \hbar \frac{\partial}{\partial t} \langle x|\Psi(t) \rangle &= [- \frac{\hbar^2}{2\mu} \frac{\partial ^2 }{\partial x^2} + U(x)] \langle x|\Psi(t) \rangle  , \quad \text{位置表象}\\
    i \hbar \frac{\partial}{\partial t} \langle p_x|\Psi(t) \rangle &= H (\hat{x}, p_x) \langle p_x|\Psi(t) \rangle  , \quad \text{动量表象}\\
    i \hbar \frac{\partial}{\partial t} \langle p_x|\Psi(t) \rangle &=  [ \frac{p^2 _x}{2\mu} + U(i \hbar \frac{\partial }{\partial p_x}) ] \langle p_x|\Psi(t) \rangle  , \quad \text{动量表象}\\
    i \hbar \frac{\partial}{\partial t} \langle n|\Psi(t) \rangle &=  \langle n|H|m \rangle \langle n |\Psi(t) \rangle  , \quad \text{Q表象}\\
    \end{aligned}
    $$
\end{frame} 

\begin{frame} 
 \frametitle{密度算符}
    
    设系统以$P_i$的概率出现$\rs{\psi_i}$ 态上, 求物理量F的平均值:\\ 
    (1) 物理量F在$\rs{\psi_i}$的均值
    \[\overline{F_i}=\ls{\psi_i}F \rs{\psi_i}
    \]
    (2) 测得$\overline{F_i}$的概率为$P_i$,所以均值为
    \[\overline{F}= \sum_i P_i \overline{F_i}\]
\end{frame}

\begin{frame} 
 \frametitle{} ~\\
      \例 [5. 若物体处于纯态, 即其状态能用一个态矢量$\rs{\psi}$描述, 求物理$F$的平均值] {
      \[ \begin{aligned}
          \overline{F} &= \lcr{\psi}{F}{\psi} \\
          &= \sum_n \lcr{\psi}{F}{n}\lr{n}{\psi}   \\
          &= \sum_n \lr{n}{\psi} \lcr{\psi}{F}{n} \\ 
          &=  \sum_n \lcr{n}{\rho F}{n} \\
          &= Tr (\rho F)
      \end{aligned}\] }
    式中定义了纯态$\rs{\psi}$的密度算符:
    \[\rho = \rl{\psi}{\psi}  \]
\end{frame}

\begin{frame} 
 \frametitle{}
 ~\\
 \例 [5'. 若物体处于混态,则要用一系列场矢量\{$\rs{\psi_i}$\}描述. 设测量其处于$\rs{\psi_i}$的概率为$P_i$, 求物理$F$的平均值] { 
      \[ \begin{aligned}
        \overline{F} &= \sum_i P_i \overline{F_i}\\
        &= \sum_i P_i \ls{\psi_i}F \rs{\psi_i} \\ 
        &= \sum_i P_i \sum_n \lcr{\psi_i}{F}{n}\lr{n}{\psi_i} \\
        &= \sum_n\ls{n} (\sum_i P_i \rl{\psi_i}{\psi_i} )F \rs{n} \\ 
        &= \sum_n\ls{n} \rho F\rs{n} = Tr (\rho F)
      \end{aligned}\] } 
      式中定义了混态的密度算符:
      $\rho = \sum_i P_i\rl{\psi_i}{\psi_i} $ 
\end{frame}

\begin{frame} 
 \frametitle{密度算符的性质}
 用密度算符求平均值, 纯态与混态的公式相同
 \[\overline{F} = Tr (\rho F) \]
 因此,上述公式应用广泛  \\ {\vspace*{2.3em}}

 有必要了解密度算符的性质
 \end{frame}
 
 \begin{frame} 
     \frametitle{}
      (1) 厄密性 $\rho ^{\dagger} = \rho$  \\ 
    \证~ 
    \[ \begin{aligned}
        \rho ^{\dagger} &= (\sum_i P_i\rl{\psi_i}{\psi_i} )^{\dagger} \\
        &= \sum_i P_i(\rl{\psi_i}{\psi_i} )^{\dagger} \\
        &= \sum_i P_i\rl{\psi_i}{\psi_i}  \\
        &= \rho
    \end{aligned}\] 
\end{frame}

\begin{frame} 
 \frametitle{}
      (2) 归一性 Tr($\rho$)=1 \\ 
      \证~
      \[ \begin{aligned}
        Tr(\rho) &= \sum_n \sum_i \lcr{n}{P_i\rl{\psi_i}{\psi_i}}{n} \\
        &= \sum_n \sum_i P_i \lr{n}{\psi_i} \lr{\psi_i}{n}  \\
        &= \sum_n \sum_i P_i a_n^* a_n \\ 
        &= \sum_i P_i \sum_n a_n^* a_n \\ 
        &= \sum_i P_i \\ 
        &=1
      \end{aligned}\] 
      量子概率: $\omega_n= a_n^* a_n$,不可消除.\\ 
      经典概率: $P_i$,没有掌握体系所有信息造成的, 可以消除.
\end{frame}

\begin{frame} 
 \frametitle{}
      (3)  $Tr(\rho^2)\leq 1$  \\ 
      \证~
      \[ \begin{aligned}
        \rho & = \sum_i P_i\rl{\psi_i}{\psi_i} \\ 
        \rho^2 & = \sum_i\sum_j P_iP_j\rl{\psi_i}{\psi_i} \rl{\psi_j}{\psi_j} \\ 
        &= \sum_i\sum_j P_iP_j\rl{\psi_i}{\psi_j} \delta_{ij} \\ 
        &= \sum_i P_iP_i\rl{\psi_i}{\psi_i}  \\
        &=\sum_i P^2_i \rl{\psi_i}{\psi_i} 
    \end{aligned}\] 
    \end{frame}
    
\begin{frame} 
          \frametitle{}  
    \[ \begin{aligned}
        Tr(\rho^2) &= \sum_n \sum_i \lcr{n}{P^2_i\rl{\psi_i}{\psi_i}}{n} \\
        &= \sum_n \sum_i P^2_i\lr{n}{\psi_i}\lr{\psi_i}{n}  \\
        &= \sum_n \sum_i P^2_i a_n^* a_n \\ 
        &= \sum_i P^2_i \sum_n a_n^* a_n \\ 
        &= \sum_i P^2_i \\ 
        &\leq \sum_i P_i \\ 
        &=1
      \end{aligned}\]
    纯态: $Tr(\rho^2) =1$, 混态: $Tr(\rho^2) < 1$. 
\end{frame}

\begin{frame} 
    \frametitle{}
       (4)  $\left\langle \rho \right\rangle   \geq 0 $ \\ 
       \证~
    \[ \begin{aligned}
        \rho & = \sum_i P_i\rl{\psi_i}{\psi_i} \\ 
        \left\langle \rho \right\rangle &= 
        \lcr{n}{\sum_i P_i\rl{\psi_i}{\psi_i}}{n} \\ 
        &= \sum_i P_i \lr{n}{\psi_i}\lr{\psi_i}{n}  \\
        &= \sum_i P_i \left|\lr{n}{\psi_i} \right|^2 \\
        & \geq 0
      \end{aligned}\]   
   \end{frame}

\begin{frame} 
 \frametitle{}
      (5) 对于纯态, 有: $\rho^2 =\rho$ \\ 
      \证~
      \[ \begin{aligned}
        \rho &= \rl{\psi}{\psi}  \\ 
        \rho^2 &= \rl{\psi}{\psi} \rl{\psi}{\psi} \\ 
        &= \rs{\psi}\left\langle \psi | \psi \right\rangle \ls{\psi} \\ 
        &= \rl{\psi}{\psi}  \\ 
        &=  \rho 
      \end{aligned}\]
\end{frame}

\begin{frame} 
\frametitle{}
   (6) 密度算符的运动方程
   \[ \begin{aligned}
    \rho &= \sum_i P_i\rl{\psi_i}{\psi_i}  \\
     i \hbar \frac{\partial \rho}{\partial t} &= \sum_i P_i \left[i \hbar\frac{\partial \rs{\psi_i}}{\partial t}\ls{\psi_i}+ \rs{\psi_i}i \hbar\frac{\partial \ls{\psi_i}}{\partial t}\right] \\ 
     &=  \sum_i P_i \left[ H \rl{\psi_i}{\psi_i} -  \rl{\psi_i}{\psi_i} H \right] \\ 
     &=   \left[ H \sum_i P_i\rl{\psi_i}{\psi_i} - \sum_i P_i \rl{\psi_i}{\psi_i} H \right] \\ 
     &= [H \rho- \rho H] \\ 
     &= [H, \rho] \\ 
     ~\\ 
     i \hbar \frac{\partial \rho}{\partial t} & = [\rho, H]
   \end{aligned}\] 
\end{frame}

\begin{frame} 
\frametitle{}
~\\
    (7) 密度算符的矩阵形式
    \[ \begin{aligned}
       \rs{\Psi} & = a_1\rs{1} + a_2\rs{2} =  \begin{pmatrix}
        a_1  \\
        a_2 
    \end{pmatrix} \\
       \rho & = \rl{\Psi}{\Psi}  = \sum_{i,j=1,2} a_i a_j ^* \rl{i} {j}\\
       \rho & = \begin{pmatrix}
        a_1  \\
        a_2 
    \end{pmatrix} \begin{pmatrix}
        a^* _1  &  a^* _2
    \end{pmatrix}\\
    &= \begin{pmatrix}
        a_1a_1 ^* & a_1a_2 ^*  \\
        a_2a_1 ^* & a_2a_2 ^*  
    \end{pmatrix} \\ 
    &= \begin{pmatrix}
        \rho_{11} & \rho_{12}  \\
        \rho_{21} & \rho_{22}  
    \end{pmatrix} \\ 
    \end{aligned}\]     
\end{frame}

\section{4.多粒子体系与张量空间}

\begin{frame}
    \frametitle{张量空间}
    \begin{tcolorbox4}[张量积]
    对于多粒子体系,比如多光子系统,其所处的空间是子系统希尔伯特空间的张量积。也称直积空间。
    \end{tcolorbox4}
\end{frame}

\begin{frame}
    \frametitle{~1. 张量空间的计算基矢}
    子系统A是n维的,计算基(某厄密算符的本征函数系)为$$\{\rs{\phi_i}\},\quad (i=1,2,3,\cdots,n)$$ 
    子系统B是m维的,计算基为$$\{\rs{\varphi_j}\},\quad (j=1,2,3,\cdots,m)$$ 
    总系统是n张m维的张量空间,计算基为:$$\{\rs{\phi_i}\otimes\rs{\varphi_j}\},\quad (i=1,2,3,\cdots,n;\quad j=1,2,3,\cdots,m)$$ 
    可简写为:$$\{\rs{\phi_i}\otimes\rs{\varphi_j}\}=\{\rs{\phi_i}\rs{\varphi_j}\}=\{\rs{\phi_i\varphi_j}\}=\{\rs{ij}\}$$
\end{frame}

\begin{frame}
    \frametitle{}
    总体系的任意态是计算基矢的叠加态:
    \[ \rs{\Psi} = \sum_{i,j} ^{n,m} a_{ij}\rs{ij}\] \vspace{0.6em}

    \例[7. 写出$C^2$空间的张量空间的计算基]{} 
    \解~基由四个基矢构成$$\{\rs{\phi_i}\otimes\rs{\varphi_j}\}=\{\rs{00},\rs{01},\rs{10},\rs{11}\}$$

矩阵表示是原矩阵的直积,例: 
\[\rs{01} = \rs{0} \otimes \rs{1} =    
\begin{pmatrix}
    1\\
    0
\end{pmatrix}
\otimes
\begin{pmatrix}
    0\\
    1
\end{pmatrix}
=
\begin{pmatrix}
    1 \otimes \begin{pmatrix}
        0\\
        1
    \end{pmatrix}\\
    0 \otimes \begin{pmatrix}
        1\\
        0
    \end{pmatrix}
\end{pmatrix}
=
\begin{pmatrix}
    0\\
    1\\
    0\\
    0
\end{pmatrix}
 \] 
\end{frame}

\begin{frame}
矩阵形式:
    \[
\rs{00} = 
\begin{pmatrix}
    1\\
    0\\
    0\\
    0
\end{pmatrix},\qquad
\rs{01} = 
\begin{pmatrix}
    0\\
    1\\
    0\\
    0
\end{pmatrix},\qquad
\rs{10} = 
\begin{pmatrix}
    0\\
    0\\
    1\\
    0
\end{pmatrix},\qquad
\rs{1} = 
\begin{pmatrix}
    0\\
    0\\
    0\\
    1
\end{pmatrix}
\] 
\end{frame}

\begin{frame}
    \frametitle{}
    双粒子态是四个计算基矢的叠加态:
    \[\rs{\psi} =\alpha_{00}\rs{00}+\alpha_{01}\rs{01}+\alpha_{10}\rs{10}+\alpha_{11}\rs{11}\]
    归一化条件:
    \[ \sum_{ij=0,1} |\alpha_{ij}|^2= 1\]
\end{frame}

\begin{frame}
      \frametitle{ 2. 张量空间的算符}
    子系统A有算符$F_A$, 子系统B有算符$F_B$,总系统可定义它们的张量积
    \[F_{AB}=F_A \otimes F_B\]
    作用于总体系的任意态时,算法为:
    \[F_A \otimes F_B \rs{\Psi} = F_A \otimes F_B \sum_{i,j} a_{ij}\rs{ij}=  \sum_{i,j} a_{ij}F_A \rs{i}\otimes F_B\rs{j}\]   
\end{frame}

\begin{frame}
    \frametitle{~3. 子系统的测量与约化密度矩阵}
    总体系任意(纯)态的密度矩阵
    \[ \rho=\rl{\Psi} {\Psi}= \sum_{i,i',j,j'} a_{i'j'}a_{ij}\rl{i'j'}{ij}\] 
    定义子体系A的约化密度矩阵(把子系统B积分丢!)
    \[ \rho(A)=\sum_{j}\lcr{j}{\rho}{j}=tr_B(\rho)\] 
    测量子体系A的物理量$F_A$的平均值为:
    \[ \bar{F}_A=tr_A(F_A\rho(A))\] 
    
\end{frame}

\section{5.量子力学基本假设}

\begin{frame}
    \frametitle{状态假设}
    \begin{tcolorbox4}[1. 状态假设]
    量子体系的状态用希尔伯特空间的态矢量完全描述。
    \end{tcolorbox4}
    \例[8. 两能级系统用量子比特描述, 它就是2维希尔伯特空间的态矢量]{ 
    \[\rs{\psi} = a_1 \rs{1} + a_2\rs{2}\]
    完全描述两能级系统的所有可能状态}
\end{frame}

\begin{frame}
    \frametitle{演化假设}
    \begin{tcolorbox4}[2. 演化假设]
    一个封闭量子体系的演化用幺正(酉)变换描述。
    \[\rs{\Psi'}=U\rs{\Psi}\]
    状态函数随时间的演化用薛定谔方程描述
    \[ i\hbar \frac{d\rs{\Psi}}{dt}=H \rs{\Psi}\]
    \end{tcolorbox4}
\end{frame}

\begin{frame}
    \例[9. 试证明薛定谔方程与酉变换等价]{}
    \证~ 定义时间演化算符:
    $$ U(t,t_0) |\Psi(t_0)\rangle = |\Psi(t)\rangle  $$
    \alert{分析}:
    (1) 因为 $ U(t_0,t_0) |\Psi(t_0)\rangle = |\Psi(t_0)\rangle  $ \\
    $$ U(t_0,t_0)=I $$
    (2):求 $ U(t,t_0)$
    $$ \begin{aligned}
        i\hbar \frac{\partial }{\partial t} |\Psi(t)\rangle &= H|\Psi(t)\rangle  \\
        i\hbar \frac{\partial }{\partial t}  U(t,t_0) |\Psi(t)\rangle &= H U(t,t_0) |\Psi(t)\rangle  \\
        i\hbar \frac{\partial }{\partial t}  U(t,t_0)  &= H U(t,t_0)  \\
        U(t,t_0)  &= e^{-\frac{i}{\hbar} H(t-t_0)}  \\
    \end{aligned} $$
\end{frame}

\begin{frame}  
    (3):$ U(t,t_0)$是幺正算符
    $$ \begin{aligned}
        U(t,t_0)  &= e^{-\frac{i}{\hbar} H(t-t_0)}  \\
        U^\dagger (t,t_0)  &= e^{\frac{i}{\hbar} H(t-t_0)}  \\
        U^\dagger (t,t_0)U(t,t_0) &= U^\dagger (t,t_0)U(t,t_0) \\
         &=e^{\frac{i}{\hbar} H(t-t_0)-\frac{i}{\hbar} H(t-t_0)} \\
         &=e^0 \\
         &=I
    \end{aligned} $$
    因此,有:
    $$ |\Psi(t)\rangle = |\Psi(t_0)\rangle e^{-\frac{i}{\hbar}H(t-t_0)}   $$
    \Note ~波函数随时间的演化服从的薛定谔方程,只是一种幺正变换。
\end{frame} 

\begin{frame}
 \frametitle{}
    试证明算符的演化方程与薛定谔方程等价 \\ \vspace*{2.3em}
    \[ i\hbar\frac{\mathrm{d A}}{\mathrm{d t}} =  [A, H ] \]
    ~\\ \vspace*{2.3em}
    *参考密度算符运动方程的推导
\end{frame}


\begin{frame}
    \frametitle{量子测量假设}
    \begin{tcolorbox4}[3. 量子测量假设]
    量子测量由一组测量算子$\{ M_m\}$ 描述,测得测量值$m$的概率(密度)为
    \[ p(m)=\lcr{\Psi}{M_m ^\dagger M_m}{\Psi} 
     \]
     测量后的状态 \[\frac{M_m\rs{\Psi}}{\sqrt{\lcr{\Psi}{M_m ^\dagger M_m}{\Psi}}}\]
    \end{tcolorbox4}
\end{frame}

\begin{frame}
    \frametitle{张量积假设}
    \begin{tcolorbox4}[4. 复合系统假设]
    复合系统的状态空间是子系统的状态空间的张量积
    \end{tcolorbox4}
\end{frame}

\begin{frame} 
    \frametitle{}
    ~ \\ 
    {\Bullet}光与两能级原子相互作用体系的两种描述方法 \\ \vspace*{0.6em}
    (1) 光场处于数态 $\rs{n}$,原子处于激发态 $\rs{2}$, 体系的状态为
    \[ \rs{\psi_1}  = \rs{n} \rs{2}\]
    原子放出一个光子后, 回到基态 $\rs{1}$, 光场则处于数态 $\rs{n+1}$, 体系的状态为 
    \[\rs{\psi_2} = \rs{n+1} \rs{1}\]
    体系的所有可能状态用叠加态描述 
    \[\rs{\psi} = a_1 \rs{\psi_1} + a_2\rs{\psi_2}\]
    \end{frame}
    
    \begin{frame} 
        \frametitle{}
        ~ \\ 
        (2) 光场处于数态 $\rs{n}$,原子处于基态 $\rs{1}$, 体系的状态
        \[ \rs{\psi_1}  = \rs{n} \rs{1}\]
        原子吸收一个光子后, 进行激发态 $\rs{2}$, 光场则处于数态 $\rs{n-1}$, 体系的状态为 
        \[\rs{\psi_2} = \rs{n-1} \rs{2}\]
        体系的所有可能状态用叠加态描述 
        \[\rs{\psi} = a_1 \rs{\psi_1} + a_2\rs{\psi_2}\]
    \end{frame}

\begin{frame}
    \frametitle{}
    基于以上4条假设,可以在希尔伯特空间推导整个量子力学(光学)! \\ 

\end{frame}

\section{6.实例}

\begin{frame}
    \frametitle{实例}
    \例 [10. 求解一维谐振子]{}
    \解 (1) 牛顿力学条件下求解: 谐振子在平衡位置附近的振动满足波动方程:
    \begin{equation*}
        u_{tt}=a^2(u_{xx}+ u_{yy}+u_{zz})
    \end{equation*}
    零边界条件求解(一维, $l$为弦长)
    \begin{enumerate}
        \IItem 固有值:$\displaystyle  \lambda~_n=\frac{n^2\pi~^2}{l~^2}$ 
        \IItem 固有解:{$\displaystyle  X~_n(x)=\sin \frac{n\pi~}{l} x=\sin \omega_n x $}
        \IItem 基本解:
        $\displaystyle u_n(x,t) = (a_n\cos\frac{ n\pi at}{l}+ b_n\sin \frac{ n\pi at}{l}) \sin \frac{ n\pi x}{l} $ 
        \IItem 叠加解:
        \[u(x,t) = \sum\limits_{n=1}^{\infty }  (a_n\cos\frac{ n\pi at}{l}+ b_n\sin \frac{ n\pi at}{l}) \sin \frac{ n\pi x}{l}\]
    \end{enumerate}
\end{frame}

\begin{frame}
    \frametitle{}
    \解 (2) 量子力学条件下求解: 先写经典哈密顿量
      \begin{equation*}
          H=T+U = \frac{p^2 }{2m} + \dfrac{1}{2} m \omega ^2 x^2 
      \end{equation*}	
      引入位置与动量算符,$\hat{x}$, $\hat{p}$, 存在对易关系:
      \[ [\hat{x}, \hat{p}] =i\hbar\] 
      得哈密顿量的算符形式
      \begin{equation*}
      \begin{aligned}
          \hat{H} &= \frac{\hat{p}^2 }{2m} + \dfrac{1}{2} m \omega ^2 x^2  \\
                 &= - \frac{\hbar ^2}{2m} \frac{\partial^2 }{\partial x^2 }+ \dfrac{1}{2} m \omega ^2 x^2   
      \end{aligned}
      \end{equation*}	
\end{frame}

\begin{frame}
    \frametitle{}
    把哈密顿量算符代入薛定谔方程, 分离出时间变量后, 得定态薛定谔方程
    \[ \hat{H}\rs{\Psi} =E \rs{\Psi} \]
    求解方程 (复杂, 略), 得: 
    \begin{enumerate}
      \IItem 固有值:$\displaystyle  E_n=(n+\frac{1}{2})\hbar \omega $ 
      \IItem 固有函数:{$\displaystyle  X~_n(x)= H(\alpha x) e^{-(\alpha x)^2 /2 }, \qquad \alpha =\sqrt{\frac{m \omega }{\hbar}} $}
      \IItem 基本解:
      \[ \Psi_n(x,t) = N_n e^{-\frac{i}{\hbar} E_n t } X~_n(x)\]  
      \IItem 叠加解:
      \[ \Psi (x,t) = \sum_n a_n \Psi_n(x,t)\] 
  \end{enumerate}
\end{frame}

\begin{frame}
    \frametitle{}
    \解 (3) 正则量子化求解, 
  \begin{equation*}
      \hat{H} = \frac{\hat{p}^2 }{2m} + \dfrac{1}{2} m \omega ^2 \hat{x}^2   \qquad \text{with} \quad [\hat{x},\hat{p}]=i\hbar 
  \end{equation*}	
  首先验证 $(\hat{x},\hat{p})$ 是一对正则共轭量, 由哈密顿正则方程得: \\ 
  \[ \dot{\hat{x}} = \frac{\partial H }{\partial \hat{p} } =  \frac{{\hat{p}}}{m}; \qquad   \dot{\hat{p}} = - \frac{\partial H }{\partial \hat{x} } = -m \omega ^2 \hat{x} \]
  由这对方程可还原谐振子方程
  \[\frac{\mathrm{d}^2 x}{\mathrm{d} t ^2}  = - \omega^2 x\]
  * 也可以构造别的对称性更好的正则量来求解方程. 比如令: 
  \[ \hat{X} = \sqrt{\frac{m\omega}{\hbar}}\hat{x}, \hat{P} = \sqrt{\frac{1}{m \hbar \omega}} \hat{p} \]
\end{frame}

\begin{frame}
    \frametitle{}
  重写哈密顿量
    \[  \hat{H}= \frac{\hbar \omega }{2} (\hat{X}^2 + \hat{P}^2 ) \qquad \text{with} \quad [\hat{X},\hat{P}]=i \]
  再令:
  \[ \hat{a}= \frac{1 }{\sqrt{2}} (\hat{X} + i\hat{P} ), \qquad \hat{a}^\dagger= \frac{1 }{\sqrt{2}} (\hat{X} - i\hat{P} ) \]
  重写哈密顿量
  \[  \hat{H}= \hbar \omega \left(\hat{a}^\dagger \hat{a} + \frac{1 }{2}\right) \qquad \text{with} \quad [\hat{a},\hat{a}^\dagger]=1 \]
  代入,有:
  \[ \hat{a}= \sqrt{\frac{m\omega}{2\hbar}}\hat{x} + i \sqrt{\frac{1}{2 m \hbar \omega}} \hat{p}, \qquad 
  \hat{a}^\dagger= \sqrt{\frac{m\omega}{2\hbar}}\hat{x} + i \sqrt{\frac{1}{2 m \hbar \omega}} \hat{p}\]
  反向可得:
  \[ \hat{x}= \sqrt{\frac{\hbar}{2m\omega}} (\hat{a}+ \hat{a}^\dagger), \qquad 
  \hat{p}= -i \sqrt{\frac{m \omega \hbar}{2}} (\hat{a}-\hat{a}^\dagger)\]
\end{frame}

\begin{frame}
    \frametitle{}
    很明显,它们是正则的. 现证明对易关系: 
  \[ \begin{aligned}
      [X,P] &=   XP- PX \\ 
      &=  \sqrt{\frac{m\omega}{\hbar}} \sqrt{\frac{1}{m \hbar \omega}}   (xp-px)  {\hspace*{8em}}~~\\
      &=  \sqrt{\frac{m\omega}{\hbar}} \sqrt{\frac{1}{m \hbar \omega}} i \hbar  \\
      &= i
  \end{aligned}\]
   
  \[ \begin{aligned}
       [\hat{a},\hat{a}^\dagger] &=   \hat{a}\hat{a}^\dagger - \hat{a}^\dagger\hat{a} \\ 
       &= \frac{1}{2} (\hat{X} + i\hat{P} )  (\hat{X} - i\hat{P} )  - \frac{1}{2} (\hat{X} - i\hat{P} ) (\hat{X} + i\hat{P} )  \\
       &=  -i (\hat{X}\hat{P}-\hat{P}\hat{X}) \\
       &= -i \times i \\
       &= 1
   \end{aligned}\]
\end{frame}


\begin{frame} 
    \frametitle{}
$ a \not =  a^\dagger $, 说明不具厄密性. 对于能量第n个本征态 \\ 
\[ H  \rs{n} = E_n  \rs{n}, \quad  aH  \rs{n} =E_n a \rs{n}\]
\[
\begin{aligned}
      H a \rs{n} &=\hbar \omega (  a ^\dagger a + \frac{1}{2} )a \rs{n} \\ 
      &= \hbar \omega (   a ^\dagger a a  + \frac{1}{2} a ) \rs{n} \\ 
      &= \hbar \omega (   (-1+a a ^\dagger )a + \frac{1}{2} a ) \rs{n} \\ 
      &= \hbar \omega  (  a a ^\dagger a -  a \frac{1}{2} ) \rs{n} \\ 
      &=  a( a ^\dagger a -  \frac{1}{2} )\hbar \omega \rs{n} 
\end{aligned}
\]
\end{frame}


\begin{frame}
\[ 
\begin{aligned}
  H a \rs{n}  &=  a( \hbar \omega   a ^\dagger a +  \frac{1}{2}\hbar \omega  - \hbar \omega ) \rs{n} \\ 
      &=  (aH - a \hbar \omega ) \rs{n} \\ 
      &=  (E_n -  \hbar \omega ) a\rs{n} \\ 
\end{aligned}
\]
也就是说 $a \rs{n} $ 也是能量本征态, 本征值为 $E_n -  \hbar \omega = E_{n-1} $ 即有一份能量$ \hbar \omega $ 被湮灭, 故称 $a$ 为 湮灭算符 \\ {\vspace*{1.3em}} 

* 存在对易关系:
\[ [H, a ] = - a \hbar \omega \] 
\end{frame}

\begin{frame}
      \frametitle{}
    同理
      \[ 
        \begin{aligned}
          H a^\dagger \rs{n}  
              &=  (a^\dagger H + a^\dagger \hbar \omega ) \rs{n} \\ 
              &=  (E_n +  \hbar \omega ) a^\dagger \rs{n} 
        \end{aligned}
        \]
也就是说 $a^\dagger\rs{n} $ 也是能量本性态, 本征值为 $E_n + \hbar \omega = E_{n+1} $ 即产生一份能量$ \hbar \omega $,
故称 $a^\dagger$ 为产生算符 \\ {\vspace*{1.3em}} 

* 存在对易关系:
\[ [H, a^\dagger ] =  a^\dagger \hbar \omega \] 
\end{frame}


\begin{frame}
本征能量不能被无限湮灭,设最小的为$E_0$, 对应本征态$\rs{0}$, 有:
\[ \boxed{a \rs{0}= 0 }\]
~~\\ 
由 \[H=(a ^\dagger a + \frac{1}{2})\hbar \omega\]  
有: 
\[H-\frac{1}{2}\hbar \omega=\hbar \omega a ^\dagger a \] 
\[ 
\begin{aligned}
(H-\frac{1}{2}\hbar \omega) \rs{0} &= \hbar \omega a ^\dagger a \rs{0} = \hbar \omega a ^\dagger 0 = 0 \\ 
H \rs{0} &= \frac{1}{2}\hbar \omega \rs{0}=E_0 \rs{0} 
\end{aligned}
\] 
即, 谐振子的最低能级为
\[ \boxed{E_0=\dfrac{1}{2}\hbar \omega} \]
\end{frame}


\begin{frame}
称$\rs{0}$ 为真空态. 从真空态出发, 相继使用产生算符,每次产生一份能量 $ \hbar \omega$, 因此谐振子的能量本征值为
\[\boxed{E_n = (n+\frac{1}{2})\hbar \omega, \qquad n=0,1,2, \cdots}  \] 

谐振子被普郞克用来解释黑体辐射场, 即能量本征态$\rs{n}$描述的是该模($\omega$)上占有$n$个激发光子, 每个光子的能量是 $\hbar \omega$ 即 :$\rs{n}$态描述的是含有$n$个光量子的态. 因此,能量本征态$\rs{n}$也称为粒子数态, 也称为$Fock$态. \\

对比 
\[  \hat{H}= \left(\hat{a}^\dagger \hat{a} + \frac{1 }{2}\right) \hbar \omega \]
说明 $a^\dagger a$ 正好描述了 $\rs{n}$态上的粒子数, 令 $\hat{N}=\hat{a}^\dagger \hat{a}$, 称为粒子数算符.
\[ \hat{H}= \left(\hat{N} + \frac{1 }{2}\right) \hbar \omega \] 
\end{frame}

\begin{frame}
  \例 [11. 求粒子数算符$\hat{N}$的本征值和本征函数]{}
  \解~ \[ 
  \begin{aligned}
      \hat{H}\rs{n} & =E_n \rs{n}  \\ 
      \left(\hat{N} + \frac{1 }{2}\right) \hbar \omega \rs{n} & =E_n \rs{n}  \\ 
     \hat{N} \hbar \omega \rs{n} & =(E_n-\frac{1}{2} \hbar \omega) \rs{n}  \\ 
     \hat{N} \hbar \omega \rs{n} & = n \hbar \omega \rs{n}  \\ 
     \hat{N}   \rs{n} & = n  \rs{n}  \\ 
  \end{aligned} 
  \]
  得 $\hat{N}$的本征值为$n$和本征态为 $\rs{n}$ \\ 
  完毕!
\end{frame}

\begin{frame}
    \frametitle{}
    产生湮灭算符分别产生和消灭这个模式的一个光子
    \[ {a \rs{n}= D_n\rs{n-1}}, \qquad a^\dagger \rs{n}= C_n \rs{n+1} \]  
    \[ 
      \begin{aligned}
        a^\dagger a \rs{n} &= a^\dagger D_n\rs{n-1} \\ 
        &= D_n C_{n-1} \rs{n}  \\
        &= n \rs{n} \\ 
        D_n C_{n-1}&=n \qquad \cdots (1)
      \end{aligned}
      \] 
      \[ 
        \begin{aligned}
          D_n &=  \lcr{n-1}{a}{n} \\
          &=  (\lcr{n}{a^\dagger }{n-1})^* \\
          &= C_{n-1} ^*   \qquad \cdots (2)
        \end{aligned}
        \] 
        联立(1)(2), 得 $C_{n-1}=\sqrt{n}= D_n, C_{n}=\sqrt{n+1} $\\  
        \[\boxed{ {a \rs{n}= \sqrt{n}\rs{n-1}}, \qquad a^\dagger \rs{n}= \sqrt{n+1} \rs{n+1}} \]   
\end{frame}

\begin{frame}
    \frametitle{}
    \[ 
  \begin{aligned}
      \rs{n+1} &= \frac{a^\dagger }{\sqrt{n+1}} \rs{n} \\
      \rs{n} &= \frac{a^\dagger }{\sqrt{n}} \rs{n-1} \\
             &= \frac{(a^\dagger)^2 }{\sqrt{n(n-1)}} \rs{n-2} \\
             & \cdots \\
      \rs{n} &= \frac{(a^\dagger)^n }{\sqrt{n!}} \rs{0} \\
  \end{aligned}    
    \]
\end{frame}

\begin{frame}
    \frametitle{}
    \例 [12. 求Fock态在位置表象中的波函数]{}
    \解~ Fock态为$\rs{n}$, 设位置本征态为 $\rs{x}$, 要求 $\psi_n (x)=\lr{x}{n} $  \\
    (1) 求$\psi_0 (x)$
    \[ 
  \begin{aligned}
    0 &= a \rs{0}  \\ 
    &= \ls{x} a \rs{0}  \\ 
    &= \ls{x} \sqrt{\frac{m\omega}{2\hbar}}\hat{x} + i \sqrt{\frac{1}{2 m \hbar \omega}} \hat{p} \rs{0}  \\ 
    &= \ls{x} \sqrt{\frac{m\omega}{2\hbar}} x + \sqrt{\frac{\hbar}{2 m \omega}} \frac{\partial }{\partial x } \rs{0}  \\ 
    &= (\sqrt{\frac{m\omega}{2\hbar}} x + \sqrt{\frac{\hbar}{2 m \omega}} \frac{\partial }{\partial x } )\lr{x}{0}  \\ 
    &= (\sqrt{\frac{m\omega}{2\hbar}} x + \sqrt{\frac{\hbar}{2 m \omega}} \frac{\partial }{\partial x } ) \psi_0 (x) \\ 
  \end{aligned} \] 
  
\end{frame}

\begin{frame}
    \frametitle{}
    解得 \[ 
      \psi_0 (x)=  (\frac{m \omega }{\pi  \hbar})^{\frac{1}{4}}  \exp(- \frac{m \omega }{2 \hbar} x^2) \]
    (2) 求$\psi_1 (x)$
    \[ 
      \begin{aligned}
        \psi_1 (x) &= \lr{x}{1}  \\ 
        &=  \ls{x} \hat{a}^\dagger \rs{0}   \\ 
        &= \ls{x} \sqrt{\frac{m\omega}{2\hbar}}\hat{x} - i \sqrt{\frac{1}{2 m \hbar \omega}} \hat{p} \rs{0}  \\ 
        &= \ls{x} \sqrt{\frac{m\omega}{2\hbar}} x - \sqrt{\frac{\hbar}{2 m \omega}} \frac{\partial }{\partial x } \rs{0}  \\ 
        &= (\sqrt{\frac{m\omega}{2\hbar}} x - \sqrt{\frac{\hbar}{2 m \omega}} \frac{\partial }{\partial x } )\lr{x}{0}  \\ 
        &= \frac{1}{\sqrt{2}} (\xi - \frac{\mathrm{d}}{\mathrm{d}\xi}) \lr{x}{0} 
      \end{aligned} \] 
\end{frame}

\begin{frame}
    \frametitle{}
    (3) 求$\psi_n (x)$
    \[ 
      \begin{aligned}
        \psi_n (x) &= \lr{x}{n}  \\ 
        &=  \frac{1}{\sqrt{n!}}\ls{x} (\hat{a}^\dagger)^n \rs{0}   \\ 
        &=  \frac{1}{\sqrt{n!}}  \frac{1}{\sqrt{2^n}} (\xi - \frac{\mathrm{d} }{\mathrm{d}\xi} )^n (\frac{m \omega }{\pi  \hbar})^{\frac{1}{4}}  e^{- \frac{1 }{2 } \xi^2}    \\ 
        &=  \frac{1}{\sqrt{n!}}  \frac{1}{\sqrt{2^n}} (\frac{m \omega }{\pi  \hbar})^{\frac{1}{4}} (-1)^n (  e ^{\frac{1}{2}\xi^2} \frac{\mathrm{d} }{\mathrm{d}\xi} e ^{-\frac{1}{2}\xi^2})^n  e^{- \frac{1 }{2 } \xi^2}  \\  
        &= \frac{1}{\sqrt{n!2^n}} (\frac{m \omega }{\pi  \hbar})^{\frac{1}{4}} e^{- \frac{1 }{2 } \xi^2} H_n(\xi)  
      \end{aligned} \] 
\end{frame}

\begin{frame}
    \frametitle{}
    \例 [13. 求量子谐振子在真空态下的位置和动量的量子涨落]{\[ \Delta x \Delta p_x =\frac{\hbar}{2} \]}
    \解 (1) 算符解法: 
    \[\begin{aligned}
       \overline{x} & = \lcr{n}{\hat{x}}{n} \\ 
       &=  \lcr{n}{ \sqrt{\frac{\hbar}{2m\omega}} (\hat{a}+ \hat{a}^\dagger)}{n} \\ 
       &=  \lcr{n}{ \sqrt{\frac{\hbar}{2m\omega}} \hat{a}}{n} + \lcr{n}{ \sqrt{\frac{\hbar}{2m\omega}} \hat{a}^\dagger}{n}  \\ 
       &=  \lcr{n}{ \sqrt{\frac{\hbar}{2m\omega}} \sqrt{n}}{n-1} + \lcr{n+1}{ \sqrt{\frac{\hbar}{2m\omega}} \sqrt{n+1} }{n} =0  \\
   \end{aligned} \]     
   \end{frame}
   
   \begin{frame}
   \[\begin{aligned}
       \overline{x^2} & = \lcr{n}{\hat{x}^2}{n} \\ 
       &=  \lcr{n}{ (\hat{a}+ \hat{a}^\dagger)^2}{n} \\ 
       &= \frac{\hbar}{2m\omega} \lcr{n}{ aa + {a}^\dagger {a}^\dagger + 2{a}^\dagger a + 1}{n} \\ 
       &= \frac{\hbar}{2m\omega} \lcr{n}{ aa + {a}^\dagger {a}^\dagger + 2 \hat{n} + 1}{n} \\ 
       &= \frac{\hbar}{2m\omega} (2n+1)  \\ 
       &= \frac{1}{m\omega^2}  (n+\frac{1}{2})\hbar\omega \\ 
       &= \frac{1}{m\omega^2} E_n
   \end{aligned} \]     
   \end{frame}
   
   \begin{frame}
       \frametitle{}
       量子涨落
       \[\begin{aligned}
           \Delta x  &= \sqrt{ \overline{x^2}- \overline{x}^2}  \\ 
           &= \sqrt{ \frac{1}{m\omega^2} E_n- 0}  \\ 
           &= \sqrt{ \frac{1}{m\omega^2} E_n}  \\ 
       \end{aligned} \]
        同理:
    \[\overline{p}_x =0, \qquad \overline{p^2}_x = m E_n\]
   \end{frame}
   
   \begin{frame}
    \frametitle{}

         量子涨落
         \[\begin{aligned}
           \Delta p_x  &= \sqrt{ \overline{p^2 _x}- \overline{p}_x ^2}  \\ 
             &= \sqrt{ m E_n}  \\ 
         \end{aligned} \]
         \[\begin{aligned}
           \Delta x \Delta p_x  &= \sqrt{ \frac{1}{m\omega^2} E_n} \sqrt{ m E_n} \\ 
           &= \frac{1}{\omega} E_n \\ 
           &= \frac{\hbar}{2}  \\
         \end{aligned} \]
         根据不确定性原理: 
         \[ \Delta x \Delta p_x \geq  \frac{\hbar}{2}  \]
         因此, 量子谐振子基态 是最小不确定度乘积态.
   \end{frame}

    \begin{frame}
        \frametitle{课堂作业}
        \解~(2)波函数解法: 位置表象下真空态波函数为\[ \psi(x)=\lr{x}{0} = (\frac{m \omega }{\pi  \hbar})^{\frac{1}{4}}  \exp(- \frac{m \omega }{2 \hbar} x^2)\] 
        \[\begin{aligned}
            \overline{x} & = \int_{-\infty}^{+\infty} \psi^{*}(x) x \psi(x) d x  \\ 
            &= (\frac{m \omega }{\pi  \hbar})^{\frac{1}{2}}  \int_{-\infty}^{+\infty} \exp(\frac{m \omega }{2 \hbar} x^2) x \exp(- \frac{m \omega }{2 \hbar} x^2) d x  \\ 
            &= ? 
        \end{aligned} \]
    \end{frame}  

\begin{frame}
    \frametitle{}
    \[\begin{aligned}
        \overline{x^2} & = \int_{-\infty}^{+\infty} \psi^{*}(x) x^2 \psi(x) d x  \\ 
        &= (\frac{m \omega }{\pi  \hbar})^{\frac{1}{2}}  \int_{-\infty}^{+\infty} \exp(\frac{m \omega }{2 \hbar} x^2) x^2 \exp(- \frac{m \omega }{2 \hbar} x^2) d x  \\ 
        &= ? 
    \end{aligned} \]
    量子涨落为 $\Delta x  = \sqrt{ \overline{x^2}- \overline{x}^2}$ \\ 
\end{frame}

\begin{frame}
    \frametitle{}
    \[\begin{aligned}
        \overline{p}_x & = \int_{-\infty}^{+\infty} \psi^{*}(x) \hat{p}_x \psi(x) d x  \\ 
        &= (\frac{m \omega }{\pi  \hbar})^{\frac{1}{2}}  \int_{-\infty}^{+\infty} \exp(\frac{m \omega }{2 \hbar} x^2) (-i\hbar \frac{\mathrm{d}}{\mathrm{d}x}) \exp(- \frac{m \omega }{2 \hbar} x^2) d x  \\ 
        &= ?  \\
        \overline{p^2}_x & = \int_{-\infty}^{+\infty} \psi^{*}(x) \hat{p}_x ^2 \psi(x) d x  \\ 
        &= (\frac{m \omega }{\pi  \hbar})^{\frac{1}{2}}  \int_{-\infty}^{+\infty} \exp(\frac{m \omega }{2 \hbar} x^2) (-i\hbar \frac{\mathrm{d}}{\mathrm{d}x})^2 \exp(- \frac{m \omega }{2 \hbar} x^2) d x  \\ 
        &= ? 
    \end{aligned} \]
\end{frame}

\begin{frame}
    \frametitle{}
    量子涨落为 $\Delta p_x  = \sqrt{ \overline{p^2 _x}- \overline{p}_x ^2}$ \\ 
\end{frame}

\begin{frame} 
\frametitle{谐振子的二次量子化}
{\Bullet}一次量子化:(物理量的算符化)
\[  \hat{H}= \hbar \omega \left(\hat{a}^\dagger \hat{a} + \frac{1 }{2}\right) \qquad \text{with} \quad [\hat{a},\hat{a}^\dagger]=1 \]
基本解:
\[ \psi_n(x,t) = \psi_n(x)e^{-\frac{i}{\hbar} E_n t }\]  
叠加解:
\[ \Psi (x,t) = \sum_n a_n \psi_n(x,t)\] 
{\Bullet}二次量子化: (态函数的算符化)
\[ \hat{\Psi } (x,t) = \sum_n \hat{a}_n \psi_n(x) e^{-i \omega_n t}\] 
\end{frame}

\begin{frame} 
\frametitle{}
称$\hat{\Psi } (x,t)$为场算符. 
\[ \hat{\Psi } (x,t) = \sum_n \hat{a}_n \psi_n(x) e^{-i \omega_n t}\]      
每一种振动模式$\{ \omega_n \}$ 都有自己的产生湮灭算符($\hat{a}_n, \hat{a}^{\dagger} _n$)\\ \vspace*{0.6em}

描述的是含无穷多粒子的场,这些粒子的能量分别为
\[ E_0, E_1, E_2, \cdots E_n, \cdots \]
频率分别为\[\omega_0, \omega_1, \omega_2, \cdots\] 
场的基态表示为$\rs{0}$ \\ 
由$\{n_i\}$个能量为$\{E_i\}$频率为 $\{\omega_i\}$的粒子构成的场为:
\[ a^{\dagger} _{n_1} a^{\dagger} _{n_2}\cdots a^{\dagger} _{n_i} \rs{0}\]
\end{frame}

\begin{frame} 
    \frametitle{* 算符的排序计号}
    通常,算符总可以表示成数个产生湮灭算符的连排形式. 不同的排序方式, 对应的算符具体形式也有所不同. 常用的有三种: \\
    \begin{enumerate}
      \item 正规(normally)排序 : ${F}^{<n>} (a, a^{\dagger})$ \[ (a^{\dagger} a)^{(n)} = a^{\dagger} a, \quad (a^{\dagger 2} a)^{(n)} = a^{\dagger} a^{\dagger} a\]
      \item 反正规(anti-normally)排序 : ${F}^{<a>} (a, a^{\dagger})$ \[ (a^{\dagger} a)^{(a)} = aa^{\dagger} , \quad (a^{\dagger 2} a)^{(s)} = a a^{\dagger} a^{\dagger} \]
      \item 对称(symmetrically)排序 : ${F}^{<s>} (a, a^{\dagger})$ \[ (a^{\dagger} a)^{(s)} = \frac{1}{2}(a^{\dagger} a + aa^{\dagger}) , \quad (a^{\dagger 2} a)^{(s)} = \frac{1}{3}(a^{\dagger} a^{\dagger} a+  a^{\dagger} a a^{\dagger} + a a^{\dagger} a^{\dagger} ) \]
    \end{enumerate}
\end{frame}

%%%%%%%%%%%%%%%%%%%%%%%%%%%%%%%%%%%%%%%%%%%%%%%%%%%%%%%%%%%%%%%%%%%%
\begin{frame}
    \frametitle{课外作业}
    \begin{enumerate}
        \item 写出常用平均值公式,并用狄拉克记号求其在动量表象中的形式
        \item 求动量表象中位置算符$\hat{x}$的具体形式
        \item 用正则量子化方法求解一维量子谐振子
        \item 求$Fock$态在正则位置表象中的波函数
                \[\left\langle q|n \right\rangle = (2^n n!)^{-1/2} ( \frac{\omega}{\pi \hbar})^{1/4} \exp(-\xi^2 /2) H_n (\xi), \quad \text{with}~~ \xi= q/\sqrt{\omega/\hbar}\]
        \item 试证明对于量子谐振子第n个能量本征态,存在 
                \[  \Delta x \Delta p_x = (n+\frac{1}{2}\hbar)   \]
    \end{enumerate}
\end{frame}
%%%%%%%%%%%%%%%%%%%%%%%%%%%%%%%%%%%%%%%%%%%%%%%%%%%%%%%%%%%%%%%%%%%   % 量子力学基础与谐振子
%\begin{frame}
    \frametitle{前情回顾}
    {\Bullet} 经典描述: 一个自由度为$n$的系统可以由成对的变量($q_1, q_2, \cdots, q_n, p_1, p_2, \cdots, p_n$)描述。$q_i$为广义坐标,$p_i$为广义速度。哈密顿有正则表示$H(q_1, q_2, \cdots, q_n, p_1, p_2, \cdots, p_n) $ 。每一组正则共轭变量($q_i, p_i$)符合哈密顿正则方程:
    \[ \frac{\mathrm{d} q_i }{\mathrm{d} t}  = \frac{\partial H }{\partial p_i}, \quad \frac{\mathrm{d} p_i }{\mathrm{d} t}  = - \frac{\partial H }{\partial q_i}\]
    {\Bullet} 正则量子化: 
    \begin{itemize}
        \item 写出经典哈密顿 H
        \item 哈密顿正则化
        \item 正则变量不对易, 算符化; 哈密顿算符化.
        \item 把哈密顿算符代入薛定谔方程求解.
    \end{itemize}     
\end{frame}

%%%%%%%%%%%%%%%%%%%%%%%%%%%%%%%%%%%%%%%%%%%%%%%%%%%%%%%%%%%%%
\begin{frame} [plain]
    \frametitle{}
    \Background[1] 
    \begin{center}
    {\huge 第4讲:光场量子化}
    \end{center}  
    \addtocounter{framenumber}{-1}   
\end{frame}
%%%%%%%%%%%%%%%%%%%%%%%%%%%%%%%%%%%%%%%%%%%%%%%%%%%%%%%%%%%% 


\section{1. 单模光场的量子化}

\begin{frame}
      \frametitle{电磁场的单模展开}
    考虑空腔中的一个线性极化的电磁波,如图 
    \begin{center}
     \includegraphics[width=0.20\textwidth]{figs/2022-04-27-12-37-33.png}
    \end{center}
    电磁场的能量密度: 
    \[ \omega = \frac{1}{2} (\epsilon_0 E^2 _x + \mu_0 H^2 _y) \]
    总能的经典哈密顿
    \[ H = \frac{1}{2} \int_V (\epsilon_0 E^2 _x + \mu_0 H^2 _y) dV \]      
\end{frame}

\begin{frame}
      \frametitle{}
    经典解为:
      \begin{enumerate}
        \IItem 电场叠加解:$\displaystyle E_{x}(z,t) = \sum\limits_{n=1}^{\infty } E_{0}  \sin \omega_n t \sin k_n z= \sum\limits_{n=1}^{\infty } a_n q_n (t) \sin (k_n z)$
        \IItem 磁场叠加解: $\displaystyle H_{y}(z,t) = \sum\limits_{n=1}^{\infty } H_{0}  \cos \omega_n t \cos k_n z = \sum\limits_{n=1}^{\infty } a_n \frac{\epsilon_0}{k_n}q_n ' (t) \cos (k_n z)$  
    \end{enumerate}	
    三维叠加解
    \[ \begin{aligned}
    \mathbf{E}( \mathbf{r},t) =& - \frac{1}{\sqrt{ \epsilon_0}} \sum_l ^\infty   q_l(t) \mathbf{E}_l( \mathbf{r}) \\
    \mathbf{H}( \mathbf{r},t) =&  \frac{1}{\sqrt{ \mu_0}} \sum_l ^\infty  \omega_l p_l(t) \mathbf{H}_l( \mathbf{r}) \\
    \end{aligned} 
    \] 
\end{frame}

\begin{frame}
      \frametitle{}
    代入电磁场的哈密顿量, 利用腔模正交性, 得:
    \[ \begin{aligned}
        H &= \frac{1}{2} \int_V (\mu_0 \mathbf{H}^2 + \epsilon_0 \mathbf{E}^2) dV \\ 
        &= \sum_l  \frac{1}{2}(p_l ^2 + \omega_l ^2 q_l ^2 ) \\ 
        &= \sum_l H_l  
     \end{aligned} 
    \] 
    电磁场哈密顿量是单模哈密顿量的线性叠加, 因此只要对单模进行量子化.
\end{frame}

\begin{frame}
      \frametitle{单模量子化}
      单模的哈密顿
      \[ H_l  =  \frac{1}{2}(p_l ^2 + \omega_l ^2 q_l ^2 ) \]
      写出密顿运动方程
    \[ \begin{aligned}
        \frac{\mathrm{d}p_l}{\mathrm{d}t} &= - \frac{\partial H_l}{\partial q_l} = - \omega ^2 _l q_l \\ 
        \frac{\mathrm{d}q_l}{\mathrm{d}t} &= \frac{\partial H_l}{\partial p_l} =p_l
        \end{aligned} 
    \] 
    说明 $ p_l $ 和$q_l$ 是 电磁场的一对正则共轭变量.  可以进行正则量子化操作!
\end{frame}

\begin{frame}    
    量子化条件
    \[  [\hat{q}_l,\hat{p}_l]=i\hbar \] 
    写在一起:
    \[ \hat{H}_l  =  \frac{1}{2}(\hat{p}_l ^2 + \omega_l ^2 \hat{q}_l ^2 )  \qquad \text{with} \quad [\hat{q_l},\hat{q_l}] =i\hbar\] 
    代入薛定谔方程,既可求得场的波函数! \\ {\vspace*{1em}}
    与谐振子相比较
    \begin{equation*}
        \hat{H} = \frac{\hat{p}^2 }{2m} + \dfrac{1}{2} m \omega ^2 \hat{x}^2   \qquad \text{with} \quad [\hat{x},\hat{p}]=i\hbar 
    \end{equation*}	
    可以发现, 如果令谐振子质量$m =1$, 则形式完全相同.  \\ 
    即单模光场可视为单位质量的谐振子, 可称为"场谐振子". \\ 
    机械振子通过动-势能的相互转换形成振荡, 场谐振子通过电场能和磁场能的相互转换形成振荡!
\end{frame}

\begin{frame}   
  求解过程与量子谐振子完全相同! \\
    令:
    \[ \hat{Q}_l = \sqrt{\frac{\omega}{\hbar}}\hat{q_l}, \qquad \hat{P}_l = \sqrt{\frac{1}{\hbar \omega}} \hat{p_l} \]
    有:
    \[  \hat{H}_l= \frac{\hbar \omega }{2} (\hat{Q}^2 _l + \hat{P}^2 _l) \qquad \text{with} \quad [\hat{Q}_l,\hat{P}_l]=i \]
    令:
    \[ \hat{a}_l= \frac{1 }{\sqrt{2}} (\hat{Q} _l+ i\hat{P}_l ), \qquad \hat{a}^\dagger _l = \frac{1 }{\sqrt{2}} (\hat{Q}_l - i\hat{P}_l ) \]
    可得:
    \[
        \hat{a}_l = \frac{1}{\sqrt{2\hbar \omega_l}} (\omega_l\hat{q}_l+i \hat{p}_l)  , \qquad
        \hat{a}_l ^\dagger = \frac{1}{\sqrt{2\hbar \omega_l}} (\omega_l\hat{q}_l-i \hat{p}_l)  
    \]  
    有
    \[  \hat{H}_l= \hbar \omega \left(\hat{a}^\dagger _l \hat{a} _l + \frac{1 }{2}\right) \qquad \text{with} \quad [\hat{a}_l,\hat{a}^\dagger _l]=1 \]
\end{frame}

\begin{frame}
      \frametitle{物理量的描述}
      由
      \[
        \hat{a}_l = \frac{1}{\sqrt{2\hbar \omega_l}} (\omega_l\hat{q}_l+i \hat{p}_l)  , \qquad
        \hat{a}_l ^\dagger = \frac{1}{\sqrt{2\hbar \omega_l}} (\omega_l\hat{q}_l-i \hat{p}_l)  
        \]  
      反向可得: 
      \[ \begin{aligned}
         \hat{q}_l &= \sqrt{\frac{\hbar}{ 2\omega_l}} (\hat{a}_l+ \hat{a}_l ^\dagger) \\ 
         \hat{p}_l &= -\sqrt{\frac{\hbar\omega_l}{2 }} (\hat{a}_l- \hat{a}_l ^\dagger)  
      \end{aligned} \] 
    若物理量F的经典表示为$F(q_l, p_l)$, 则其量子算符为 $F(\hat{q}_l, \hat{p}_l)$, 即电磁场的所有经典物理量都实现了量子化!     
\end{frame}

\begin{frame}
    \frametitle{}
    \例 [1.求一维单模腔场电磁场分量的算符形式] {}
    \解 ~把 
      \[ \begin{aligned}
        \hat{q}_l &= \sqrt{\frac{\hbar}{ 2\omega_l}} (\hat{a}_l+ \hat{a}_l ^\dagger) \\ 
        \hat{p}_l &= -\sqrt{\frac{\hbar\omega_l}{2 }} (\hat{a}_l- \hat{a}_l ^\dagger)  
     \end{aligned} \] 
    代入一维单模腔场解, 得:
    \[ \begin{aligned}
      \hat{\mathbf{E}}_{x,l}( z,t) =& \sqrt{\frac{\hbar\omega_l}{ 2\epsilon_0 L }} (\hat{a}_l(t)+ \hat{a}_l ^\dagger(t)) \sin(k_l z) = E^0 _l (\hat{a}_l(t)+ \hat{a}_l ^\dagger(t)) E_l(z)\\
      \hat{\mathbf{B}}_{y,l}( z,t) =& -\frac{i}{c} \sqrt{\frac{\hbar\omega_l}{ 2\epsilon_0 L }} (\hat{a}_l(t)- \hat{a}_l ^\dagger(t)) \cos(k_l z) \\
      \hat{\mathbf{H}}_{y,l}( z,t) =& -\frac{i}{c\mu_0} \sqrt{ \frac{\hbar\omega_l}{2\epsilon_0 L }} (\hat{a}_l(t)- \hat{a}_l ^\dagger(t)) \cos(k_l z) \\
      \end{aligned} 
      \] 
\end{frame}

\section{2. 自由光场的量子化}

\begin{frame}
      \frametitle{行波展开}
      自由空间电磁场模为平面波(行波), 解的形式为:
      \[  \hat{e}_\sigma exp (\pm i (\omega_k t - \mathbf{k}\cdot \mathbf{r})), \qquad k^2 =\frac{\omega_k ^2}{c^2} \]
      式中 $\hat{e}_\sigma $ 为偏振方向上的单位矢量, $\sigma=\pm$ 代表两个振动方向, 它们相互正交且都与波矢$\mathbf{k}$正交. \\ \vspace*{1em} 
      经箱归一化, 可离散化行波, 得 
      \[ \mathbf{k} = \frac{2\pi}{L} (l_1 \mathbf{i} + l_2 \mathbf{j}+ l_3 \mathbf{k}), \qquad l_i= 0, \pm 1, \pm 2, \cdots \]
      行波本征模为
      \[ \mathbf{u}_{k\sigma} (\mathbf{r}) = \hat{e}_\sigma e^{i \mathbf{k}\cdot \mathbf{r}}\]
\end{frame} 

\begin{frame}
      \frametitle{}
    自由空间电磁场行波展开
      \[   \begin{aligned}
        \mathbf{E} (\mathbf{r},t) &=i \sum^\infty _{k,\sigma} (\frac{\hbar\omega_k}{2 \epsilon_0 V } )^{1/2} \hat{e}_\sigma [ a_{k\sigma} (t) e^{i \mathbf{k}\cdot \mathbf{r}} - a ^* _{k\sigma} (t)  e^{-i \mathbf{k}\cdot \mathbf{r}}] \\
      \mathbf{H} (\mathbf{r},t) &=i \sum^\infty _{k,\sigma} (\frac{\hbar\omega_k}{2 \mu_0 V } )^{1/2} (\hat{e}_k \times \hat{e}_\sigma) [ a_{k\sigma} (t) e^{i \mathbf{k}\cdot \mathbf{r}} - a ^* _{k\sigma} (t)  e^{-i \mathbf{k}\cdot \mathbf{r}}] 
      \end{aligned} \]
    箱内总能量:
      \[ \begin{aligned}
        H &= \frac{1}{2} \int_V (\mu_0 \mathbf{H}^2 + \epsilon_0 \mathbf{E}^2) dV \\ 
        &= \frac{1}{2}\sum^\infty _{k,\sigma} \hbar \omega_k (a_{k\sigma} a_{k\sigma} ^* + a_{k\sigma} ^*a_{k\sigma} )   \\ 
        &= \frac{1}{2}\sum^\infty _{k,\sigma} (p_{k\sigma} ^2 + \omega_k ^2 q_{k\sigma} ^2 )  = \sum^\infty _{k,\sigma} H_{k\sigma}
      \end{aligned} 
      \] 
\end{frame}

\begin{frame}
      \frametitle{}
      单色行波的哈密顿
      \[ H_{k\sigma} = \frac{1}{2} (p_{k\sigma} ^2 + \omega_k ^2 q_{k\sigma} ^2)\]
      哈密顿运动方程
        \[ \begin{aligned}
        \frac{\mathrm{d}p_{k\sigma}}{\mathrm{d}t} &= - \frac{\partial H_{k\sigma}}{\partial q_{k\sigma}} = - \omega ^2 _{k\sigma} q_{k\sigma} \\ 
        \frac{\mathrm{d}q_{k\sigma}}{\mathrm{d}t} &= \frac{\partial H_{k\sigma}}{\partial p_{k\sigma}} =p_{k\sigma}
        \end{aligned} 
      \] 
\end{frame}

\begin{frame}        
      同理,可正则量子化:
      \[ \hat{H}_{k\sigma} = \frac{1}{2} (\hat{p}_{k\sigma} ^2 + \omega_k ^2 \hat{q}_{k\sigma} ^2) \qquad \text{with} \quad [\hat{p}_{k\sigma},\hat{q}_{k\sigma}] =i\hbar \]

      \[ \hat{H}_{k\sigma}= \hbar \omega_{k\sigma} \left(\hat{a}^\dagger _{k\sigma} \hat{a} _{k\sigma}+ \frac{1 }{2}\right) \qquad \text{with} \quad [\hat{a} _{k\sigma},\hat{a}^\dagger _{k\sigma}]=1 \]
      比如: 行波的电场和磁场算符为:
      \[   \begin{aligned}
        \hat{\mathbf{E}} (\mathbf{r},t) &=i \sum^\infty _{k,\sigma} (\frac{\hbar\omega_k}{2 \epsilon_0 V } )^{1/2} \hat{e}_\sigma [ \hat{a} _{k\sigma} (t) e^{i \mathbf{k}\cdot \mathbf{r}} - \hat{a} ^\dagger _{k\sigma} (t)  e^{-i \mathbf{k}\cdot \mathbf{r}}] \\
        &= \hat{\mathbf{E}}^+(\mathbf{r},t)+ \hat{\mathbf{E}}^-(\mathbf{r},t) \\
      \hat{\mathbf{H}} (\mathbf{r},t) &=i \sum^\infty _{k,\sigma} (\frac{\hbar\omega_k}{2 \mu_0 V } )^{1/2} (\hat{e}_k \times \hat{e}_\sigma) [ \hat{a} _{k\sigma} (t) e^{i \mathbf{k}\cdot \mathbf{r}} - \hat{a} ^\dagger _{k\sigma} (t)  e^{-i \mathbf{k}\cdot \mathbf{r}}] 
      \end{aligned} \]
\end{frame}

\section{3. 光场的量子涨落}

\begin{frame}
      \frametitle{}
    \例[1.试证明$Fock$ 态下电磁场的电场强度平均值为零]{}
    \证~
    设电磁场处于$Fock$ 态 $\rs{n}$, 取三维腔模进行计算
    \[ 
      \begin{aligned}
        \lcr{n}{\mathbf{E}(\mathbf{r,t})}{n} &= \lcr{n}{E^0  (\hat{a}+ \hat{a} ^\dagger)\mathbf{E}(\mathbf{r})}{n}   \\ 
        &= \lcr{n}{E^0 \mathbf{E}( \mathbf{r})\hat{a} }{n} - \lcr{n}{E^0 \mathbf{E}( \mathbf{r})\hat{a}^\dagger }{n}  \\ 
        &= 0-0 \\ 
        &=0 
      \end{aligned}
      \]   
      * 相位随机性导致测量平均值为零! Fock表象一般用于处理小粒子数的情况.  \\      
\end{frame}

\begin{frame}
    \frametitle{}
    \例[2.考虑一维单模驻波场, 求电场和磁场强度的量子涨落]{}
    \解~一维单模驻波场的电场和磁场算符为  
    \[ \begin{aligned}
      E_x(z,t)
      &= E_0  (a^\dagger(t)+a(t) )\sin (kz) \\ 
      H_y(z,t)  
      &= H_0 (a(t) - a^\dagger(t)) \cos (kz)
   \end{aligned} \]
   设光场处于$Fock$态$\rs{n}$, 有:
   \[ \lcr{n}{E_x(z,t)}{n} = \lcr{n}{H_y(z,t)}{n} =0\]
  \end{frame}

  \begin{frame}
   \[ \begin{aligned}
     \lcr{n}{E^2_x}{n} &= \lcr{n}{E^2_0 \sin^2 (kz) (a^\dagger(t)-a(t))^2}{n} \\
     &= E^2_0 \sin^2 (kz) \lcr{n}{2a^\dagger a+1}{n} \\ 
     &= 2 E^2_0 \sin^2 (kz) \lcr{n}{n+\frac{1}{2}}{n} \\ 
     &= 2 (n+\frac{1}{2})E^2_0 \sin^2 (kz)  
  \end{aligned} \]
  量子涨落: 
  \[
 \begin{aligned}
        \Delta E_x &= \sqrt{\langle E^2_x\rangle - \langle E_x\rangle ^2} \\
        &= \sqrt{2} \sqrt{n+\frac{1}{2}} E_0 \left|\sin kz \right| 
 \end{aligned} \]
 即使没有激发(n=0),依然存在真空涨落$ E_0 \left|\sin kz \right|  $  
\end{frame}

%%%%%%%%%%%%%%%%%%%%%%%%%%%%%%%%%%%%%%%%%%%%%%%%%%%%%%%%%%%%%%%%%%%
\begin{frame}
  \frametitle{课堂作业}
   \begin{block}{试证明如下重要结论}
   \[
  \begin{aligned}
   \lcr{n}{\hat{a} ^\dagger\hat{a} ^\dagger}{n} & = \sqrt{(n+1)(n+2)} \lr{n}{n+2}=0   \\     
   \lcr{n}{\hat{a} \hat{a} }{n} & = \sqrt{n(n-1)} \lr{n}{n-2}=0   \\  
   \lcr{n}{\hat{a} ^\dagger\hat{a} }{n} & = n  \lr{n}{n}=n   \\   
   \lcr{n}{\hat{a} \hat{a} ^\dagger}{n} & = (n+1)  \lr{n}{n}=n+1   \\    
  \end{aligned}
  \]
   \end{block}
\end{frame}
%%%%%%%%%%%%%%%%%%%%%%%%%%%%%%%%%%%%%%%%%%%%%%%%%%%%%%%%%%%%%%%%%%%

\begin{frame}
    \frametitle{}
    \例[3.考虑单模行波场, 求电场和磁场强度的量子涨落]{}
    \解~ 单模行波场的电场为
    \[ \mathbf{E}_l(\mathbf{r},t) = E_{0} (\hat{a}_l\exp^{-i \omega t + i \mathbf{k}\cdot \mathbf{r}} + \hat{a}_l ^\dagger \exp^{i \omega t - i \mathbf{k}\cdot \mathbf{r}})\]
    设光场处于$Fock$态$\rs{n}$, 电场的平均值: 
  \[ 
  \begin{aligned}
          \left\langle \mathbf{E}_l(\mathbf{r},t) \right\rangle &=\lcr{n}{\mathbf{E}_l(\mathbf{r},t)}{n} \\  
          &= \lcr{n}{E_{0} (\hat{a}_l \exp^{-i \omega t + i \mathbf{k}\cdot \mathbf{r}} +\hat{a}_l ^\dagger \exp^{i \omega t - i \mathbf{k}\cdot \mathbf{r}})}{n}\\ 
          &=0 
  \end{aligned}\]
\end{frame}

\begin{frame}
    \frametitle{}
    \[ \begin{aligned}
      \left\langle \mathbf{E}^2_l(\mathbf{r},t) \right\rangle &=\lcr{n}{\mathbf{E}^2_l(\mathbf{r},t)}{n} \\  
      &= E^2_{0} \lcr{n}{(\hat{a}_l \exp^{-i \omega t + i \mathbf{k}\cdot \mathbf{r}} +\hat{a}_l ^\dagger \exp^{i \omega t - i \mathbf{k}\cdot \mathbf{r}})^2}{n}\\ 
      &= E^2_{0} \lcr{n}{\hat{a}_l\hat{a}_l ^\dagger + \hat{a}_l ^\dagger \hat{a}_l}{n}\\ 
      &= E^2_{0} \lcr{n}{\sqrt{nn} + \sqrt{(n+1)(n+1)}}{n}\\ 
      &= E^2_{0} (2n+1)
\end{aligned}\]
量子涨落: 
\[
\begin{aligned}
      \Delta \mathbf{E}_l &= \sqrt{\langle \mathbf{E}^2 _l \rangle - \langle \mathbf{E}_l \rangle ^2} \\
      &= E_0 \sqrt{2n+1}
\end{aligned} \]
即使没有激发(n=0),依然存在真空涨落$ E_0 $  
\end{frame}

\section{4. 光场随时间的演化}

\begin{frame}
 \frametitle{}
 海森堡方程描述算符随时间的演化规律:
 \[ \frac{\mathrm{d}F}{\mathrm{d}t} = \frac{i}{\hbar}[H,F] \]
 把$F= a $ 和光场哈密顿 \[\hat{H}_l= \hbar \omega \left(\hat{a}^\dagger _l \hat{a} _l + \frac{1 }{2}\right) \]
 代入上式, 求得:
 \[ a_{l}(t)=a_{l}(0) e^{-i \omega_{l} t} \]  
 同时,  
 \[ a ^\dagger _{l}(t)=a ^\dagger _{l}(0) e^{-i \omega_{l} t} \]  
\end{frame}

\begin{frame}
 \frametitle{}
 场算符随时间的演化:
\[ q_{l}(t)=\sqrt{\frac{\hbar}{2 m_{l} \omega_{l}}}\left(a_{l} e^{-i \omega_{l} t}+a_{l}^{\dagger} e^{i \omega_{l} t}\right) \]
\[ p_{l}(t)=-i \sqrt{\frac{m_{l} \hbar \omega_{l}}{2}}\left(a_{l} e^{-i \omega_{l} t}-a_{l}^{\dagger} e^{i \omega_{l} t}\right) \]   
\end{frame}

\begin{frame}
 \frametitle{}
  场强随时间的演化 
  \[ 
    \begin{aligned}
      &E_{x}(z, t)=\sum_{j} \mathcal{E}_{l}^{(s)}\left(a_{l} e^{-i \omega t}+a_{l}^{\dagger} e^{i \omega t}\right) \sin \left(k_{l} z\right) \\
      &H_{y}(z, t)=-i \varepsilon_{0} c \sum_{l} \mathcal{E}_{l}^{(s)}\left(a_{l} e^{-i \omega t}+a_{l}^{\dagger} e^{i \omega t}\right) \cos \left(k_{l} z\right)
    \end{aligned} 
  \]
式中 $\mathcal{E}_{l}^{(s)}=\sqrt{\frac{\hbar \omega_{l}}{\varepsilon_{0} V}}$
\end{frame}

\section{5. 相图量子化}

\begin{frame}
  \frametitle{经典相图1}
  考虑空腔中的一个线性极化的电磁波,如图
    \begin{center}
       \includegraphics[width=0.22\textwidth]{figs/2022-04-27-12-37-33.png}
    \end{center}
   场分量如下:  
   \[ \begin{cases}
    E_{x}(z, t)=E_{0} \sin k z \sin \omega t \\ 
    B_{y}(z, t)=B_{0} \cos k z \cos \omega t  \quad \text{with} \quad B_{0} = E_{0} /c 
   \end{cases} \]
 \end{frame}
 
%\begin{frame}
% \frametitle{}
% 电磁场的能量密度为:
% \[ H=\frac{1}{2}\left(\epsilon_{0} {E}^{2}+\frac{1}{\mu_{0}} B^{2}\right) \]
% 电场能 (设模面积为A): 
% \[ \begin{aligned}
%  E_{\text {electric }} &=\frac{1}{2} \epsilon_{0} A \int_{0}^{L} {E}_{0}^{2} \sin ^{2} k z \sin ^{2} \omega t \mathrm{~d} z \\
%  &=\frac{1}{4} \epsilon_{0} A {E}_{0}^{2} \sin ^{2} \omega t \int_{0}^{L}(1-\cos 2 k z) \mathrm{d} z \\
%  &=\frac{1}{4} \epsilon_{0} V {E}_{0}^{2} \sin ^{2} \omega t
%  \end{aligned}
%    \]       
%\end{frame}
%
%\begin{frame}
% \frametitle{}
% 磁场能 :  
% \[\begin{aligned}
%  E_{\text {magnetic }} &=\frac{1}{2 \mu_{0}} A \int_{0}^{L} B_{0}^{2} \cos ^{2} k z \cos ^{2} \omega t \mathrm{~d} z \\
%  &=\frac{1}{4 \mu_{0}} A B_{0}^{2} \cos ^{2} \omega t \int_{0}^{L}(1+\cos 2 k z) \mathrm{d} z \\
%  &=\frac{1}{4 \mu_{0}} V B_{0}^{2} \cos ^{2} \omega t
%  \end{aligned} \] 
%总能量 :
%\[ H=\frac{V}{4}\left(\epsilon_{0} \mathcal{E}_{0}^{2} \sin ^{2} \omega t+\frac{B_{0}^{2}}{\mu_{0}} \cos ^{2} \omega t\right)
%   \]
%\end{frame}
%
%begin{frame}
% \frametitle{}
% 令: \[ q(t)=\left(\frac{\epsilon_{0} V}{2 \omega^{2}}\right)^{1 / 2} \mathcal{E}_{0} \sin \omega t \]
% \[ p(t)=\left(\frac{V}{2 \mu_{0}}\right)^{1 / 2} B_{0} \cos \omega t \equiv\left(\frac{\epsilon_{0} V}{2}\right)^{1 / 2} \mathcal{E}_{0} \cos \omega t \]
% 代回, 得总能:
%     \[ H=\frac{1}{2}\left(p^{2}+\omega^{2} q^{2}\right) \]
% 说明电磁场的振荡是电场能与磁场能之间的转换
%end{frame}

\begin{frame}
 \frametitle{}
 初始条件决定一个初始相位 $ \varphi$ \\ 
  {\Bullet}复函表示:  \[
    \begin{aligned}
         E_{x}(z, t) &=E_{0} (z) e^ {i\varphi}  \\
         &=  E_{0}(z) \cos\varphi  + i E_{0}(z) \sin\varphi \\ 
         &= E_1 (z,t) + i E_2 (z,t)
    \end{aligned}
    \]
  其中 $E_1(z,t), E_2(z,t) $是电场的实部和虚部 \\   
\end{frame}

\begin{frame}
   \begin{center}
      \includegraphics[width=0.4\textwidth]{figs/2.png}
   \end{center}
 \[ \begin{cases}
  Re(E)= E_{0}(z) \cos\varphi\\ 
  Im(E)= E_{0}(z) \sin\varphi\\ 
 \end{cases} \]
 电磁场的大小和相位角都是确定的.  
\end{frame}

\begin{frame}
  \frametitle{经典相图2}
  定义电场的正交分量(field quadratures)
  \[ \begin{cases}
    X_{1}(t) = \sqrt{\frac{\omega}{2\hbar}}q(t)= \sqrt{\frac{\epsilon_{0} V}{4 \hbar \omega}} {E}_{0} \sin \omega t\\ 
    X_{2}(t) = \sqrt{\frac{1}{2\hbar \omega}}p(t)= \sqrt{\frac{\epsilon_{0} V}{4 \hbar \omega}} {E}_{0} \cos \omega t\\ 
   \end{cases} \]
   代入
   \[
  \begin{aligned}
       E_{x}(z, t) &=E_{0} \sin k z \sin (\omega t + \varphi) \\
       &=  E_{0} \sin k z (\cos\varphi \sin\omega t + \sin\varphi \cos\omega t) \\ 
       &= \sqrt{\frac{4 \hbar \omega}{\epsilon_{0} V}} \sin k z\left(\cos \varphi X_{1}(t)+\sin \varphi X_{2}(t)\right) \\ 
       &=  X_{1}(z,t)+ i X_{2}(z,t)
  \end{aligned}
  \]
\end{frame}

\begin{frame}
  \begin{center}
     \includegraphics[width=0.5\textwidth]{figs/3.png}
  \end{center}
\end{frame}


\begin{frame}
  \frametitle{量子化相图}
  场正交分量可量子化:
  \[ \begin{cases}
    \hat{X}_{1}(t) = \sqrt{\frac{\omega_l}{2\hbar}}\hat{q}(t)=\frac{1}{2}\left(a+a^{^{\dagger}}\right)\\ 
    \hat{X}_{2}(t) = \sqrt{\frac{1}{2\hbar \omega_l}}\hat{p}(t)= \frac{1}{2 i}\left(a-a^{^{\dagger}}\right)\\ 
   \end{cases} \]
  \[ \text{with} \, [X_1, X_2] =\frac{i}{2} \]
  计算不确定度:
  \[ \Delta X_1 \Delta X_2  = \frac{1}{2\hbar}  \Delta q \Delta p \geq \frac{1}{2\hbar} \frac{\hbar}{2}=\frac{1}{4}  \]
\end{frame}

\begin{frame}
    \frametitle{} 
    \begin{center}
       \includegraphics[width=0.4\textwidth]{figs/2022-04-27-15-34-34.png}
    \end{center}
    \[ \Delta X_1 \Delta X_2  \geq  \frac{1}{4}  \]
    电磁场的大小和相位角都有一定的不确定性. 灰色区描述不可区分的简并态. 
\end{frame}

\begin{frame}
  \frametitle{}
  \例[4.试证明真空态是最小不确定度乘积态]{
   \[ \Delta X_1 \Delta X_2 =\dfrac{1}{4}, \, \Delta X_1 = \Delta X_2 = \frac{1}{2} \] 
  }
  \解~不确定度计算公式\[ \Delta x  = \sqrt{ \overline{x^2}- \overline{x}^2}\] 
\[ 
\begin{aligned}
  \overline{ X_1}  &= \lcr{n}{\frac{1}{2}\left(a+a^{\dagger}\right)}{n} \\ 
        &= 0 \\
  \overline{ X_2 } &= \lcr{n}{\frac{1}{2i}\left(a-a^{\dagger}\right)}{n} \\ 
        &= 0 
\end{aligned}\]
\end{frame}

\begin{frame}
  \frametitle{}
\[ 
\begin{aligned}
  \overline{X_1 ^2} &= \lcr{n}{\frac{1}{4}\left(a+a^{\dagger}\right)^2}{n} \\
        &= \frac{1}{4} \lcr{n}{\left(aa+a^{\dagger}a^{\dagger}+aa^{\dagger}+ a^{\dagger}a\right)^2}{n} \\
        &= \frac{1}{4} (2n+1)
\end{aligned}\]
\[
\begin{aligned}
  \overline{X_2 ^2} &= \lcr{n}{-\frac{1}{4}\left(a-a^{\dagger}\right)^2}{n} \\
  &= -\frac{1}{4} \lcr{n}{\left(aa+a^{\dagger}a^{\dagger}-aa^{\dagger}- a^{\dagger}a\right)^2}{n} \\
  &= \frac{1}{4} (2n+1)
\end{aligned}\]
\[ \Delta X_1 = \Delta X_2  =  \sqrt{ \overline{X^2}- \overline{X}^2} =\frac{1}{2}\sqrt{2n+1}\geq \frac{1}{2}\] 
\end{frame}

\begin{frame}
 \frametitle{}
  对于真空态
 \[ \Delta X_1 = \Delta X_2  = \frac{1}{2}\]
 \[ \Delta X_1 \Delta X_2  = \frac{1}{4}\]    
 证毕!\\ 
\end{frame}

\begin{frame}
 \frametitle{真空态相图}
真空态的经典电场能$E_0=0$, 电场矢量大小为零, 处于原点.  
      \begin{center}
         \includegraphics[width=0.4\textwidth]{figs/4.png}
      \end{center}
      \[ \Delta X_1 = \Delta X_2  = \frac{1}{2}\]
真空态的量子涨落为量子噪音的极限(极小值$\dfrac{1}{4}$), 是场强完全确定而相位完全不确定的态. 
\end{frame}

\begin{frame} 
\frametitle{小结:}
  (1) 光场哈密顿可表示为一系列量子化的场谐振子的能量 (一次量子化)
  \[ \hat{H}_{k\sigma}= \sum_{k,\sigma}\hbar \omega_{k\sigma} \left(\hat{a}^\dagger _{k\sigma} \hat{a} _{k\sigma}+ \frac{1 }{2}\right) \qquad \text{with} \quad [\hat{a} _{k\sigma},\hat{a}^\dagger _{k\sigma}]=1 \]
  (2) 自由光场的电场可表示为一系列量子化的平面波 (二次量子化)
  \[\begin{aligned}
    \hat{\mathbf{E}} (\mathbf{r},t) &=i \sum^\infty _{k,\sigma} (\frac{\hbar\omega_k}{2 \epsilon_0 V } )^{1/2} \hat{e}_\sigma \left[ \hat{a} _{k\sigma} (t) e^{i \mathbf{k}\cdot \mathbf{r}} - \hat{a} ^\dagger _{k\sigma} (t)  e^{-i \mathbf{k}\cdot \mathbf{r}}\right] \\
    &= \hat{\mathbf{E}}^+(\mathbf{r},t)+ \hat{\mathbf{E}}^-(\mathbf{r},t)
  \end{aligned} \]
\end{frame}
%%%%%%%%%%%%%%%%%%%%%%%%%%%%%%%%%%%%%%%%%%%%%%%%%%%%%%%%%%%%%%%%%%%
\begin{frame}
    \frametitle{课外作业}
    \begin{enumerate}
        \item  利用腔模正交性证明 \[ H= \sum_l H_l\]
        \item  计算电磁场真空态的量子涨落
    \end{enumerate}
\end{frame}
%%%%%%%%%%%%%%%%%%%%%%%%%%%%%%%%%%%%%%%%%%%%%%%%%%%%%%%%%%%%%%%%%%%       % 光场量子化
%
%%%%%%%%%%%%%%%%%%%%%%%%%%%%%%%%%%%%%%%%%%%%%%%%%%%%%%%%%%%%%
\begin{frame} [plain]
    \frametitle{}
    \Background[1] 
    \begin{center}
    {\huge 第5-6讲:相干态}
    \end{center}  
    \addtocounter{framenumber}{-1}   
\end{frame}
%%%%%%%%%%%%%%%%%%%%%%%%%%%%%%%%%%%%%%%%%%%%%%%%%%%%%%%%%%%% 

\section{1. 相干态的基本知识}

\begin{frame}
 \frametitle{相干态的定义}
 相干态: 也叫做格劳伯态, 是罗伊·格劳伯于1963年提出来的一种量子力学纯态, 它是数态的相干叠加态. (2005年诺贝尔物理学奖)。
 \[ \rs{\alpha} = \sum_{n=0} ^{+\infty} c_n  \rs{n} = e^{-\frac{1}{2}\left|\alpha\right|^2}  \sum_{n=0} ^{+\infty}  \frac{\alpha^n}{\sqrt{n!}} \rs{n} \]
   \begin{center}
        \includegraphics[width=0.5\textwidth]{figs/2022-04-27-11-50-32.png}
   \end{center}
\end{frame}

\begin{frame}
 \frametitle{相干的经典经验}
    经典相干三条件: 频率相同, 相差恒定, 传播方向相同.
 \begin{itemize}
     \item 电谐振动的振动模式的空间传播形成电磁波
     \item 频率相同的波是相干波
     \item 谐振子拉离平衡位置(平移), 然后舒放 (起振)
     \item 起振后谐振子的频率与拉离的距离无关 (固有频率)
     \item 谐振子的能量与拉离的距离有关
 \end{itemize}
    经验: 平移(拉离平衡位置) 产生相干态!    
\end{frame}

\begin{frame} 
    \begin{tcolorbox4}[相干态基本命题:]
        \begin{enumerate}
            \item 湮灭算符的本征态是光场相干态
            \[ \hat{a}\rs{\alpha}= \alpha\rs{\alpha}\]
            \item 光场相干态源于真空态的平移
            \[ \hat{a} =  D(\alpha)  \rs{0}  \]
            \item 光场相干态具有超完备性
            \[ \frac{1}{\pi} \int \rl{\alpha}{\alpha}d ^2 \alpha =1\]
        \end{enumerate}  
    \end{tcolorbox4}
\end{frame}


\section{2. 湮灭算符的本征态}

\begin{frame}
    \frametitle{湮灭算符的本征态}
    设湮灭算符有属于本征值$\alpha$的本征态 $\rs{\alpha}$, 写本征方程: 
    \[ \hat{a}\rs{\alpha}= \alpha\rs{\alpha}\]
    \[ \ls{\alpha}\alpha^* =\ls{\alpha}\hat{a}^\dagger\]
    * 相干态是湮灭一个光子后,场的状态没有发生改变的态 (如何理解?)\\ 
    由于产生湮灭算符不厄密, 本征值不是实数(复数), 对应经典的复振幅. 
    \[ \alpha =\left|\alpha\right| e^{i\varphi} = X_1 + i X_2\]
    \[ \alpha^* =\left|\alpha\right| e^{-i\varphi} = X_1 - i X_2\]
\end{frame}

\begin{frame}
    \frametitle{}     
    \例 [1. 试在$Fock$表象求湮灭算符的本征态]{}
    \解 ~对于本征方程
    \[ a \rs{\alpha} = \alpha \rs{\alpha} \]
    把本征态在$Fock$表象展开 
  \[\begin{aligned}
      a \rs{\alpha} &=\sum_{n=0} ^{+\infty} c_n a \rs{n} \\
      &=  \sum_{n=1} ^{+\infty} c_n \sqrt{n} \rs{n-1} \\
      &=  \sum_{n=0} ^{+\infty} c_{n+1} \sqrt{n+1} \rs{n} \\
      &=   \alpha \sum_{n=0} ^{+\infty} c_n  \rs{n}  
  \end{aligned} \]
\end{frame}

\begin{frame}
    \frametitle{}
    得递推公式
    \[\begin{aligned}
      c_{n+1} \sqrt{n+1} &=\alpha c_n \\
      c_{n} \sqrt{n} &=\alpha c_{n-1} \\
      c_{n}  &=\frac{\alpha}{\sqrt{n}} c_{n-1} \\
              &=\frac{\alpha^2}{\sqrt{n(n-1)}} c_{n-2} \\
              & \cdots  \\ 
              &=\frac{\alpha^n}{\sqrt{n!}} c_{0} \\
  \end{aligned} \]
  \[ \rs{\alpha} = \sum_{n=0} ^{+\infty} c_n  \rs{n} = \sum_{n=0} ^{+\infty} \frac{\alpha^n}{\sqrt{n!}} c_{0} \rs{n} \]
\end{frame}

\begin{frame}
    \frametitle{}
    归一化
    \[ \begin{aligned}
      1=\lr{\alpha}{\alpha} &=\ls{n}|\sum_{n=0} ^{+\infty} \frac{\alpha^n}{\sqrt{n!}} c_{0}|^2 \rs{n} \\ 
      &= \ls{n}\left|c_0\right|^2 \sum_{n=0} ^{+\infty} \frac{\left|\alpha^2\right|^n}{n!}  \rs{n} \\ 
      &= \ls{n}\left|c_0\right|^2 e^{\left|\alpha\right|^2}  \rs{n} \\  
      &= \left|c_0\right|^2 e^{\left|\alpha\right|^2}  \\ 
      \to c_0 &=   e^{-\frac{1}{2}\left|\alpha\right|^2} \\
      \to c_n &=  e^{-\frac{1}{2}\left|\alpha\right|^2} \frac{\alpha^n}{\sqrt{n!}}  
  \end{aligned}\]
  \[ \boxed{\rs{\alpha} = e^{-\frac{1}{2}\left|\alpha\right|^2}  \sum_{n=0} ^{+\infty}  \frac{\alpha^n}{\sqrt{n!}} \rs{n}}\]
\end{frame}

\begin{frame}
    \frametitle{}
    \[
  \begin{aligned}
          \rs{\alpha} &= \sum_{n=0} ^{+\infty} c_n  \rs{n}  \\
          &=  e^{-\frac{1}{2}\left|\alpha\right|^2}  \sum_{n=0} ^{+\infty}  \frac{\alpha^n}{\sqrt{n!}} \rs{n} \\ 
          &= e^{-\frac{1}{2}\left|\alpha\right|^2}  \sum_{n=0} ^{+\infty}  \frac{\alpha^n}{\sqrt{n!}} \frac{(\hat{a}^\dagger)^n}{\sqrt{n!}}\rs{0} \\ 
          &= e^{-\frac{1}{2}\left|\alpha\right|^2}  \sum_{n=0} ^{+\infty}  \frac{(\alpha\hat{a}^\dagger)^n}{n!}\rs{0} \\ 
          &=  e^{-\frac{1}{2}\left|\alpha\right|^2 + \alpha \hat{a}^\dagger }  \rs{0} \\ 
          &=  e^{-\frac{1}{2}\left|\alpha\right|^2 + \alpha \hat{a}^\dagger -\alpha^{*} a}  \rs{0}  \\ 
          &=  D(\alpha)  \rs{0} 
  \end{aligned}    
    \]
\end{frame}

\section{3. 平移算子}

\begin{frame}
      \frametitle{平移算子}
      \[ \rs{\alpha} =  D(\alpha)  \rs{0} \]
      上式定义了新的算符$D(\alpha)$,称之为平移(位移)算符
      \[D(\alpha)= e^{-\frac{1}{2}\left|\alpha\right|^2 + \alpha \hat{a}^\dagger -\alpha^{*} a}\]
      结论: 湮灭算符的本征态是真空态发生平衡后的态. \\ 
\end{frame}
    

\begin{frame}[allowframebreaks=] 
    \frametitle{}      
    ~\\
    \例 [2. 试证明位平移符可写成 ]{ \[ D(\alpha)=\exp \left(\alpha a^{\dagger}-\alpha^{*} a\right) \]}
    \解 ~(1) 先证明BH算符公式 :如果两算符满足
        \[[A,[A,B]] = [B, [A,B]]=0\]
        则有 
        \[e^{A+B}= e^A e^B e^{-\frac{1}{2}[A,B]} =  e^B e^A e^{\frac{1}{2}[A,B]}\]  
    \证~令 $ f(\xi) = e^{\xi A} e^{\xi B}$ 
       \[ \begin{aligned}
           \frac{\mathrm{d}f}{\mathrm{d}\xi} &=Ae^{\xi A} e^{\xi B} + e^{\xi A} Be^{\xi B} \\
           &= Ae^{\xi A} e^{\xi B} + e^{\xi A} Be^{\xi B} e^{-\xi A} e^{-\xi B} e^{\xi A} e^{\xi B} \\
           &= (A+e^{\xi A} Be^{-\xi A})e^{\xi A} e^{\xi B} \\ 
           &= (A+e^{\xi A} Be^{-\xi A}) f(\xi)
       \end{aligned}\] 
             
       令 $g(\xi)= e^{\xi A} Be^{-\xi A}$, 并做泰勒展开
       \[\begin{aligned}
               g(\xi) &= g(0)+ \xi \frac{\mathrm{d}g}{\mathrm{d}\xi}|_{\xi=0}+ \frac{1}{2!}\xi^2 \frac{\mathrm{d}^2g}{\mathrm{d}\xi^2}|_{\xi=0}+ \cdots \\ 
               &= B + e^{\xi A}[A,B]e^{-\xi A}|_{\xi=0} + e^{\xi A}[A,[A,B]]e^{-\xi A}|_{\xi=0} + \cdots \\ 
               &= B + \xi[A,B] \\ 
           \frac{\mathrm{d}f}{\mathrm{d}\xi} &= ((A+ B) + \xi[A,B] )f(\xi) \\ 
           \frac{\mathrm{d}f}{f(\xi)} &= ((A+ B) + \xi[A,B] ) \mathrm{d}\xi\\ 
           f(\xi) &=  e^{((A+ B)\xi + \frac{\xi^2}{2} [A,B] )}\\ 
           e^{\xi A} e^{\xi B} &= e^{(A+ B)\xi} e^{\frac{\xi^2}{2} [A,B]}
           \end{aligned} \]
           取 $\xi=1$, 得:
           \[e^{A+B}= e^A e^B e^{\frac{1}{2}[A,B]}\] 
           同理, 有 
           \[e^{A+B} = e^{B+A} = e^B e^A e^{\frac{1}{2}[B,A]} =e^B e^A e^{-\frac{1}{2}[A,B]} \] 
           证毕! \\ 
    (2) 再证明原题
    \[ 
  \begin{aligned}
      D(\alpha) &=\exp \left(\alpha a^{\dagger}-\alpha^{*} a\right) \\  
      &= e^{\alpha a^{\dagger}} e^{-\alpha^{*} a} e^{-[\alpha a^{\dagger},-\alpha^{*} a]/2}  \\
      &= e^{\alpha a^{\dagger}} e^{-\alpha^{*} a} e^{-[\left|\alpha\right| e^{i\varphi} a^{\dagger},-\left|\alpha\right| e^{-i\varphi} a]/2}  \\
      &= e^{\alpha a^{\dagger}} e^{-\alpha^{*} a} e^{-(\left|\alpha\right| e^{-i\varphi} a\left|\alpha\right| e^{i\varphi} a^{\dagger}- \left|\alpha\right| e^{i\varphi} a^{\dagger}\left|\alpha\right| e^{-i\varphi} a)/2}  \\
      &= e^{\alpha a^{\dagger}} e^{-\alpha^{*} a} e^{-(\left|\alpha\right|^2  (a a^{\dagger}-  a^{\dagger} a))/2}  \\
      &= e^{\alpha a^{\dagger}} e^{-\alpha^{*} a} e^{-\left|\alpha\right|^2 /2} \\
  \end{aligned}  
    \]
 \end{frame}

 \begin{frame}
    \frametitle{}
    \例 [3. 试证明平移算符有如下性质 ]{
        \[ D (\alpha_1) \rs{\alpha_2}= \rs{\alpha_1+ \alpha_2} \] 
    }
    \证 
  \[ 
    \begin{aligned}
        D(\alpha) &=\exp \left(\alpha a^{\dagger}-\alpha^{*} a\right) \\
        D (\alpha_1+ \alpha_2) &= \exp \left((\alpha_1+ \alpha_2) a^{\dagger}-(\alpha_1+\alpha_2)^{*} a\right) \\
        &= D (\alpha_1) D(\alpha_2) \\
        & ~ \\
        D (\alpha) \rs{0} &= \rs{\alpha} \\
        D (\alpha_1+ \alpha_2) \rs{0} &= \rs{\alpha_1+ \alpha_2} \\
        D (\alpha_1) D(\alpha_2)\rs{0} &= \rs{\alpha_1+ \alpha_2} \\
        D (\alpha_1) \rs{\alpha_2} &= \rs{\alpha_1+ \alpha_2} \\
  \end{aligned}  
  \] 
  这正是"平移"算符的名称来源.
\end{frame}
 
 \begin{frame}
       \frametitle{} 
       \例 [4. 试证明位移算符是幺正算符 ]{}
       \证 ~
    \[ 
    \begin{aligned}
        D^\dagger (\alpha) &= (\exp \left(\alpha a^{\dagger}-\alpha^{*} a\right) )^\dagger \\
        &= \exp \left(\alpha^* a-\alpha a^\dagger \right)  \\
        &= \exp \left( (-\alpha) a^\dagger- (-\alpha)^* a \right)  \\
        &= D (-\alpha)  \\
    \end{aligned}  
    \]  
    \[ 
    \begin{aligned}
        D (\alpha)D^\dagger (\alpha) &= D^\dagger (\alpha)D (\alpha) \\
        &= \exp \left(\alpha^* a-\alpha a^\dagger \right) \exp \left(\alpha a^{\dagger}-\alpha^{*} a\right)  \\
        &=1
    \end{aligned}  
    \]  
    即
    \[   D^\dagger (\alpha)= D^{-1} (\alpha)  \]
    证毕!
\end{frame}

\begin{frame}
      \frametitle{}
      \例 [5. 试证明位移算符有如下性质 ]{
          \[ a D (\alpha) =  D (\alpha) a  +  D (\alpha)\alpha \]
       \[ D^\dagger (\alpha) a D (\alpha) = a + \alpha; \qquad D^\dagger (\alpha) a^\dagger D (\alpha) = a^\dagger + \alpha^* \]   
      }
      \证 (1)~
    \[ 
      \begin{aligned}
        (a D (\alpha) -  D (\alpha) a) \rs{\alpha} &= a D (\alpha) \rs{\alpha}- D (\alpha) a \rs{\alpha} \\ 
        &= a \rs{\alpha+\alpha} - \alpha D (\alpha) \rs{\alpha} \\
        &= 2\alpha D (\alpha) \rs{\alpha} -  \alpha D (\alpha) \rs{\alpha} \\
        &= \alpha D (\alpha) \rs{\alpha} \\
        (a D (\alpha) -  D (\alpha) a) \sum \rs{\alpha}\ls{\alpha } &=  \alpha D (\alpha) \sum\rs{\alpha}\ls{\alpha }  \\ 
        a D (\alpha) -  D (\alpha) a &=  D (\alpha) \alpha  \\ 
        a D (\alpha) &= D (\alpha)  a + D (\alpha) \alpha  \\ 
    \end{aligned}  
    \] 
\end{frame}

\begin{frame}
 \frametitle{}
   (2) 
   \[ 
    \begin{aligned}
        a D (\alpha) &=  D (\alpha) a  +  D (\alpha)\alpha  \\
        D^\dagger  a D (\alpha) &= D^\dagger D (\alpha) a  + D^\dagger D (\alpha)\alpha  \\
        &= a + \alpha
  \end{aligned}  
  \] 
\end{frame}

\section{4. 相干态性质}

\begin{frame}
    \frametitle{光子数均值}
        \例[6.求处于相干态 $\rs{\alpha}$光场的平均光子数]{}
        \解 ~光子数的平均值为    
    \[ \begin{aligned}
     \overline{n} &=\lcr{\alpha}{\hat{N}}{\alpha} \\ 
     &= \lcr{\alpha}{a^{\dagger} a  }{\alpha}  \\ 
     &= \alpha^* \alpha \\ 
     &= \left| \alpha\right|^2  \\ 
    \end{aligned}\]
\end{frame}

\begin{frame}
    \frametitle{光子数分布}
        \例[7.光场处于相干态 $\rs{\alpha}$, 求测量导致其处于第n个能量激发态的概率]{}
        \解 ~    
    \[ \begin{aligned}
     P (n) &=\left|\lr{n}{\alpha}\right|^2 \\ 
    &= \left|\ls{n} e^{-\frac{1}{2}\left|\alpha\right|^2}  \sum_{m=0} ^{+\infty}  \frac{\alpha^m}{\sqrt{m!}} \rs{m} \right|^2\\
    &=  e^{-\left|\alpha\right|^2}  \sum_{m=0} ^{+\infty}  \frac{\alpha^{2m}}{m!} \delta_{nm} ^2  \\ 
    &=  e^{-\left|\alpha\right|^2}  \frac{\alpha^{2n}}{n!} \\
    &=  \frac{ \overline{n} ^{n} }{n!}e^{-\overline{n}} 
    \end{aligned}\]
    相干态在$Fock$空间服从Poisson分布! 
\end{frame} 

\begin{frame}
    \frametitle{光子数的量子涨落}
        \例[8.光场处于相干态 $\rs{\alpha}$, 求光子数的量子涨落]{}
        \解 ~光子数方的均值 
    \[ \begin{aligned}
    \overline{n^2}&=\lcr{\alpha}{\hat{N}^2}{\alpha} \\ 
    &= \lcr{\alpha}{a^{\dagger} a a^{\dagger} a  }{\alpha}  \\ 
    &= \lcr{\alpha}{a^{\dagger} (a^{\dagger}  a+1) a  }{\alpha}  \\ 
    &= \lcr{\alpha}{a^{\dagger} a^{\dagger}  a a  }{\alpha} + \lcr{\alpha}{a^{\dagger} a  }{\alpha}  \\ 
    &= \left| \alpha\right|^4 + \left| \alpha\right|^2
    \end{aligned}\]

\end{frame}

\begin{frame}
      \frametitle{}  
    光子数的量子涨落
    \[
    \begin{aligned}
    \Delta n & = \sqrt{ \overline{n^2}- \overline{n}^2} \\ 
    & = \sqrt{ \left| \alpha\right|^4 + \left| \alpha\right|^2- (\left| \alpha\right|^2)^2} \\ 
    & = \sqrt{  \left| \alpha\right|^2} \\ 
    & = \sqrt{ \overline{n}} \\ 
    &= \left| \alpha\right|
    \end{aligned}    
    \]
\end{frame} 

\begin{frame}
    \frametitle{相干光场物理图像}
    {\Bullet}泊松分布: 描述单位时间(区间)随机事件发生的概率分布, 是二项分布的一种特例, 事例发生概率低而样本空间很大时, 即稀有事件的二项分布无限逼近泊松分布
      \begin{center}
           \includegraphics[width=0.55\textwidth]{figs/2022-04-27-09-17-50.png}
      \end{center}
   \[ P(X=k)=\frac{\lambda^{k}}{k !} e^{-\lambda} ; \qquad P (n) =  \frac{ \overline{n} ^{n} e^{-\overline{n}}}{n!} \]
   相干态: $ \rs{\alpha} =  D(\alpha)  \rs{0} $, 
   峰值位置: $ \overline{n} = \left| \alpha\right|^2 $, 展宽(量子涨落): $ \Delta n = \left| \alpha\right| $ 
\end{frame}

\begin{frame}
 \frametitle{相干态场的量子涨落}
 \例[9.试证明所有的相干态都是最小不确定度乘积态]{
   \[ \Delta X_1 \Delta X_2 =\dfrac{1}{4}, \, \Delta X_1 = \Delta X_2 = \frac{1}{2} \] 
  }
 \解~湮灭算符不厄密, 总可以写成两厄密算符的形式
 \[ a = X_1 + i X_2, \, a ^\dagger = X_1 - i X_2, \qquad \text{with}\qquad  [X_1, X_2]= \frac{i}{2} \]
 对应场的正交分量算符: 
 \[ \hat{X}_{1} =\frac{1}{2}\left(a + a^{\dagger}\right)\]
 \[ \hat{X}_{2} = \frac{1}{2 i}\left(a - a^{\dagger}\right)\]
\end{frame}



\begin{frame}
 \frametitle{}
      先证明对易关系: 
    \[ \begin{aligned}
        \hat{X}_{1} \hat{X}_{2} & =\mathrm{i}\left(\hat{a}^{\dagger}+\hat{a}\right)\left(\hat{a}^{\dagger}-\hat{a}\right) / 4\\ 
        &=\mathrm{i}\left(\hat{a}^{\dagger} \hat{a}^{\dagger}+\hat{a} \hat{a}^{\dagger}-\hat{a}^{\dagger} \hat{a}-\hat{a} \hat{a}\right) / 4 \\ 
        \hat{X}_{2} \hat{X}_{1} & = \mathrm{i}\left(\hat{a}^{\dagger}-\hat{a}\right)\left(\hat{a}^{\dagger}+\hat{a}\right) / 4 \\ 
        & =\mathrm{i}\left(\hat{a}^{\dagger} \hat{a}^{\dagger}-\hat{a} \hat{a}^{\dagger}+\hat{a}^{\dagger} \hat{a}-\hat{a} \hat{a}\right) / 4 \\ 
        \left[\hat{X}_{1}, \hat{X}_{2}\right] & =\hat{X}_{1} \hat{X}_{2}-\hat{X}_{2} \hat{X}_{1} \\ 
        &= \mathrm{i}\left(\hat{a} \hat{a}^{\dagger}-\hat{a}^{\dagger} \hat{a}\right) / 2 \\ 
        & =\mathrm{i}\left[\hat{a}, \hat{a}^{\dagger}\right] / 2  \\ &=  \frac{i}{2} \\ 
        \Delta X_1 \Delta X_2 & \geq  \left| \left[\hat{X}_{1}, \hat{X}_{2}\right]  \right| /2  \\ 
        &= \frac{1}{4}
    \end{aligned}\]

\end{frame}

\begin{frame} 
 \frametitle{}
      电场的行波展开
      \[ \begin{aligned}
          E(\mathbf{r},t)  &= i  \sqrt{\frac{\hbar \omega}{2 \epsilon_0 V}} [a e^{i(\mathbf{k}\cdot \mathbf{r} -\omega t) }- 
          a^{\dagger} e^{-i(\mathbf{k}\cdot \mathbf{r}-\omega t)}]\\
          &=i  \sqrt{\frac{\hbar \omega}{2 \epsilon_0 V}} [X_1 \sin(\mathbf{k}\cdot \mathbf{r}-\omega t) + X_2 \cos(\mathbf{k}\cdot \mathbf{r}-\omega t)] 
      \end{aligned}\] 
      电场的驻波展开
      \[ \begin{aligned}
        E(\mathbf{r},t)  &= \frac{1}{2} E_0 [ a e^{ -\omega t}+ a^{\dagger} e^{ \omega t}] \\ 
        &= E_0 [X_1 \cos(\omega t)+ X_2 \sin(\omega t)]
      \end{aligned}\]
    式中
    \[ X_1 = \sqrt{\frac{\omega}{2\hbar}}q, \qquad X_2 = \sqrt{\frac{1}{2\hbar\omega}}p\]
\end{frame}
\begin{frame} 
    对相干态求均值
    \[
    \begin{aligned}
        \overline{ X_1}  &= \lcr{\alpha}{\frac{1}{2}\left(a+a^{\dagger}\right)}{\alpha} = \frac{1}{2}(\alpha+\alpha^*) \\
        \overline{X_2}  &= \lcr{\alpha}{\frac{1}{2i}\left(a-a^{\dagger}\right)}{\alpha} = \frac{1}{2i}(\alpha-\alpha^*) \\
    \end{aligned}\]
    对相干态求方的均值
    \[
    \begin{aligned}
      \overline{X^2 _1} &= \lcr{\alpha}{\frac{1}{4}\left(a+a^{\dagger}\right)^2} {\alpha}  \\ 
      &= \frac{1}{4}\lcr{\alpha}{\left(aa+a^{\dagger}a^{\dagger}+ aa^{\dagger} + a^{\dagger}a\right)} {\alpha}  \\
      &= \frac{1}{4}\lcr{\alpha}{\left(aa+a^{\dagger}a^{\dagger}+ 2a^{\dagger}a + 1 \right)} {\alpha} \\
      &= \frac{1}{4}(\alpha + \alpha^*)^2+ \frac{1}{4}
    \end{aligned}   
    \]
\end{frame}

\begin{frame}
      \frametitle{}
      \[
        \begin{aligned}
          \overline{X^2 _2} &= \lcr{\alpha}{-\frac{1}{4}\left(a-a^{\dagger}\right)^2} {\alpha}  \\ 
          &= -\frac{1}{4}\lcr{\alpha}{\left(aa+a^{\dagger}a^{\dagger}- aa^{\dagger} - a^{\dagger}a\right)} {\alpha}  \\
          &= -\frac{1}{4}\lcr{\alpha}{\left(aa+a^{\dagger}a^{\dagger}- 2a^{\dagger}a - 1 \right)} {\alpha} \\
          &= -\frac{1}{4}(\alpha - \alpha^*)^2 + \frac{1}{4}
        \end{aligned}   
        \]  
        求场的量子涨落:
        \[ \Delta X_1 = \sqrt{\overline{X^2 _1}- \overline{X_1}^2} = \frac{1}{2}\]
        \[ \Delta X_2 = \sqrt{\overline{X^2 _2}- \overline{X_2}^2} = \frac{1}{2}\]
        \[ \Delta X_1 \Delta X_2=\frac{1}{4}\]
    证毕!
\end{frame}

\begin{frame}
      \frametitle{相干态的相图}
    与真空态不同, 相干态的场矢量大小为$\left|\alpha\right|$ 
   \begin{center}
        \includegraphics[width=0.35\textwidth]{figs/13.png}
   \end{center}
   相干态相对于真空态,只是进行了平移,其涨落性质并没有改变!\\ 
   但随着场强增大,其相位的不确定度从真空态的无限大变得越来越小,\\ 当场强足够大时, 相位的不确定度趋零,对应于有确定相位的经典电磁场.
\end{frame}

\begin{frame}
    \frametitle{相干态的产生}
  理论上,就是要构造一个系统, 具有如下平移算符. 
  \[ D(\alpha)=\exp \left(\alpha a^{\dagger}-\alpha^{*} a\right) \] 
\end{frame}

\begin{frame}
 \begin{enumerate}
    \item 电流源辐射: 激发真空态产生相干辐射场.
    \item 激光: 大量同一种原子受激辐射, 产生与激源光子物理特性完全相同的光.  
     \begin{center}
      \includegraphics[width=0.35\textwidth]{figs/8.png}
    \end{center}   
 \end{enumerate}
 原子自发辐射发的光子,不具有相位、偏振态、传播方向上的一致,是非相干光. 当有信号光子入射时, 处于高能级的原子可辐射一个与信号光子频率、相位、偏振态以及传播方向都相同的光子. 如果大量原子被激励在高能级,则可产生大量频率、相位、偏振态以及传播方向相同的光子, 即为相干光.  
\end{frame}



\begin{frame}
    \frametitle{相干态演化规律}
        \例[10.已知$t=0$时刻的相干态为$\rs{\alpha (0)}$, 试求t时刻的相干态$\rs{\alpha (t)}$  ]{}
        \解 ~ 光场的哈密顿不显含时间, 其时间演化算符为
    \[ \begin{aligned}
     \hat{U}(t, 0)  &= e^{-i \hat{H}t / \hbar}\\ 
     &= e^{-i \hbar \omega ( a^\dagger a+\frac{1}{2} ) t / \hbar} \\ 
     &= e^{-i \omega t (a^\dagger a +\frac{1}{2} )}  \\ 
    \end{aligned}\]
    在Fock表象:
    \[ \begin{aligned}
        \rs{\alpha (t)} &= \hat{U}(t, 0) \rs{\alpha (0)} \\ 
        &= e^{-i \omega t ( a^\dagger a +\frac{1}{2} )} \rs{\alpha (0)} \\ 
        &= e^{-i \omega t ( a^\dagger a +\frac{1}{2} )} e^{-\frac{1}{2}\left|\alpha(0)\right|^2}  \sum_{n=0} ^{+\infty}  \frac{\alpha(0)^n}{\sqrt{n!}} \rs{n}  \\  
       \end{aligned}\]
\end{frame}

\begin{frame}
    \frametitle{}
    \[ \begin{aligned}
        \rs{\alpha (t)} &= e^{-i\frac{1}{2} \omega t}  e^{-\frac{1}{2}\left|\alpha(0)\right|^2} \sum_{n=0} ^{+\infty}  \frac{\alpha(0)^n}{\sqrt{n!}} e^{-i \omega t  a^\dagger a}  \rs{n}  \\
        &= e^{-i\frac{1}{2} \omega t}  e^{-\frac{1}{2}\left|\alpha(0)\right|^2}\sum_{n=0} ^{+\infty}  \frac{\alpha(0)^n}{\sqrt{n!}} e^{-i n \omega t }  \rs{n}  \\ 
        &= e^{-i\frac{1}{2} \omega t}  e^{-\frac{1}{2}\left|\alpha(0)\right|^2} \sum_{n=0} ^{+\infty}  \frac{(\alpha(0) e^{-i \omega t })^n}{\sqrt{n!}} \rs{n}  \\ 
        &= e^{-i\frac{1}{2} \omega t} e^{\left|e^{-i \omega t }\right|^2}   e^{-\frac{1}{2}\left|\alpha(0)e^{-i \omega t }\right|^2} \sum_{n=0} ^{+\infty}  \frac{(\alpha(0) e^{-i \omega t })^n}{\sqrt{n!}} \rs{n}  \\ 
        &= e^{-i\frac{1}{2} \omega t}  e^{\left|e^{-i \omega t }\right|^2}  \rs{\alpha(0) e^{-i \omega t} } \\ 
        &= \rs{\alpha(0) e^{-i \omega t}} \\ 
       \end{aligned}\]
\end{frame}

\begin{frame}
      \frametitle{}
      \[ \rs{\alpha (t)}= \rs{\alpha(0) e^{-i \omega t}}  \]
       说明: 随着时间的推移, 相干态的本征值只是获得一个相位 $e^{-i \omega t}$, 相干态本身变成了湮灭算符的属于本征值$ \alpha(0) e^{-i \omega t} $的本征态. \\ {\vspace*{0.6em}} 
       即, 随着时间的推移, 相干态只是获得一个相位振荡, 它还是湮灭算符的本征态, 所以相干特性不随时间发生变化. 这正是实验上发现稳定相干条纹的前提.
\end{frame}

\begin{frame}
 \frametitle{}
   \begin{center}
        \includegraphics[width=0.7\textwidth]{figs/5.png}
   \end{center}
    在相图中,表现为:
 \begin{enumerate}
     \item 场矢大小保持不变
     \item 相位解随时间发生周期性振荡(频率$\omega$)
 \end{enumerate}
  位置表象里, 表现为: 
  \begin{enumerate}
    \item 波包的形状保持不变
    \item 波包的中心位置发生周期性振荡(频率$\omega$)
\end{enumerate}   
\end{frame}

\section{5. 相干态表象}

\begin{frame}
    \frametitle{相干态不正交}
        \例[11.试证明相干态不具正交性]{}
        \证 ~ 把本征态
        \[\rs{\alpha}=e^{-\frac{1}{2}\left|\alpha\right|^2}  \sum_{n=0} ^{+\infty}  \frac{(\alpha\hat{a}^\dagger)^n}{n!}\rs{0} \] 
    代入内积公式
    \[ \begin{aligned}
     \lr{\alpha}{\beta} &= \lr{0\left|\sum_{n=0} ^{+\infty}\frac{(\alpha\hat{a}^\dagger)^n}{n!}}{\sum_{n=0} ^{+\infty}\frac{(\beta\hat{a}^\dagger)^n}{n!}\right|0} e^{-\frac{1}{2}(\left|\alpha\right|^2 +\left|\beta\right|^2)}\\ 
     &= e^{-\frac{1}{2}(\left|\alpha\right|^2 +\left|\beta\right|^2)} \sum_{n=0} ^{+\infty}\frac{(\alpha^*)^n (\beta)^n}{n!} \\ 
     &= e^{\alpha^* \beta -\frac{1}{2}(\left|\alpha\right|^2 +\left|\beta\right|^2)} \\ 
     \left|\lr{\alpha}{\beta} \right|^2 &= e^{-  \left| \alpha - \beta\right|^2} \not = 0 
    \end{aligned}\]
\end{frame}

\begin{frame}
    Analysing the two-slit experiment:
      \begin{center}
           \includegraphics[width=0.6\textwidth]{figs/Etwoslitexp.png}
      \end{center}
    \begin{itemize}
        \Item Using wavefunction $\rs{\psi_1}$ to describe the state of the electron running across slit-1 and $\rs{\psi_2}$ for slit-2. when the both slits opened, the electron locates at the superposition state
            \[ \rs{\Psi}=c_1 \rs{\psi_1}+ c_2\rs{\psi_2} \]
    \end{itemize}
\end{frame}

\begin{frame}
    \tikzstyle{na} = [baseline=-.5ex]
    \begin{itemize}
        \Item based on statistical interpretation, the possiblity density of electron reaches certain point of screen should be
        \begin{equation*}
        \begin{split}
            \omega &=|\rs{\Psi}|^2 \\
            &= (\ls{\psi_1}c_1 + \ls{\psi_2}c_2 ) (c_1 \lr{\psi_1}+ c_2\lr{\psi_2}) \\
            & = |c_1|^2 |\rs{\psi_1}|^2 + |c_2|^2 |\rs{\psi_2}|^2  
            + \Myitem{t1}{red}{c_1 c_2 ^* \lr{\psi_2}{\psi_1} + c_1 ^* c_2 \lr{\psi_1}{\psi_2} } \\
        \end{split} 
        \end{equation*}
    \end{itemize}
    \begin{itemize}
        \Item 存在相干项(后两项),形成干涉条纹
        \Item 相干项不为零的条件: 
        \begin{enumerate}
            \item 叠加态: 即同时过两个缝, 否则有$\psi_1$ 或$\psi_2$为零,相干项为零
            \item 不正交性: $\rs{\psi_1}$与 $\rs{\psi_2}$不正交, 否则相干项为零
        \end{enumerate}
        \Item $\left|\lr{\alpha}{\beta} \right|^2 = e^{-  \left| \alpha - \beta\right|^2}$ 表明, 当$\alpha - \beta \to \infty $, 
        趋于正交, 相干项趋零.
    \end{itemize}
\end{frame}

\begin{frame}
 \frametitle{}
   \begin{center}
        \includegraphics[width=1.0\textwidth]{figs/11.png}
   \end{center}
 当$\alpha - \beta \to \infty $, 两位移量差别大, 两波函数虽不正交但已
 重叠部分趋零, 内积趋零, 干涉项趋零, 干涉条纹变弱趋无
 
\end{frame}

\begin{frame}
 \frametitle{相干态的超完备性}
 \例[12.试证明相干态具有如下完备性关系]{
    \[ \frac{1}{\pi} \int \rl{\alpha}{\alpha}d ^2 \alpha =1 \] }
 \证 ~ 把相干态的数态展开式
    \[\rs{\alpha} = e^{-\frac{1}{2}\left|\alpha\right|^2}  \sum_{n=0} ^{+\infty}  \frac{\alpha^n}{\sqrt{n!}} \rs{n} \]
 代入上式左边:
 \[
    \begin{aligned}
        \frac{1}{\pi} \int \rl{\alpha}{\alpha}d ^2 \alpha &=  \frac{1}{\pi} \int \sum_{n,m=0} ^{+\infty} e^{-\left|\alpha\right|^2}  
        \frac{\alpha^n(\alpha^*)^m}{\sqrt{n!m!}}  \rs{n}\ls{m }d ^2 \alpha \\
        &= \sum_{n,m=0} ^{+\infty}  \frac{\rs{n}\ls{m }}{\pi \sqrt{ n!m!}}  \int  e^{-\left|\alpha\right|^2}  \alpha^n(\alpha^*)^m  d ^2 \alpha
    \end{aligned}    
 \]    
\end{frame}

\begin{frame}
 \frametitle{}
  转换到极坐标系进行计算
  \[ 
    \begin{aligned}
        \frac{1}{\pi} \int \rl{\alpha}{\alpha}d ^2 \alpha 
        &= \sum_{n,m=0} ^{+\infty}  \frac{\rs{n}\ls{m }}{\pi \sqrt{ n!m!}}  \int  e^{-\left|\alpha\right|^2}  \alpha^n(\alpha^*)^m  d ^2 \alpha \\ 
        &= \sum_{n,m=0} ^{+\infty}  \frac{\rs{n}\ls{m }}{\pi \sqrt{ n!m!}} 
        \int _0 ^ {+\infty}  e^{-r^2}  r^{n+m}  r dr   \int _0 ^ {2 \pi} e^{i(n-m)\theta} d\theta  \\ 
        &= \sum_{n,m=0} ^{+\infty}  \frac{\rs{n}\ls{m }}{\pi \sqrt{ n!m!}} 
        \int _0 ^ {+\infty}  e^{-r^2}  r^{n+m}  r dr   2 \pi \delta(n-m)  \\ 
        &= \sum_{n=0} ^{+\infty}  \frac{\rs{n}\ls{n }}{\sqrt{ n!n!}} 
        \int _0 ^ {+\infty}  e^{-r^2}  r^{n+n}  2r dr  \\ 
        &= \sum_{n=0} ^{+\infty}  \frac{\rs{n}\ls{n }}{n!} 
        \int _0 ^ {+\infty}  e^{-(r^2)}  (r^2)^{n}   d(r^2)  
    \end{aligned}    
 \]        
\end{frame}

\begin{frame}
 \frametitle{}
 \[ 
    \begin{aligned}
        \frac{1}{\pi} \int \rl{\alpha}{\alpha}d ^2 \alpha    
        &= \sum_{n=0} ^{+\infty}  \frac{\rs{n}\ls{n }}{n!} 
        \int _0 ^ {+\infty}  e^{-t}  t^n   dt  \\ 
        &= \sum_{n=0} ^{+\infty}  \frac{\rs{n}\ls{n }}{n!} 
        \Gamma(n+1)  \\ 
        &= \sum_{n=0} ^{+\infty}  \frac{\rs{n}\ls{n }}{n!} 
        n!  \\ 
        &= \sum_{n=0} ^{+\infty}  \rs{n}\ls{n } \\
        &= 1
    \end{aligned}    
 \] 
 证毕!
\[  \boxed{ \frac{1}{\pi} \int \rl{\alpha}{\alpha}d ^2 \alpha =1} \]
\end{frame}

\begin{frame} 
\frametitle{}
     * 相干态表象的过完备性: $\{\rs{\alpha}\}$的一个子集就可以构成一组完备基, 因此一个态函数在相干态表象中的展开系数是不唯一的.也称超完备性. \\ {\vspace*{0.3em}}
     \证~(1)线性不独立性
     \[ \begin{aligned}
         \int \alpha ^m \rs{\alpha} d^2 \alpha &= \sum_n \frac{\rs{n}}{\sqrt{n!}} \int_0 ^\infty \left|\alpha\right|^{n+m+1} e ^ {- \left|\alpha\right|^2 /2} \int_0 ^{2\pi} e ^{i(n+m)\varphi}  d \varphi =0\\
     \end{aligned}\] 
     (2) 相干态可以相互展开
     \[ \begin{aligned}
        \rs{ \beta } & = \frac{1}{\pi} \int \rs{\alpha} \left\langle \alpha | \beta \right\rangle  d ^2 \alpha \\ 
        &= \frac{1}{\pi} \int \rs{\alpha} e^{-  \left| \alpha - \beta\right|^2}  d ^2 \alpha
    \end{aligned}\] 
\end{frame}

\begin{frame} 
\frametitle{}
     (3) 对于$\{\rs{\alpha}\}$,设$\alpha = r e ^{i \varphi}$, 则$r$取一个固定值的子集就是一组完备基
      \[ \begin{aligned}
          \int _0 ^{2\pi} \rs{\alpha} e ^{-i m \varphi} d \varphi &= e^{- r^2 /2 \sum_n \frac{\rs{n} r^n}{\sqrt{n!}} } \int _0 ^{2\pi} e ^{i (n-m) \varphi} d \varphi \\
          &= 2\pi r^m e^{- r^2 /2}  \frac{\rs{m}}{\sqrt{m!}} \\
        \to \rs{m} &= \frac{r^{-m}}{2\pi}e^{ r^2 /2}  \int _0 ^{2\pi} \rs{\alpha} e ^{-i m \varphi} d \varphi 
      \end{aligned}\] 
      数态可以在这个子集上展开, 因此是完备的. \\ 
      (4) 算符在相干态表象中的矩阵表示是不唯一的, 比如展开在数态上, 可得到一组展开系数.
      \[ F_{\alpha' \alpha} = \left\langle \alpha' | F |\alpha \right\rangle = F(\alpha^*, \alpha')e^{-(\left|\alpha\right|^2+ \left|\alpha'\right|^2)/2}\]
      式中$ F(\alpha^*, \alpha')$也称矩阵元的生成函数.
\end{frame}

\begin{frame} 
 \frametitle{}
        \begin{center}
             \includegraphics[width=1.0\textwidth]{figs/2022-05-11-11-10-06.png}
        \end{center}
    \begin{enumerate}
        \item 正交完备空间~表述唯一
        \item 非正交完备空间~表述唯一
        \item 非正交超完备空间~表述不唯一
    \end{enumerate}
\end{frame}

\begin{frame}  
 \frametitle{}
    求相干态在正则位置表象中的波函数
    \[ \begin{aligned}
       \lr{q}{\alpha} & = \psi_\alpha(q) \\
       \lcr{q}{a}{\alpha} & = \alpha\psi_\alpha(q) \\
       \lcr{q}{\frac{1}{\sqrt{2\hbar \omega}}\left[ \omega q +\hbar \frac{\partial }{\partial q}\right]}{\alpha} & = \alpha\psi_\alpha(q) \\
       \frac{1}{\sqrt{2\hbar \omega}}\left[ \omega q +\hbar \frac{\partial }{\partial q}\right]\lr{q}{\alpha} & = \alpha\psi_\alpha(q) \\ 
       \frac{1}{\sqrt{2\hbar \omega}}\left[ \omega q +\hbar \frac{\partial }{\partial q}\right]\psi_\alpha(q) & = \alpha\psi_\alpha(q)\\   
    \end{aligned}\]  
解方程, 得:
\[\psi_\alpha(q) = (\frac{\omega}{\pi \hbar})^\frac{1}{4} \exp{[(Im \alpha)^2]}\exp\left\{ -\frac{\omega}{\pi \hbar} \left[ q-(\frac{2 \hbar}{\omega})^\frac{1}{2} \alpha \right]^2 \right\} \]
\end{frame}

\begin{frame} 
    \frametitle{相位涨落}
         经典光场的复振幅 
         \[ a= \left|a\right| e^{i\varphi}\]
         经典光场的正则分量
         \[ q(t) = a e^{-\omega t} + a e^{\omega t} = 2\left|a \right|\cos(\omega t + \varphi)\]
         (1) 那是否可写出一个相位算符呢?, 比如把湮灭算符写成
         \[ \hat{a} = \hat{g} e ^ i \hat{\varphi} \]
         这种写法是不行的, 因为
         $e ^ i \hat{\varphi}$与$e ^ {-i \hat{\varphi}}$ 并不是幺正的, 会导致 $\hat{\varphi}$不是厄密的,因而没有可测量意义. 
\end{frame}

\begin{frame} 
      \frametitle{}
         (2) 为保证厄密, 有人定义了新的相位算符
         \[ \hat{S}=\sin \hat{\varphi}, \hat{C}=\cos \hat{\varphi}\]
         指数形式为
         \[\hat{C}= \frac{1}{2} [e^{i \hat{\varphi}} + e^{- i \hat{\varphi}}]\]
         \[\hat{S}= \frac{1}{2i} [e^{i \hat{\varphi}} - e^{- i \hat{\varphi}}]\]
         与产生湮灭算符有如下关系
         \[\hat{a}= \sqrt{\hat{N}+1}(\hat{C}+i\hat{S})\]
         \[\hat{a}^{\dagger}= (\hat{C}-i\hat{S})\sqrt{\hat{N}+1}\]
         但是, $\hat{C}, \hat{S}$不对易, 说明如果能描述相位,它们两也对应两个不同的相位.\\ 
         
\end{frame}

\begin{frame} 
 \frametitle{}
 反向求得:
         \[ C= \frac{1}{2}\left[ \frac{1}{\sqrt{N+1}}a + a^{\dagger}  \frac{1}{\sqrt{N+1}}\right]\]
         \[ S= \frac{1}{2i}\left[ \frac{1}{\sqrt{N+1}}a - a^{\dagger}  \frac{1}{\sqrt{N+1}}\right] \]
相位均值:
      \[  \left\langle \alpha |C| \alpha   \right\rangle = \left|\alpha\right|\cos\theta \exp(-\left|\alpha\right|^2) \sum_n \frac{\left|\alpha\right|^{2n}}{n! \sqrt{n+1}} \]
      当 $\left|\alpha\right|^2\gg 1$时
      \[  \left\langle \alpha |\cos \varphi| \alpha   \right\rangle = \cos \theta (1+ \frac{1}{8\left|\alpha\right|^2}+ \cdots )\]
      \[  \left\langle \alpha |\sin \varphi| \alpha   \right\rangle = \cos^2 \theta -\frac{\cos^2 \theta -\frac{1}{2}}{2\left|\alpha\right|^2 } \]
\end{frame}

\begin{frame} 
 \frametitle{相位算符的性质}
 \[ \begin{aligned}
     S \rs{n} &= \frac{1}{2i} [\rs{n-1}+\rs{n-1}]\\
     S \rs{0} &=-\frac{1}{2i} \rs{1}\\
     S^2 \rs{n}&= \frac{1}{4} [\rs{n}+\rs{n}- \rs{n-2}-\rs{n+2}] \\  
 \end{aligned}\] 
      
\end{frame}
%%%%%%%%%%%%%%%%%%%%%%%%%%%%%%%%%%%%%%%%%%%%%%%%%%%%%%%%%%%%%%%%%%%
\begin{frame}
    \frametitle{课外作业}
    \begin{enumerate}
        \item 试证明相干态是最小不确定度乘积态
        \item 试求相干态在位置表象中的波函数
        \item 试求位置表像里,相干态随时间的振荡表达式
        \item 证明
        \[D^\dagger (\alpha) a^\dagger D (\alpha) = a^\dagger + \alpha^* \]
        \[ \rs{\alpha }\ls{\alpha } a= \left( \alpha + \frac{\partial }{\partial \alpha^*} \rs{\alpha }\ls{\alpha } \right)\]
    \end{enumerate}
\end{frame}
%%%%%%%%%%%%%%%%%%%%%%%%%%%%%%%%%%%%%%%%%%%%%%%%%%%%%%%%%%%%%%%%%%%   % 相干态
%\include{lecture_7and8}   % 压缩态
%\include{lecture_9and10}  % 光子探测与计数
%\begin{frame}
    \frametitle{前情回顾}
    \begin{itemize}
        \item 量子化光场~~真空态~~产生湮灭算符 
        \item 相干态~~ 平移算符 ~~相图
        \item 压缩态~~ 压缩算符
        \item 数态表象~~数态展开
        \item 参量下转换~~平衡零差探测 
        \item 光子计数 ~~ 亚泊松~~泊松~~超泊松光场 
    \end{itemize}     
\end{frame}

%%%%%%%%%%%%%%%%%%%%%%%%%%%%%%%%%%
\begin{frame} [plain]
    \frametitle{}
    \Background[1] 
    \begin{center}
    {\huge 第11-12讲:场关联函数}
    \end{center}  
    \addtocounter{framenumber}{-1}   
\end{frame}

\section{1. 星光测量HB-T 实验}

\begin{frame} 
 \frametitle{实验原理}
 汉伯里·布朗(Hanbury Brown)和特维斯(Twiss)是天文学家,他们想通过星光测量恒星的直径.
 为此他们革新了迈克耳逊恒星干涉仪.
   \begin{center}
        \includegraphics[width=1.0\textwidth]{figs/2022-05-08-12-30-39.png}
   \end{center}
\end{frame}

\begin{frame}
 \frametitle{}
 迈克耳逊恒星干涉仪: 点光源的光是相干光, 平行光是相干光, 小角度的非平行光是部分相干光. \\ 
 由于地球与恒星的距离(L)远大于恒星的直径(D), 可以满足小角度条件.
 \[ \delta \theta_s =\frac{D}{L}\] 
 由于时间和空间相干性的限制, 迈克耳逊恒星干涉仪存在极限:
 \[ \delta \theta_r\leq 1.22 \frac{\lambda}{d}\] 
 当 $d> 1.22 \lambda /\delta \theta_s$, 则没有干涉条纹.  \\ 
 因此, 把d从小依次变大, 会出现在某个d时,干涉条纹刚好消失, 由此可得到$\theta_s$, 进而得到恒星的直径D. 
\end{frame}

\begin{frame} 
 \frametitle{}
 但是$d$增大时,保持两收集镜片的稳定性会变得越来越困难(不稳定性会导致角度误差变大). 因此, 迈克耳逊恒星干涉仪不能用来测量很大的恒星. \\ {\vspace*{2.3em}}
 计算迈克耳逊恒星干涉仪的光强
 \[ \begin{aligned}
     E&= E_k (e^{\mathbf{ik}\cdot \mathbf{r}_1}+ e^{\mathbf{ik}\cdot \mathbf{r}_2}) + E_{k'} (e^{\mathbf{ik'}\cdot \mathbf{r}_1}+ e^{\mathbf{ik'}\cdot \mathbf{r}_2}) \\ 
     I&= \kappa \left\langle E^* E \right\rangle \\
     &= \kappa \left\langle 2 \left| E_k\right|^2 + 2\left| E_{k'}\right|^2 + \left|E_k\right|^2 (e ^{i \mathbf{k}\cdot(\mathbf{r}_1-\mathbf{r}_2)+c.c.})  \right\rangle \\
 \end{aligned}\]
\end{frame}

\begin{frame} 
\frametitle{}
 取 $\left| E_{k}\right|^2 \approx \left| E_{k'}\right|^2 = I_0$, $\mathbf{r}_0 = \dfrac{\mathbf{r}_1 - \mathbf{r}_2}{2}$    
 光强为
 \[ I = 4 \kappa I_0 \left\{ 1+\cos\left[ (\mathbf{k} + \mathbf{k}')\cdot \mathbf{r}_0\right] \cos\left( (\mathbf{k} - \mathbf{k}')\cdot \mathbf{r}_0 \right)  \right\}\]
分析: \\ 
 (1) $\mathbf{k} - \mathbf{k}'$大气的扰动相互抵消, 因此, 调节位置, 可使 \[\cos\left( (\mathbf{k} - \mathbf{k}')\cdot \mathbf{r}_0 \right)=1 \]
(2) 问题在
\[ \cos\left[ (\mathbf{k} + \mathbf{k}') \cdot \mathbf{r}_0 \right] \]
没有办法处理. 而这是一个变化很快的敏感项, 大气的扰动和仪器的噪声涨落可导致其平均值归零, 从而限制了迈克耳逊恒星干涉仪的使用 
\end{frame}

\begin{frame} 
\frametitle{}
 汉伯里·布朗(Hanbury Brown)和特维斯(Twiss) 在1954-1957年间,提出可HB-T实验来测量恒星的直径. \\
 革新点:
 \begin{enumerate}
     \item 改平面镜为凹面镱, 可以收集更多的光, 可测量暗些的恒星
     \item 改相位干涉(关联)为强度干涉(关联)
 \end{enumerate}
 HB-T实验是整个量子光学的奠基性实验!
\end{frame}

\begin{frame} 
 \frametitle{}
    现在要证明两光场的强度之间存在关联!   
    \begin{center}
        \includegraphics[width=0.6\textwidth]{figs/2022-05-08-13-20-54.png}
   \end{center}
   量子力学中,如果两函数不正交(内积不为零),则它们线性相关, 是有关联的.
   \[ \int \psi^*_1(\mathbf{r_1},t) \psi_2(\mathbf{r_2},t+ \tau) \mathbf{dr}dt \not = 0 \]
因此, HB-T 实验本质上是两光场波函数的内积不为零的问题.
\end{frame}

\section{2. 半经典关联函数}

 \begin{frame} 
  \frametitle{一阶关联函数}
  根据半经典测量理论,若光场强度函数为$I(\mathbf{r},t)$ , 则在  $ \Delta t $ 时间内,测到发生光电发射事件的差分概率为 
\[  P (\mathbf{r},t) \Delta t = \eta \left\langle I(\mathbf{r},t) \right\rangle \Delta t \]  
代入电场强度, 有:   
\[ \begin{aligned}
    P (\mathbf{r},t) \Delta t &= \eta \left\langle I(\mathbf{r},t) \right\rangle \Delta t \\ 
    & = \eta \left\langle E^* (\mathbf{r},t) E (\mathbf{r},t)\right\rangle \Delta t \\
    &=\eta G^{1} (\mathbf{r},t) \Delta t  \\
\end{aligned}\] 
 \end{frame}

 \begin{frame} 
  \frametitle{}
  上式中,定义了内积函数:
\[ \boxed{ G^{(1)} (\mathbf{r},t) = \left\langle E^* (\mathbf{r},t) E (\mathbf{r},t)\right\rangle} = \int E^* (\mathbf{r},t) E (\mathbf{r},t) d \mathbf{r} dt \]
  称为一阶关联函数, 它描述了光场的自相关性.\\ 
  \[ \begin{aligned}
    P (\mathbf{r},t) \Delta t &=\eta G^{1} (\mathbf{r},t) \Delta t  \\
   \end{aligned}\] 
   表明: 测量发生光电发射事件的概率,用一阶关联函数描述. 
 \end{frame}

 \begin{frame}   
    \frametitle{二阶关联函数}
   如果有两个探测器, 则
    \[ \begin{aligned}
      P (\mathbf{r_1},t_1) \Delta t_1 &=\eta \left\langle E^*_1 (\mathbf{r_1},t_1) E_1 (\mathbf{r_1},t_1)\right\rangle \Delta t_1  \\
      P (\mathbf{r_2},t_2) \Delta t_2 &=\eta \left\langle E^*_2 (\mathbf{r_2},t_2) E_2 (\mathbf{r_2},t_2)\right\rangle  \Delta t_2  \\
     \end{aligned}\] 
    两探测器相互独立, 则两探测器都测得发生光电发射事件的概率为 (没有归一化):
    \[ \begin{aligned}
        P_2 (\mathbf{r_1},t_1; \mathbf{r_2},t_2)  &= P (\mathbf{r_1},t_1) \Delta t_1 P (\mathbf{r_2},t_2) \Delta t_2\\
        &= \eta_1 \eta_2 \left\langle E^*_1 (\mathbf{r_1},t_1) E_1 (\mathbf{r_1},t_1)\right\rangle  \left\langle E^*_2 (\mathbf{r_2},t_2) E_2 (\mathbf{r_2},t_2)\right\rangle \Delta t_1 \Delta t_2 \\
    \end{aligned}\] 
    如果两探测器相互不独立, 则:
    \[ \begin{aligned}
        P_2 (\mathbf{r_1},t_1; \mathbf{r_2},t_2)  
        &= \eta_1 \eta_2 \left\langle E^*_1 (\mathbf{r_1},t_1) E_1 (\mathbf{r_1},t_1) E^*_2 (\mathbf{r_2},t_2) E_2 (\mathbf{r_2},t_2)\right\rangle \Delta t_1 \Delta t_2 \\
        &= \eta_1 \eta_2  G^{(2)} (\mathbf{r_1},t_1; \mathbf{r_2},t_2) \Delta t_1 \Delta t_2 
    \end{aligned}\]
\end{frame}

\begin{frame} 
 \frametitle{}
 上式中,定义了函数:
 \[ \boxed{ \begin{aligned}   
 G^{(2)} (\mathbf{r_1},t_1; \mathbf{r_2},t_2)  &= \left\langle E^*_1 (\mathbf{r_1},t_1) E_1 (\mathbf{r_1},t_1) E^*_2 (\mathbf{r_2},t_2) E_{2} (\mathbf{r_2},t_2)\right\rangle \\ 
 &= \left\langle I_1 (\mathbf{r_1},t_1) I_2 (\mathbf{r_2},t_2)\right\rangle
\end{aligned}} \]
称为二阶关联函数, 它描述了两光场强度之间的相关性. 或者说两个探测器测得的光场强度的相关性. \\ 
如果只考虑时间关联, 则描述同一电场不同时刻相关性的函数为:
\[ \begin{aligned}
    G^{(2)}(\tau) &= G^{(2)}(t; t+ \tau) \\
    &= \left\langle E^* (t) E (t) E^* (t+ \tau) E (t+ \tau)\right\rangle \\
    &= \left\langle I (t)  I (t+ \tau)\right\rangle \\ 
\end{aligned}\] 
\end{frame}

\begin{frame}
\frametitle{}
 归一化的场关联函数为:
\[  g^{(1)} (\mathbf{r},t) =  \frac{\left\langle E^* (\mathbf{r},t) E (\mathbf{r},t)\right\rangle} {\left\langle E^* (\mathbf{r},t)\right\rangle\left\langle E (\mathbf{r},t)\right\rangle}  \]
\[ \begin{aligned}   
    g^{(2)} (\mathbf{r_1},t_1; \mathbf{r_2},t_2)  
    &= \frac{\left\langle I_1 (\mathbf{r_1},t_1) I_2 (\mathbf{r_2},t_2)\right\rangle}{\left\langle I_1 (\mathbf{r_1},t_1)\right\rangle\left\langle I_2 (\mathbf{r_2},t_2)\right\rangle}
   \end{aligned} \]
\end{frame}

 \begin{frame} 
  \frametitle{分析}
  同一时刻的相关性
  \[  g^{(2)}(0) = \frac{\left\langle I (t)  I (t+ \tau)\right\rangle} {\left\langle I (t) \right\rangle \left\langle I (t+ \tau)\right\rangle} =  \frac{\left\langle I^2 (t)  \right\rangle}{\left\langle I (t)\right\rangle^2 }  \geq 1\]
  对于相干态, $  \left\langle I^2 (t)  \right\rangle = \left\langle I (t) \right\rangle^2, \quad g^{(2)}(0) =1 $ \\ 
  对于混沌光束, $ \left\langle I^2 (t)  \right\rangle > \left\langle I(t) \right\rangle^2, \quad g^{(2)}(0) >1 $
\end{frame}

\begin{frame} 
      \frametitle{}   
  设光束的相干时间为$\tau_c$ (原子能级的寿命),当$\tau\gg \tau_c$, 没有时间相干性, 则 
  \[ I (t) = \left\langle I \right\rangle  + \Delta I (t) =  \left\langle I  \right\rangle  \] 
  \[ \begin{aligned}
    g^{(2)}(\tau) &= \frac{\left\langle I (t)  I (t+ \tau)\right\rangle }{\left\langle I (t) \right\rangle \left\langle  I (t+ \tau)\right\rangle }\\ 
    &= \frac{\left\langle \left\langle I  \right\rangle  \left\langle I  \right\rangle \right\rangle}{\left\langle I \right\rangle^2} \\
    &= 1 
\end{aligned}\]     
\end{frame}
  
 \begin{frame} 
  \frametitle{}
      对于相干态, 恒有 $ I (t)= I (0) =I$ 
      \[ \begin{aligned}
        g^{(2)}(\tau) &= \frac{\left\langle I (t)  I (t+ \tau)\right\rangle}{\left\langle I (t) \right\rangle \left\langle I (t+ \tau)\right\rangle} \\ 
        &= \frac{\left\langle I^2\right\rangle}{ \left\langle I\right\rangle ^2} \\
        &= 1   
    \end{aligned}\] 
      \begin{center}
           \includegraphics[width=0.5\textwidth]{figs/2022-05-08-15-34-39.png}
      \end{center} 
    * 半经典二阶关联函数只能描述经典光和相干光的光强相关性,不能给出非经典光的物理图像.
 \end{frame}

 \begin{frame} 
  \frametitle{}
  热辐射光场的一阶关联函数:
  \[ g^{(1)}(\tau) = \frac{\sum_k \overline{n}_k \omega_k \exp{(-i \omega_k \tau)} }{\sum_k \overline{n}_k \omega_k}, \quad g^{(2)}(\tau) =1+ \left|g^{(1)}(\tau) \right|^2\]
  对于高斯混沌光束: 
       \[g^{(2)}(\tau) =1 + e^{-\pi (\frac{\tau}{\tau_c})^2 } \]
  对于洛伦兹混沌光束
       \[g^{(2)}(\tau) =1 + e^{-2 \frac{\left|\tau\right|}{\tau_0} } \]
         \begin{center}
              \includegraphics[width=1.0\textwidth]{figs/2022-05-08-16-01-24.png}
         \end{center}
 \end{frame}

\section{3. 量子关联函数}

 \begin{frame} 
  \frametitle{ 光子HB-T实验}
         \begin{center}
              \includegraphics[width=0.8\textwidth]{figs/2022-05-08-16-13-39.png}
         \end{center}
     如光强很弱, 则要使用单光子探测器进行计数, 那两计数之间是否存在关联?
 \end{frame}

 \begin{frame} 
    \frametitle{}
    平稳随机过程假设\\
    (1) 统计的稳定性(statistically stationary)\\ 
    对任意时间$T$,下面的联合测量概率恒成立.
         \[ P_{n}\left(x_{n}, t_{n} ; x_{n-1}, t_{n-1} ; \ldots ; x_{1}, t_{1}\right)=P_{n}\left(x_{n}, t_{n}+T ; x_{n-1}, t_{n-1}+T ; \ldots ; x_{1}, t_{1}+T\right)\]
    (2) 性质: \\ 
    时间原点选取不影响概率
    \[ P_1(x,t)=P_1(x,0)\]
    对空间的概率积分不依赖于时间原点选取
    \[ \left\langle x(t) \right\rangle = \int x P_1(x,t)dx=  \int x P_1(x,0)dx  \]
    \end{frame}

 \begin{frame} 
  \frametitle{}
    分析: 单光子探测器, 测得的光子数(计数)服从泊松分布(平稳随机过程)
       \[  P_n = \frac{\overline{n}^n }{n!} \mathrm{e}^{- \overline{n} }\]
       计数均值:
    \[ \begin{aligned}
        \overline{n} &= \sum_n n P_n\\
        &=\sum_n n \frac{\overline{n}^n }{n!} \mathrm{e}^{- \overline{n} } \\
        &= \left\langle \sum_n n \frac{(\eta I)^n }{n!} \mathrm{e}^{- \eta I }  \right\rangle \\ 
        &= \left\langle (\eta I) \frac{\mathrm{d}}{\mathrm{d}(\eta I)}\sum_n  \frac{(\eta I)^n }{n!} \mathrm{e}^{-\eta I }  \right\rangle \\
        &=  \left\langle \eta I \right\rangle = \eta \left\langle I \right\rangle
    \end{aligned}\] 
 \end{frame}

 %\begin{frame} 
 % \frametitle{}
 % 平方数均值:
 % \[ \begin{aligned}
 %   \overline{n^2} &= \sum_n n^2 P_n\\
 %   &= \sum_n n(n-1) P_n +\sum_n n  P_n \\ 
 %   &= \eta \left\langle I \right\rangle + \eta^2 \left\langle I^2 \right\rangle 
 %   \end{aligned}\] 
 %   均方差:
 %   \[ \begin{aligned}
 %       \left\langle (\Delta n)^2 \right\rangle &= \overline{n^2} -\overline{n}^2 \\ 
 %       &= \overline{n} + \eta^2 \left\langle I^2 \right\rangle 
 %   \end{aligned}\]     
 %\end{frame}

 \begin{frame}  
  \frametitle{}
      若两探测器量子效率相同,根据二阶关联函数的定义, 有:
      \[ \begin{aligned}
        g^{(2)}(\tau) &= \frac{\left\langle I_1 (t)  I_2 (t+ \tau)\right\rangle }{\left\langle I_1 (t) \right\rangle  \left\langle I_2 (t+ \tau)\right\rangle }\\ 
        &=  \frac{\left\langle n_1 (t)  n_2 (t+ \tau)\right\rangle}{\left\langle n_1 (t)\right\rangle \left\langle n_2 (t+ \tau)\right\rangle}
    \end{aligned}\] 
    ~\\ {\vspace*{2.3em}} 
    * 计数相关体现光强相关, 必须采用全量子理论进一步处理.
 \end{frame}

 \begin{frame} 
  \frametitle{}
       {\Bullet}考虑理想光子计数器. 它在时空点$(\mathbf{r}, t)$处吸收了一个光子(即探测到一个光子), 导致场从初态$\rs{i }$ 跃迁到末态$\rs{f}$. \\
       光场可表示为正频部分和负频部分:
       \[ \hat{E} = \hat{E}^{(+)} + \hat{E}^{(-)} =\sum_j \epsilon_j a_j e^{-i(v_j t - \mathbf{k}\cdot \mathbf{r})} + \sum_j \epsilon_j a^{\dagger} _j e^{i(v_j t - \mathbf{k}\cdot \mathbf{r})}\]
       正频是湮灭算符随时间振荡的总和, 与光子被吸收(湮灭)有关,\\ 
       探测概率等于跃迁概率:
       \[T_{if} = \left| \lcr{f}{E^{(+)}(\mathbf{r}, t)}{i} \right|^2\] 
       总探测概率是对所有可能末态求和
       \[ \begin{aligned}
           \omega'_1(\mathbf{r}, t) = \sum_f T_{if} &= \sum_f \left| \lcr{f}{E^{(+)}(\mathbf{r}, t)}{i} \right|^2 \\ 
           &= \sum_f \lcr{i}{E^{(-)}(\mathbf{r}, t)}{f} \lcr{f}{E^{(+)}(\mathbf{r}, t)}{i}   \\
           &= \lcr{i}{E^{(-)}(\mathbf{r}, t)E^{(+)}(\mathbf{r}, t)}{i}  
       \end{aligned}\] 
 \end{frame}

 \begin{frame} 
  \frametitle{}
       若初态是真空态:
       \[ \rho = \rl{0 }{0 }\]
       \[\omega'_1(\mathbf{r}, t)= \lcr{0}{E^{(-)}(\mathbf{r}, t)E^{(+)}(\mathbf{r}, t)}{0} =0 \]
       表示对真空态,探测到一个光子的概率为零. \\ 
      通常,得考虑初态是叠加态$\rs{\psi} = \sum_i a_i\rs{i}$, 或混态, 其密度算符分别为
       \[\rho = \rl{\psi}{\psi} = \sum_i a^*_i a_i\rl{i}{i}; \quad \rho = \sum_n P_n\rl{\psi_n}{\psi_n} = \sum_{n,i} P_n a^*_i a_i\rl{i}{i};  \]
       对应的概率为:
    \[\begin{aligned}
        \omega_1(\mathbf{r}, t)
        &= \sum_i \lcr{i}{a^* _i E^{(-)}(\mathbf{r}, t)E^{(+)}(\mathbf{r}, t) a_i}{i} \\ 
        &= \sum_i a^*_i a_i\rl{i}{i}{E^{(-)}(\mathbf{r}, t)E^{(+)}(\mathbf{r}, t)} \\ 
        &= Tr( \rho E^{(-)}(\mathbf{r}, t)E^{(+)}(\mathbf{r}, t))
    \end{aligned} \] 
 \end{frame}

\begin{frame} 
    \frametitle{}
    基于$\omega_1(\mathbf{r}, t)$, 定义一阶关联函数:
    \[ G^{(1)}(\mathbf{r_1}, \mathbf{r_2},; t_1, t_2) = Tr( \rho E^{(-)}(\mathbf{r_1}, t_1)E^{(+)}(\mathbf{r_2}, t_2)) \]   
    使用记号 $x_i= (\mathbf{r}_i, t_i)$,简写为:
    \[ G^{(1)}(x_1; x_2) = Tr( \rho E^{(-)}(x_1)E^{(+)}(x_2)) \]  
    令$\mathbf{r_1}=\mathbf{r_2}=r,t_1=t, t_2=t+\tau$, 
    \[ G^{(1)}(\mathbf{r} ;t, t+\tau) = Tr( \rho E^{(-)}(\mathbf{r}, t)E^{(+)}(\mathbf{r}, t+\tau)) \] 
    令$\mathbf{r_1}=\mathbf{r_2}=r,t_1=t_2=t$, 退化为测得一个光子的概率
    \[ G^{(1)}(\mathbf{r}, t) = Tr( \rho E^{(-)}(\mathbf{r}, t)E^{(+)}(\mathbf{r}, t))=\omega_1(\mathbf{r}, t) \] 
    很明显, 一阶关联函数描述各种时位条件下测得一个光子事件的概率
   \end{frame}

   \begin{frame} 
    \frametitle{}
    {\Bullet}考虑有两个探测器,在两个不同时-位点各探测到一个光子,则两事件的联合概率为
    \[ \begin{aligned}
        \omega'_2(\mathbf{r_1}, t_1;\mathbf{r_2}, t_2) &= \sum_f \left| \lcr{f}{E^{(+)}(\mathbf{r_1}, t_1)E^{(+)}(\mathbf{r_2}, t_2)}{i} \right|^2 \\ 
        &= \lcr{i}{E^{(-)}(\mathbf{r_1}, t_1)E^{(-)}(\mathbf{r_2}, t_2)E^{(+)}(\mathbf{r_1}, t_1)E^{(+)}(\mathbf{r_2}, t_2)}{i} \\ 
        \omega_2(\mathbf{r_1}, t_1;\mathbf{r_2}, t_2)&=  Tr(\rho E^{(-)}(\mathbf{r_1}, t_1)E^{(-)}(\mathbf{r_2}, t_2)E^{(+)}(\mathbf{r_1}, t_1)E^{(+)}(\mathbf{r_2}, t_2) )
    \end{aligned}\] 
    可定义二阶关联函数:\\ 
    $~ {\hspace*{2em}} G^{(2)}(\mathbf{r_1}, \mathbf{r_2}, \mathbf{r_3}, \mathbf{r_4}; t_1, t_2, t_3, t_4) = $ \[ ~ {\hspace*{3em}} Tr(\rho E^{(-)}(\mathbf{r_1}, t_1)E^{(-)}(\mathbf{r_2}, t_2)E^{(+)}(\mathbf{r_3}, t_3)E^{(+)}(\mathbf{r_4}, t_4) ) \]
    使用记号 $x_i= (\mathbf{r}_i, t_i)$, 简写为:
    \[ G^{(2)}(x_1, x_2; x_3,x_4) = Tr(\rho E^{(-)}(x_1)E^{(-)}(x_2)E^{(+)}(x_3)E^{(+)}(x_4)) \]
\end{frame}

\begin{frame} 
      \frametitle{}
      {\Bullet} 描述$n$个探测器,则使用$n$阶关联函数, 定义为:\\ 
        $~ {\hspace*{2em}} G^{(n)}(x_1,x_2,\cdots, x_n;x_{n+1},\cdots, x_{2n})= $ 
        \[ Tr(\rho E^{(-)}(x_1)E^{(-)}(x_2)\cdots E^{(-)}(x_n) E^{(+)}(x_1)\cdots E^{(+)}(x_n)) \]
        若令 $\tau = t_2-t_1$
    \[\begin{aligned}
        G^{(1)}(\mathbf{r_1}, \mathbf{r_2},; t_1, t_2) &= G^{(1)}(\mathbf{r_1}, \mathbf{r_2},; \tau) \\ 
        &= Tr( \rho E^{(-)}(\mathbf{r_1})E^{(+)}(\mathbf{r_2}, \tau)) 
    \end{aligned} \]
    若令 $\tau_1 = t_3-t_1; \tau_2 = t_4-t_2$ \\
    \[\begin{aligned}
    G^{(2)}(\mathbf{r_1}, \mathbf{r_2}, \mathbf{r_3}, \mathbf{r_4}; & t_1, t_2, t_3, t_4) \\ 
    &= G^{(2)}(\mathbf{r_1}, \mathbf{r_2}, \mathbf{r_3}, \mathbf{r_4};\tau_1,\tau_2 )  \\
    &=  Tr(\rho E^{(-)}(\mathbf{r_1})E^{(-)}(\mathbf{r_2})E^{(+)}(\mathbf{r_3}, \tau_1)E^{(+)}(\mathbf{r_4}, \tau_2) ) 
   \end{aligned} \]
   \end{frame}

   \begin{frame} 
    \frametitle{}
    归一化的场关联函数:
        \[g^{(1)} (x_1,x_2)= \frac{G^{(1)}(x_1; x_2) }{[G^{(1)}(x_1) G^{(1)}(x_2)]^{1/2} }\]
        \[g^{(2)} (x_1, x_2; x_3,x_4)= \frac{G^{(2)}(x_1,x_2; x_3, x_4) }{[G^{(1)}(x_1) G^{(1)}(x_2)G^{(1)}(x_3) G^{(1)}(x_4)]^{1/2} }\]
    ~\\ {\vspace*{1.3em}}
   * 一阶关联函数是一个探测器测得一个光子的概率\\ 
   * 二阶关联函数是二个探测器各测得一个光子的联合概率 \\ 
   * $n$阶关联函数是n个探测器各测得一个光子的联合概率 
   \end{frame}
\begin{frame} [label=current] 
\frametitle{}
    关联函数的产生湮灭算符表示 (平稳随机过程假设)
    \[ \begin{aligned}
        g^{(2)}(\tau) 
        &=  \frac{\left\langle n_1 (t)  n_2 (t+ \tau)\right\rangle}{\left\langle n_1 (t)\right\rangle \left\langle n_2 (t+ \tau)\right\rangle} \\ 
        &=  \frac{\left\langle a^{\dagger} (t)  a^{\dagger} (t+ \tau)\right\rangle}{\left\langle a^{\dagger} a \right\rangle ^2} \\
        g^{(1)}(\tau) &=  \frac{\left\langle a^{\dagger} (t)  a (t+ \tau)\right\rangle}{\left\langle a^{\dagger} a \right\rangle } \\
    \end{aligned}\] 
\end{frame}

\section{4. 应用实例~~反聚束}

   \begin{frame} 
    \frametitle{单光子干涉}
        单光子干涉实验是延迟选择实验的简化版\\ 
        装置如下.
          \begin{center}
               \includegraphics[width=0.4\textwidth]{figs/19.png}
          \end{center}
        实验现象: 通过改变光程差,可有效控制光子计数.
   \end{frame}

   \begin{frame} 
    \frametitle{}
    分析: \\ 
    (1) 这是单光子单模场, 分束器在$t$时刻对光子做第一次测量, 关联函数为:
    \[ \begin{aligned}
        g^{(1)} (\mathbf{r},t) &= \frac{G^{(1)}(x_1) }{[G^{(1)}(x_1)]^{1/2} } \\
        G^{(1)}(x_1) &= \omega_1(\mathbf{r}, t)  \\
        &= Tr( \rho E^{(-)}(\mathbf{r}, t)E^{(+)}(\mathbf{r}, t))\\ 
        &= \lcr{1}{a^{\dagger}(t)a(t)}{1}\\
        &=1\\ 
        g^{(1)} (\mathbf{r},t) &=1
    \end{aligned}\] 
   \end{frame}

   \begin{frame} 
    \frametitle{}
    (2) 探测器在$t$时刻对光子做测量, 光走第一条光路, 时间为 $t-s_1/c$, 光走第二条光路, 时间为 $t-s_2/c$,光程差为$\delta s = s_2-s_1$, 关联函数为可写为:
    \[ \begin{aligned}
        g^{(1)} (\delta s) &= \frac{G^{(1)}(t-s_1/c, t-s_2/c ) }{[G^{(1)}(t-s_1/c) G^{(1)}(t-s_2/c ) ]^{1/2} } \\
        &= \frac{\left\langle a^{\dagger}(t-s_1/c) a(t-s_2/c) \right\rangle}{\left\langle a^{\dagger} a \right\rangle}  \\
        &= e^{i \omega  \frac{\delta s}{c}} \\
        &= \cos{\omega  \frac{\delta s}{c}}+ i \sin {\omega  \frac{\delta s}{c}}
    \end{aligned}\] 
    若 $ \delta s= n\lambda $ 时, 测得光子的概率为100\% \\ 
    若 $ \delta s= \frac{n}{2}\lambda $ 时, 测得光子的概率为0! \\
    * 说明单个光子的确同时走两条路径, 并与自己发生干涉. 这不可能用经典光学进行解释.也不可能用半经典光学进行解释!
   \end{frame}

   \begin{frame}
    \frametitle{ HB-T 符合计数测量}
           \begin{center}
                \includegraphics[width=0.5\textwidth]{figs/2022-05-09-13-44-06.png}
           \end{center}
        关心的是两个探测器测得的光强的相关性问题. 对应二阶关联函数$g^{(2)}$
   \end{frame}

   \begin{frame} 
    \frametitle{}
    \[\begin{aligned}
     g^{(2)} (t_1, t_2) &= \frac{G^{(2)}(t_1,t_2) }{[G^{(1)}(t_1) G^{(1)}(t_2)]^{1 /2} }\\ 
     &=  \frac{\left\langle a_3 ^{\dagger}(t_1)a_4^{\dagger}(t_2)a_3(t_1)a_4(t_2)  \right\rangle}{ \left\langle a_3^{\dagger}(t_1)a_3(t_1) \right\rangle  \left\langle a_4^{\dagger}(t_2)a_4(t_2) \right\rangle}  \\ 
     g^{(2)} (\tau) &= \frac{\left\langle a_3^{\dagger}(0)a_4^{\dagger}(\tau)a_3(0)a_4(\tau)  \right\rangle}{ \left\langle a_3^{\dagger}(0)a_3(0) \right\rangle  \left\langle a_4^{\dagger}(\tau)a_4(\tau) \right\rangle} \\ 
     g^{(2)} (0) &= \frac{\left\langle a_3^{\dagger}a_4^{\dagger}a_3a_4  \right\rangle}{ \left\langle a_3^{\dagger}a_3 \right\rangle  \left\langle a_4^{\dagger}a_4 \right\rangle} \\ 
    \end{aligned} \]
    下面的计算与输入的具体态相关. 
\end{frame}

\begin{frame} 
 \frametitle{}
 分束器处 in-场与out-场之间的场强关系:
 \[\begin{aligned}
    &\mathcal{E}_{3}=\left(\mathcal{E}_{1}-\mathcal{E}_{2}\right) / \sqrt{2} \\
    &\mathcal{E}_{4}=\left(\mathcal{E}_{1}+\mathcal{E}_{2}\right) / \sqrt{2}
    \end{aligned}\]
对应in-场与out-场的湮灭算符关系:
\[\begin{aligned}
    &a_{3}=\left(a_{1}-a_{2}\right) / \sqrt{2} \\
    &a_{4}=\left(a_{1}+a_{2}\right) / \sqrt{2}
    \end{aligned}\]
设1端的输入态函数为$\rs{\psi_1}$, 2端的输入态函数是真空态$\rs{0_2}$
总的输入态:
\[\rs{\Psi}= \rs{\psi_1}\rs{0_2} \]
\end{frame}

\begin{frame} 
 \frametitle{}
    \[\begin{aligned}
        \left\langle a_3^{\dagger}a_3 \right\rangle &=  \left\langle \psi_1|\ls{0_2}  a_3^{\dagger}a_3 |\rs{\psi_1}0_2\right\rangle \\ 
        &=\frac{1}{2}\left\langle \psi_1|\ls{0_2} \left(a^{\dagger}_{1}-a^{\dagger} _{2}\right)\left(a_{1}-a_{2}\right) |\rs{\psi_1}0_2\right\rangle  \\ 
        &=\frac{1}{2}\left\langle \psi_1|\ls{0_2} \left(a^{\dagger}_{1}a_{1} - a^{\dagger}_{1}a_{2} - a^{\dagger}_{2}a_{1} + a^{\dagger}_{2}a_{2} \right) \rs{\psi_1}|0_2\right\rangle  \\
        &= \frac{1}{2}\left\langle \psi_1| a^{\dagger}_{1}a_{1}  |\psi_1\right\rangle \\ 
        &= \frac{1}{2}\left\langle \psi_1| n_1 |\psi_1\right\rangle \\ 
    \left\langle a_4^{\dagger}a_4 \right\rangle &= \frac{1}{2}\left\langle \psi_1| n_1 |\psi_1\right\rangle       
    \end{aligned} \]
\end{frame}

\begin{frame} 
 \frametitle{}
 \[\begin{aligned}
    \left\langle a_3^{\dagger}a_4^{\dagger}a_3 a_4\right\rangle &= \frac{1}{4}\left\langle \Psi|(a^{\dagger} _{1}-a^{\dagger}_{2}) (a^{\dagger}_{1} + a^{\dagger}_{2})(a_{1}-a_{2})(a_{1}+a_{2})   |  \Psi \right\rangle \\ 
    &= \frac{1}{4}\left\langle \psi_1|a^{\dagger} _{1}a^{\dagger} _{1}a_{1}a_{1}   |  \psi_1 \right\rangle \\ 
    &= \frac{1}{4}\left\langle \psi_1|a^{\dagger} _{1}(a_{1}a^{\dagger} _{1} - 1)a_{1}   |  \psi_1 \right\rangle \\ 
    &= \frac{1}{4}\left\langle \psi_1| n_1(n_1-1) |  \psi_1 \right\rangle
\end{aligned} \]  
\end{frame}

\begin{frame} 
 \frametitle{}
 \[\begin{aligned}
    g^{(2)} (0) &= \frac{\left\langle a_3^{\dagger}a_4^{\dagger}a_3a_4  \right\rangle}{ \left\langle a_3^{\dagger}a_3 \right\rangle  \left\langle a_4^{\dagger}a_4 \right\rangle} \\
    &=  \frac{\frac{1}{4}\left\langle \psi_1| n_1(n_1 -1) |  \psi_1 \right\rangle}{\frac{1}{2}\left\langle \psi_1| n_1 |\psi_1\right\rangle \frac{1}{2}\left\langle \psi_1| n_1 |\psi_1\right\rangle } \\ 
    &=  \frac{\left\langle \psi_1| n_1(n_1-1) |  \psi_1 \right\rangle}{\left\langle \psi_1| n_1 |\psi_1\right\rangle^2 } \\ 
    &= \frac{\left\langle \hat{n}_1(\hat{n}_1-1)   \right\rangle}{\left\langle \hat{n}_1 \right\rangle^2 }
   \end{aligned} \] 
\end{frame}

\begin{frame} 
      \frametitle{}
    讨论:(1) 若是单光子源, 光场是$Fock$场
    \begin{enumerate}
        \item 若$\rs{\psi_1  }=\rs{0} $ \\ 
         \[g^{(2)}(0) = \frac{\left\langle 0| n_1(n_1-1) |  0 \right\rangle}{\left\langle 0| n_1 |0\right\rangle^2 } =0\]

        \item 若$\rs{\psi_1  }=\rs{n} $  \\ 
         \[g^{(2)}(0)= \frac{\left\langle n| n_1(n_1-1) |  n \right\rangle}{\left\langle n| n_1 |n \right\rangle^2 }=\frac{n(n-1)}{n^2}=1-\dfrac{1}{n}<1\]
    \end{enumerate}
   \end{frame}

   \begin{frame}
    \frametitle{}
    讨论:(2) 若是理想激光光源, 光场是相干场, 上式要对$\rs{\alpha}$态求平均. 
    \[ g^{(2)}(0) = \frac{\left\langle \alpha | a^{\dagger} _{1}a^{\dagger} _{1}a_{1}a_{1} | \alpha  \right\rangle}{ \left\langle \alpha |a^{\dagger} _{1} | \alpha\right\rangle ^2} =  \frac{\alpha^* \alpha^* \alpha \alpha }{(\alpha^* \alpha)^2 } = 1 \] 
    讨论:(3) 若是混沌光束(非理想激光光源)
    \[\begin{aligned}
        g^{(2)}(0) = \frac{\left\langle n_1(n_1-1) \right\rangle}{\left\langle n_1 \right\rangle^2 } &=\frac{\left\langle  (a^{\dagger}a)^2 \right\rangle - \left\langle a^{\dagger}a  \right\rangle}{ \left\langle  a^{\dagger} a \right\rangle ^2} \\
        &=\frac{(\Delta n)^2 + \left\langle  a^{\dagger} a \right\rangle ^2 - \left\langle a^{\dagger}a  \right\rangle}{ \left\langle  a^{\dagger} a \right\rangle ^2} \\
        &= 1 + \frac{(\Delta n)^2 - \left\langle a^{\dagger}a  \right\rangle}{ \left\langle  a^{\dagger} a \right\rangle ^2} \\ 
        &= 1 + \frac{\left\langle  a^{\dagger} a \right\rangle ^2 + \left\langle a^{\dagger}a  \right\rangle - \left\langle a^{\dagger}a  \right\rangle}{ \left\langle  a^{\dagger} a \right\rangle ^2} =2     
    \end{aligned} \]
\end{frame}

%\begin{frame} 
%      \frametitle{}
%      
%    \[\begin{aligned}
%        g^{(2)}(0)&= \frac{\left\langle a^{\dagger}a^%{\dagger}a a   \right\rangle}{ \left\langle  a^%{\dagger} a \right\rangle ^2} \\ 
%        &=\frac{\left\langle a^{\dagger}(a^{\dagger}a) %a   \right\rangle}{ \left\langle  a^{\dagger} %a \right\rangle ^2} \\
%        &=\frac{\left\langle a^{\dagger}(aa^{\dagger}%-1) a   \right\rangle}{ \left\langle  a^%{\dagger} a \right\rangle ^2} \\
%        &= \frac{\left\langle  a^{\dagger}aa^{\dagger} %a -a^{\dagger}a  \right\rangle}{ \left\langle  %a^{\dagger} a \right\rangle ^2} 
%    \end{aligned} \]
%    \end{frame}
%    
%    \begin{frame} 
%          \frametitle{}
%    \[\begin{aligned}
%        g^{(2)}(0)
%        &=\frac{\left\langle  (a^{\dagger}a)^2 %\right\rangle - \left\langle a^{\dagger}a  %\right\rangle}{ \left\langle  a^{\dagger} a %\right\rangle ^2} \\
%        &=\frac{(\Delta n)^2 + \left\langle  a^%{\dagger} a \right\rangle ^2 - \left\langle a^%{\dagger}a  \right\rangle}{ \left\langle  a^%{\dagger} a \right\rangle ^2} \\
%        &= 1 + \frac{(\Delta n)^2 - \left\langle a^%{\dagger}a  \right\rangle}{ \left\langle  a^%{\dagger} a \right\rangle ^2} \\ 
%        &= 1 + \frac{\left\langle  a^{\dagger} a %\right\rangle ^2 + \left\langle a^{\dagger}a  %\right\rangle - \left\langle a^{\dagger}a  %\right\rangle}{ \left\langle  a^{\dagger} a %\right\rangle ^2} \\
%        &=2     
%    \end{aligned} \]
%   \end{frame}

\begin{frame} 
    \frametitle{小结}
    基于二阶关联函数对光场分类:
    \begin{itemize}
        \item 反聚束光(antibunched light): $g^{(2)}(0)<1$, 非经典光, 单光子光源, 亚泊松分布 
        \item 相干光 (Coherent light) : $g^{(2)}(0)=1$, 激光, 泊松分布 (随机分布)
        \item 聚束光 (bunched light) : $g^{(2)}(0)>1$ , 经典光, 混沌光束, 过泊松分布 
    \end{itemize}
      \begin{center}
           \includegraphics[width=0.4\textwidth]{figs/2022-05-09-10-30-03.png}
      \end{center}       
\end{frame}

\begin{frame}  
    \frametitle{}
    光强与光子聚束 
      \begin{center}
           \includegraphics[width=0.5\textwidth]{figs/2022-05-09-11-42-49.png}
      \end{center}
\end{frame}

\begin{frame}  
    \frametitle{}
    二阶关联函数随时间的变化关系
      \begin{center}
           \includegraphics[width=0.8\textwidth]{figs/21.png}
      \end{center}
\end{frame}

\begin{frame} 
    \frametitle{反聚束光实验制备}
    ~ \\
    \begin{enumerate}
        \item 单原子
        \item 掺杂在晶体中的荧光染料分子
        \item 半导体量子点
        \item 金刚石色心
    \end{enumerate}
          \begin{center}
               \includegraphics[width=0.43\textwidth]{figs/2022-05-09-12-08-59.png}
          \end{center}
\end{frame}

\begin{frame} 
\frametitle{HB-T 星光测量}
     \[ \begin{aligned}
        G^{(2)}&(x_1, x_2; x_3,x_4) = \left\langle E^{(-)}(x_1)E^{(-)}(x_2)E^{(+)}(x_3)E^{(+)}(x_4) \right\rangle \\
        &= \left\langle 1_k, 1_{k'} \left| E^{(-)}(x_1)E^{(-)}(x_2)E^{(+)}(x_3)E^{(+)}(x_4) \right| 1_{k'}, 1_k \right\rangle \\
        &= \sum_{\{ n\}} \left\langle 1_k, 1_{k'} \left| E^{(-)}(x_1)E^{(-)}(x_2)  \left| \{ n\} \left\rangle  \right\langle \{ n\} \right| E^{(+)}(x_3)E^{(+)}(x_4) \right| 1_{k'}, 1_k \right\rangle \\
        &= \left\langle 1_k, 1_{k'} \left| E^{(-)}(x_1)E^{(-)}(x_2)  \left| 0 \left\rangle  \right\langle 0 \right| E^{(+)}(x_3)E^{(+)}(x_4) \right| 1_{k'}, 1_k \right\rangle \\ 
        &= \left\langle 0 \left| E^{(+)}(x_1)E^{(+)}(x_2)  \left| 1_k, 1_{k'}  \left\rangle^*  \right\langle 0 \right| E^{(+)}(x_3)E^{(+)}(x_4) \right| 1_{k'}, 1_k \right\rangle
     \end{aligned}\] 
    代入:
    \[\hat{E}^{(+)} (x_i) + \hat{E}^{(-)}(x_i) =\epsilon_k a_k e^{-i(v_k t - \mathbf{k}\cdot \mathbf{r}_i)} + \epsilon_k a^{\dagger} _k e^{i(v_k t - \mathbf{k}\cdot \mathbf{r}_i)}\]
    进行计算, 并令 $x_{1}=x_3, x_2=x_4$, 可得: 
    \[ G^{(2)} = 2 \left|\epsilon_k\right|^4 \left\{ 1+ \cos (\mathbf{k} - \mathbf{k}')\cdot (\mathbf{r}- \mathbf{r}')  \right\}\]
\end{frame}


\begin{frame} 
\frametitle{}
     分析: \\ 
     \[ G^{(2)} = 2 \left|\epsilon_k\right|^4 \left\{ 1+ \cos (\mathbf{k} - \mathbf{k}')\cdot (\mathbf{r}- \mathbf{r}')  \right\}\]
     (1) 干涉项源于夹角(对应恒星直径) \\ 
     \[\cos (\mathbf{k} - \mathbf{k}')\cdot (\mathbf{r}- \mathbf{r}') \]
     (2) 公式中没有敏感项
     \[ \cos\left[ (\mathbf{k} + \mathbf{k}') \cdot \mathbf{r}_0 \right] \]
     因此可以测得更大更远的恒星星光 \\ 
     (3) 公式可写成平均粒子数形式(设两光束光强相同)
     \[ G^{(2)} = 2 \left|\epsilon_k\right|^4 \left\{ \left|n^2\right| - \left|n\right|+ \left| n \right|^2\cos (\mathbf{k} - \mathbf{k}')\cdot (\mathbf{r}- \mathbf{r}')  \right\}\]
     光强关联的本质在于光子数均值及涨落的关联 \\
     (4) 所有的HB-T测量都是基于二阶关联函数的测量
\end{frame}

\begin{frame} 
\frametitle{}
     (5) 对于两电子(费米子), 其反对易关系导致二阶关联函数取如下形式:
     \[ G^{(2)} = 2 \left|\epsilon_k\right|^4 \left\{ 1- \cos (\mathbf{k} - \mathbf{k}')\cdot (\mathbf{r}- \mathbf{r}')  \right\}\]
     两电子之间也存在二阶关联.
\end{frame}
%%%%%%%%%%%%%%%%%%%%%%%%%%%%%%%%%%%%%%%%%%%%%%%%%%%%%%%%%%%%%%%%%%%
\begin{frame} 
    \frametitle{课外作业}
    对于热辐射场, 混沌光束, 激光, 压缩光, Fock光, 真空态
    \begin{enumerate}
        \item 写出 $\overline{n}$ 和 $\overline{n^2}$
        \item 写出分布函数 $P(n)$
        \item 写出二阶关联函数 $g^{(2)}(0)$
    \end{enumerate}
\end{frame}
%%%%%%%%%%%%%%%%%%%%%%%%%%%%%%%%%%%%%%%%%%%%%%%%%%%%%%%%%%%%%%%%%%% % 光场统计 关联函数
%\include{lecture_13and14} % 光场的表示
%\include{lecture_15and16} % 光与原子的相互作用
\begin{frame}
    \frametitle{前情回顾}
    \begin{itemize}
        \item 光场量子化~~真空态~~数态~~产生湮灭算符 
        \item 相干态~~压缩态~~辐射场 
        \item 光子计数 ~~ 关联函数~~反聚束
        \item 光场表象
        \item 光与原子相互作用
    \end{itemize}     
\end{frame}

%%%%%%%%%%%%%%%%%%%%%%%%%%%%%%%%%%
\begin{frame} [plain]
    \frametitle{}
    \Background[1] 
    \begin{center}
    {\huge 第17-18讲:光腔中的原子}
    \end{center}  
    \addtocounter{framenumber}{-1}   
\end{frame}


\begin{frame} 
\frametitle{引入}
  \begin{center}
       \includegraphics[width=0.6\textwidth]{figs/25.png}
  \end{center}
  想要观察并利用拉比振荡, 必须增强光与原子之间的耦合 $g$. 腔场可最大限度地减少作用体积,增强耦合.
\end{frame}

\section{1. 光学谐振腔}

\begin{frame} 
\frametitle{}   
{\Bullet}光学谐振腔具有选频功能.当不同频率的入射波在腔镜上来回反射时,同频的反射波及入射波之间会相互干涉,腔内稳定传播的都是干涉相长的波.
\begin{columns}[T,onlytextwidth]
    \column{0.59\textwidth}
     考虑fp干涉仪型平面光腔, 透射系数
     \[ T = \frac{1}{1+(4F^2/\pi^2)\sin ^2(\varphi /2)}\]
    式中, 往返相移
    \[ \varphi = \frac{4\pi n L_{cav}}{\lambda}\]
    空腔的精细度
    \[ F = \frac{\pi (R_1 R_2)^{\frac{1}{4}}}{1-\sqrt{R_1 R_2}}\]
    \column{0.4\textwidth}
        \begin{center}
             \includegraphics[width=0.9\textwidth]{figs/2022-05-22-23-47-55.png}
        \end{center}
  \end{columns}
\end{frame}

  \begin{frame} 
  \frametitle{}
  \begin{columns}[T,onlytextwidth]
    \column{0.59\textwidth}
       共振透射 ($T=1$) 条件
       \[ \varphi =\frac{4\pi n L_{cav}}{\lambda} = 2 m \pi, \quad m=0, 1, 2, \cdots  \]
       \[L_{cav} = m \frac{\lambda}{2n}\]
       共振模
       \[ \omega_m = m \frac{\pi c }{n L_{cav}}\]
       谱宽
       \[ \Delta \omega = \frac{\pi c }{n F L_{cav}}\]
       品质因数(Q值)
       \[ Q= \frac{\omega}{\Delta \omega }\]
       \column{0.4\textwidth}
       \begin{center}
            \includegraphics[width=0.9\textwidth]{figs/2022-05-23-00-15-53.png}
       \end{center}
    \end{columns}
  \end{frame}

  \begin{frame} 
  \frametitle{}
    光子衰减率(损失率)
  \[ \boxed{\kappa \equiv  \frac{1}{\tau_{cav}} = \Delta \omega } \]
       光腔共振模的谱宽由光子衰减率决定 \\ {\vspace*{1.3em}}
       原子光谱的谱宽由原子的自发发射率决定.
       \[ \boxed{A_{21} \equiv  \frac{1}{\tau_{a}} = \Delta \omega } \]
  \end{frame}

  \section{2. 原子与腔的相互作用}

\begin{frame} 
\frametitle{}
     
\end{frame}  % 光腔中的原子
%--------------------------------------%

%%%%%%%%%%%%%%%%%%%%%%%%%%%%%%%%%%%%%%%%

\begin{frame}[plain]
    \Background[2] 
	\begin{center}
		{\huge \color{deepred} \textrm{Thanks for your attention!  \\ \vspace{1.0em}
         A \& Q}}
	\end{center}
    \addtocounter{framenumber}{-1} 
\end{frame}

%%%%%%%%%%%%%%%%%%%%%%%%%%%%%%%%%%%%%%%%%
\end{document}
%---------------THE END-----------------%
