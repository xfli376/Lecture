
\section{输运(热传导)方程}
\subsection{方程的建立}
\begin{example} %1
实验上发现,热量总是从温度高的地方传向温度低的地方,并发现有傅里叶热传导定律:
\begin{equation*}
q=-k\nabla u
\end{equation*}
式中,q是热流强度 (定义为单位时间通过单位横截面积的热量); k 是材料的导热系数 ;$\nabla $ 是梯度算子~$(\frac{\partial }{\partial x} +\frac{\partial }{\partial y} +\frac{\partial }{\partial z})$。\\
现对于介质中任意小体积($d\tau $),试建立温度函数u(x,y,z,t)所满足的方程 。\\
~~~\hspace*{\fill} \\

\end{example}
%\usetikzlibrary {3d} 
\tikzset{math3d/.style={x={(-0.353cm,-0.353cm)},z={(0cm,1cm)},y={(1cm,0cm)}}}
 \opencutright 
\def\windowpagestuff{\flushright 
\begin{tikzpicture}[math3d]
	\def\k{1.5}
	\draw [ ->] (-\k,-\k,-\k) --  (\k,-\k,-\k)  node[below] {$z$};
	\draw [ ->]  (-\k,-\k,-\k) -- (-\k, \k,-\k)   node[above] {$x$};
	\draw [ ->]  (-\k,-\k,-\k) -- (-\k,-\k, \k)    node[left] {$y$};
	 \def\a{0.5}
	\def\b{0.5}
	\def\c{0.5}
	\coordinate (A01) at ( \a, \b, \c);
	\coordinate (A02) at ( \a,-\b, \c);
	\coordinate (A03) at (-\a,-\b, \c);
	\coordinate (A04) at (-\a, \b, \c);
	\coordinate (A05) at ( \a, \b,-\c);
	\coordinate (A06) at ( \a,-\b,-\c);
	\coordinate (A07) at (-\a,-\b,-\c);
	\coordinate (A08) at (-\a, \b,-\c);
	\draw(A01)--(A02)--(A03)--(A04)--cycle;
	\draw(A06)--(A05)--(A08);
	\draw[dash dot](A06)--(A07)--(A08);
	\draw(A01)--(A05)(A02)--(A06)(A04)--(A08);
	\draw[dash dot](A03)--(A07);
    \draw[line width =1pt]  (-\k,-1,-\k) node[below] {$x$}-- (-\k, 0.0,-\k)  node[below] {$x+dx$}; 
\end{tikzpicture}}

\begin{cutout} {0}{8cm}{0pt}{6}
\begin{proof} 
建立如下坐标系:对于各向同性介质,\\
傅里叶热传导定律有分量形式 \\
	$~~~~~~q_{x_i}=-k \frac{\partial }{\partial x_{i}} u$ \\
考虑单位时间 x 方向的净流入:\\
  $\displaystyle \begin{array}{llll}
	&- (q_x|_{x+dx}-q_x|_x)dydz\\
	&=-\frac{\partial q_x }{\partial x}  dxdydz\\
	&= \frac{\partial }{\partial x} (k\frac{\partial u}{\partial x})dxdydz \\
\end{array}$ \\   
则总的净流入为:
\begin{equation*}
	[\frac{\partial }{\partial x} (k\frac{\partial u}{\partial x}) + \frac{\partial }{\partial y} (k\frac{\partial u}{\partial y}) + \frac{\partial }{\partial z} (k\frac{\partial u}{\partial z}) ]dxdydz 
\end{equation*}
\end{proof}
\end{cutout} 	
流入的热量导致介质温度发生变化(热量守恒定律)\\
\begin{equation*}
	c \rho \frac{\partial u}{\partial t}dxdydz=[\frac{\partial }{\partial x} (k\frac{\partial u}{\partial x}) + \frac{\partial }{\partial y} (k\frac{\partial u}{\partial y}) + \frac{\partial }{\partial z} (k\frac{\partial u}{\partial z}) ]dxdydz 
	\end{equation*}
其中c是比热,$\rho$ 是质量密度。 对于各向同性的均匀介质:
\begin{equation*}
	c \rho \frac{\partial u }{\partial t}=k[\frac{\partial }{\partial x} (\frac{\partial u}{\partial x}) + \frac{\partial }{\partial y} (\frac{\partial u}{\partial y}) + \frac{\partial }{\partial z} (\frac{\partial u}{\partial z}) ]	
\end{equation*}
\begin{equation*}
 \frac{\partial u }{\partial t}=\frac{k}{c\rho}[\frac{\partial }{\partial x} (\frac{\partial u}{\partial x}) + \frac{\partial }{\partial y} (\frac{\partial u}{\partial y}) + \frac{\partial }{\partial z} (\frac{\partial u}{\partial z}) ]	
\end{equation*}
\begin{equation*}
	u_t=a^2 [u_{xx}   +u_{yy}  +u_{zz}] 
\end{equation*}
\begin{equation*}
u_t=a^2 \nabla ^2 u = a^2 \triangle u	
\end{equation*}
\begin{note}
$\frac{\partial u }{\partial t}$ 是时间元(单位时间)升高的温度,$ \rho  dxdydz $ 是体系元质量 ,$c$ 是单位质量物体温度升高单位度所需热量。
\end{note}
对于一维导线:
{ $\displaystyle u_t= a^2 u_{xx}$ }  \\ 
如果有热源F(x,y,z,t),令 $f=\frac{F}{c\rho}$:	{$\displaystyle u_t= a^2 u_{xx}+f$ }  \\
如果时间足够长,温度应不再随时间变化 ($u_t =0$), \\ 
得无源Laplace 方程: { $\displaystyle   \nabla ^2 u =0$ }  \\ 
和有源Poisson 方程: { $\displaystyle   \nabla ^2 u =f$ }  \\ 

\begin{remark}
	传导方程描述了,热、电等传输的基本规律。
\end{remark}

\subsection{热传导方程的求解}
\begin{example} %2
对于有限长的导线,求解如下一维热传导方程的初边值问题  \\
 $\displaystyle \begin{cases}
 u_{t}=a^2u_{xx} ,~~~~ (0<x<l, t>0)\\
u(x,t)|_{t=0}= \psi (x)  \\
u(x,t)|_{x=0}= 0, ~~~  u(x,t)|_{x=l}= 0 
\end{cases}$ \\
\begin{proof} 方程可分离变量:
设 $\displaystyle  u(x,t)=T(t)X(x) $,代回方程,有  \\
 $\displaystyle  \begin{cases}
  T~^{'}(t)X(x) =a~^2 T(t)X~^{''}(x) \\ \vspace{0.3cm}
 \frac{T~^{'}}{a~^2 T}=\frac{X~^{''} }{X} =-\lambda 
\end{cases}$ \\
即偏微分方程转化为两常微分方程 \\
方程(I):\\
 $\displaystyle  \begin{cases}
	X~^{''} +\lambda X=0  ~~,~~ 0<x<l\\
   X(0)=0 ~,~X(l)=0
\end{cases}$ \\	
方程(II):\\
 $\displaystyle  \begin{cases}
	T~^{'} +\lambda {a~^2 T}=0 \\
	......
\end{cases}$ \\	

\begin{remark}
	热传导方程分离变量后,得到的是两个什么方程?
\end{remark}

方程(I)是固有值问题,...\\
固有值:$\displaystyle  \lambda~_n=\frac{n^2\pi~^2}{l~^2}$ \\ 
固有函数:{\large $\displaystyle  X~_n= \sin \frac{n\pi~}{l} x$}\\

解方程II : 	$\displaystyle  T~^{'} +\lambda {a~^2 T}=0 $ \\ 
代入$\lambda_n$, 得:
$\displaystyle  T~^{'} +\lambda~_n a~^2 ~T=0 $ \\
变形成:$\displaystyle  T~^{'} + r_nT=0 $ \\ 
这是衰减数学模型,有公式:
\begin{equation*}
	T= B_n exp(-rt)
\end{equation*}
得通解 :$\displaystyle T~_n=B_n  \exp(-(\frac{n\pi a}{l})^2 t)$ \\  \vspace{0.3cm}

原方程的基本解为:\\
   $\begin{array}{llll}
	u_n(x,t) &= T_n(t)X_n(x)\\
	&= B_n  \exp(-(\frac{n\pi a}{l})^2 t) \sin \frac{n\pi~}{l} x \\
\end{array}$ \\ 
 叠加解 (解函数):\\ 
 $ \displaystyle u(x,t)=\sum_{n=1}^{\infty } B_n  \exp(-(\frac{n\pi a}{l})^2 t) \sin \frac{n\pi~}{l} x$ \\
代入定解条件:\\ 
 $ \displaystyle u(x,0)= \psi(x)$ ~~=> ~~$\psi (x)=\sum_{n=1}^{\infty } B_n \sin \frac{ n\pi }{l} x$\\  
由傳里叶变换(非对称公式),得系数:\\  
$ \displaystyle B_n=  \frac{2}{l}\int_{0 }^{l}  \psi (x) \sin \frac{ n\pi }{l} x dx , ~~~ (n=1,2,3,...) $\\   
\end{proof}
\end{example}

\subsection{热传导方程初值问题}
\begin{example} %4
	求解如下初边值问题\\
	$\displaystyle  \begin{cases}
		u_{t} =u_{xx} ~~,~~ 0<x<L, t>0\\
		u(0,t) =0, u(L,t)=0 \\
		u(x,0) =x(L-x)
	\end{cases}$ \\	
	\begin{proof} 
	由边界条件知固有值和固有函数分别为:\\
	固有值:$\displaystyle  \lambda~_n=\frac{n^2\pi~^2}{l~^2 }= (\frac{n\pi }{L}) ^2$ \\ 
	固有函数:$\displaystyle  X~_n=\sin \frac{n\pi~}{l} x=\sin \frac{n\pi~}{L} x $\\
	解函数:\\ 
		$\displaystyle \begin{array}{llll}
			u(x,t)&=\sum_{n=1}^{\infty } B_n  \exp(-(\frac{n\pi a}{l})^2 t) \sin \frac{n\pi~}{l} x\\
			        &= \sum_{n=1}^{\infty } B_n  \exp(-(\frac{n\pi a}{L})^2 t) \sin \frac{n\pi~}{L} x \\
		\end{array}$ \\ 
	
		$\displaystyle \begin{array}{lllllllll}
		B_n&= \frac{2}{l}\int_{0 }^{l}  \psi (x) \sin \frac{ n\pi }{l} x dx  \\
		       &= \frac{2}{L}\int_{0 }^{L}  x(L-x) \sin \frac{ n\pi }{L} x dx  \\
		       &=\frac{2}{L} \times2\times (\frac{L2}{n\pi})^3  [1-\cos n \pi ]  \\
		       &= 4 \frac{L^2}{(n\pi)^3}[1-(-1)^n ]  \\
		\end{array}$ \\ 
		解函数:$\displaystyle  u(x,t) =  (\frac{4L^2}{\pi ^3}) \sum_{n=1}^{\infty} \frac{1}{n^3} [1-(-1)^n ]  \exp(-(\frac{n\pi a}{L})^2 t) \sin \frac{n\pi~}{L} x  $\\
	\end{proof}
\end{example}

\subsection{固有函数与边界条件的相关性}
固有函数的具体形式是与边界条件相关的,如果边界条件改变,则要先求对应的固有值和固有函数
\begin{example} %4
	求解如下初边值问题\\
	$\displaystyle  \begin{cases}
		u_{t} =u_{xx} ~~,~~ 0<x<l, t>0\\
		u_x (0,t) =0, u_x (l,t)=0 \\
		u(x,0) =\psi(x)
	\end{cases}$ \\	
\begin{proof} 
分离变量后,偏微分方程可转化为两常微分方程 \\
方程(I):\\
$\displaystyle  \begin{cases}
	X~^{''} +\lambda X=0  ~~,~~ 0<x<l\\
	X'(0)=0 ~,~X' (l)=0
\end{cases}$ \\	
方程(II):\\
$\displaystyle  \begin{cases}
	T~^{'} +\lambda {a~^2 T}=0 \\
	......
\end{cases}$ \\	
注意到方程(I)是导数边界条件,与原有的固值问题不同,先求方程(I)。根据以前的讨论,只有在
$\lambda >0 $ 即特征方程有虚根时,方程才有非零解。因此,方程的通解可写成:
\begin{equation*}
	X=A\cos \sqrt{\lambda} x + B \sin \sqrt{\lambda} x
\end{equation*}
\begin{equation*}
	X' (x)=\sqrt{\lambda} [-A\sin \sqrt{\lambda} x + B \cos \sqrt{\lambda} x]
\end{equation*}

代入边界条件(分别取$x=0, x=l$),得方程组 \\
$\left[
\begin{array}{lll}
	0&1\\
	-\sin( {\sqrt{\lambda}~l}) &\cos ({\sqrt{\lambda}~l})
\end{array}
\right]$
$\left[
\begin{array}{ll}
	A\\
	B
\end{array}
\right]$
=$\left[
\begin{array}{ll}
	0\\
	0
\end{array}
\right]$\\ 
由方程组有非零解的充要条件为系数矩阵行列式为零,即得$ \sin\sqrt{\lambda}~l =0$,得固有值:
\begin{equation*}
	\lambda_n =(\frac{n\pi}{l})^2 ~~,~~ (n=0,1,2,......)
\end{equation*}
代回方程组,得解待定系数:$ [A, B] ^T =[1, 0] ^T$,即通解为:
\begin{equation*}
	X_n=\cos \frac{n\pi}{l} x~,~~~  (n=0,1,2,......)
\end{equation*}
对应的级数解为:
\begin{equation*}
	u(x,t)=\sum_{n=0}^{\infty } B_n  \exp(-(\frac{n\pi a}{l})^2 t) \cos \frac{n\pi~}{l} x
\end{equation*}
系数计算公式为:
$\displaystyle  \begin{cases}
 B_0&= \frac{1}{l} \int_{0}^{l} \psi(x) dx \\
 B_n&= \frac{2}{l} \int_{0}^{l} \psi(x) \cos \frac{n\pi}{l} xdx ,~~ (n=1,2,......)
\end{cases}$ \\	
\end{proof}
\end{example}

\begin{remark}
	边界条件导致固有函数由sin变成了cos, 进而导致 n=0,变得有意义 。$X_0(x) = \cos \frac{n\pi}{l} =1$。
\end{remark}

\begin{example} %4
	求解如下初边值问题\\
	$\displaystyle  \begin{cases}
		u_{t} =u_{xx} ~~,~~ 0<x<\pi, t>0\\
		u_x (0,t) =0, u_x (l,t)=0 \\
		u(x,0) =x^2 (\pi-x)^2
	\end{cases}$ \\	
	\begin{proof} 
		注意到本方程是导数边界条件,由边界条件知固有值和固有函数分别为:\\
	固有值:$\displaystyle  \lambda~_n=\frac{n^2\pi~^2}{l~^2 }= (\frac{n\pi }{\pi}) ^2 = n^2$ \\ 
	固有函数:$\displaystyle  X~_n=\cos \frac{n\pi~}{l} x=\cos nx $\\

	解函数:\\ 
$\displaystyle \begin{array}{llll}
	u(x,t)&= \frac{B_0}{2} + \sum_{n=1}^{\infty } B_n  \exp(-(\frac{n\pi a}{l})^2 t) \cos \frac{n\pi~}{l} x\\
	        &= \frac{B_0}{2} + \sum_{n=1}^{\infty } B_n  \exp(-(na)^2 t) \cos n x \\
	\end{array}$ \\ 
	
	$\displaystyle \begin{array}{lllllllll}
	B_0&=\frac{1}{l} \int_{0}^{l} \psi(x) dx   \\
	&= \frac{1}{\pi} \int_{0}^{\pi}  x^2 (\pi-x)^2 dx  \\
	&=\frac{\pi ^4}{15}\\
	B_n&=\frac{2}{l} \int_{0}^{l} \psi(x) \cos \frac{n\pi}{l} xdx \\
	&= \frac{2}{\pi} \int_{0}^{\pi}   x^2 (\pi-x)^2   \cos nx dx \\
	&= \frac{2}{\pi} \times (-\frac{12 \pi}{n^4})  [\cos n\pi +1 ] \\
	&= -\frac{24}{n^4} [ (-1)^n +1 ] \\
\end{array}$ \\ 
解函数:$\displaystyle  u(x,t) =  \frac{\pi ^4}{30} -24 \sum_{n=1}^{\infty } \frac{(-1)^n +1 }{n^4}  \exp(-(na)^2 t) \cos n x$\\
	\end{proof}
\end{example}


\begin{remark}
定积分的计算:$B_n= \frac{2}{\pi} \int_{0}^{\pi}   x^2 (\pi-x)^2   \cos nx dx$ \\
令 $\psi (x) = x^2 (\pi-x)^2$ ,求导得:\\
		$\displaystyle \begin{array}{lllllllll}
		  \psi ' (x) &=  2x(2x - \pi )(x - \pi ) \\
		   \psi '' (x) &=  2\pi ^2 -12\pi x+12x^2 \\
		     \psi ''' (x) &=  24x -12 \pi \\
		     \psi^{(4)} (x) &=  24  \\
	\end{array}$ \\ 
令 $\nu^{(4)} (x) = \cos nx$ , 则:\\
	$\displaystyle \begin{array}{lllllllll}
		\nu ''' (x) &= \frac{1}{n} \sin nx \\
		\nu '' (x) &= -\frac{1}{n^2} \cos nx \\
		\nu ' (x) &= -\frac{1}{n^3} \sin nx \\
		\nu (x) &= \frac{1}{n^4} \cos nx \\
	\end{array}$ \\ 
我们有:\\
\begin{equation*}
	\int_{0}^{\pi}  \psi^{(4)} (x)  \nu (x)  dx = \frac{24}{n^4} \int_{0}^{\pi}  \cos nx dx =0
\end{equation*}
应用分部积分公式,有 \\
	$\displaystyle \begin{array}{lllllllll}
	\int_{0}^{\pi}  \psi (x) \nu^{(4)} (x)  dx &= [ \psi \nu''' - \psi' \nu'' + \psi'' \nu' -\psi''' \nu ]|_0 ^\pi +\int_{0}^{\pi}  \psi^{(4)} (x)  \nu (x)  dx  \\
	&=  [ \psi \nu''' - \psi' \nu'' + \psi'' \nu' -\psi''' \nu ]|_0 ^\pi 
	\end{array}$ \\ 
所以有
\begin{equation*}
	\int_{0}^{\pi}  \psi (x) \nu^{(4)} (x)  dx = [-\psi''' \nu ]|_0 ^\pi =  -\frac{12 \pi}{n^4}  [\cos n\pi +1 ]
\end{equation*}
\end{remark}

\subsection{作业:}
~~~\hspace*{\fill} \\
1、求解热传导方程初边值问题\\
	$\begin{array}{lllllllll}
	& \begin{cases}
		u_{t} =a^2u_{xx} ~~,~~ 0<x<l, t>0\\
		u(0,t) =u(l,t)=0 \\
		u(x,0) =x (l-x)
	    \end{cases}\\	
	&\begin{cases}
		u_{t} =a^2u_{xx} ~~,~~ 0<x<l, t>0\\
		u_x(0,t) =u_x(l,t)=0 \\
		u(x,0) =x(l-x/2)
	\end{cases} \\	
\end{array}$ \\ 

 ~~~\hspace*{\fill} \\
2、求固有值问题\\
	$\begin{array}{lllllllll}
	& \begin{cases}
		X'' (x)  + \lambda X =0   ~~,~~ 0<x<L\\
        X' (0) =0, X (L) =0
	\end{cases}\\	
	&\begin{cases}
		X'' (x)  + \lambda X =0   ~~,~~ 0<x<L\\
       X (0) =0, X' (L) =0
	\end{cases} \\	
\end{array}$ \\ 
3. 什么是固有值?有何用处?
\\
4. 什么是固有函数?与固有值有何关系? 
\\
5. 分离变量法的数学思想是什么?
\\
6. 什么是叠加原理?与分离变量法有何关系
\\
7. 正交性是什么意思?有何用处?
\\
8. 较复杂的分部积分法怎么用?
\\

