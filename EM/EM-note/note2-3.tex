\section{拉普拉斯(场)方程}
\subsection{方程的建立}
\begin{example} %1
对于位于原点的质量为M的质点,试建立其引力势函数
u(x,y,z,t) 所满足的方程 .
\end{example}
%%\usetikzlibrary {3d} 
%\tikzset{math3d/.style={x={(-0.353cm,-0.353cm)},z={(0cm,1cm)},y={(1cm,0cm)}}}
% \opencutright 
%\def\windowpagestuff{\flushright 
\begin{tikzpicture}
	\def\k{1.5}
	\draw [ ->] (0,0,0) --  (0,0,2.5)  node[below] {$z$};
	\draw [ ->]  (0,0,0) -- (\k, 0,0)   node[right] {$x$};
	\draw [ ->]  (0,0,0) -- (0,\k, 0)    node[right] {$y$};
	\fill[black!100] (0,0,0) node[right]{$M$} circle(0.8ex);
		\begin{scope}[canvas is zy plane at x=0] \draw (0,0) circle (1cm);
			\draw (-1,0) -- (1,0) (0,-1) -- (0,1);
		\end{scope};
		\begin{scope}[canvas is zx plane at y=0] \draw (0,0) circle (1cm);
			\draw (-1,0) -- (1,0) (0,-1) -- (0,1);
		\end{scope};
		\begin{scope}[canvas is xy plane at z=0] \draw (0,0) circle (1cm);
			\draw (-1,0) -- (1,0) (0,-1) -- (0,1);
		\end{scope};
\end{tikzpicture}
%\begin{cutout} {0}{8cm}{0pt}{6}
\begin{proof} 
建立如图坐标系后,在空间任一点(x,y,z)放置试验质点m, m感受的力为:\\
{	$\displaystyle  \overrightarrow{F} =-G\frac{Mm}{r^3} \overrightarrow {r} $ }  ~~,~~ $r=\sqrt{x^2+y^2+z^2}$\\ 
说明M激发的引力场场强为 \\
{ $  \displaystyle  \overrightarrow{A} =\frac{GM}{r^3} \overrightarrow{r} $ }\\
取无穷远处场强为零,则引力势为\\
$\displaystyle  u = -\int_{r}^{\infty} \overrightarrow{A}\cdot d \overrightarrow{r} =- \int_{r}^{\infty} \frac{GM}{r^2} dr =- \frac{GM}{r} $ \\
即有: $\overrightarrow{A} =-\nabla u$ \\
封闭球面S内的质量通量为 \\
 $\displaystyle \oint_{S} \overrightarrow{A} \cdot d \overrightarrow{S} = \frac{GM}{r^2} 4\pi r^2 =\int  4\pi G  \rho d\tau$  \\
 由高斯定理可知:\\
 $ \displaystyle \oint_{S} \overrightarrow{A} \cdot d \overrightarrow{S} =\int  \nabla \cdot \overrightarrow{A} d\tau $\\
 因此: 
 \begin{equation*}
 	\nabla \cdot \overrightarrow{A} = 4\pi G \rho
 \end{equation*}
由于 $\nabla \cdot \overrightarrow{A} = \nabla \cdot \left(-\nabla u\right)= -\nabla ^2 u$ \\
得泊松方程:
 \begin{equation*}
\nabla ^2 u= -4\pi G \rho
\end{equation*}
 对于无源区域,得拉普拉斯方程
 \begin{equation*}
\nabla ^2  u =0 
\end{equation*}
定义拉普拉斯算子:
 \begin{equation*}
\triangle  = \nabla ^2 = \frac{\partial ^2}{\partial x^2} +\frac{\partial^2 }{\partial y^2} +\frac{\partial^2  }{\partial z^2}
\end{equation*}
\end{proof} 

\begin{remark}
拉普拉斯(场)方程是物理学描述各种势场的基本方程。
\end{remark}

\subsection{矩形区域上拉普拉斯方程的求解}
拉普拉斯方程(泊松方程)是描述各种场的基本方程。求解拉普拉斯方程是数理方程的一个基本任务。
\begin{example} %2
 对于矩形区域 ,  求解有如下边值问题 \\
 $\displaystyle \begin{cases}
u_{xx} +u_{yy} =0 ,~~~~ (0<x<a, 0<y<b)\\
u(x,0)= f_1 (x) ,  u(x,b)= f_2 (x) \\
u(0,y)= g_1 (y) ,  u(a,y)= g_2 (y) 
\end{cases}$ \\
\end{example}

\begin{proof} 这是第一类边界条件,可将问题分解成两个具有零边界条件的问题 \\
$\displaystyle \begin{cases}
	u_{xx} +u_{yy} =0 ,~~~~ (0<x<a, 0<y<b)\\
	u(x,0)= 0,  u(x,b)= 0 \\
	u(0,y)= g_1 (y) ,  u(a,y)= g_2 (y) 
\end{cases}$ \\
$\displaystyle \begin{cases}
	u_{xx} +u_{yy} =0 ,~~~~ (0<x<a, 0<y<b)\\
	u(x,0)= f_1 (x) ,  u(x,b)= f_2 (x) \\
	u(0,y)= 0,  u(a,y)= 0 
\end{cases}$ \\
同理,还可以再分解,变成四个问题。\\
$\displaystyle \begin{cases}
	u_{xx} +u_{yy} =0 ,~~~~ (0<x<a, 0<y<b)\\
	u(x,0)= 0,  u(x,b)= 0 \\
	u(0,y)= g_1 (y) ,  u(a,y)= 0
\end{cases}$ \\
$\displaystyle \begin{cases}
	u_{xx} +u_{yy} =0 ,~~~~ (0<x<a, 0<y<b)\\
	u(x,0)= f_1 (x) ,  u(x,b)= 0 \\
	u(0,y)= 0,  u(a,y)= 0 
\end{cases}$ \\
$\displaystyle \begin{cases}
	u_{xx} +u_{yy} =0 ,~~~~ (0<x<a, 0<y<b)\\
	u(x,0)= 0,  u(x,b)= 0 \\
	u(0,y)= 0,  u(a,y)= g_2 (y) 
\end{cases}$ \\
$\displaystyle \begin{cases}
	u_{xx} +u_{yy} =0 ,~~~~ (0<x<a, 0<y<b)\\
	u(x,0)= 0,  u(x,b)= f_2 (x) \\
	u(0,y)= 0,  u(a,y)= 0 
\end{cases}$ \\
\end{proof}

\begin{example} %3
 求解有如下边值问题 \\
$\displaystyle \begin{cases}
	u_{xx} +u_{yy} =0 ,~~~~ (0<x<1, 0<y<1)\\
	u(x,0)= 0,  u(x,1)= 0 \\
	u(0,y)= g_1 (y) =\sin \pi y,  u(1,y)= 0
\end{cases}$ \\
\end{example}
\begin{proof}
设有 	$\displaystyle  u(x,y)=X(x) Y(y)$ ,代回原方程,得
 \begin{equation*}
X~^{''}(x)Y(y) +X(x)Y~^{''}=0
\end{equation*}
 \begin{equation*}
-\frac{X~^{''}}{X}=\frac{Y~^{''} }{Y} =-\lambda 
\end{equation*}
得两个常微分方程:\\ 
方程(I):\\
$\displaystyle  \begin{cases}
	Y~^{''} +\lambda Y=0  ~~,~~ 0<y<1\\
	Y(0)=0 ~~,~~Y(1)=0 
\end{cases}$ \\	
方程(II):\\
$\displaystyle  \begin{cases}
	X~^{'} -\lambda X=0  ~~,~~ 0<x<1 \\
	X(0)=\sin \pi y~~,~~X(1)=0 
\end{cases}$ \\	

方程(I)是固有值问题,\\
固有值:$\displaystyle  \lambda~_n=\frac{n^2\pi~^2}{l~^2} =n^2\pi~^2$ \\ 
固有函数: $\displaystyle  Y~_n= \sin \frac{n\pi~}{l} y = \sin n \pi y $ \\

求解方程II : 	$\displaystyle  X~^{''} -\lambda {Y}=0 $ \\ 
代入$\lambda_n$, 得:
$\displaystyle  X~^{''} - n^2\pi~^2~X=0 $ \\
特征方程有两相异实根,方程的通解为: \\ 
{ $\displaystyle  X_n(x)=C_n exp(n\pi x )+ D_n exp(-n\pi x )$}\\ 

结合(I)(II),原方程的解为:\\ 
	$\begin{array}{llll}
		u_n(x,y) &=& [C_n exp(n\pi x )+ D_n exp(-n\pi x )] \sin (n \pi y)  \\ 
		&=& [a_n \cosh (n\pi x )+ b_n \sinh(n\pi x ) ]\sin (n \pi y)  \\ 
		u(x, y)    &=& \sum_{n=1}^{\infty }  [a_n \cosh (n\pi x )+ b_n \sinh (n\pi x ) ] \sin (n \pi y)  
	\end{array}$ \\
 
 代入 定解条件:$ u(0,y)=0$, 得 \\
  $ \sum_{n=1}^{\infty }  a_n  \sin (n \pi y) =0 $ ,=> $ a_n=0$ \\
 因此:\\
  $	u(x,y)    = \sum_{n=1}^{\infty }  b_n \sinh (n\pi x )  \sin (n \pi y)  $ \\ 
  代入 定解条件:$u(1,y) = \sin \pi y $, 得
   \begin{equation*}
  u(1,y)    = \sum_{n=1}^{\infty }  b_n \sinh (n\pi )  \sin (n \pi y)  = \sin \pi y
  \end{equation*}
  由正交性,得:$ b_1\sinh\pi =1~~,~~ b_n=0~,~ (n>1)$	\\ 
  原方程得解:
  \begin{equation*}
  u(x,y)    = \frac{\sinh \pi x}{\sinh \pi}  \sin ( \pi y) 
  \end{equation*}
\end{proof}  

\begin{remark}	
双曲函数: 
	$\displaystyle \begin{cases}
		&\sinh(x) = -i \sin(ix) = \frac{e^x -e^{-x}}{2} \\
		&\cosh(x) = \cos(ix) = \frac{e^x +e^{-x}}{2} 
	\end{cases}$ 
\end{remark}

\begin{tikzpicture} 
\centering
% draw the axis
\draw[eaxis] (-2,0) -- (2,0) node[below] {$x$};
\draw[eaxis] (0,-2) -- (0,4) node[left] {$y$};
\draw[elegant, orange, domain= -2:2]  plot (\x, {  exp(\x) * 0.5  } ) ;
\draw[elegant, orange]  (-2,0) node[above] {$y=\frac{e^ x}{2}$};
\draw[elegant, blue, domain= -2:2]  plot (\x, {  exp(-\x) * 0.5  } );
\draw[elegant, blue]  (2,0) node[above] {$y=\frac{e^ {-x}}{2}$};
\draw[elegant, red, domain= -1.4:2]  plot (\x, {  exp(\x) * 0.5 -  exp(-\x) * 0.5 } ) ;
\draw[elegant, red]  (3,2) node[above] {$y=\sinh(x)$};
\draw[elegant, black, domain= -2:2]  plot (\x, {  exp(\x) * 0.5 +exp(-\x) * 0.5 } );
\draw[elegant, black]  (-0.8,3) node[above] {$y=\cosh(x)$};
\end{tikzpicture} 

\begin{remark}	
其他三个边值问题也需构造出 x或y 的固有值问题,原理同上。
\end{remark}

\subsection{圆域上拉普拉斯方程的求解}
直角坐标~ $(x,y,z): $	{ 	$ \displaystyle  \nabla ^{2}  = \frac{\partial ^2}{\partial x^2} +\frac{\partial^2 }{\partial y^2} +\frac{\partial^2  }{\partial z^2}$}\\ 	
球坐标~ $	(r,\theta, \varphi )$ :
{ 	$ \displaystyle  \nabla ^{2} =\frac{1}{r^2} \frac{\partial }{\partial r} (r^2\frac{\partial }{\partial r} )+
	\frac{1}{r^2 \sin \theta  } \frac{\partial }{\partial \theta } (\sin \theta \frac{\partial }{\partial \theta } )
	+\frac{1}{r^2 \sin \theta  } \frac{\partial^2}{\partial\varphi ^2}$ } \\ 	
 极坐标~ $	(r,\theta)$ :
 { 	$ \displaystyle  \nabla ^{2} =\frac{\partial ^2 }{\partial r^2 } +\frac{1}{r } \frac{\partial }{\partial r } +
	\frac{1}{r^2 } \frac{\partial ^2 }{\partial \theta ^2 } $ }\\ 	

\begin{example} %4
	求解如下边值问题\\
{ $  \displaystyle  \left \{ 
	\begin{array}{cc}
		\displaystyle {	\frac{\partial^2 u }{\partial r^2 } +\frac{1}{r } \frac{\partial u }{\partial r } +
			\frac{1}{r^2 } \frac{\partial ^2 u }{\partial \theta ^2
		} } =0, ~~ 0<r<r_0\\
		u(r_0,\theta )=f(\theta ) ,~~~~~~~~~ 0<\theta <2\pi 
	\end{array}
	\right. $}  
\end{example}

\begin{proof} 
方程可分离变量,令 $\displaystyle  u(r,\theta)=R(r) \Theta(\theta)$,代入原方程  \\ 
{ $\displaystyle  R''\Theta +\frac{1}{r^2} R\Theta '' +\frac{1}{r}R'\Theta=0 $} \\ 
{ $\displaystyle  \frac{r^2R''+rR'}{R}=-\frac{\Theta '' }{\Theta} =\lambda $} \\ 
分离变量后,得两常微分方程:\\ 
{I、 $\displaystyle 	\Theta '' + \lambda \theta =0 $ } \\ 
{ 定解条件:$\displaystyle 	\Theta(\theta +2 \pi )=\Theta (\theta)  $ } \\ 
{II、$\displaystyle  r^2 R'' +r R' -\lambda R =0 $  } \\ 
{ 定解条件:......}


求解方程I:  根据以前的分析,在 $\lambda > 0 $ 时,有通解\\ 
{ $\displaystyle  \Theta(\theta)=A\cos \sqrt{\lambda \theta} +B\sin \sqrt{\lambda \theta}$}\\ 
由定解条件:$\displaystyle 	\Theta(2 \pi )=\Theta (0)~,~~ 	\Theta' (2 \pi )=\Theta' (0)  $ 得方程: \\ 
	$ \left [
\begin{array}{lll}
	\cos (\sqrt {\lambda} 2\pi )-1  & \sin (\sqrt {\lambda} 2\pi )\\
	-\sin (\sqrt {\lambda} 2\pi ) & \cos (\sqrt {\lambda} 2\pi )-1
\end{array} \right] 
\left [
\begin{array}{lll}
	A\\
	B
\end{array} \right] 
=
\left [
\begin{array}{lll}
	0\\
	0
\end{array} \right]
$ \\
系数行列式应为零,得\\
$(\cos (\sqrt {\lambda} 2\pi )-1 ) ^2 + \sin ^2 (\sqrt {\lambda} 2\pi ) =0$ \\ 
$\cos (\sqrt {\lambda} 2\pi)=1$   \\ 
固有值为:$\lambda _n =n^2 ~,~~ (n=0,1,2,...)$  \\ 
解为:$\displaystyle  \Theta(\theta)=A_n\cos n \theta +B_n \sin n \theta $\\ 

求解方程II:
{$\displaystyle  r^2 R'' +r R' -\lambda R =0 $  } \\ 
把$\lambda_n =n^2 $代入, 得 \\ 
{$\displaystyle  r^2 R'' +r R' -n^2R =0 $  } \\ 
令 $ r=exp(t) $ ,有 $t=\ln r$, 先求导 \\ 
$ \displaystyle \frac{dR}{dr} =\frac{dR}{dt} \frac{dt}{dr} =\frac{1}{r} \frac{dR}{dt} $ \\ 
$ \displaystyle \frac{d^2R}{dr^2} =-\frac{1}{r^2}\frac{dR}{dt} + \frac{1}{r} \frac{d}{dr} (\frac{dR}{dt} )$ ,得:\\ 
$ \displaystyle \frac{d^2R}{dr^2} =\frac{1}{r^2} (\frac{d^2R}{dt^2}-\frac{dR}{dt} )$ \\ 
代回方程,得:\\ 
$ \displaystyle   \frac{d^2R}{dt^2} -n^2 R =0 $ \\ 
由特征方程有两相异实根,得通解:\\ 
$ R=C_nexp(nt)+D_n exp(-nt) $\\
把 $t=\ln r$ 代回,得\\
$R=C_n r^n +D_nr^{-n}$ \\ 
第二项发散,应删除,得\\
$R= C_n r^n,  ~~ (n=0,1,2,......) $		

利用叠加原理, 得解:\\ 
	$\begin{array}{llll}
		u_n(r,\theta) &=& (a_n\cos n\theta +b_n \sin n \theta ) r^n  \\ 
		u(r, \theta) &=& \frac{1}{2} a_0 +\sum_{n=1}^{\infty } (a_n\cos n\theta +b_n \sin n \theta ) r^n
	\end{array}$ \\ 
 代入定解条件:$ u(r_0,\theta)=f (\theta)   $ \\ 
 $ =  \frac{1}{2} a_0 +\sum_{n=1}^{\infty } (a_n\cos n\theta +b_n \sin n \theta ) r_0^n $\\ 
 获得系数:\\ 
   $  \displaystyle  a_n = \frac{1}{r_0 ^n \pi }  \int\limits_{0}^{2\pi} f(\theta) \cos n \theta d\theta $ \\ 
  $  \displaystyle  b_n = \frac{1}{r_0 ^n \pi }  \int\limits_{0}^{2\pi} f(\theta) \sin n \theta d\theta $  \\ 
\end{proof}

\begin{example} %4
	求解如下边值问题\\
	{ $  \displaystyle  \left \{ 
		\begin{array}{cc}
			\displaystyle {	\frac{\partial^2 u }{\partial r^2 } +\frac{1}{r } \frac{\partial u }{\partial r } +
				\frac{1}{r^2 } \frac{\partial ^2 u }{\partial \theta ^2
			} } =0, ~~ 0<r<r_0\\
			u(r_0,\theta )=A\cos(\theta),~~~~~~~~~ 0<\theta <2\pi 
		\end{array}
		\right. $}  
\end{example}
\begin{proof} 
	求系数:\\
  $  \displaystyle  a_1 = \frac{1}{r_0 ^1 \pi }  \int\limits_{0}^{2\pi} A\cos(\theta) \cos  \theta d\theta  $ \\ 	
    $  \displaystyle  a_1 = \frac{A}{r_0  \pi }  \int\limits_{0}^{2\pi} \cos ^2 (\theta)  d\theta  = \frac{A}{r_0 1n \pi }  \int\limits_{0}^{2\pi} (1+\cos2\theta) d\theta$ = $\frac{A}{r_0}$ \\ 
  $  \displaystyle  a_n = \frac{1}{r_0 ^n \pi }  \int\limits_{0}^{2\pi} A\cos(\theta) \cos n \theta d\theta =0~,~ (n\ne 1)$ \\ 
 $  \displaystyle  b_n = \frac{1}{r_0 ^n \pi }  \int\limits_{0}^{2\pi} A\cos(\theta) \sin n \theta d\theta =0 $  \\ 
叠加解:\\
	$\begin{array}{llll}
	u(r, \theta) &=& \frac{1}{2} a_0 +\sum_{n=1}^{\infty } (a_n\cos n\theta +b_n \sin n \theta ) r^n \\
	                   &=& \frac{1}{2} a_0+ a_1 r\cos \theta \\
	                   &=&  \frac{A}{r_0} r \cos \theta 
\end{array}$ \\ 
\end{proof} 

\begin{remark}
若将边界条件修改为: $A \cos 2\theta$ ,或 $A \sin 2\theta $ ,考虑解会如何变化。
\end{remark}

\subsection{作业:}
~~~\hspace*{\fill} \\
1、求解固有值问题\\ 
	$\begin{array}{lllllllll}
	& \begin{cases}
		Y~^{''} +\lambda Y=0  ~~,~~ 0<y<2\pi \\
        Y(0) =Y(2\pi) , ~~ Y'(0) =Y'(2\pi)
	    \end{cases}\\	
\end{array}$ \\ 

 ~~~\hspace*{\fill} \\
2、求解边值问题\\
	$\displaystyle  \begin{array}{lllllllll}
	&\begin{cases}
	\frac{\partial^2 u }{\partial r^2 } +\frac{1}{r } \frac{\partial u }{\partial r } +
	\frac{1}{r^2 } \frac{\partial ^2 u }{\partial \theta ^2  } =0, ~~ 0<r<1\\
	u(1,\theta)= A\cos 2 \theta +B \cos 4 \theta \\	
\end{cases} \\	
\end{array}$ \\ 

 ~~~\hspace*{\fill} \\
3、求解边值问题\\
$\begin{array}{lllllllll}
	& \begin{cases}
		u_{xx} +u_{yy} =0 ,~~~~ (0<x, y<1)\\
		u(x,0)= u(0,y)=u(x,1)= 0 \\
		u(1,y)= \sin 2\pi y
	\end{cases}\\	
	&\begin{cases}
		u_{xx} +u_{yy} =0 ,~~~~ (0<x, y<1)\\
		u(1,y)= u(0,y)=u(x,0)= 0 \\
		u(x,1)= \sin n\pi x
	\end{cases} \\	
\end{array}$ \\ 

