\renewcommand{\thechapter}{}
\chapter{第三章~~薛定谔方程 I(6学时)} 
\renewcommand{\thechapter}{3}
 ~~~\hspace*{\fill} \\
\begin{itemize}
\item \textbf{主要内容}:Schrödinger方程分离变量法、量子谐振子与Hermite多项式,Hermite多项式的性质。
\item \textbf{重点和难点:}一维Schrödinger方程分解的物理意义,无限深势井的Schrödinger方程推导,Hermite多项式性质。
\item \textbf{掌握:}一维Schrödinger方程的分离变量法、量子谐振子模型与分析方法、Hermite多项式性质。
\item \textbf{理解:}Hermite多项式的递推公式。
\item \textbf{了解:}Hermite方程的求解过程。
\end{itemize}

薛定谔方程是量子力学的基本方程,地位相当于牛顿力学中的牛顿第二定理。薛丁谔方程的解称为波函数,完全描述体系的状态。

\section{薛定谔方程分离变量法}
\subsection{方程的建立}
\begin{example} %1
物质波的状态由波函数$\psi(\overrightarrow{r},t)$完全描述,试建立波函数所满足的方程 。\\
\textbf{ 观点1:}  最小作用量原理方法  \\ 
\textbf{ 观点2:}  波动性和粒子性相结合的方法  \\ 
\textbf{ 观点3:}  基本假设,不能从其他原理推导出来 \\
\end{example}

\subsection{含时薛定谔方程}
1926年,奥地利物理学家薛定谔,提出描述物质状态的波函数$\psi(\overrightarrow{r},t)$服从如下方程:\\
 \begin{equation*}
  i\hbar \frac{\partial }{\partial t} \Psi (\overrightarrow{r},t ) =\left [ -\frac{\hbar^2}{2\mu }\nabla ^2 + V(\overrightarrow{r}t ) \right ]\Psi (\overrightarrow{r}, t ) 
 \end{equation*}
若取 $ \displaystyle  i\hbar \frac{\partial }{\partial t} ~~\to ~~ \hat{E} ~~, ~~  -\frac{\hbar^2}{2\mu }\nabla ^2 + V(\vec{r},t ) ~~\to ~~ \hat{H} $ ,
方程有如下算符形式:   \\ 
 \begin{equation*}
  \hat{E} ~ \Psi (\overrightarrow{r},t )  = \hat{H} ~ \Psi (\overrightarrow{r},t )  
\end{equation*}

\begin{remark}
薛定谔方程为什么难求解?\\ 
实际体系的势函数:$V(\vec{r},t ) $一般比较复杂\\ 
多粒子体系的波函数:$ \displaystyle  \Psi (\vec{r},t ) =\Psi (\vec{r_1},\vec{r_2},...,\vec{r_n},t ) $  \\ 
多粒子体系的哈密顿量: $ \displaystyle  \hat{H} ~=\sum_{i=1}^{n} -\frac{\hbar^2}{2\mu }\nabla ^2 _i + \sum_{i,j=1, i\ne j}^{n}  V(\vec{r_i},\vec{r_j},t ) $ \\ 
分离变量法可能吗?条件呢?
\end{remark}

\subsection{定态薛定谔方程}
若势函数不显含时间 t,则时间变量可分离 \\
方程: { $ \displaystyle i \hbar \frac{\partial }{\partial t} \Psi (\vec{r},t ) =\left [- \frac{\hbar^2}{2\mu }\nabla ^2 + V(\vec{r}) \right ]\Psi (\vec{r},t ) $}  \\ 
\textbf{解:} 设  $\Psi (\vec{r},t )  = \Psi (\vec{r} ) f(t) $ , 代回方程 \\ 
{ $ \displaystyle i\hbar \Psi (\vec{r})  \frac{\partial }{\partial t} f(t)=f(t) \left [ -\frac{\hbar^2}{2\mu }\nabla ^2 + V(\vec{r}) \right ]\Psi (\vec{r}) $}  \\ 	
{ $ \displaystyle i\hbar \frac{1}{f(t)}  \frac{\partial }{\partial t} f(t)= \frac{1}{\Psi (\vec{r}) } \left [ -\frac{\hbar^2}{2\mu }\nabla ^2 + V(\vec{r}) \right ]\Psi (\vec{r}) =E $}  \\ 	
得两个微分方程:\\ 
{I、 $ \displaystyle  i\hbar \frac{1}{f(t)}  \frac{\partial }{\partial t} f(t)=E $ } \\ 
解方程,得:$\displaystyle  f(t) =e^{-iEt/\hbar}$ \\
{II、$\displaystyle   \left [ -\frac{\hbar^2}{2\mu }\nabla ^2 + V(\vec{r}) \right ]\Psi (\vec{r}) =E \Psi (\vec{r})  $  } \\ 
方程II称为定态薛定谔方程。
\subsection{单粒子定态薛定谔方程}
	{$\displaystyle   \left [ -\frac{\hbar^2}{2\mu }\nabla ^2 + V(x,y,z) \right ]\Psi (x,y,z) =E \Psi (x,y,z)  $  } \\ \vspace{0.3cm}
能否求解,依赖于势函数的具体形式..... \\ \vspace{0.3cm}
如果 $ V(x,y,z)=V_1(x)+V_2(y)+V_3(z) $,则方程可分离变量: \\ 
设: $\Psi(x,y,z)=X(x)Y(y)Z(z)$~~,~~ $ E=E_x+E_y+E_z$, 则: \\ 
{$\displaystyle  \frac{\partial^2  \Psi}{\partial  x^2}  =  \frac{\partial ^2 }{\partial  x^2} (XYZ) = \frac{d^2X}{ x^2} YZ ; ~~  \frac{\partial^2  \Psi}{\partial  y^2}  =\frac{d^2Y}{ y^2} ZX ; ~~  \frac{\partial^2  \Psi}{\partial  z^2}  = \frac{d^2Z}{ z^2} XY  $  } \\ 
方程变为: \\ 
{$\displaystyle  YZ \left [ -\frac{\hbar^2}{2\mu } \frac{d^2X}{dx^2} \right ]  +  ZX \left [ -\frac{\hbar^2}{2\mu } \frac{d^2Y}{dy^2} \right ]  + XY \left [ -\frac{\hbar^2}{2\mu } \frac{d^2Z}{dz^2} \right ] $}  \\ 
{$\displaystyle  =\left [ -(V_1-E_x) - (V_1-E_y) - (V_1-E_z)  \right ]  XYZ  $  } \\ 
两边同除以XYZ,得到三个一维薛定谔方程:\\ 
{ $ \displaystyle
\begin{cases}
		\left [ -\frac{\hbar^2}{2\mu} \frac{\mathrm{d} ^2}{\mathrm{d} x^2} +V_1(x) \right ]X(x)=E_xX(x) \\  
		\\
		\left [ -\frac{\hbar^2}{2\mu} \frac{\mathrm{d} ^2}{\mathrm{d} y^2} +V_2(y) \right ]Y(y)=E_yY(y) \\  
		\\
		\left [ -\frac{\hbar^2}{2\mu} \frac{\mathrm{d} ^2}{\mathrm{d} z^2} +V_3(z) \right ]Z(z)=E_zZ(z) \\
\end{cases}
	$}\\

\begin{remark}
	定态薛定谔方程容易求解吗?\\ 
	实际体系的势函数:$V(\vec{r}) $一般比较复杂\\ 
	多粒子体系的波函数:$ \displaystyle  \Psi (\vec{r},t ) =\Psi (\vec{r_1},\vec{r_2},...,\vec{r_n},t ) $  \\ 
	多粒子体系的哈密顿量: $ \displaystyle  \hat{H} ~=\sum_{i=1}^{n} -\frac{\hbar^2}{2\mu }\nabla ^2 _i + \sum_{i,j=1, i\ne j}^{n}  V(\vec{r_i},\vec{r_j},t ) $ \\ 
	分离变量法依然很难!!!
\end{remark}


\begin{example} %3
设有一粒子处于如下一维无限深势阱中,求解薛定谔方程\\
 { $ \displaystyle 
	V(x)=\left \{ 
	\begin{array}{cccc}
	     0	~~ ~~ 0<x<a \\  
		+\infty ~~x<0, x>a\\
	\end{array}
	\right.
	$} \\
\end{example}
\begin{proof}
势函数不显含时间t,只要求解如下定态薛定谔方程\\
{  $ \displaystyle 
	\left \{ 
	\begin{array}{cccc}
		\left [ -\frac{\hbar^2}{2\mu} \frac{\mathrm{d} ^2}{\mathrm{d} x^2} +0 \right ]\Psi(x)=E\Psi(x)  ~~ ~~ 0<x<a,~~~~~ (1)  \\ 
	     \\	
		\left [ -\frac{\hbar^2}{2\mu} \frac{\mathrm{d} ^2}{\mathrm{d} x^2} +\infty \right ]\Psi(x)=E\Psi(x)  ~~ ~~ x<0,~ x>a ~~~~~(2)  \\
	\end{array}
	\right.
	$} \\
很明显,方程(2)的解为  $\Psi(x) = 0$ \\ 
令 $ k^2= \frac{2\mu E}{\hbar ^2} $, 方程(1)可变为如下边值问题:  \\ 
{ $ \displaystyle 
\begin{cases}
		\Psi''(x) + k^2	\Psi(x)=0  \\
		\Psi(0)=0~,~~ \Psi(a)=0 ~~~~~
\end{cases}
	$} \\
很明显,特征方程有两虚根,所以方程的通解为:\\
$  \displaystyle	\Psi(x) = A\cos(kx) +B\sin(kx) $   \\ 
代入定解条件,得: A=0, $\sin ka =0$ \\ 
  $ka=n\pi $ => $k=\frac{n\pi}{a} = \sqrt{\frac{2\mu E}{\hbar ^2}}$    \\ 
得量子化能级:{ $E_n = \frac{n^2\pi^2\hbar^2}{2\mu a^2} (n=1,2,3,...)$}\\
固有函数为: $ \Psi_n(x) = B_n\sin(\frac{n\pi}{a}x) $ 	 \\ 
***归一化: $ \displaystyle \int_{0}^{a}  \Psi_n ^*(x)  \Psi_n(x)dx = \int_{0}^{a}  |B_n| ^2 \sin^2(\frac{n\pi}{a}x) dx =1$  \\ 
得系数:$B_n=\sqrt{\frac{2}{a}}$ \\ 
归一化固有函数: $ \Psi_n(x)= \sqrt{\frac{2}{a}} \sin(\frac{n\pi}{a}x) $ \\ 
则波函数为: \\
{  $ \displaystyle 
	\Psi_n(x,t)= \left \{ 
	\begin{array}{cccc}
		\sqrt{\frac{2}{a}} \sin(\frac{n\pi}{a}x) e^{-\frac{i}{\hbar} E_n t} ~~~~   0<x<a \\
		0 ~~~~~~~~~~~~~~~~~~~~~~~ x<0,x>a  
	\end{array}
	\right.
	$} \\
\end{proof}	

\begin{remark}
1、 金属中的自由电子被限制在一个有限的范围(金属内)运动,在金属外出现的概率为零,可粗略近似为无限深势阱 \\ 
2、 能级间隔:  $\triangle E = E_{n+1}-E_n=\frac{\pi^2 \hbar^2}{2\mu a^2} (2n+1)$   \\ 
$\bullet $ 当导线的直径(a)很小时,能级量子化非常明显。 \\ 
$\bullet $ 宏观物体的质量 ($\mu$) 很大,能级量子化现象消失。 \\ 
\\
%\usetikzlibrary {datavisualization.formats.functions} 
%\usetikzlibrary {datavisualization.formats.functions} 
\begin{tikzpicture}[baseline, scale=.6]
	\datavisualization [ scientific axes, 
	visualize as smooth line/.list={sin,cos,tan}, style sheet=strong colors,
	style sheet=vary dashing,
	sin={label in legend={text=$\Psi_1$}}, 
	cos={label in legend={text=$\Psi_2 $}}, 
	tan={label in legend={text=$\Psi_3 $}}, 
	data/format=function ]
	data [set=sin] {
		var x : interval [0:1];
        func y = sin(pi* \value x r ) * 1.414;
	}
	data [set=cos] {
		var x : interval [0:1];
       func y = sin(2*pi* \value x r ) * 1.414;
	}
	data [set=tan] {
		var x : interval [0:1];
       func y = sin(3*pi* \value x r ) * 1.414;
    };
\end{tikzpicture}
\begin{tikzpicture}[baseline, scale=.6]
	\datavisualization [ scientific axes, 
	visualize as smooth line/.list={sin,cos,tan}, style sheet=strong colors,
	style sheet=vary dashing,
	sin={label in legend={text=$|\Psi_1|^2$}}, 
	cos={label in legend={text=$|\Psi_2|^2$}}, 
	tan={label in legend={text=$|\Psi_3|^2$}}, 
	data/format=function ]
	data [set=sin] {
		var x : interval [0:1];
		func y = sin(pi* \value x r ) *sin(pi* \value x r ) *2;
	}
	data [set=cos] {
		var x : interval [0:1];
		func y = sin(2*pi* \value x r ) *sin(2*pi* \value x r ) *2;
	}
	data [set=tan] {
		var x : interval [0:1];
		func y = sin(3*pi* \value x r ) * sin(3*pi* \value x r ) *2;
	};
\end{tikzpicture}
\end{remark}

\begin{example} %3
	设有一粒子处于如下一维无限深势阱中,求解薛定谔方程\\
	{ $ \displaystyle 
		V(x)=\left \{ 
		\begin{array}{cccc}
			0	~~ ~~ |x|<a \\  
			+\infty ~~|x|>a\\
		\end{array}
		\right.
		$} \\
\end{example}
\begin{proof}	
势函数与上例存在平移关系, $x' =x+a/2$\\
有:  $  \sin(\frac{n\pi}{a}x') =\sin(\frac{n\pi}{a} (x+a/2)) =\sin \frac{n\pi}{a} x \cos \frac{n\pi}{2} + \cos \frac{n\pi}{a} x \sin \frac{n\pi}{2}  $ \\ 
n为偶数时, $ \Psi_{2m}(x)= B_{2m} \sin(\frac{2m\pi}{a}x) $,  $E_{2m} = \frac{2m^2\pi^2\hbar^2}{\mu a^2} $ \\
n为奇数时, $ \Psi_{2m+1}(x)= B_{2m+1} \cos(\frac{(2m+1)\pi}{a}x) $,  $E_{2m+1} = \frac{(2m+1)^2\pi^2\hbar^2}{2\mu a^2} $ \\
归一化,求出系数 ... 
\end{proof}	

\begin{example} %3
求解一维无限深势阱的非定常问题 \\
{ $ \displaystyle 
\begin{cases}
i\hbar \frac{\partial }{\partial t} \Psi = -\frac{\hbar^2}{2\mu } \frac{\partial ^2 \Psi }{\partial ^2  x ^2 } , ~~ (0<x<L, t>0) \\
\Psi (0,t) =0, ~~ \Psi (L,t) =0 \\
\Psi (x,0) =f(x)  \\
\end{cases}
$} \\
\end{example}
\begin{proof}	
令$\Psi (x,t) =\Psi (x) T(t) $ ,  代回方程, 得:\\
{ $ \displaystyle 
	\begin{cases}
		\Psi''(x) + k^2	\Psi(x)=0  \\
		\Psi(0)=0~,~~ \Psi(a)=0 ~~~~~
	\end{cases}
	$} \\
固有值:{ $E_n = \frac{n^2\pi^2\hbar^2}{2\mu a^2} (n=1,2,3,...)$}\\
固有函数: $ \Psi_n(x) = \sin(\frac{n\pi}{a}x) $ 	 \\ 
时间函数: $T_n(t)  = \exp(-i E_n t /\hbar) $ \\
方程的级数解为:  $ \Psi(x,t)  = \sum_{n=1}^{\infty}  B_n \exp(-i E_n t /\hbar)  \sin(\frac{n\pi}{a}x)  $ \\
取 t=0, 代入初值条件, 得:  $ f(x)= \sum_{n=1}^{\infty}  B_n \sin(\frac{n\pi}{a}x)  $ \\
得系数:  $ B_n= \frac{2}{a} \int_{0} ^{a}  \sin(\frac{n\pi}{a}x) dx, ~~ (n=1,2,3,...) $ \\
\end{proof}	

\subsection{作业:}
~~~\hspace*{\fill} \\
1、求定态薛定谔方程\\ 
$\begin{array}{lllllllll}
	& \begin{cases}
		\Psi'' (x) +\frac{2\mu E}{\hbar ^2} \Psi(x) =0,~~ |x|<a/2 \\
		\Psi(-a/2) =\Psi(a/2) =0\\
	\end{cases}\\	
\end{array}$ \\ 
2、求解初边值问题\\
$\begin{array}{lllllllll}
	& \begin{cases}
		i\hbar \frac{\partial }{\partial t} \Psi = -\frac{\hbar^2}{2\mu } \frac{\partial ^2 \Psi }{\partial ^2  x ^2 } , ~~ (0<x<L, t>0) \\
		\Psi (0,t) =0, ~~ \Psi (L,t) =0 \\
		\Psi (x,0) =f(x)  \\
	\end{cases}\\
\end{array}$ \\ 
3、求三维无限势阱问题\\
$ ~~~~	V(x,y,z)=\left \{ 
\begin{array}{cccc}
	0	~~ ,~~ 0<x,y,z<a \\  
	+\infty ,~~others\
\end{array}
\right.
 $ 